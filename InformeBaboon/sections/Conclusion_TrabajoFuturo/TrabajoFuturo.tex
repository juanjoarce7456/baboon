\section{Trabajo Futuro}
Durante el desarrollo del framework y su posterior utilización, surgieron nuevos
aspectos y posibles mejoras a desarrollar. A continuación se presentan dichas
mejoras:
\begin{itemize}
  \item Performance: el uso de técnicas de reflection y de programación
  orientada a aspectos brinda la posibilidad de realizar la inversión de control
  y de ofrecer una interfaz de usuario amigable para el programador de sistemas
  reactivos pero sacrifica performance. Es de interés investigar la existencia 
  de otras alternativas que permitan la implementación de la inversión de
  control de manera más performante.
  \item Utilización para desarrollo de sistemas distribuidos: analizar la
  posibilidad de centralizar la ejecución de la RdP y exponerla como un servicio
  para la coordinación de ejecución de sistemas remotos.
  \item Agregar soporte para otros dialectos de PNML: existen otros software de
  edición de RdP más potentes que TINA, cada uno con su dialecto de PNML. JPCM
  está preparado para agregar soporte para nuevos dialectos de forma simple.
  \item Soporte para ejecución de múltiples RdP: Se puede paralelizar la toma
  de decisiones mediante el uso de RdP jerárquicas o simplemente mediante el
  uso de múltiples RdP donde cada una modela una parte del sistema.
  \item Soporte para ejecución de otros tipos de modelo: en principio, dentro
  del monitor JPCM se podría ejecutar múltiples tipos de modelos (máquinas de
  estado finitas, máquinas de Turing, etc). Esta capacidad amplía la gama de
  usuarios interesados en la utilización de \nombreFramework Framework.
  \item Embeber el manejo e intercambio de datos entre procesos dentro del
  framework: Actualmente, el framework controla el flujo de control de las
  instrucciones del programa, pero el manejo de los datos es responsabilidad
  total del usuario. En la versión actual del framework, el manejo de la
  dinámica de los datos resulta engorroso y agrega una complejidad innecesaria
  al concepto de acción de software. Además de la acción concreta debe
  programarse el manejo del flujo de datos. El framework podría ofrecer
  interfaces para el intercambio de datos entre las acciones de software,
  facilitando aún más la creación de sistemas con \nombreFramework Framework.
\end{itemize}


