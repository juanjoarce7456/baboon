\section{Conclusión}

En este trabajo se realizó el diseño e implementación
\textit{\nombreFramework}, un framework para el desarrollo de sistemas
reactivos. Como resultado de este desarrollo se logró la centralización de la
gestión de los recursos, la concurrencia y la sincronización de hilos y,
además, se obtuvo la conducción del flujo de ejecución utilizando un modelo de
Red de Petri que implementa la lógica del sistema. A su vez, se desarrolló un
mecanismo de gestión de prioridad de ejecución de los hilos por medio de
políticas configurables por el usuario.

La centralización de gestión de recursos, manejo de concurrencia y
sincronización de hilos mencionada en el párrafo previo se obtuvo mediante:
\begin{itemize}
 \item La transformación del modelo de RdP en código interpretado.

 \item El desarrollo mecanismo de suscripción a disparos de transiciones.

 \item La implementación de colas de espera y suspensión de hilos.
\end{itemize}

Estas características fueron implementadas como parte de Java Petri Concurrency
 Monitor (JPCM), un monitor de concurrencia que ejecuta Redes de Petri haciendo
 uso de la ecuación generalizada desarrollada en
 \cite{Ecuacion_generalizada_LAC}.

Se obtuvo una arquitectura de framework que gestiona los eventos y mecanismos
de comunicación necesarios para desacoplar el modelo de RdP, el código de
usuario y el entorno.
Como resultado, se simplifica el diseño del software de usuario, el que queda
definido por:

\begin{itemize}
    \item El modelo de RdP
    \item El conjunto de acciones: Son porciones de código con una
    responsabilidad concreta y simple. Intercambian eventos con el entorno.
    \item El conjunto de eventos de acción: Contienen las reglas de traducción
    entre los eventos del entorno y eventos comprensibles para el modelo de
    RdP.
    \item El conjunto de suscripciones de acciones a eventos de acción
\end{itemize}

Se logró un diseño de framework no restrictivo sobre las herramientas
que ofrece el lenguaje de programación. En consecuencia, el usuario dispone
de todas las características de la programación orientada a objetos para el
diseño de su sistema.

Se implementó la inversión de control del framework mediante la utilización de
prácticas de \textit{Reflection} y \textit{Aspect Oriented Programming}. Como
resultado, el flujo de control del programa es responsabilidad del framework y
el código de usuario se centra en las funcionalidades concretas del sistema a implementar.

El código del framework se encuentra disponible de forma pública en repositorios
en la red (ver sección \ref{config_baboon_env}). En los repositorios mencionados
se encuentra también la documentación en formato \textit{Javadoc} y los casos de
test automatizados para las funcionalidades implementadas.

La utilización de Baboon Framework en conjunto con el proceso de diseño de
sistemas reactivos expuesto en \cite{Bentivegna-Ludemann} permite el diseño y
desarrollo de sistemas reactivos confiables, mantenibles y portables a
múltiples plataformas.
