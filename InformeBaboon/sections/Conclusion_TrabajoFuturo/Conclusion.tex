\section{Conclusión}

En este trabajo se realizó el diseño e implementación de
\textit{\nombreFramework \ Framework}, un framework para el desarrollo de
sistemas reactivos. Como resultado de este desarrollo se logró la centralización de la
gestión de los recursos, la concurrencia y la sincronización de hilos y,
además, se obtuvo la conducción del flujo de ejecución utilizando un modelo de
Red de Petri que implementa la lógica del sistema. A su vez, se desarrolló un
mecanismo de gestión de prioridad de ejecución de los hilos por medio de
políticas configurables por el usuario. En consecuencia, se cumplió con
el objetivo principal del proyecto, descripto en la sección
\ref{sec:objetivo_principal}.

El proceso de diseño e implementación también estuvo guiado por los objetivos
secundarios planteados en la sección \ref{sec:objetivos_secundarios}. A lo largo
de los siguientes párrafos se explica cómo se cumplieron estos objetivos.

La centralización de gestión de recursos, manejo de
concurrencia y sincronización de hilos mencionada previamente se obtuvo
mediante:
\begin{itemize}
 \item La transformación del modelo de RdP en código interpretado.

 \item El desarrollo de un mecanismo de suscripción a disparos de transiciones.

 \item La implementación de colas de espera y suspensión de hilos.
\end{itemize}

Estas características fueron implementadas como parte de
\javapetriconcurrencymonitor (JPCM), un monitor de concurrencia que ejecuta
Redes de Petri haciendo uso de la ecuación generalizada desarrollada en
\cite{Ecuacion_generalizada_LAC}.

Se obtuvo una arquitectura de framework que gestiona los eventos y mecanismos
de comunicación necesarios para desacoplar el modelo de RdP, el código de
usuario y el entorno.
Como resultado, se simplifica el diseño del software de usuario, el que queda
definido por:

\begin{itemize}
    \item El modelo de RdP
    \item El conjunto de acciones: Son porciones de código con una
    responsabilidad concreta y simple. Intercambian eventos con el entorno.
    \item El conjunto de eventos de acción: Contienen las reglas de traducción
    entre los eventos del entorno y eventos comprensibles para el modelo de
    RdP.
    \item El conjunto de suscripciones de acciones a eventos de acción
\end{itemize}

Se logró un diseño de framework no restrictivo sobre las herramientas
que ofrece el lenguaje de programación. En consecuencia, el usuario dispone
de todas las características de la programación orientada a objetos para el
diseño de su sistema.

Se implementó la inversión de control del framework mediante la utilización de
prácticas de \textit{Reflection} y \textit{Aspect Oriented Programming}. Como
resultado, el flujo de control del programa es responsabilidad del framework y
el código de usuario se centra en las funcionalidades concretas del sistema a implementar.

El código del framework se encuentra disponible de forma pública en repositorios
en la red (ver sección \ref{config_baboon_env}) bajo licencia Apache 2.0. En los
repositorios mencionados se encuentra también la documentación en formato \textit{Javadoc} y los casos de
test automatizados para las funcionalidades implementadas.

El desarrollo del framework se realizó siguiendo los requerimientos planteados
en la sección \ref{sec:definicion_reqs}. A continuación se presenta un listado
de los resultados obtenidos:
\begin{itemize}
  \item El framework interactúa con JPCM, el monitor de Redes de Petri
  desarrollado en este proyecto.
  \item El control del flujo de ejecución es delegado al monitor de RdP
  mediante la utilización de prácticas de \textit{Reflection} y \textit{Aspect
  Oriented Programming}.
  \item Se requiere aprender menos de diez conceptos para desarrollar un
  sistema utilizando Baboon Framework (task controller, happening
  controller, complex sequential task controller, interfaz declare, interfaz
  subscribe, transition policies, topic, guard provider, RdP).
  \item Se desarrollaron diversos casos de uso, disponibles de forma pública en
  el repositorio de ejemplos (\url{\repoEjemplos}). En estos ejemplos se muestra
  la forma de uso de todas las interfaces de usuario del framework.
  \item El framework utiliza las interfaces que ofrece JPCM para interactuar con
  el modelo de RdP.
  \item Actualmente el framework no es compatible con otros monitores
  desarrollados en el Laboratorio de Arquitectura de Computadoras, pero su
  arquitectura fue diseñada para adaptarse a un cambio de monitor de forma
  simple.
  \item El reporte de cobertura de los casos de prueba arroja un resultado del
  82,5\% de líneas de código cubiertas. Este porcentaje supera al requerido.
  \item El framework tiene al menos un test automatizado por cada funcionalidad
  implementada.
  \item Se realizó documentación de código utilizando el formato estándar
  \textit{Javadoc}.
  \item Todos los objetos del framework tienen documentación sobre sus métodos,
  explicando la forma de uso y su función.
  \item La documentación no brinda detalles de implementación.
  \item La documentación se encuentra disponible de forma pública. Ver la
  sección \ref{genJavadoc}
\end{itemize}

A modo de corolario se menciona que la utilización de Baboon Framework
en conjunto con el proceso de diseño de sistemas reactivos expuesto en
\cite{Bentivegna-Ludemann} permite el diseño y desarrollo de sistemas reactivos
confiables, mantenibles y portables a múltiples plataformas.
