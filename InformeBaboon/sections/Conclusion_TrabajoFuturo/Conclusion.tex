\section{Conclusión}

Como se planteó inicialmente en los objetivos, se logró construir un framework
para el desarrollo de sistemas reactivos capaz de manejar la concurrencia de
forma centralizada ejecutando una red de Petri. El software desarrollado por un
usuario se ejecuta concurrentemente según el modelo lógico provisto (RdP),
gestionando la concurrencia de manera transparente.

El desarrollo del monitor de concurrencia para redes de Petri (JPCM) cumple con
el objetivo de ejecutar RdP utilizando la ecuación generalizada presentada por
\cite{Ecuacion_generalizada_LAC}.

Las interfaces expuestas por el framework permiten que el código del usuario
explote todas las características de la orientación a objetos (no así el
desarrollado en \cite{chimp}) y abstraen al usuario de la RdP gracias a la capa
intermedia provista por los eventos de acción (ver sección
\ref{sec:diseno_topicos}). Además, se implementó la inversión de control del
framework mediante la utilización de prácticas de Reflection y de Aspect
Oriented Programming.

Por otro lado, se cumplieron las expectativas de complejidad de uso de Baboon
Framework. El usuario debe aprender 10 conceptos para poder utilizarlo:
\begin{enumerate}
  \item HappeningController
  \item TaskController
  \item GuardProvider
  \item ComplexSecuentialTask
  \item Políticas de Prioridad de Transición
  \item Tópicos
  \item Interfaz BaboonApplication
  \item Interfaz de suscripción de acciones simples.
  \item Interfaz de creación de tareas complejas.
  \item Interfaz para agregar una tarea a una tarea compleja.
\end{enumerate}
De esta manera, se cumple con el límite planteado inicialmente en los
requerimientos.

Los requerimientos de testing, disponibilidad de código y documentación también
fueron resueltos. El código del framework se encuentra disponible
públicamente en repositorios en la nube. Dentro de dichos repositorios se
encuentra también la documentación en formato Javadoc y los casos de test
automatizados para las funcionalidades implementadas.

La utilización de Baboon Framework en conjunto con el proceso de diseño de
sistemas reactivos expuesto en \cite{Bentivegna-Ludemann} permite el diseño y
desarrollo de sistemas reactivos confiables, mantenibles y portables a
múltiples plataformas (por estar desarrollados en lenguaje Java).
