\section{Conclusión}

En este trabajo se logró construir \textit{\nombreFramework}, un framework para
el desarrollo de sistemas reactivos capaz de manejar la gestión de recursos, la
concurrencia y la sincronización de hilos de forma centralizada. El flujo de
ejecución es conducido a través de un modelo de Red de Petri que implementa la
lógica del sistema.
La gestión de prioridad de ejecución de los hilos se implementa a través de
políticas de prioridad a cargo del usuario, de forma separada al modelo.

El framework transforma el modelo de RdP en un código interpretado simplificando
el diseño del software del usuario. En consecuencia la implementación de la
concurrencia y gestión de recursos se realiza únicamente a través de la RdP.

Dada la necesidad de ejecutar Redes de Petri se construyó \textit{Java Petri
Concurrency Monitor} (JPCM), un monitor de concurrencia que implementa la
ecuación generalizada desarrollada en \cite{Ecuacion_generalizada_LAC}.

JPCM cuenta con un mecanismo de suscripción a disparos de transiciones que
permite la existencia de observers asíncronos a la ejecución de la RdP.
Además implementa las colas de espera y suspensión de hilos necesarias para 
la gestión de concurrencia y sincronización mencionadas previamente.

La comunicación entre los componentes de \textit{\nombreFramework Framework} y
los componentes del software del usuario se realiza mediante el intercambio de
eventos. De forma similar, el software del usuario es responsable de emitir y
recibir eventos del mundo físico.

El sistema de usuario queda comprendido por:
\begin{itemize}
    \item El modelo de RdP
    \item El conjunto de acciones: Son porciones de código con una
    responsabilidad concreta y simple. Intercambian eventos con el mundo físico.
    \item El conjunto de eventos de acción: Contienen las reglas de traducción
    entre los eventos del mundo físico y eventos comprensibles para el modelo de
    RdP.
    \item El conjunto de suscripciones de acciones a eventos de acción
\end{itemize}

Los eventos de acción funcionan como una capa de abstracción para el programador
de acciones, evitando la necesidad de conocer detalles del modelo de RdP.

\textit{\nombreFramework Framework} está diseñado para no imponer restricciones
sobre las herramientas que ofrece el lenguaje de programación. Permitiendo así
al usuario utilizar todas las características de la programación orientada a
objetos en el diseño de su sistema.

Se implementó la inversión de control del framework mediante la utilización de
prácticas de \textit{Reflection} y \textit{Aspect Oriented Programming}. Como
resultado, el flujo de control del programa es responsabilidad únicamente del
framework y el código de usuario se centra en las funcionalidades concretas del
sistema a implementar.

El código del framework se encuentra disponible de forma pública en repositorios
en la red (ver sección \ref{config_baboon_env}). En dichos repositorios se
encuentra también la documentación en formato \textit{Javadoc} y los casos de
test automatizados para las funcionalidades implementadas.

La utilización de Baboon Framework en conjunto con el proceso de diseño de
sistemas reactivos expuesto en \cite{Bentivegna-Ludemann} permite el diseño y
desarrollo de sistemas reactivos confiables, mantenibles y portables a
múltiples plataformas.
