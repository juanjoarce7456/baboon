\section{Disparos perennes}
Un disparo de transición perenne se define como un disparo que no ocasiona el
bloqueo del hilo que lo realiza. Es decir que por más que las condiciones no
estén dadas para disparar una transición, si se la dispara de forma perenne el
hilo no será dormido en la cola de condición de la transición (ni en ninguna
otra cola).
De la definición de disparo perenne se llega a la conclusión de que este
tipo de disparo no es útil a la hora de realizar petición de ejecución al
monitor. En cambio, es una herramienta que permite realizar avisos asincrónicos
o realizar cambios en el flujo de ejecución desde fuera del monitor. A su vez,
al realizar un disparo perenne se puede despertar a otros hilos durmiendo en una
transición para que realicen una tarea si es que están disponibles para
hacerlo. A la tarea o hilo que realiza el disparo perenne no le interesa el
resultado del mismo, y este resultado no modifica su propia ejecución debido a
que el disparo siempre arrojará un resultado exitoso a quien lo realice y no
bloqueará dicho hilo.

Dentro del contexto de Baboon, un evento interno puede definirse como el
disparo de una transición 
