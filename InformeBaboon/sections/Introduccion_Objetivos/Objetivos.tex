\section{Objetivo General}
El objetivo general de este proyecto integrador es diseñar e implementar un
framework que permita desarrollar sistemas reactivos utilizando modelos basados
en Redes de Petri no autónomas. El framework resultante debe aislar el control
de la aplicación en un módulo que ejecute a la RdP. Este módulo debe manejar el
asincronismo del sistema de forma transparente al código de software de la
aplicación, delegando las decisiones a la RdP, tanto de la lógica del sistema
como de la política.


\section{Objetivos Secundarios}
\label{sec:objetivos_secundarios}
A continuación se mencionan los objetivos secundarios de este proyecto:
\begin{itemize}
  \item Reutilizar en la etapa de implementación de un sistema reactivo el
  modelo lógico del mismo para garantizar su correcto funcionamiento.
  \item Separar la lógica del sistema del código que implementa sus
  funcionalidades.
  \item Estudiar Redes de Petri ordinarias, orientadas a procesos y no
  autónomas.
  \item Investigar implementaciones en Proyectos Integradores previos y analizar
  la reutilización o reimplementación del código existente.
  \item Obtener un software capaz de ejecutar RdP utilizando la ecuación de
  estado generalizada descrita en \cite{Ecuacion_generalizada_LAC}.
  \item Implementar la inversión de control del framework.
  \item Priorizar la mantenibilidad del código generado respetando estándares y
  estilos de programación.
  \item Generar tests automáticos para garantizar el correcto funcionamiento de
  las funcionalidades del software generado
  \item Documentar detalladamente el código fuente siguiendo una metodología
  estándar.
  \item Ofrecer interfaces de programación sencillas.
  \item Resolver problemas conocidos de programación concurrente para poner a
  prueba el framework.
  \item Resolver un problema real de concurrencia utilizando el framework
  desarrollado.
  \item Documentar el proceso de desarrollo de aplicaciones particulares
  utilizando el framework.
  \item Ofrecer ejemplos de uso del framework para facilitar la asimilación de
  los usuarios.
  \item Ofrecer el software resultante de forma pública y accesible para
  cualquier usuario en cualquier parte del mundo.
\end{itemize}

\section{Análisis Previo}
Antes de la definición del objetivo general se realizó un trabajo de
investigación y comparación para determinar el tipo de herramienta a
desarrollar (Biblioteca de software, API, Framework, Generación de código,
etc). (ver capítulo \ref{cap:investigacion})