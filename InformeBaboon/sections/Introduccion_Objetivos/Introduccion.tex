\section{Introducción}
En la actualidad, existe un crecimiento en la complejidad de los sistemas
reactivos debido a un constante aumento en la interacción entre los procesos y
su entorno. Dichas interacciones se representan como eventos de software
asincrónicos respecto a la ejecución de dichos sistemas. Estos eventos
desencadenan cambios en el estado global de los sistemas, que a su vez,
acarrean problemas de concurrencia.
El manejo del asincronismo y la concurrencia de estos sistemas sin las
herramientas adecuadas es una tarea por demás compleja, propensa a la
generación de errores de ejecución si no se realiza con especial precaución.
En general, los sistemas más interactivos son: los sistemas embebidos, las
interfaces gráficas, los servicios en la nube y los dispositivos IoT. Todos
estos sistemas entran en la clasificación de aplicaciones dirigidas por eventos
o event-driven applications.\cite{chimp} Los mismos se presentan en
cantidades masivas y crecientes en la actualidad.
Este proyecto integrador pretende diseñar e implementar un framework para el
desarrollo de aplicaciones, utilizando redes de Petri como mecanismo de
modelado y ejecución para procesar eventos y representar los estados globales y
locales del sistema. Es decir, se pretende utilizar una Red de Petri como
modelo de la lógica del sistema y de sus interacciones con el medio.

En el marco del Laboratorio de Arquitecturas de Computadoras se han realizado
numerosos proyectos siguiendo esta línea de trabajo. Se destacan:
“Implementación de un sistema multicore heterogéneo embebido con procesador de
Petri sobre FPGA” \cite{Gallia-Pereyra}, “Desarrollo de un IP core con
procesamiento de Redes de Petri Temporales para sistemas multicore en FPGA”
\cite{Nonino-Pisetta-Micolini}, “Reducción de recursos en un procesador de redes
de Petri implementado en un IP Core” \cite{Birocco-Arlettaz-Micolini},
“Modularización del Procesador de Petri y Optimización para Sistemas Embebidos”
\cite{Daniele} y “Estudio e Implementación de un Caso Testigo para el
Desarrollo de Sistemas Embebidos, Críticos y Reactivos” \cite{Bentivegna-Ludemann}.
Por otro lado y dentro del mismo marco, dos trabajos sentaron las bases de este
proyecto integrador. En primer lugar “Generación de Código de Sistemas Reales,
Paralelos y Concurrentes a partir de Redes de Petri Orientadas a Procesos”
\cite{codegen} propone un esquema básico de comunicación entre el software y la
red de Petri mediante colas de entrada, de salida y etiquetas.También permite
generar código estático a partir de una red de Petri y una configuración. En
segundo lugar, “Desarrollo de un Framework para Aplicar el Paradigma de
Programación Reactiva Utilizando Redes de Petri como Procesador de
Eventos”\cite{chimp} donde se introduce la idea de un framework para el
desarrollo de sistemas reactivos dirigido por RdP.
El objetivo de este trabajo es diseñar e implementar un nuevo framework. Se
tienen en cuenta ciertos aspectos de diseño desarrollados en \cite{chimp},
pero a su vez se realiza un amplio trabajo de rediseño. Los principales puntos
a destacar del nuevo diseño son: el desacoplamiento entre la red de petri (RdP)
y la aplicación de usuario, la modificación del modo de sincronización utilizado
para las comunicaciones con la RdP, el énfasis en mantener la inversión de
control, la no imposición de restricciones al usuario desarrollador sobre las
herramientas propias del lenguaje, la expansión de funcionalidades del monitor
de RdP para adoptar la ecuación de estado generalizada descrita en
\cite{Ecuacion_generalizada_LAC} y la simplificación de las interfaces de
programación ofrecidas al usuario desarrollador. Debido a los cambios en el
diseño mencionados se realiza una re-implementación total del código del
framework. A lo largo de este informe se presenta el proceso de diseño e
implementación del framework en su totalidad, se brindan las bases teóricas
necesarias para la comprensión del mismo y se ejemplifica con casos de uso en
aquellos puntos donde los autores lo consideran necesario.
