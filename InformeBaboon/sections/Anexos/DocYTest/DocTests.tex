\section{Generación de la documentación en formato HTML}
\label{genJavadoc}
Junto con el código fuente desarrollado se encuentra en formato Javadoc la
documentación del mismo. El código mencionado se obtiene de los repositorios
ubicados en:
\begin{itemize}
  \item JPCM: \url{\repoMonitor}
  \item \nombreFramework Framework: \url{\repoFramework}
\end{itemize}

 Al ser Javadoc un formato estándar existen intérpretes
que lo llevan a una forma legible para desarrolladores de software.
La IDE Eclipse incluye una herramienta que realiza esta tarea, mediante la cual
se construye la documentación de \nombreFramework Framework.

Para utilizar esta herramienta deben realizarse los siguientes pasos:
\begin{itemize}
  \item Dirigirse al menú \textit{Project}
  \item Seleccionar la opción \textit{Generate Javadoc\ldots}
  \item Del árbol del proyecto seleccionar el directorio \textit{/src}
  \item Elegir el nivel de visibilidad deseado (\textit{public} para usuarios
  del framework, \textit{private} para desarrolladores)
  \item Elegir la ubicación donde se generará la documentación
  \item Hacer click en \textit{Finish}
\end{itemize}

\section{Generación de la documentación de Test}
En los repositorios mencionados previamente se encuentra el código fuente de los
casos de prueba automatizados. A fin de conocer estos casos, se genera la
documentación de los mismos.
Esto se realiza siguiendo los pasos descriptos en la sección \ref{genJavadoc}
con la salvedad de que se elige la carpeta \textit{/test} del árbol del proyecto.

De esta manera se genera una página HTML que contiene la descripción de los casos
de test desarrollados. La descripción mencionada explicita \textit{precondiciones}, \textit{acciones} y
\textit{resultados esperados} de cada caso de prueba.

Para ejecutar los casos de prueba, debe ejecutarse el directorio \textit{/test}
utilizando JUnit.
