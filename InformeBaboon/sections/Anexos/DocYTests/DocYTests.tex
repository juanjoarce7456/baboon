\section{Documentación del Código Fuente}
\label{genJavadoc}
Junto con el código fuente desarrollado se encuentra en formato Javadoc la
documentación del mismo. El código mencionado se obtiene de los repositorios
ubicados en:
\begin{itemize}
  \item JPCM: \url{\repoMonitor}
  \item \nombreFramework: \url{\repoFramework}
\end{itemize}

Cada repositorio tiene un botón en la página de inicio que lleva a su
documentación online.

\subsection{Generación de la Documentación}
Si se desea construir la documentación, al ser Javadoc un formato estándar,
existen intérpretes que lo llevan a una forma legible para desarrolladores de
software. Múltiples IDEs incluyen herramientas que realiza esta tarea.
Para utilizar esta herramienta deben realizarse los siguientes pasos:

\subsubsection* {Eclipse}
\begin{itemize}
  \item Dirigirse al menú \textit{Project}
  \item Seleccionar la opción \textit{Generate Javadoc\ldots}
  \item Del árbol del proyecto seleccionar el directorio \textit{/src}
  \item Elegir el nivel de visibilidad deseado (\textit{public} para usuarios
  del framework, \textit{private} para desarrolladores)
  \item Elegir la ubicación donde se generará la documentación
  \item Hacer click en \textit{Finish}
\end{itemize}

\subsubsection* {IntelliJ}
\begin{itemize}
  \item Dirigirse al menú \textit{Tools}
  \item Seleccionar la opción \textit{Generate JavaDoc\ldots}
  \item Sleccionar \textit{Custom Scope} y especificar \textit{Project
  Production Files}
  \item Destildar la opción \textit{Include test sources}
  \item Elegir el nivel de visibilidad deseado (\textit{public} para usuarios
  del framework, \textit{private} para desarrolladores)
  \item En el campo \textit{Output Directory} Elegir la ubicación donde se
  generará la documentación
  \item Hacer click en \textit{Ok}
\end{itemize}

\section{Generación de la documentación de Test}
En los repositorios mencionados previamente se encuentra el código fuente de los
casos de prueba automatizados. A fin de conocer estos casos, se genera la
documentación de los mismos.
Esto se realiza siguiendo los pasos descriptos en la sección \ref{genJavadoc}
con la salvedad de que se elige la carpeta \textit{/test} del árbol del proyecto
en Eclipse, y se especifica \textit{Custom Scope} como \textit{Project
  Test Files} en IntelliJ.

De esta manera se genera una página HTML que contiene la descripción de los casos
de test desarrollados. La descripción mencionada explicita
\textit{precondiciones}, \textit{acciones} y \textit{resultados esperados} de
cada caso de prueba.

Para ejecutar los casos de prueba, debe ejecutarse el directorio \textit{/test}
utilizando JUnit ó mediante Maven ejecutando el comando \textit{mvn test}.
