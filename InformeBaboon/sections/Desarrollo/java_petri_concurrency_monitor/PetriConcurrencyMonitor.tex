\section{Introduccion}
\label{sec:petri_concurrency_monitor_intro}
Como se dijo en la sección~\ref{sec:objetivos_secundarios}, uno de los
objetivos de este proyecto integrador es modelar sistemas reactivos para garantizar su
correcto diseño.
Para esto se propone modelar el sistema con RdP (ver sección \ref{redes_de_petri}).
Otro objetivo planteado es separar la lógica de un sistema del código que
implementa sus funcionalidades y su política de gestión de hilos.
Teniendo ambos objetivos es cuenta es que se propone que sea el propio modelo
quien guíe la ejecución lógica del sistema.

Por su naturaleza, los sistemas reactivos son concurrentes y envían y reciben
eventos del mundo exterior.
Por esto se debe hacer una buena gestión de la concurrencia para asegurar su correcto
funcionamiento, utilizando alguna de las técnicas vistas en
la sección \ref{ProgramacionConcurrente} y para modelarlos se debe incluir en
el modelo el envío y recepción de eventos.

Como el modelo de concurrencia obtenido por las RdP es centralizado, resulta
casi natural ejecutar una RdP como lógica secuencial de un sistema reactivo
dentro de un monitor de concurrencia. Luego, para relacionar los eventos con el
monitor se utilizan RdP no autónomas.

\section{\javapetriconcurrencymonitor}
\label{sec:java_petri_concurrency_monitor}

\javapetriconcurrencymonitor es un monitor de concurrencia que ejecuta Redes
de Petri, hecho en lenguaje de programación Java.
Provee al usuario de una interfaz de programación de aplicaciones (API) para
ejecutar una RdP, pretegiéndola de los problemas de concurrencia con la
exclusion mutua del monitor.

Brinda soporte para:
\begin{itemize}
  \item Redes de Petri:
  \begin{itemize}
    \item Plaza-Transición
    \item Temporales
  \end{itemize}
  
  \item Tipos de arcos;
  \begin{itemize}
    \item Normal
    \item Lector o de Prueba
    \item Inhibidor
    \item Reset
  \end{itemize}
  
  \item Carga de RdP en lenguaje PNML (dialecto de TINA)
  \item Guardas
  \item Transitiones automáticas
  \item Subscripción a eventos en transiciones informadas
  \item Políticas intercambiables y extensibles de prioridad de disparo
  \item Disparos perennes y no-perennes

\end{itemize}
