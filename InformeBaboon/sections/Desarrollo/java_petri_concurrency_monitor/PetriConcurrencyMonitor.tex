\section{Introduccion}
\label{sec:petri_concurrency_monitor_intro}
Como el modelo de concurrencia obtenido por las RdP es centralizado, resulta
directo ejecutar una RdP como lógica secuencial de un sistema reactivo dentro
de un monitor de concurrencia. Luego, para relacionar los eventos con el
monitor se utilizan RdP no autónomas. Por lo cual en este apartado desarrollamos
un monitor de concurrencia con Redes de Petri.

\section{Requerimientos del monitor}
Del estudio de \cite{Ecuacion_generalizada_LAC}, \cite{codegen}, \cite{chimp} y
\cite{Bentivegna-Ludemann} emergen los requerimientos del monitor:
\begin{enumerate}
  \item Debe ofrecer interfaces para la carga de una Red de Petri en formato
  estándar PNML (dialecto Tina).
  \item Debe soportar Redes de Petri ordinarias y temporales.
  \item Debe ofrecer interfaces para el disparo de transiciones.
  \item Debe implementar la ecuación de estado generalizada descrita en
  \cite{Ecuacion_generalizada_LAC} para la ejecución de RdP.
  \item Debe soportar las etiquetas de transición definidas en \cite{codegen}
  (transiciones automáticas/disparadas y transiciones informadas/no
  informadas).\\
  Esto implica los siguientes requerimientos:
  \begin{enumerate}
    \item Debe ofrecer interfaces para la suscripción a informes de transiciones
    informadas.
    \item Debe prohibir el disparo manual de transiciones automáticas.
    \item Debe disparar automáticamente transiciones automáticas sensibilizadas.
  \end{enumerate}
  \item Debe ofrecer interfaces para la carga de políticas de prioridad de
  disparo de transiciones.
  \item Debe brindar alguna forma de asociar guardas con transiciones.
  \item Debe ofrecer interfaces para la modificación del valor de guardas.
  \item Debería soportar disparos perennes y no perennes.
  \item Debería soportar arcos normales, de reset, lectores e inhibidores.
\end{enumerate}

\section{\javapetriconcurrencymonitor}
\label{sec:java_petri_concurrency_monitor}

\javapetriconcurrencymonitor es un monitor de concurrencia que ejecuta Redes
de Petri, hecho en lenguaje de programación Java.
Provee al usuario de una interfaz de programación de aplicaciones (API) para
ejecutar una RdP, pretegiéndola de los problemas de concurrencia con la
exclusion mutua del monitor.
