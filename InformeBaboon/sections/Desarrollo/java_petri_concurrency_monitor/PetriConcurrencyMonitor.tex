\section{Introduccion}
\label{sec:petri_concurrency_monitor_intro}
Como el modelo de concurrencia obtenido por las RdP es centralizado, resulta
directo ejecutar una RdP como lógica secuencial de un sistema reactivo dentro
de un monitor de concurrencia. Luego, para relacionar los eventos con el
monitor se utilizan RdP no autónomas. Por lo cual en este apartado desarrollamos
un monitor de concurrencia con Redes de Petri.

\section{Requerimientos del monitor}
Del estudio de \cite{Ecuacion_generalizada_LAC}, \cite{codegen}, \cite{chimp} y
\cite{Bentivegna-Ludemann} emergen los requerimientos del monitor:
    \begin{enumerate}
      \item El monitor debe ofrecer interfaces para la carga de una Red de
      Petri en formato PNML (dialecto Tina).
      \item El monitor debe implementar la ecuación de estado generalizada
      descrita en \cite{Ecuacion_generalizada_LAC}.
      \item El monitor debe ofrecer interfaces para la carga de políticas de
      prioridad de disparo de transiciones.
      \item El monitor debe soportar Redes de Petri temporales y de
      plaza-transición.
      \item El monitor debe ofrecer interfaces para el disparo de
      transiciones.
      \item El monitor debe ofrecer interfaces para la modificación del
      valor de guardas.
      \item El monitor debe soportar disparos perennes y no perennes.
      \item El monitor debe soportar etiquetas de transición Disparada -
      Informada \cite{codegen}.
      \item El monitor debe soportar etiquetas de guarda en las
      transiciones.
      \item El monitor debe soportar arcos normales, de reset, lectores e
      inhibidores.
      \item El monitor debe ofrecer interfaces para la suscripción a
      informes de transiciones informadas.
      \item El monitor debe ser desarrollado en Java, solicitado por el
      Director del proyecto.
    \end{enumerate}

\section{\javapetriconcurrencymonitor}
\label{sec:java_petri_concurrency_monitor}

\javapetriconcurrencymonitor es un monitor de concurrencia que ejecuta Redes
de Petri, hecho en lenguaje de programación Java.
Provee al usuario de una interfaz de programación de aplicaciones (API) para
ejecutar una RdP, pretegiéndola de los problemas de concurrencia con la
exclusion mutua del monitor.

Brinda soporte para:
\begin{itemize}
  \item Redes de Petri:
  \begin{itemize}
    \item Plaza-Transición
    \item Temporales
  \end{itemize}
  
  \item Tipos de arcos;
  \begin{itemize}
    \item Normal
    \item Lector o de Prueba
    \item Inhibidor
    \item Reset
  \end{itemize}
  
  \item Carga de RdP en lenguaje PNML (dialecto de TINA)
  \item Guardas
  \item Transitiones automáticas
  \item Subscripción a eventos en transiciones informadas
  \item Políticas intercambiables y extensibles de prioridad de disparo
  \item Disparos perennes y no-perennes

\end{itemize}
