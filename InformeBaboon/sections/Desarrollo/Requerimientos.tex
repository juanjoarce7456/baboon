\section{Introducción}
En este capítulo se hace una definición priorizada de los requerimientos que
debe cumplir el framework a desarrollar.
\section {Prioridades de los requerimientos}
    Los requerimientos se priorizan utilizando la semántica expuesta en la
    Tabla~\ref{tab:semantica_requerimientos}, descripta en
    \cite{programacionUml}.
    
\begin{table}[H]
    \centering
    \begin{tabularx}{\textwidth}{ | p{4cm} | X | X | }
    \hline
        \textbf{Atributo de Prioridad} & \textbf{Semántica} 
        \\[10pt]\hline
        Debe Tener & Requisitos obligatorios que son fundamentales para el
        sistema
        \\[10pt] \hline
        Debería tener & Requisitos importantes que se pueden omitir
        \\[10pt] \hline
        Podría tener & Requisitos que son opcionales (se realizan si hay tiempo)
        \\[10pt] \hline
        Quiere tener & Requisitos que pueden esperar para versiones posteriores
        del sistema
        \\[10pt] \hline
    \end{tabularx}
    \caption{Semántica de los Requerimientos}
    \label{tab:semantica_requerimientos}
\end{table}
    
\section{Definición de Requerimientos}
\label{sec:definicion_reqs}
A continuación se definen los requerimientos del framework a desarrollar:
\begin{enumerate}
    \item El framework debe contar con un monitor de Redes de Petri.
    \item El framework debe delegar el control del flujo de ejecución en el
    monitor de Redes de Petri.
    \item Para un usuario con conocimiento intermedio en Java y Redes de Petri, el
    framework podría aprender a usarse en una semana o menos.
        \begin{itemize}
            \item La utilización del sistema podría incorporar como máximo diez
            conceptos nuevos a aprender por un usuario con un nivel intermedio
            de conocimientos en Java y redes de Petri.
            \item El sistema debería ser acompañado con al menos dos ejemplos de uso
            en los cuales se muestre de un mínimo del 80\% de las interfaces del
            mismo.
        \end{itemize}
    \item El sistema debería ser compatible con las versiones actuales de
    monitores de Petri desarrollados en el Laboratorio de Arquitectura de Computadoras de la
    Facultad de Ciencias Exactas y Naturales de la Universidad Nacional de Córdoba.
        \begin{itemize}
            \item El framework quiere tener la posibilidad de elegir el monitor de
            Petri que desea usar (monitor en Java, monitor en IP Core, monitor en
            driver, etc.)
            \item El framework debe utilizar las interfaces expuestas por el monitor
            de Petri para hacer manejo de la RdP.
        \end{itemize}
    \item El framework debe tener casos de test para probar sus funcionalidades.
        \begin{itemize}
            \item Las cobertura de los tests debería ser del 90\% de las
            funcionalidades.
            \item El framework debe tener al menos un test unitario automatizado
            por cada funcionalidad implementada.
        \end{itemize}
     \item El framework debe tener documentación del código
        \begin{itemize}
          \item La documentación debe tener un formato estándar (por ejemplo
          Javadoc).
          \item La documentación debe realizarse sobre cada objeto y método del
          framework.
          \item La documentación debe explicar la funcionalidad y la forma de
          uso del elemento documentado. 
          \item La documentación debe evitar brindar detalles de implementación.
          \item La documentacion debe estar disponible de forma pública.
        \end{itemize}
\end{enumerate}