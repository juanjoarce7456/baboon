\section{Introducción}
En este capítulo se hace una definición priorizada de los requerimientos que
debe cumplir el framework a desarrollar.
\section {Prioridades de los requerimientos}
    Los requerimientos se priorizan utilizando la semántica expuesta en la
    Tabla~\ref{tab:semantica_requerimientos}, descripta en
    \cite{programacionUml}.

\begin{table}
    \begin{tabularx}{\textwidth}{ | p{2cm} | X | X | }
    \hline
        \textbf{Atributo de Prioridad} & \textbf{Semántica} 
        \\[10pt]\hline
        Debe Tener & Requisitos obligatorios que son fundamentales para el
        sistema
        \\[10pt] \hline
        Debería tener & Requisitos importantes que se pueden omitir
        \\[10pt] \hline
        Podría tener & Requisitos que son opcionales (se realizan si hay tiempo)
        \\[10pt] \hline
        Quiere tener & Requisitos que pueden esperar para versiones posteriores
        del sistema
        \\[10pt] \hline
    \end{tabularx}
    \caption{Semántica de los Requerimientos}
    \label{tab:semantica_requerimientos}
\end{table}
    
\section{Definición de Requerimientos}
A continuación se definen los requerimientos del framework a desarrollar:
\begin{enumerate}
    \item El framework debe contar con un monitor de Redes de Petri.
        \begin{itemize}
          \item El monitor debe ofrecer interfaces para la carga de una Red de
          Petri en formato PNML (dialecto Tina).
          \item El monitor debe implementar la ecuación de estado generalizada
          descrita en \cite{Ecuacion_generalizada_LAC}.
          \item El monitor debe ofrecer interfaces para la carga de políticas de
          prioridad de disparo de transiciones.
          \item El monitor debe soportar Redes de Petri temporales y de
          plaza-transición.
          \item El monitor debe ofrecer interfaces para el disparo de
          transiciones.
          \item El monitor debe ofrecer interfaces para la modificación del
          valor de guardas.
          \item El monitor debe soportar disparos perennes y no perennes.
          \item El monitor debe soportar etiquetas de transición Disparada -
          Informada \cite{codegen}.
          \item El monitor debe soportar etiquetas de guarda en las
          transiciones.
          \item El monitor debe soportar arcos normales, de reset, lectores e
          inhibidores.
          \item El monitor debe ofrecer interfaces para la suscripción a
          informes de transiciones informadas.
        \end{itemize}
    \item El framework debe delegar el control del flujo de ejecución en el
    monitor de Redes de Petri.
        \begin{itemize}
            \item El framework debe ofrecer interfaces para mapear eventos lógicos
            de una RdP a eventos de acción especificados por los usuarios.
            \item El framework debe ofrecer interfaces para especificar si un
            controlador de acción responde a un evento físico de entrada o a un
            evento físico de salida.
            \item Un evento de acción debe definir todos los eventos lógicos
            necesarios para la sincronización de la ejecución de un controlador de
            acción:
                \begin{itemize}
                  \item Permiso de ejecución del controlador de acción (disparo de
                  transición).
                  \item Cambio de estado del sistema (seteo de guardas).
                  \item Aviso de finalización de ejecución del controlador de acción
                  (disparo de transición).
                \end{itemize}
            \item El framework debe ofrecer interfaces para la suscripción de
            controladores de acción a eventos de acción.
            \item El framework debe ser responsable de crear y controlar la
            ejecución de los hilos de ejecución para controladores de acción que
            generen eventos físicos de salida.
            \item El framework debe ser responsable de controlar la ejecución de los
            hilos creados por el usuario para ejecutar controladores de acción que
            manejen eventos físicos de entrada.
            \item El monitor de RdP debe manejar los eventos lógicos del sistema,
            mediante las interfaces de disparo y seteo de guardas.
        \end{itemize}
    \item Para un usuario con conocimiento intermedio en Java y Redes de Petri, el
    framework podría aprender a usarse en una semana o menos.
        \begin{itemize}
            \item La utilización del sistema podría incorporar como máximo diez
            conceptos nuevos a aprender por un usuario con un nivel intermedio
            de conocimientos en Java y redes de Petri.
            \item El sistema debería ser acompañado con al menos dos ejemplos de uso
            en los cuales se muestre de un mínimo del 80\% de las interfaces del
            mismo.
        \end{itemize}
    \item El sistema debería ser compatible con las versiones actuales de
    monitores de Petri desarrollados en el Laboratorio de Arquitectura de Computadoras de la
    Facultad de Ciencias Exactas y Naturales de la Universidad Nacional de Córdoba.
        \begin{itemize}
            \item El framework quiere tener la posibilidad de elegir el monitor de
            Petri que desea usar (monitor en Java, monitor en IP Core, monitor en
            driver, etc.)
            \item El framework debe utilizar las interfaces expuestas por el monitor
            de Petri para hacer manejo de la RdP.
        \end{itemize}
    \item El framework debe tener casos de test para probar sus funcionalidades.
        \begin{itemize}
            \item Las cobertura de los tests debería ser del 90\% de las
            funcionalidades.
            \item El framework debe tener al menos un test unitario automatizado
            por cada funcionalidad implementada.
        \end{itemize}
     \item El framework debe tener documentación del código
        \begin{itemize}
          \item La documentación debe tener un formato estándar (por ejemplo
          Javadoc).
          \item La documentación debe realizarse sobre cada objeto y método del
          framework.
          \item La documentación debe explicar la funcionalidad y la forma de
          uso del elemento documentado. 
          \item La documentación debe evitar brindar detalles de implementación.
          \item La documentacion debe estar disponible de forma pública.
        \end{itemize}
\end{enumerate}