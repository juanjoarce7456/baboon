\section{Introducción}
En este capítulo se describe el proceso de investigación realizado con el
objetivo de determinar las características de la herramienta de software a
desarrollar en este Proyecto Integrador.

\section{Objetivos de la Investigación}
Con la intención de definir la orientación del proyecto, se realizó
una primera definición aproximada del objetivo principal:\\
\emph{``El sistema a desarrollar es una herramienta para facilitar a
desarrolladores de software la utilización de una Red de Petri como lógica de su
producto.''}

En este sentido, se realizó un listado de herramientas candidatas para
cumplir con dicho objetivo:
\begin{itemize}
  \item Generación de código
  \item APIs
  \item Frameworks
\end{itemize}

Se desea que la herramienta resultante posea las siguientes condiciones:
\begin{itemize}
    \item La lógica del sistema debe quedar expresada en una Red de Petri.
    \item El grado de acoplamiento entre el código de usuario
    y la Red de Petri debe ser el mínimo.
    \item La arquitectura de la herramienta tiene que contemplar el uso de
    patrones de diseño en el código de usuario.
    \item El flujo de ejecución debe quedar definido por la herramienta.
    \item Determinar el grado de escalabilidad de los sistemas desarrollados
    utilizando la herramienta.
    \item La herramienta debe favorecer la mantenibilidad del código de usuario.
    \item En caso de ser posible, analizar experiencias previas de sistemas
    similares utilizando dichas herramientas.
\end{itemize}

\section{Desarrollo de la Investigación}
\label{sec:investigacion_desarrollo}
La investigación se basa en el análisis de generación de código, APIs y
Frameworks expuestas en el capítulo~\ref{generacion_frameworks_apis}. Además se
estudiaron las experiencias previas desarrolladas en \cite{codegen}, \cite{chimp} y
\cite{Bentivegna-Ludemann}:
\begin{itemize}
  \item En \cite{codegen} se propone una solución basada en la generación de
  código. Tras analizar los ejemplos de uso de la herramienta se pudieron
  detectar desventajas importantes. El código generado utiliza las interfaces
  del monitor directamente, provocando un alto grado de acoplamiento entre el
  software del usuario y la Red de Petri. Esto provoca que los sistemas
  desarrollados tienen reducida escalabilidad y mantenibilidad.
  \item En \cite{chimp} se propone una solución basada en el desarrollo de un
  framework como herramienta superadora a la generación de código. Los ejemplos
  de uso muestran un claro avance respecto a \cite{codegen}. Los
  sistemas desarrollados son mantenibles y la herramienta permite la
  utilización de patrones de diseño en las aplicaciónes.
  Sin embargo, la implementación presenta problemas de acoplamiento a la Red de
  Petri ya que carece de una capa de abstracción entre eventos físicos y eventos
  lógicos (ver sección~\ref{sec:sincronizacion_RdP_por_eventos}).
  \item En \cite{Bentivegna-Ludemann} se desarrolla un sistema dirigido por RdP
  y se utilizan las interfaces del monitor de Petri como una API o
  biblioteca de funciones. Si bien se logra obtener un sistema funcional, en la
  conclusión del proyecto sus autores expresan que es necesario un framework
  para desacoplar el código de la red. Además el flujo de ejecución es
  determinado por el desarrollador, aumentando el grado de acoplamiento.
\end{itemize}

\section{Conclusión de la Investigación}
Tras lo expuesto en la sección~\ref{sec:investigacion_desarrollo} se determinó
que la herramienta a desarrollar debe tener la forma de un Framework por las
siguientes razones:
\begin{itemize}
  \item Funciona como una capa de abstracción entre las acciones de software,
  los eventos lógicos de la red de Petri, los eventos físicos del mundo
  exterior al sistema, las secuencias de acciones de software, las políticas de
  prioridad y los estados de la Red de Petri.
  \item El flujo de control queda contenido dentro de la estructura del
  Framework.
  \item El desacoplamiento facilita la mantenibilidad y legibilidad del sistema.
\end{itemize} 
