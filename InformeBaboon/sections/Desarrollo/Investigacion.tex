\section{Introducción}
En este capítulo se describe el proceso de investigación realizado con el
objetivo de determinar las características de la herramienta de software a
desarrollar en este Proyecto Integrador.

\section{Objetivos de la Investigación}
Con la intención de definir la orientación del proyecto, se realizó
una primera definición aproximada del objetivo principal:\\
\emph{``El sistema a desarrollar es una herramienta para facilitar a
desarrolladores de software la utilización de una Red de Petri como lógica de su
producto.''}

En este sentido, se realizó un listado de herramientas de interés que pudieran
cumplir con dicho objetivo:
\begin{itemize}
  \item Generación de código
  \item APIs
  \item Frameworks
\end{itemize}

A su vez, se definieron objetivos de investigación para determinar la
factibilidad de desarrollo, condiciones para cumplir el objetivo principal,
mantenibilidad y escalabilidad de dichas herramientas:
\begin{itemize}
    \item Determinar la posibilidad de utilizar un monitor de Redes de Petri
    como motor lógico de un sistema desarrollado con la herramienta.
    \item Determinar el grado de acoplamiento entre el código desarrollado
    utilizando la herramienta y la Red de Petri.
    \item Determinar la posibilidad y facilidad de aplicar patrones de
    diseño en los sistemas desarrollados utilizando la herramienta.
    \item Determinar el grado de escalabilidad de los sistemas desarrollados
    utilizando la herramienta.
    \item Determinar la mantenibilidad del código desarrollado
    utilizando la herramienta.
    \item En caso de ser posible, analizar experiencias previas de sistemas
    similares utilizando dichas herramientas.
\end{itemize}

\section{Desarrollo de la Investigación}
\label{sec:investigacion_desarrollo}
La investigación se basó en el análisis de información acerca de los tres tipos
de herramientas consideradas. Esta información se expone de forma resumida en
el capítulo~\ref{generacion_frameworks_apis}. Además se estudiaron las
experiencias previas desarrolladas en \cite{codegen}, \cite{chimp} y
\cite{Bentivegna-Ludemann}:
\begin{itemize}
  \item En \cite{codegen} se propone una solución basada en la generación de
  código. Tras analizar los ejemplos de uso de la herramienta se pudieron
  detectar desventajas importantes. En primer lugar, el código generado 
  utiliza las interfaces del framework directamente, provocando un alto grado de
  acoplamiento entre el software del usuario y la Red de Petri. A su vez, los
  sistemas desarrollados con esta herramienta tienen reducida escalabilidad y
  mantenibilidad.
  \item En \cite{chimp} se propone una solución basada en el desarrollo de un
  framework como herramienta superadora a la generación de código. Los ejemplos
  de uso muestran un claro avance respecto a \cite{codegen}. Los
  sistemas desarrollados son mantenibles y la herramienta permite la
  reutilización de patrones de diseño en las aplicaciónes.
  Sin embargo, la implementación presenta problemas de acoplamiento a la Red de
  Petri ya que carece de una capa de abstracción entre eventos físicos y eventos
  lógicos (ver sección~\ref{sec:sincronizacion_RdP_por_eventos}).
  \item En \cite{Bentivegna-Ludemann} se desarrolla un sistema dirigido por RdP
  y se utilizan las interfaces del monitor de Petri como una API o
  biblioteca de funciones. Si bien se logra obtener un sistema funcional, en la
  conclusión del proyecto sus autores expresan que es necesario un framework
  para desacoplar el código de la red.
\end{itemize}

\section{Conclusión de la Investigación}
Tras lo expuesto en la sección~\ref{sec:investigacion_desarrollo} se determinó
que la herramienta a desarrollar debe tener la forma de un Framework, que
funcione como una capa de abstracción entre las acciones del código del software
de usuario y las transiciones de la red de Petri.