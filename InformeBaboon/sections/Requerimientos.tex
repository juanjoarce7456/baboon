\section{Requerimientos}

\begin{enumerate}
	\item El sistema debe delegar flujo de ejecución a un motor de
	petri.
	\begin{itemize}
		\item El sistema debe mapear transiciones de una red de petri a eventos
		especificados por los usuarios. Un evento puede ser equivalente a un conjunto de transiciones.
		\item Cuando un evento es desencadenado por el disparo de un conjunto de
		transiciones el sistema debe ejecutar todas las tareas que se encuentran registradas al evento.
		\item Cuando un suceso definido por el usuario ocurre, el sistema debe
		notificar todos los eventos asociados a este suceso al motor de petri.
		\item El sistema debe proveer una interface para que el usuario pueda
		suscribir sucesos, tareas y fines de tareas a eventos especificados por el usuario.
		\item El sistema debe proveer una interface para que el usuario pueda definir
		eventos.
		\item Cuando una tarea termina el sistema debe notificar al motor de petri
		acerca de todos los eventos asociados a la finalización de la tarea.
	\end{itemize}
	\item Para un usuario con conocimiento intermedio en Java y Redes de Petri, el
	framework puede aprender a usarse en una semana o menos.
	\begin{itemize}
	    \item La utilización del sistema puede incorporar como máximo diez
	    conceptos nuevos a aprender por un usuario con un nivel intermedio en Java
	    y redes de Petri.
	    \item El sistema debe ser acompañado con al menos dos ejemplos de uso en
	    los cuales se muestre al menos un 80\% de las funciones del mismo.
	\end{itemize}
	\item El sistema debe ser compatible con las versiones actuales de motores de
	Petri desarrollados en el Laboratorio de Arquitectura de Computadoras de la
	Facultad de Ciencias Exactas y Naturales de la Universidad Nacional de Córdoba.
	\begin{itemize}
	    \item El sistema debe proveer una interfaz para que el usuario ingrese un
	    archivo PNML con la descripción de una red de Petri.
	    \item El sistema puede instanciar un entorno de ejecución de redes de
	    Petri dado que el usuario ha ingresado un archivo PNML conteniendo la
	    descripción de la red y ha elegido el motor de Petri que desea usar.
	    \item El sistema debe utilizar la interface expuesta por el motor de
	    petri.
	\end{itemize}
	\item El sistema quiere tener una interfaz gráfica de usuario para inicializar
	un nuevo proyecto.
	\begin{itemize}
	    \item La interfaz de usuario quiere contener una pantalla 'PNML Loader'
	    	\begin{itemize}
	    	    \item La pantalla debe dejar al usuario buscar en su disco local y
	    	    elegir un archivo.
	    	    \item La pantalla debe dejar al usuario ingresar la dirección a un
	    	    archivo manualmente mediante la escritura con el teclado.
	    	    \item La pantalla debe permitir confirmar la elección de un archivo.
	    	    \item Si el usuario confirma un archivo y el archivo es un PNML válido
	    	    entonces puede usarse para configurar el entorno de ejecución de
	    	    Petri.
	    	    \item Si el usuario confirma un archivo y el archivo no es un PNML
	    	    válido entonces debe mostrarse un error en pantalla y el usuario debe
	    	    ser capaz de elegir otro archivo.
	    	\end{itemize}
	    \item La interfaz de usuario quiere contener una pantalla de creación de
	    eventos
	    \begin{itemize}
	    	    \item La pantalla debe dejar al usuario definir un evento y asociarlo
	    	    con una o más transiciones definidas en un archivo PNML cargado
	    	    previamente por el usuario.
	    	    \item La pantalla de creación de eventos quiere permitir guardar las
	    	    decisiones del usuario en un archivo.
	    	    \item La pantalla de creación de eventos quiere permitir al usuario
	    	    cargar configuraciones a partir de un archivo.
	    	    \item Si un archivo guardado previamente se selecciona para ser
	    	    cargado y su contenido tiene un formato inválido, la pantalla
	    	    quiere mostrar un texto de error especificando el problema y el
	    	    archivo no debe ser cargado.
	    	    \item Si un archivo guardado previamente se selecciona para ser
	    	    cargado y el contenido del archivo contiene uno omás eventos que
	    	    mapean a transiciones inexistentes, la pantalla quiere mostrar un
	    	    texto de error especificando el problema y sólo debe cargarse la
	    	    configuración de los eventos fuera de conflicto.
	    \end{itemize}
	     \item La interfaz de usuario quiere contener una pantalla de selección de
	     el motor de Petri.
	     \begin{itemize}
	         \item La pantalla debe permitir elegir entre un motor de Petri Java,
	         un motor de Petri en hardware o un motor de Petri en driver.
	         \item La pantalla debe comunicar la decisión del usuario para preparar
	         el entorno de ejecución de Petri de acuerdo al motor elegido.
	     \end{itemize}
	         
	\end{itemize}
\end{enumerate}