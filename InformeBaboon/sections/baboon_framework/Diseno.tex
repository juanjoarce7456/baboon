\section{Diseño de Baboon Framework}

\subsection{Introducción}
Baboon Framework funciona en su mayor parte como un framework de caja negra.
Tal como se explicó en la sección \autoref{sec:tipos_framework}, este tipo de
frameworks permite al usuario reutilizar las interfaces y sus reglas para
interconectar componentes.
A su vez también presenta algunas características de caja blanca,
ya que el usuario puede extender ciertas clases del framework para especificar
funcionalidades particulares.\\
Baboon Framework fue diseñado para implementar la arquitectura detallada
\emph{\color{red} DETALLAR ARQUITECTURA}.
En consecuencia, el diseño expone interfaces que le permiten a un desarrollador
determinar las tareas y sucesos, y asociarlas a topicos. De esta manera, el desarrollador
se abstrae de la comunicación entre la Red de Petri y el software y, por lo
tanto, del manejo de la concurrencia del sistema y de las condiciones
necesarias para la ejecución de cada tarea en particular.

\subsection{Clases y Responsabilidades}
El diseño de Baboon Framework está basado en patrones y en conceptos de la
programación orientada a aspectos para cumplir con los requerimientos y
objetivos explicados previamente.


