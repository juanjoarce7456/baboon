\section{Relación entre Eventos Lógicos, Eventos de Acción y Controladores de
Acción}
\label{sec:relacion_evento_controlador}
En la sección~\ref{sec:sincronizacion_RdP_por_eventos} se explica que las
acciones de software, embebidas en controladores de acción, son desencadenadas
por eventos de acción. A su vez, en la sección
\ref{sec:controladores_de_acciones} se definen dos tipos de controladores de
acción, cuyo modo de ejecución se explica en las secciones
\ref{sec:ejecucion_task_controller} y \ref{sec:ejecucion_happening_controller}.

La ejecución de ambos tipos de controladores tiene tres etapas marcadas:
\begin{itemize}
  \item Petición de permiso de ejecución al monitor.
  \item Ejecución de controlador de acción.
  \item Aviso de finalización de ejecución al monitor.
\end{itemize}

Si incorporamos el concepto de Guard Provider visto en la
sección~\ref{sec:guard_providers}, la ejecución incorpora una etapa más,
resultando en:
\begin{itemize}
  \item Petición de permiso de ejecución al monitor.
  \item Ejecución de controlador de acción.
  \item Ejecución del método Guard Provider y seteo de la guarda.
  \item Aviso de finalización de ejecución al monitor.
\end{itemize}

En consecuencia, podemos definir a un evento de acción como el conjunto de tres
eventos lógicos, claves para definir la sincronización de la ejecución de un
controlador de acción y sus influencias sobre el estado de la RdP:
\begin{labeling}{description}
  \item [Permiso de ejecución: ] Consiste en un evento lógico de disparo
  de transición al monitor de manera bloqueante (perenne).
  \item [Callback de guardas: ] Consiste en un evento lógico de seteo de
  guarda. El sistema de ejecución obtiene el valor a setear en la guarda a
  partir de la ejecución automática de un método de tipo Guard Provider. Este
  método está asociado a la guarda y al controlador de acción desencadenado por
  este evento de acción.
  \item [Callback de aviso de finalización: ]
  Consiste en un evento lógico de disparo de transición al monitor de manera no
  bloqueante (no perenne). Esto se debe a que se trata de una devolución de
  recursos al modelo de RdP y no de una petición de sincronización.
\end{labeling}

\begin{framed}
\textbf{Nota:} Un evento de acción debe contener un permiso de ejecución
obligatoriamente, ya que la petición de ejecución es el principio de la
inversión de control del framework. Sin embargo los callbacks de guardas y de
aviso de finalización son opcionales y están sujetos a las características del
modelo para el controlador de acción correspondiente.
\end{framed}

\subsection{Tópicos}
\label{sec:diseno_topicos}
Un tópico es una representación de un evento de acción. Los controladores de
acciones se suscriben a tópicos.
Un tópico esta compuesto por:
\begin{itemize}
  \item Un nombre único: identifica al tópico y se utiliza al momento de
  realizar la suscripción al mismo.
  \item Una lista ordenada de nombres de transición. Constituye el permiso
  de ejecución de cada controlador de acción suscripto al tópico. En el caso de
  Task Controllers simples y de Happening Controllers, esta lista contiene
  un solo nombre de transición mientras que en el caso de ComplexSecuentialTask
  Controllers contiene uno por cada sub tarea.
  \item Una lista ordenada de listas de nombres guardas. Cada lista
   dentro de la lista ordenada constituye el callback de guardas de cada
   controlador de acción suscripto al tópico. En un ComplexSecuentialTask
   Controller las guardas se setean al finalizar cada una de las acciones
   individuales que componen la tarea compleja. En el caso de
  Task Controllers simples y de Happening Controllers, la lista ordenada
  contiene una sola lista de nombres de guardas mientras que en el caso de
  ComplexSecuentialTask Controllers contiene una lista por cada sub tarea.
  \item Una lista de nombres de transiciones que constituye el callback de
  aviso de finalización de ejecución. Contiene los nombres de todas las
  transiciones que se disparan de manera no bloqueante al finalizar la ejecución
  del controlador de acción. En un  ComplexSecuentialTask
  Controller el callback de transiciones se dispara luego de finalizar la última
  sub tarea.
\end{itemize}
