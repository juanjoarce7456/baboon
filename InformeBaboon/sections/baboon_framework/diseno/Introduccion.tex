\section{Introducción}
En este capítulo se detalla el diseño de \nombreFramework \ Framework.
En primer lugar se fundamenta la decisión de elaborar un Framework teniendo en
cuenta el análisis de experiencias previas. Se realiza una clasificación de los
eventos que se intercambian en sistemas reactivos desarrollados utilizando el
monitor de RdP.
Se detalla el diseño de la arquitectura del framework en base a dicho
intercambio de eventos.

Se realiza un análisis de las formas en que se pueden sincronizar las
acciones de un sistema utilizando un monitor de RdP. Este análisis tiene el
objetivo de definir el modo de sincronización más adecuado para la arquitectura
del framework.

Se define el concepto de controlador de acción y sus clasificaciones. A su vez,
se define el concepto de Guard Provider. Finalmente se define la relación entre
los eventos y los controladores de acción, formalizando el intercambio de
eventos entre el software de usuario y el framework.


