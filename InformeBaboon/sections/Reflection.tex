\section{Reflection}

\subsection{Concepto}

El concepto general de Reflection ha sido definido por Brian Smith en
\cite{Smith84}:

``Por 'Reflection' en su estado más general, Me refiero a la habilidad de un
agente de razonar no solo introspectivamente, acerca de si mismo y de los
procesos internos de pensamiento, sino también externamente, acerca de sus
comportamientos y su situacion en el mundo. El pensamiento ordinario es externo
en un sentido simple; el punto de la reflección es darle a un agente una postura
más sofisticada desde donde considerar su propia presencia en el mundo donde
esta inserto.''

En los lenguajes de programación, esta definición toma la siguiente forma
\cite{BGW93}:
``Refleccion'' es la habilidad de un programa de manipular como datos algo que
representa el estado del programa durante su propia ejecución. Hay dos aspectos
de dicha manipulación: introspección e intercesión.
\begin{itemize}
  \item Introspección es la habilidad de un programa para observar y, por lo
tanto, razonar acerca de su propio estado.
  \item Intercesión es la habilidad de un programa para modificar su propio
  estado de ejecución o alterar su propia interpretación o significado.
\end{itemize}

Ambos aspectos requieren un mecanismo para codificar estados de ejecución como
si fueran datos. Proveer dicha codificación se denomina reificación.
