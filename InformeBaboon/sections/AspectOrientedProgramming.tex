\section{Programación Orientada a Aspectos}


\subsection{Concepto}

La programación orientada a aspectos es un paradigma de programación que tiene
como objetivo incrementar la modularidad mediante la separación de intereses
transversales (cross-cutting concerns). 
Los intereses transversales son aspectos de un programa que afectan a otros
intereses, son partes de un programa que dependen de, o afectan, muchas
otras partes del sistema.
Estos intereses usualmente no pueden separarse claramente del resto del
sistema, y pueden resultar en duplicación de código o un alto grado de
dependencia entre partes del sistema.


Los intereses transversales son la base para el desarrollo de aspectos. Estos no
pueden ser representados claramente en los paradigmas de programación orientado
a objetos o programación procedural.
La separación de intereses transversales se realiza añadiendo comportamientos
adicionales al codigo existente (llamados advices) sin cambiar el
código existente. En cambio especifica separadamente que codigo debe modificarse
mediante la definición de pointcuts. 
 
\subsection{Ejecución}


Un pointcut es un conjunto de joinpoints, que permiten
determinar donde aplicar exactamente un advice de un aspecto. Generalmente
se especifican utilizando nombres de clases o de métodos, en algunos casos
utilizando expresiones regulares que hacen juego con estos nombres. Cuando la
ejecución del programa alcanza uno de los joinpoints descriptos en el pointcut,
una porción de código asociada al pointcut (llamada advice) es ejecutada. Esto
permite al programador describir dónde y cuándo el codigo adicional debe
ejecutarse. Esto proporciona la posibilidad de añadir aspectos a un código
existente, o al diseño de un software con una clara separación de intereses.
