
\section{Frameworks}


\subsection{Definición}

Los frameworks son una técnica de reutilización de prácticas, conceptos y
criterios orientadas a facilitar la solución de un tipo de problemáticas en
particular.Son estructuras concretas de software, que proveen una manera
estándar de construir aplicaciones. Sirven como base para el diseño y
desarrollo de software orientado a resolver problemas de dicho tipo.
De acuerdo a \cite{Johnson97} Dos de las definiciones más comunes de framework
son:
\begin{itemize}
  \item ``Un framework es un diseño reusable de todo o parte de un sistema que es
  representado por un conjunto de clases abstractas y la forma en que esas
  instancias interactúan''
  \item ``Un framework es el esqueleto de una aplicación que puede ser
  personalizado por un desarrollador de aplicaciones.''
\end{itemize} 

 La primer definición describe la estructura de un framework, mientras que la
 segunda describe su propósito. 

Un framework es una técnica de reutilización de código porque facilita la
creación de una aplicación a partir de una biblioteca de componentes existentes.
Además es posible la creación de nuevos componentes, que heredan la mayor parte
de su implementación de clases abstractas definidas en el framework.

Debe pensarse en frameworks y componentes de software como tecnologías
diferentes, pero que cooperan entre sí \cite{JohnsonFeb97}:
\begin{itemize}
  \item Un framework provee un contexto reusable para los componentes.
  \item Un framework es más abstracto y flexible que los componentes.
\end{itemize} 

Por otro lado, los frameworks son más concretos y simples de reutilizar
que un diseño puro \cite{JohnsonFeb97}.

\subsection{Inversión de Control}
\label{sec:inversion_control}
La inversión de control es una de las características de los frameworks.
Es un principio de diseño en el cual porciones de código personalizado por el
usuario reciben el flujo de control (orden de ejecución de las instrucciones)
por parte de un framework genérico.

Al implementar un sistema sin utilizar un framework, generalmente el
desarrollador reutiliza componentes de una biblioteca escribiendo el código de
un programa principal que realiza llamadas a los componentes. El desarrollador
decide en el código cuándo llamar al componente y se encarga de la estructura y
el flujo de control del programa.

En un framework el programa principal es reutilizado, el
desarrollador solamente conecta componentes existentes al framework o
implementa nuevos componentes para conectar. Las porciones de código del
desarrollador son llamadas por el framework. El framework determina la
estructura y el flujo de control del programa.

La inversión de control sirve para los siguientes propósitos de diseño:
\begin{itemize}
  \item Desacoplar la ejecución de una tarea de su implementación.
  \item Mantener el foco en la tarea para la que fue diseñado el módulo.
  \item Guiar el diseño respetando las interfaces entre módulos.
  \item Evitar efectos colaterales al reemplazar un módulo.
\end{itemize}

\subsection {Ventajas de los frameworks}
\begin{itemize}
	\item Aplica técnicas de reutilización: La reutilización ideal de tecnología
	provee componentes que pueden conectarse fácilmente para hacer un nuevo sistema.
	El desarrollador de software no tiene que conocer la implementación del
	componente. El sistema resultante es eficiente, fácil de mantener y
	confiable.
	
	\item Son personalizables: Los frameworks son más
	personalizables que la mayoría de los componentes. Tienen interfaces más
	complejas.
	
	\item Sirven para múltiples aplicaciones: Un framework está orientado a
	facilitar la implementación de aplicaciones de un tipo determinado. En
	consecuencia, puede ser utilizado para implementar diversas aplicaciones que
	pertenezcan a dicho tipo.
	
	\item Facilitan el trabajo del desarrollador: Un framework puede reducir
	el esfuerzo para desarrollar una aplicación personalizada en un orden de magnitud.
	
	\item La uniformidad reduce los costos de mantener el código: Los programadores
	encargados de mantenerlo pueden cambiar de una aplicación a otra que utiliza el
	mismo framework sin tener que aprender un nuevo diseño.
	
	\item Los frameworks obligan al usuario a respetar patrones de diseño en las
	aplicaciones.

\end{itemize}

\subsection {Desventajas de los frameworks}
\begin{itemize}
    \item Curva de Aprendizaje: Los programadores deben aprender las interfaces
    antes de poder utilizar el framework. Generalmente aprender un nuevo
    framework es difícil.
    
    \item Restricción de elección del lenguaje de programación: Uno de los
    problemas de utilizar un framework implementado en un lenguaje en particular
    es que restringe a los sistemas a utilizar dicho lenguaje. La relación
    efectividad-costo es baja al construir una aplicación en un lenguaje con un
    framework escrito en otro.
    
    \item Debido a que los frameworks son descritos con lenguajes de
    programación, es difícil para los desarrolladores aprender los patrones
    colaborativos de un framework mediante la lectura del código.
    
\end{itemize}

\subsection{Frameworks desde la perspectiva del usuario}
\label{sec:tipos_framework}
\begin{itemize}

    \item Black-Box Frameworks: La forma más sencilla de utilizar un framework
    consiste en conectar componentes ya existentes. De esta forma no se modifica
    el framework ni se crean nuevas clases concretas.
	Se Reutilizan las interfaces del framework y sus reglas para interconectar
	componentes. Es lo más parecido a construir un circuito. El desarrollador
	necesita conocer la interfaz de conexión entre un objeto A y un objeto B, pero
	no es necesario que conozca las especificación exacta de A o B.

    \item White-Box Frameworks: Consiste en definir clases concretas, que
    extienden de clases abstractas definidas en el framework, y utilizarlas para
    implementar una aplicación. Las subclases están estrechamente acopladas a
    sus superclases, de esta forma se requiere más conocimiento acerca de la
    implementación de las clases que conforman el framework.
	
	\item Extensión o Modificación del núcleo del framework: Consiste en extender
	el framework cambiando las clases abstractas que forman su núcleo, usualmente
	para añadir nuevas variables u operaciones. Requiere mayores
    conocimientos acerca del diseño del framework. Cambiar las clases abstractas
	puede provocar fallos en las clases concretas existentes. Esta forma de
	utilizar un framework no es aplicable si el propósito es crear un
	sistema abierto.
\end{itemize}

Existen intermedios entre Black-Box frameworks y White-Box frameworks. Es común
que los frameworks puedan ser usados como Black-Box la mayor parte del tiempo y
ser extendidos cuando la ocasión lo demande.

\subsection{Frameworks VS APIs}

\begin{table}
	\renewcommand{\arraystretch}{1.5}
	\centering
	\begin{tabularx}{\textwidth}{ | p{2cm} | X | X | }
	\hline
	Categoría & Framework & API (Biblioteca) \\[10pt] \hline
	Extensibilidad & Por parte de los desarrolladores del framework. Si es de
	código abierto cualquiera puede extenderlo. & Por parte del fabricante de la
	librería.
	O cualquiera si es de código abierto \\[10pt] \hline
	Reusabilidad & Objetivo principal del diseño de un framework.
	Se aplica a nivel de arquitectura de software & Es reutilizable siempre que se
	requiera la funcionalidad que brinda\\[10pt] \hline
	Complejidad de Utilizar & Gran complejidad al principio, se simplifica a medida
	que el usuario aprende el framework & Complejidad inicial menor que un
	framework\\[10pt] \hline 
	Gestión del Flujo Principal & El framework toma el
	flujo principal del software & A cargo del programador\\[10pt] \hline
	Confiabilidad & El flujo principal está ampliamente testeado por todos los
	usuarios del framework & No brinda ninguna garantía de flujo\\[10pt] \hline
	Aplicación de Patrones de diseño & Usualmente un framework fuerza al usuario a
	utilizar uno o varios patrones & No obliga al usuario a utilizar ningún patrón
	de diseño\\[10pt] \hline 
	Especificidad / Generalización & Son de uso específico,
	están diseñados para resolver una familia de problemas. Por esto mantienen una
	arquitectura & De uso general donde una funcionalidad pueda ser
	utilizada\\[10pt] \hline 
	Impacto en el software del usuario ante un cambio &
	Poco o nulo mientras se respete la arquitectura global & Puede dejar
	incompatible cambiando o deprecando las funciones expuestas\\[10pt] \hline
	Acoplamiento a un determinado lenguaje & Obliga al usuario a desarrollar en el
	mismo lenguaje en el que está hecho el framework & No restringe a un lenguaje.
	Permite llamadas desde cualquier lenguaje mientras se respete la firma de las funciones expuestas\\[10pt] 
	\hline
	\end{tabularx}
	\caption{Comparación entre Frameworks y APIs}
\end{table}

