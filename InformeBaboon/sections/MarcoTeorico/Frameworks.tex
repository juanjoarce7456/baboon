
\section{Frameworks}


\subsection{Definición}


Los frameworks son una técnica de reutilización orientada a objetos.  Una de las
definiciones más utilizadas es ``Un framework es un diseño reusable de todo o
parte de un sistema que es representado por un conjunto de clases abstractas y
la forma en que esas instancias interactúan''. Otra definición común es ``Un
framework es el esqueleto de una aplicación que puede ser personalizado por un
desarrollador de aplicaciones.''. La primer definición describe la estructura de
un framework, mientras que la segunda su propósito. (Frameworks= Components +
Patterns, R. E. Johnson).

Se dice que un framework es una técnica de reutilización de código porque provee
medios para facilitar la creación de una aplicación a partir de una biblioteca
de componentes existentes que exponen las interfaces del framework. Además
permiten la creación de componentes nuevos que pueden heredar la mayor parte de
su implementación de superclases abstractas del framework.


Debe pensarse en frameworks y componentes de software como tecnologías
diferentes, pero que cooperan entre sí. Los frameworks proveen un contexto
reusable para los componentes y son más abstractos y flexibles que los
componentes. A su vez, los frameworks son  más concretos y simples de reutilizar
que un diseño puro. (Components, Frameworks, Patterns, R. E. Johnson).

\subsection{Inversión de Control}
\label{sec:inversion_control}
Una de las características de los frameworks es la inversión de control.
Generalmente, un desarrollador reutiliza componentes de una biblioteca 
escribiendo un programa principal que llama a los componentes cuando es 
necesario. El desarrollador decide cuándo llamar al componente y se encarga de
la estructura y el flujo de control del programa. En un framework el programa
principal es reutilizado, el desarrollador solamente conecta componentes
existentes al framework o implementa nuevos componentes para conectar. El código
del desarrollador es llamado por el framework. El framework determina la estructura
y el flujo de control del programa.

\subsection {Ventajas de los frameworks}
\begin{itemize}
	\item Aplica técnicas de reutilización: La reutilización ideal de tecnología
	provee componentes que pueden conectarse fácilmente para hacer un nuevo sistema.
	El desarrollador de software no tiene que saber cómo está implementado el
	componente y es fácil para el desarrollador aprender a usarlo. El sistema
	resultante será eficiente, fácil de mantener y confiable.
	
	\item Los frameworks son personalizables: Los frameworks son más
	personalizables que la mayoría de los componentes y tienen interfaces más
	complejas.
	
	\item Sirven para múltiples aplicaciones, dentro del grupo para
	las cuales fue desarrolado el framework, y facilitan el trabajo al
	desarrollador. Un buen framework puede reducir el esfuerzo
	para desarrollar una aplicación personalizada en un orden de magnitud.
	
	\item La uniformidad reduce los costos de mantener el código: Los programadores
	encargados de mantenerlo pueden moverse de una aplicación a otra que utiliza el
	mismo framework sin tener que aprender un nuevo diseño.
	
	\item Los frameworks fuerzan patrones en las aplicaciones que los utilizan.

\end{itemize}

\subsection {Desventajas de los frameworks}
\begin{itemize}
    \item Curva de Aprendizaje: Los programadores deben aprender las interfaces
    antes de poder utilizar el framework. Generalmente aprender un nuevo
    framework es difícil.
    
    \item Uno de los problemas de utilizar un lenguaje en particular es que
    restringe al framework a los sistemas que utilizan dicho lenguaje.
    
    \item La relación efectividad-costo es baja al construir una aplicación en
    un lenguaje con un framework escrito en otro.
    
    \item Debido a que los frameworks son descritos con lenguajes de
    programación, es difícil para los desarrolladores aprender los patrones
    colaborativos de un framework mediante la lectura del código.
    
\end{itemize}

\subsection{Frameworks desde la perspectiva del usuario}
\label{sec:tipos_framework}
\begin{itemize}

    \item Black-Box Frameworks: La forma más fácil de usar un framework es
	conectando componentes ya existentes. Esto no modifica el framework ni crea
	nuevas clases concretas.
	Reutiliza las interfaces del framework y sus reglas para interconectar
	componentes. Es lo más parecido a construir un circuito. El desarrollador sólo
	tiene que saber que un objeto A se conecta con un objeto B a través de una
	interfaz y no necesita conocer las especificación exacta de A o B. 

    \item White-Box Frameworks: No todos los frameworks pueden funcionar de esta
	manera. La siguiente forma más fácil de usar un framework es definir nuevas
	clases concretas y utilizarlas para implementar una aplicación. Las subclases están
	estrechamente acopladas a sus superclases, de esta forma se requiere más
	conocimiento acerca de la clase abstracta que en el primer modo. 
	
	\item Extensión o Modificación del núcleo del framework: La forma de usar el
	framework que requiere mayores conocimientos consiste en extender el mismo
	cambiando las clases abstractas que forman su núcleo, usualmente para añadir
	nuevas variables u operaciones. Cambiar las clases abstractas puede provocar
	fallos en las clases concretas existentes, esta forma de utilizar un framework
	no es aplicable si el propósito del mismo es crear un sistema abierto.
\end{itemize}

Existen intermedios entre Black-Box frameworks y White-Box frameworks. Es común
que los frameworks puedan ser usados como Black-Box la mayor parte del tiempo y
ser extendidos cuando la ocasión lo demande.

La mejor forma de aprender a usar un nuevo framework es mediante ejemplos. La
mayoría de los frameworks vienen con una serie de ejemplos que pueden ser
estudiados. Los ejemplos son concretos y fáciles de comprender en comparación a
aprender todo el framework en conjunto

\subsection{Frameworks VS APIs}

\begin{table}
	\renewcommand{\arraystretch}{1.5}
	\centering
	\begin{tabularx}{\textwidth}{ | p{2cm} | X | X | }
	\hline
	Categoría & Framework & API (Biblioteca) \\[10pt] \hline
	Extensibilidad & Por parte de los desarrolladores del framework. Si es de
	código abierto cualquiera puede extenderlo. & Por parte del fabricante de la
	librería.
	O cualquiera si es de código abierto \\[10pt] \hline
	Reusabilidad & Objetivo principal del diseño de un framework.
	Se aplica a nivel de arquitectura de software & Es reutilizable siempre que se
	requiera la funcionalidad que brinda\\[10pt] \hline
	Complejidad de Utilizar & Gran complejidad al principio, se simplifica a medida
	que el usuario aprende el framework & Complejidad inicial menor que un
	framework\\[10pt] \hline 
	Gestión del Flujo Principal & El framework toma el
	flujo principal del software & A cargo del programador\\[10pt] \hline
	Confiabilidad & El flujo principal está ampliamente testeado por todos los
	usuarios del framework & No brinda ninguna garantía de flujo\\[10pt] \hline
	Aplicación de Patrones de diseño & Usualmente un framework fuerza al usuario a
	utilizar uno o varios patrones & No obliga al usuario a utilizar ningún patrón
	de diseño\\[10pt] \hline 
	Especificidad / Generalización & Son de uso específico,
	están diseñados para resolver una familia de problemas. Por esto mantienen una
	arquitectura & De uso general donde una funcionalidad pueda ser
	utilizada\\[10pt] \hline 
	Impacto en el software del usuario ante un cambio &
	Poco o nulo mientras se respete la arquitectura global & Puede dejar
	incompatible cambiando o deprecando las funciones expuestas\\[10pt] \hline
	Acoplamiento a un determinado lenguaje & Obliga al usuario a desarrollar en el
	mismo lenguaje en el que está hecho el framework & No restringe a un lenguaje.
	Permite llamadas desde cualquier lenguaje mientras se respete la firma de las funciones expuestas\\[10pt] 
	\hline
	\end{tabularx}
	\caption{Comparación entre Frameworks y APIs}
\end{table}

