
\section{Frameworks}


\subsection{Definición}

Los frameworks son una técnica de reutilización de prácticas, conceptos y
criterios orientadas a facilitar la solución de un tipo de problemáticas en
particular. Son estructuras concretas de software, que proveen una manera
estándar de construir aplicaciones. Sirven como base para el diseño y
desarrollo de software orientado a resolver problemas específicos.
De acuerdo a \cite{Johnson97} dos de las definiciones más comunes de framework
son:
\begin{itemize}
  \item ``Un framework es un diseño reusable de todo o parte de un sistema que es
  representado por un conjunto de clases abstractas y la forma en que sus
  instancias interactúan''
  \item ``Un framework es el esqueleto de una aplicación que puede ser
  personalizado por un desarrollador de aplicaciones.''
\end{itemize} 

 La primer definición describe la estructura de un framework, mientras que la
 segunda describe su propósito.

Un framework es una técnica de reutilización de código porque facilita la
creación de una aplicación a partir de una biblioteca de componentes existentes.
Es posible la creación de nuevos componentes extendiendo los provistos por el
framework. Una característica que distingue a los frameworks de otras
técnicas de reutilización es la inversión de control (ver
sección\ref{sec:inversion_control}).

Debe pensarse en frameworks y componentes de software como tecnologías
diferentes, pero que cooperan entre sí \cite{JohnsonFeb97}:
\begin{itemize}
  \item Un framework provee un contexto reusable para los componentes.
  \item Un framework es más abstracto y flexible que los componentes.
\end{itemize} 

Por otro lado, los frameworks son más concretos y simples de reutilizar
que un diseño puro \cite{JohnsonFeb97}.

\subsection{Inversión de Control}
\label{sec:inversion_control}
La inversión de control es una característica principal de los
frameworks. Es un principio de diseño en el cual porciones de código
personalizado por el usuario son controladas por un framework.

Al implementar un sistema sin utilizar un framework, generalmente el
desarrollador escribe el código de un programa principal que realiza llamadas a
componentes de una biblioteca. El desarrollador decide en el código cuándo
llamar al componente y se encarga de la estructura y el flujo de control del
programa.

En un software basado en un framework el programa principal es reutilizado. 
El desarrollador solamente conecta componentes existentes al framework, o
implementa nuevos componentes para conectar. Las porciones de código del
desarrollador son llamadas por el framework. De esta manera, el framework
determina la estructura y el flujo de control del programa.

La inversión de control sirve para los siguientes propósitos de diseño:
\begin{itemize}
  \item Desacoplar la ejecución de una tarea de su implementación.
  \item Mantener el foco en la tarea para la que fue diseñado el módulo.
  \item Guiar el diseño respetando las interfaces entre módulos.
  \item Evitar efectos colaterales al reemplazar un módulo.
\end{itemize}

\subsection {Ventajas de los frameworks}
\begin{itemize}
	\item Al utilizar un framework se aplican técnicas de reutilización de
	software y de diseño.
	
	\item Son personalizables: Los frameworks son más
	personalizables que la mayoría de los componentes. Tienen interfaces más
	complejas.
	
	\item Sirven para múltiples aplicaciones: Un framework está orientado a
	facilitar la implementación de aplicaciones de un tipo determinado. En
	consecuencia, puede ser utilizado para implementar diversas aplicaciones que
	pertenezcan a dicho tipo.
	
	\item Facilitan el trabajo del desarrollador.
	
	\item La uniformidad reduce los costos de mantener el código: Los programadores
	encargados de mantenerlo pueden cambiar de una aplicación a otra que utiliza el
	mismo framework sin tener que aprender un nuevo diseño.
	
	\item Los frameworks obligan al usuario a respetar patrones de diseño en las
	aplicaciones.

\end{itemize}

\subsection {Desventajas de los frameworks}
\begin{itemize}
    \item Curva de Aprendizaje: Los programadores deben aprender las interfaces
    antes de poder utilizar el framework. Generalmente aprender un nuevo
    framework es difícil.
    
    \item Restricción de elección del lenguaje de programación: Uno de los
    problemas de utilizar un framework implementado en un lenguaje en particular
    es que restringe a los sistemas a utilizar dicho lenguaje. La relación
    efectividad-costo es baja al construir una aplicación en un lenguaje con un
    framework escrito en otro.
    
    \item Debido a que los frameworks son descritos con lenguajes de
    programación, es difícil para los desarrolladores aprender los patrones
    colaborativos de un framework mediante la lectura del código.
    
\end{itemize}

\subsection{Frameworks desde la perspectiva del usuario}
\label{sec:tipos_framework}
    Según \cite{JohnsonFeb97} existen tres formas de utilizar un framework
    desde la perspectiva de un usuario desarrollador de software: 
\begin{itemize}
    \item Black-Box Frameworks: Consiste en conectar componentes ya existentes.
    De esta forma no se modifica el framework ni se crean nuevas clases
    concretas sino que se reutilizan las interfaces del framework y sus reglas
    para interconectar componentes. Este método es similar a la construcción de
    un circuito eléctrico. El desarrollador necesita conocer la interfaz de
    conexión entre un objeto A y un objeto B, pero no es necesario que conozca
    la especificación exacta de A o B.

    \item White-Box Frameworks: Consiste en definir clases concretas, que
    extienden de clases abstractas definidas en el framework, y utilizarlas
    para implementar una aplicación. Las subclases están estrechamente
    acopladas a sus superclases. De esta forma se requiere más conocimiento
    acerca de la implementación de las clases que conforman el framework.
	
	\item Extensión o Modificación del núcleo del framework:  Consiste en extender
	el framework cambiando las clases abstractas que forman su núcleo para añadir
	nuevas variables u operaciones. Requiere conocimientos avanzados acerca del
	diseño del framework. Cambiar las clases abstractas puede provocar fallos en
	las clases concretas existentes. Este modo de utilización no es aplicable si
	el propósito es crear un sistema abierto.
\end{itemize}

Entre las formas de utilización mencionadas existen combinaciones intermedias. Es común
que los frameworks se utilicen como Black-Box la mayor parte del tiempo y
sean extendidos cuando la ocasión lo demande.

\section{Comparación entre Frameworks y APIs}

En la tabla \ref {tab:comparacion_frameworks_apis} se observa una comparación
entre frameworks y APIs.
\begin{table}[H]
	\centering
	\begin{tabular}{|>{\raggedright}m{3cm}|m{6cm}|m{6cm}|}%{
	% | p{2.5cm} | X | X | }
	\hline
	\multicolumn{1}{|>{\centering\arraybackslash}m{3cm}|}{\textbf{Categoría}}
	& \multicolumn{1}{>{\centering\arraybackslash}m{6cm}|}{\textbf{Framework}} 
    & \multicolumn{1}{>{\centering\arraybackslash}m{6cm}|}{\textbf{API (Biblioteca)}}\\
	%\textbf{Categoría} & \textbf{Framework} & \textbf{API (Biblioteca)} \\[10pt]
	\hline
    \textbf{Gestión del Flujo Principal} & El framework toma el
    flujo principal del software & A cargo del programador\\[10pt] \hline
    \textbf{Confiabilidad} & El flujo principal está ampliamente testeado por todos
    los usuarios del framework & No brinda ninguna garantía de flujo\\[10pt] \hline
    \textbf{Extensibilidad} & Por parte de los desarrolladores del
	framework.
	Si es de código abierto cualquiera puede extenderlo. & Por parte del fabricante de la
	librería.
	O cualquiera si es de código abierto \\[10pt] \hline
	\textbf{Reusabilidad} & Objetivo principal del diseño de un framework.
	Se aplica a nivel de arquitectura de software & Es reutilizable a nivel
	de llamada a métodos\\[10pt] \hline 
	\textbf{Complejidad de Uso} & Gran
	complejidad al principio, se simplifica a medida que el usuario aprende el framework & Complejidad inicial menor que un
	framework\\[10pt] \hline 
	\textbf{Aplicación de Patrones de diseño} & Usualmente un framework fuerza al
	usuario a utilizar uno o varios patrones & No obliga al usuario a utilizar ningún patrón
	de diseño\\[10pt] \hline 
	\textbf{Especificidad / Generalización} & Son de uso específico,
	están diseñados para resolver una familia de problemas. Por esto mantienen una
	arquitectura & De uso general donde una funcionalidad pueda ser
	utilizada\\[10pt] \hline 
	\textbf{Restricciones de lenguaje} & Obliga al usuario a
	desarrollar en el mismo lenguaje en el que está hecho el framework & No restringe a un lenguaje.
	Permite llamadas desde cualquier lenguaje mientras se respete la firma de las funciones expuestas\\[10pt] 
	\hline
	\end{tabular}
	\caption{Comparación entre Frameworks y APIs}
	\label{tab:comparacion_frameworks_apis}
\end{table}

