\section{Programación Orientada a Objetos}

No es del interés de este proyecto integrador ahondar en el campo de la
programación orientada a objetos, y se asume que el lector tiene conocimientos
afianzados sobre este paradigma de programación. Aún así, es necesario nombrarlo
para poder introducir los conceptos desarrollados en la sección \ref{reflection}

\subsection{Reflection}
\label{reflection}

Existen escenarios en la programación donde es útil tener la opción de conocer
los datos disponibles y las operaciones que se pueden aplicar sobre estos datos
en tiempo de ejecución. Además, resulta ventajoso poder tomar decisiones sobre
estos datos y operaciones en base al flujo del programa para modificar el
comportamiento del mismo. Tener esta posibilidad permite escribir software
flexible, reutilizable y capaz de adaptarse a múltiples escenarios. Estas
capacidades se pueden obtener por medio de la programación basada en la
\textit{reflexión} del programa.

Reflexión (o reflection) es la habilidad de un programa de examinarse a sí
mismo y a su entorno en tiempo de ejecución, y de cambiar su comportamiento
dependiendo de lo que encuentra.

Para realizar esta autoexaminación, un programa necesita tener una
representación de sí mismo. Esta información se llama \textit{metainformación} o
\textit{metadata}. En particular, en un entorno de programación orientada a
objetos la metadata se organiza en objetos, llamados \textit{metaobjetos}. La
revisión en tiempo de ejecución de los metaobjetos se llama
\textit{introspección} o \textit{introspection}.
\cite{Forman04javareflection}

En general, la introspección está seguida de un cambio del comportamiento.
Existen tres técnicas que una interfaz de programación de reflection puede
ofrecer para generar cambios de comportamiento:
\begin{itemize}
    \item Modificación de los metaobjetos
    \item Operaciones con la metadata: como la invocación dinámica de métodos 
    \item Intercesión: Se le permite al código interceder en varias fases de la
    ejecución para alterar el comportamiento del programa.
\end{itemize}

Estas características hacen que el uso de reflection permita diseñar software
más flexible, que se adapte más fácilmente a cambio de requerimientos y que a
la vez mantenga una estructura ordenada y buena legibilidad de código.
Esto favorece a la mantenibilidad.

Para poder hacer introspection, un programa que aplique reflection debe poder
acceder a su metainformación. Por esto, esta representación es el elemento
estructural más importante de un sistema reflectivo. Examinando su
autorepresentación, un programa puede obtener información acerca de su
estructura y comportamiento para tomar decisiones importantes.

Existen tres problemas relacionados al uso de reflection en el diseño de un
programa que deben ser tenidos en cuenta para poder asegurar la calidad del
mismo. Estos son:
\begin{itemize}
    \item Seguridad
    \item Complejidad del código
    \item Performance
\end{itemize}
Todos ellos se pueden mitigar mediante el uso de buenas prácticas de programación y un correcto diseño del software.