\section{Programación Orientada a Objetos}

Existe extensa bibliografía de consulta acerca del paradigma de Programación
Orientada a Objetos. Es de interés en particular aquella bibliografía basada en
el lenguaje Java (ver \cite{flanagan2005java} \cite{lafore2003data}). 
Dentro del marco de la progrmaación orientada a objetos, tiene
gran relevancia en este proyecto el concepto de Reflection, expuesto en la
sección~\ref{reflection}.

\subsection{Reflection}
\label{reflection}

En determinados escenarios de programación es útil conocer en
tiempo de ejecución la estructura interna de los objetos y las operaciones del
programa. De esta manera, en base a esta información y al flujo de ejecución,
el programa modifica su propio comportamiento.

Estas capacidades se obtienen por medio de la programación basada en la
\textit{reflexión} del programa.

Reflexión (o reflection) es la habilidad de un programa de examinarse a sí
mismo y a su entorno en tiempo de ejecución, y de cambiar su propio
comportamiento.

Para realizar esta autoexaminación, un programa necesita tener una
representación de sí mismo. Esta información se llama \textit{metainformación} o
\textit{metadata}. En particular, en un entorno de programación orientada a
objetos la metadata se organiza en objetos, llamados \textit{metaobjetos}. La
revisión en tiempo de ejecución de los metaobjetos se llama
\textit{introspección} o \textit{introspection}.
\cite{Forman04javareflection}

La metainformación es el elemento estructural más importante de un
sistema reflectivo. Examinando su autorepresentación, un programa obtiene
información acerca de su estructura y comportamiento para tomar decisiones
importantes.

En general, la introspección está seguida de un cambio del comportamiento del
propio programa. Existen tres técnicas que una interfaz de reflection ofrece
para generar cambios de comportamiento:
\begin{itemize}
    \item Modificación de los metaobjetos
    \item Operaciones con la metadata: como la invocación dinámica de métodos 
    \item Intercesión: El código intercede en varias fases de la ejecución para
    alterar el comportamiento del programa.
\end{itemize}

Estas características permiten diseñar software más flexible, que se adapta a
cambio de requerimientos, favoreciendo a la mantenibilidad.

Existen tres problemas relacionados al uso de reflection en el diseño de un
programa que deben tenerse en cuenta:
\begin{itemize}
    \item Seguridad
    \item Complejidad del código
    \item Performance
\end{itemize}

Los problemas mencionados se mitigan mediante el uso de buenas prácticas de
programación y un correcto diseño del software.
