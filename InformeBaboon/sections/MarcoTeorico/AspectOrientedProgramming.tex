\section{Programación Orientada a Aspectos}
\label{sec:aop}

\subsection{Concepto}

La programación orientada a aspectos es un paradigma de programación que tiene
como objetivo incrementar la modularidad mediante la separación de intereses
transversales (cross-cutting concerns). 
Los intereses transversales son aspectos de un programa que afectan a otros
intereses. Son partes de un programa que afectan o dependen de muchas
otras partes del sistema.
Estos intereses usualmente no pueden separarse claramente del resto del
sistema, y pueden resultar en duplicación de código o un alto grado de
dependencia entre partes del sistema.


Los intereses transversales son la base para el desarrollo de aspectos. Estos no
pueden ser representados claramente en los paradigmas de programación orientado
a objetos o programación procedural. \cite{AspectJInAction}
La separación de intereses transversales se realiza añadiendo comportamientos
adicionales al código existente, llamados advices o consejos, sin modificar el
mismo. Para lograrlo, se especifican puntos de ejecución (mediante la definición
de pointcuts) donde se aplican los advices previamente mencionados.

La programación orientada a aspectos complementa a la programación orientada a
objetos al permitir al desarrollador modificar dinamicamente el modelo estático
orientado a objetos para crear un sistema que puede crecer para cumplir nuevos
requerimientos. Tal como los objetos en el mundo real pueden cambiar sus estados
a lo largo de su vida, una aplicación puede adoptar nuevas características a
medida que se va desarrollando. \cite{Introduction_To_Aspect}


\subsection{Terminología}
\label{sec:aop_terminologia}
\begin{itemize}
  \item Intereses Transversales (Cross-cutting concerns): Aunque la mayoría de
  las clases en un modelo orientado a objetos está destinada a perfeccionar una función única y
  específica, usualmente comparten requerimientos secundarios en común con otras
  clases. Por ejemplo, se puede desear añadir mecanismos de logueo a las clases
  dentro de la capa de acceso de datos y también a las clases en la capa de
  interfaz de usuario cada vez que un hilo entre o salga de un método. Aunque la
  funcionalidad principal de cada clase es muy diferente, el código necesario
  para realizar la tarea secundaria es usualmente
  idéntico.\cite{Introduction_To_Aspect}
  
  \item Consejos (Advices): Es el código adicional que se desea aplicar al
  modelo existente. Siguiendo con el ejemplo anterior, es el código de logueo
  que se quiere aplicar cada vez que un hilo ingrese o salga de un
  método.\cite{Introduction_To_Aspect}
  
  \item Punto de unión (Join-point): Es el término que se le otorga al punto
  de ejecución en la aplicación en el cual los intereses transversales deben ser
  aplicados. En el ejemplo, un punto de unión es alcanzado cuando un hilo
  ingresa a un método, y un segundo punto de unión es alcanzado cuando un hilo
  sale de un método.
  
  \item Punto de corte (Point-cut): Un punto de corte es un conjunto de puntos
  de unión. Un point-cut permite definir dónde aplicar exactamente un consejo,
  lo cual permite la separación de intereses y ayuda a modularizar la lógica de
  negocios \cite{Classification_Of_Pointcut_Language_Constructs}.
  
  \item Aspecto (Aspect): La combinación de un punto de corte y un consejo se
  denomina aspecto. \cite{Introduction_To_Aspect}
  
  \item Tejido (Weaving): Proceso de aplicar aspectos a los objetos
  destinatarios para crear los nuevos objetos resultantes en los puntos de
  unión especificados. De acuerdo al momento del ciclo de vida del sistema en
  el cual se aplica el tejido, se realiza la siguiente clasificación:
  	\begin{itemize}
		\item Aspectos en Tiempo de Compilación.
		\item Aspectos en Tiempo de Carga.
		\item Aspectos en Tiempo de Ejecución.
	\end{itemize}
\end{itemize} 