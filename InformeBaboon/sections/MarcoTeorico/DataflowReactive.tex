\section{Paradigma Dataflow}

El paradigma de programación \textit{Dataflow} se basa en la idea de evitar que
el programador piense en términos del flujo de control del programa y se centre
en el flujo de los datos que son procesados.
De esta manera, las aplicaciones son representadas como un conjunto de nodos (o
bloques) con puertos de entrada y/o salida. Estos nodos pueden ser productores,
consumidores o bloques de procesamiento de información que fluye por el sistema. Los nodos
están conectados por aristas que definen el flujo de información por el
sistema. La mayoría de los lenguajes de programación visuales que usan una
arquitectura basada en bloques están basados en el paradigma dataflow.
\cite{DataflowTiagoSousa}

Los nodos son ejecutados únicamente cuando reciben y/o envían mensajes, lo que
sucede asíncronamente respecto de los demás nodos. Por esto, las aplicaciones
dataflow son inherentemente paralelas.\cite{DataflowRichardHarter}

La programación dataflow es capaz de proveer paralelismo sin la complejidad de
la gestión de hilos. Esto es posible gracias a que cada nodo es un bloque de
procesamiento independiente de los demás y no produce efectos colaterales
\cite{DataflowTiagoSousa}

Hay una amplia variedad de lenguajes dataflow. Se pueden mencionar
software de hojas de cálculo, Labview y Erlang, entre otros. La programación se
hace alterando diagramas de flujo. Una característica principal de este tipo de
lenguajes es que contienen un sistema de ejecución (runtime
system).\cite{DataflowRichardHarter}

Los programas imperativos tradicionales están compuestos de rutinas que se
llaman entre sí. Por ejemplo, en una secuencia de llamada un llamador
construye un paquete de datos y transfiere el control y el paquete de datos a
una rutina llamada. Cuando la rutina llamada termina, contruye otro paquete de
datos para pasar de vuelta al llamador y le transfiere nuevamente el control.

En los programas dataflow las “rutinas” no se llaman entre sí, en su lugar son
activadas por el sistema de ejecución cuando hay entrada para ellos. Cuando se
crean salidas, el sistema de ejecución se hace cargo de mover la salida al
destino que requiere esas salidas. Cuando las “rutinas” terminan, transfieren el
control de vuelta al sistema de ejecución.

Una diferencia entre la programación imperativa y dataflow es
la semántica utilizada para la transferencia de datos. Mientras la programación
imperativa utiliza semántica LIFO, la dataflow usa semántica FIFO
\cite{DataflowRichardHarter}. Eso es, un programa imperativo pone datos en una
pila y obtiene datos desde la misma pila.
En cambio en programas dataflow, cada elemento obtiene datos de una cola y pone
datos en otras colas.
Otra diferencia es que la conectividad de los programas procedurales está
embebida en el código. Para pasar datos de la rutina $A$ a $B$, $A$ debe
llamar explícitamente a $B$, es decir que un llamado tiene que especificar el
destino de los datos.
Por otro lado, en programas dataflow la conectividad puede estar separada
del código, $A$ no pasa datos directamente a $B$; en su lugar, le pasa datos al
sistema de ejecución, quien le pasa los datos a $B$.
El llamador no tiene que especificar hacia dónde van los datos y hasta puede
no saberlo. \cite{DataflowRichardHarter}

\subsection*{Ventajas y Desventajas del paradigma Dataflow}

Entre las ventajas de utilizar el paradigma dataflow se encuentran:

\begin{itemize}
  \item La concurrencia y paralelismo son naturales. El código se puede distribuir entre cores y a través de redes. Algunos
  problemas relacionados a hilos desaparecen
  \item Las redes dataflow son representaciones naturales e intuitivas para
  representar procesos.
  \item El paso de mensajes permite deshacerse de problemas asociados a memoria compartida y locks.
  \item Los elementos pueden ser agrupados en elementos compuestos, mejorando
  la escalabilidad de los sistemas.
\end{itemize}

Por otro lado, resulta poco ventajoso utilizar este paradigma por los siguientes
motivos:

\begin{itemize}
  \item El modelo de pensamiento de programación dataflow es poco familiar para
  la mayoría de los programadores profesionales.
  \item La mayoría de los lenguajes de programación dataflow son lenguajes de
  un nicho usado por programadores no profesionales.
  \item La intervención del sistema de ejecución puede tener aparejado un alto
  costo computacional. La gran ventaja de la semántica LIFO es que se implementa
  en código de manera inmediata y poco costosa.
  \item No utilizar memoria compartida tiene sus costos. Los mensajes deben ser
  copiados o deben ser inmutables.
  \item No son compatibles con otros lenguajes de programación. De esta
  manera, convertir programas tradicionales en programas dataflow es difícil
  porque la estructura es diferente.
\end{itemize}

\section{Paradigma Reactivo}

El paradigma reactivo es un paradigma de programación construído en torno a
flujos de datos, y la propagación de los cambios sobre ellos. Esto significa que
los lenguajes que implementan este paradigma deben permitir expresar flujos de
datos de manera estática o dinámica con facilidad, y el modelo de ejecución
debe propagar automáticamente los cambios en los datos cuando ocurran,
actualizando todos los valores correspondientes de manera transparente para el
programador.

A fin de comprender las características principales de este paradigma se
presenta el siguiente ejemplo:

\begin{figure}[h!]
\centering
\begin{minted}{perl}
a = 1
b = 2
c = a + b
a = 3
\end{minted}
\end{figure}

En programación imperativa, terminada la ejecución de esta sección de código,
la variable $c$ contiene el valor $3$ y se mantendrá de esta manera
indefinidamente hasta que el programador le asigne un nuevo valor. En cambio en
programación reactiva el valor de $c$ se mantiene siempre actualizado, es
decir, la expresión declarada como $c$ se vuelve a computar automáticamente
ante un cambio en $a$ o en $b$, y en este ejemplo pasa a valer $5$. Se dice que
$c$ es \textit{dependiente} de $a$ y $b$.
\cite{Bainomugisha:2013:SRP:2501654.2501666}

Al igual que en el paradigma dataflow, en el paradigma reactivo tiene relevancia
el flujo de datos, en lugar del flujo de control.
La diferencia radica en que, bajo el paradigma reactivo, las ``conexiones'' de
datos pueden ser alteradas dinámicamente en tiempo de ejecución.
Además se introducen restricciones de tiempo real blando, para lo cual se
definen dos conceptos:
\begin{itemize}
  \item \textit{Behaviours (Comportamientos)} representan eventos de variación
  contínua en el tiempo. El behaviour por excelencia es el tiempo, de hecho los
  lenguajes reactivos ofrecen primitivas para representar al tiempo.
  \item \textit{Events (Eventos)} representan eventos discretos. Suelen estar
  representados en forma de flujos de cambios de valores. A diferencia de los
  behaviours, los eventos cambian en instantes puntuales del tiempo. Los
  lenguajes reactivos ofrecen primitivas para combinar y procesar eventos.
\end{itemize}
\cite{Bainomugisha:2013:SRP:2501654.2501666}

\subsection*{El Paradigma Reactivo y El Patrón Observer}

El patrón de diseño \textit{observer} \cite{Gamma:1995:DPE:186897} nace de la
necesidad de mantener consistencia de datos en sistemas particionados, sin generar
acoplamiento entre capas de dichos sistemas.
Permite que un \textit{sujeto} publique cambios en su estado a sus
\textit{observadores}, quienes se susbribieron previamente a estas
actualizaciones.

El patrón observer se debe utilizar en alguna de las siguientes situaciones:
\begin{itemize}
  \item Cuando una abstracción tiene dos partes, una dependiente de la otra.
  \item Cuando el cambio en un objeto implica el cambio en otros, y no se sabe
  de antemano cuántos ni quiénes deben aplicar estos cambios.
  \item Cuando un objeto debe notificar a otros sin conocer nada de ellos, es
  decir sin generar acomplamiento.
\end{itemize}
\cite{Gamma:1995:DPE:186897}

Analizando este patrón de diseño se encuentran similitudes con el paradigma
reactivo. Por ejemplo, el patrón observer describe flujos de datos a nivel de
clases, mientras que el paradigma reactivo describe flujos a nivel de clases,
miembros de clases y variables.

En programación reactiva, cuando se forma una expresión dependiente de otras, se
genera una suscripción implícita de manera automática y el modelo de ejecución
es el encargado de propagar los cambios de manera transparente para el
programador.

