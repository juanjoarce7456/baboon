\section*{Introducción}

En esta parte del presente informe se expone el marco teórico necesario para
la comprensión de los temas tratados en este proyecto integrador.

En el capítulo \ref{cap:modelos} se introduce al lector a \textit{Autómatas} o
\textit{Máquinas de Estado Finitas (FSM)} y se generaliza este concepto para dar
lugar a \textit{Redes de Petri}. Se formaliza la definición de RdP y la
semántica del disparo de transiciones. Luego, se extiende esta semántica hasta
llegar a la ecuación de estado desarrollada en \cite{Ecuacion_generalizada_LAC}.
Finalmente se presenta una comparación entre FSM y RdP.

En el capítulo \ref{cap:paradigmas_programacion} se desarrollan paradigmas y
herramientas de la programación necesarias para entender los detalles del diseño
detallado en las próximas secciones. Se expone el paradigma \textit{Dataflow} y
el \textit{Reactivo}, luego se explican los conceptos de la \textit{Programación
Orientada a Aspectos (AOP)} y finalmente se presenta el concepto y utilización
de la \textit{Reflexión}.

En el capítulo \ref{generacion_frameworks_apis} se comienza desarrollando el
concepto de un generador automático de código fuente para luego presentar
\textit{Frameworks}. Más adelante se brinda una comparación entre estos y la
generación automática de código.

Finalmente, en el capítulo \ref{cap:concurrencia} se introduce al lector en el
concepto de \textit{Concurrencia} junto con sus propiedades y problemas. A
continuación se explican los distintos mecanismos de sincronización entre hilos
y procesos. Luego, se desarrollan \textit{Semáforos} y \textit{Monitores} como
herramientas de sincronización, haciendo especial foco en Monitores.
