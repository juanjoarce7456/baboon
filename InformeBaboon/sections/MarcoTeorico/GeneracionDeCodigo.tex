\section{Generación de Código Fuente}

La idea de generación automática de código fuente y de código ejecutable es casi
tan antigua como la programación en sí misma. Debido a que ahorra mucho
tiempo y costo de desarrollo de sistemas, ha sido y sigue siendo foco de
investigación.

La generación automática de código fuente está englobada por el concepto de
\textit{Programación Automática}. El significado de este término ha avanzado
junto a la programación a lo largo de los años:
\begin{itemize}
  \item En la década de 1940, se llamó de esta forma a la automatización del
  proceso de perforar cintas de papel para escribir el programa
  \cite{AutomaticProgrammingGorn}. Lo que Gorn llamó programación automática, es
  en realidad un lenguaje assembler.
  \item En el comienzo de los lenguajes de alto nivel se le llamó de esta manera
  a los compiladores. Tanto es así que uno de los primeros compiladores se llamó
  Autocode.
  \item Actualmente se identifica este término como la generación de código
  fuente escrito en un lenguaje de programación (compilable o interpretable a
  código máquina) a partir de una descripción de más alto nivel.
\end{itemize}

Algunos ejemplos de generadores de código son:
\begin{itemize}
  \item \underline{Apache Thrift:} Desarrollado por Facebook y actualmente
  liberado bajo licencia Apache, Thrift es un generador de servicios para múltiples lenguajes
  orientado a la comunicación por medio de llamadas a procedimiento remoto
  (remote procedure call – RPC). Para lograr esto expone un lenguaje de
  definición de interfaces (interface definition language – IDL) propio,
  utilizado para describir el servicio que luego Thrift generará en alguno de
  los múltiples lenguajes que soporta.\cite{ApacheThrift}
  \item \underline{Acceleo}: Es un generador de código que implementa el
  estándar \textit{MOFM2T} desarrollado por The Eclipse Foundation y su código fuente
  está liberado bajo licencia EPL. Acceleo permite la especificación del
  software en modelos como UML (v1 y v2), EMF (eclipse model framework) y
  lenguajes de modelado personalizados (DSL). Además permite especificar
  plantillas definidas por el usuario. Genera código en múltiples lenguajes de
  programación.\cite{Acceleo}
  \item \underline{Actifsource:} Desarrollado por actifsource GmbH y de código
  cerrado, es un generador de código a partir de modelos similares a UML. Soporta la
  creación de múltiples modelos y la unión de estos, y la utilización de modelos
  generados en cualquier software que tolere formato ecore, definido por el
  Eclipse Modeling Framework. Está desarrollado como un plugin para
  Eclipse.\cite{Actifsource}
  \item \underline{Spring Roo:} Desarrollado conjuntamente por DISID y Pitvotal
  bajo licencia Apache 2.0, es un generador enfocado al desarrollo acelerado de
  software empresarial en Java. La aplicación generada utiliza tecnologías Java
  comunes como Spring Framework, Java Persistence API, Apache Maven, etc. A
  diferencia de otros generadores de código, Roo expone una interfaz por línea
  de comandos con sus propios comandos.\cite{SpringRoo}
  \item \underline{GeneXus:} Desarrollado por ARTech bajo licencia cerrada y con
  primer lanzamiento en 1988, es un generador de código fuente a partir de un
  lenguaje declarativo de alto nivel. A partir de este lenguaje, se genera
  código fuente en C\#, COBOL, Java, Objective-C, RPG, Visual Basic,
  Ruby y Visual FoxPro. Además tolera múltiples lenguajes para gestión de bases de
  datos como Microsoft SQL, Oracle, DB2, Informix, PostgreSQL y
  MySQL.\cite{Genexus}
  \end{itemize}
  
La programación automática siempre ha sido un eufemismo para la
programación con un lenguaje de más alto nivel del disponible para el
programador. Investigar en programación automática es simplemente desarrollar
la implementación de lenguajes de programación de más alto nivel
\cite{Parnas:1985:SAS:214956.214961}.

En conclusión, los generadores automáticos de código fuente son en realidad
traductores de un lenguaje de programación a otro. Esto brinda un mayor nivel
de abstracción para el programador pero lo obliga a especificar su software en
el lenguaje provisto por el generador.
