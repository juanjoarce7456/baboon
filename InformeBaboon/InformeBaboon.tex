\documentclass{report}
\usepackage[spanish]{babel}
\usepackage[utf8]{inputenc}
\usepackage{amsmath, array}
\usepackage{amssymb}
\usepackage{tabularx}
\usepackage{minted}
\RecustomVerbatimEnvironment{Verbatim}{BVerbatim}{}
\usepackage{cite}
\usepackage{graphicx}
\usepackage{graphbox}
\usepackage{color}
\usepackage{float}
\usepackage{subfigure}
\usepackage{amsfonts}
\usepackage[section]{placeins}
\usepackage{hyperref}
\hypersetup{
     colorlinks,
     citecolor=black,
     filecolor=black,
     linkcolor=black,
     urlcolor=black
}
\date{}
\setcounter{secnumdepth}{3} % Add number to subsubsections
\usepackage{framed}

% counter used for definitions. Every time you want to add a definition, use
% \stepcounter{definitionsCounter} before the definition
\newcounter{definitionsCounter}

\begin{document}
    %Nombre templetizado para poder cambiarlo fácil hasta tenerlo definido
    \newcommand{\nombreTesis}{Framework de Sincronización de Tareas Coordinado
    por Redes de Petri}
    %Nombre templetizada por si lo tenemos que cambiar
    \newcommand{\nombreFramework}{Baboon Framework}
    \title{\nombreTesis}
    \author{Ariel Iván Rabinovich \\ \href{mailto:airabinovich@gmail.com}{airabinovich@gmail.com}
        \and Juan José Arce Giacobbe \\ \href{mailto:juanjo.arce7546@gmail.com}{juanjo.arce7456@gmail.com}}
    \graphicspath{ {resources/images/} }
    
    \maketitle
    
    \tableofcontents
    
    \listoffigures
    \listoftables
    
    \setcounter{definitionsCounter}{0}
    
    \part{Marco Teórico}
        \chapter{Introducción}
            \section{Introducción}
En este capítulo se detalla el diseño de \nombreFramework \ Framework.
En primer lugar se fundamenta la decisión de elaborar un Framework teniendo en
cuenta el análisis de experiencias previas. Se realiza una clasificación de los
eventos que se intercambian en sistemas reactivos desarrollados utilizando el
monitor de RdP.
Se detalla el diseño de la arquitectura del framework en base a dicho
intercambio de eventos.

Se realiza un análisis de las formas en que se pueden sincronizar las
acciones de un sistema utilizando un monitor de RdP. Este análisis tiene el
objetivo de definir el modo de sincronización más adecuado para la arquitectura
del framework.

Se define el concepto de controlador de acción y sus clasificaciones. A su vez,
se define el concepto de Guard Provider. Finalmente se define la relación entre
los eventos y los controladores de acción, formalizando el intercambio de
eventos entre el software de usuario y el framework.



        \chapter{Modelos}
            \section{Autómatas y Redes de Petri}

\subsection{Autómatas o Máquinas de Estado}

Existen muchas formas de modelar el comportamiento de los sistemas, y el uso de
máquinas de estado finitas es una de las más antiguas y más conocidas.
Las máquinas de estado finitas o autómatas nos permiten pensar acerca del
``estado'' de un sistema en un instante en particular y caracterizar el comportamiento de dicho
sistema basado en ese estado. El uso de esta técnica de modelado no está
limitada al desarrollo de sistemas de software.\cite{FSM_Wright}

\subsubsection{Definición Conceptual de Máquina de Estado}

Si una máquina de estados M, en un instante dado, se encuentra en el estado
$E_{0}$ y ocurre un evento $e_{0}$ que lleva a M al estado $E_{1}$, se
dice que ocurrió una \textit{transición} del estado $E_{0}$ al estado
$E_{1}$.
A partir de esto se puede deducir que M no puede estar en $E_{0}$ y $E_{1}$
a la vez, y por lo tanto los estados de una máquina de estados, son
\textbf{estados globales} del sistema modelado.

Analizando la semántica de las máquinas de estado, se pueden
identificar algunas características clave de un sistema que puede ser modelado con máquinas de
estados finitas:
\begin{itemize}
  \item El sistema debe ser descripto por conjunto finito de estados.
  \item El sistema debe tener una cantidad finita de entradas y/o eventos que
  puedan disparar transiciones entre estados.
  \item El comportamiento del sistema en un instante dado depende del estado
  actual y de sus entradas o eventos que ocurran en ese instante.
  \item Para cada estado posible en que el sistema pueda encontrarse existe un
  comportamiento definido para cada posible entrada o evento.
  \item El sistema tiene un estado inicial único y definido.
\end{itemize} \cite{FSM_Wright}

\subsubsection{Definición Formal de Máquina de Estado}

A fin de eliminar la ambigüedad existente en una definición conceptual, se
introduce una definición formal de Autómata Finito:
\newline\newline\emph{Definición:} Un autómata finito M está definido por una
tupla $(\Sigma, Q, q_{0}, F, \sigma)$, donde:
\begin{itemize}    
  \item $\Sigma$ es el conjunto de símbolos de entrada de M
  \item $Q$ es el conjunto de estados de M
  \item $q_{0}$ es el estado inicial de M
  \item $F \subseteq Q$ es el conjunto de estados finales de M
  \item $\sigma : Q  \times \Sigma \rightarrow Q$ es la función de
  transición
\end{itemize} \cite{FSM_Wright}

\subsection{Redes de Petri}

Tomando el concepto de transición en una máquina de estados, se lo puede
extender a una entidad propia.
Esta transitión $t_{i}$ será denotada por una barra, un rectángulo o un
cuadrado, y puede tener múltiples arcos de entrada (entrantes) y de salida
(salientes) a la vez. Esta transición, representa la \textit{transición} básica
de una Red de Petri (RdP).\cite{PetriNetsFundamentals}

De la misma forma que en una máquina de estados los círculos denotan estados
del sistema, en una RdP se utilizan círculos para denotar las \textit{plazas} o
\textit{lugares} de la red. Estas plazas no representan estados globales, sino
\textbf{estados locales}. \cite{PetriNetsFundamentals}

El estado local de una plaza, está dado por la cantidad de \textit{tokens} o
\textit{marcas} que esta contiene.

Como consecuencia de su estructura, una Red de Retri puede ser representada como
un grafo bipartito, donde los tipos de nodo existentes son \textit{plazas} y
\textit{transiciones}. Estos nodos se unen entre dos de distinto tipo
únicamente (de ahí el calificativo de bipartito), utilizando \textit{arcos}.\\

\begin{figure}[h]
	\centering
	\includegraphics[width=75mm]{Partes_De_Una_Red}
	\caption{Partes de una Red de Petri}
	\label{fig:partes_de_una_red}
\end{figure}

Se pueden visualizar las partes de una Red de Petri en la figura
\ref{fig:partes_de_una_red}.\\


\begin{figure}[h]
    \centering
    \includegraphics[height=40mm]{Automata_Y_Petri}
    \caption{Equivalencia entre una Máquina de Estados y una Red de Petri}
    \label{fig:automata_y_petri}
\end{figure}

En la figura \ref{fig:automata_y_petri} se aprecia:\\
\begin{itemize}
  \item[(a)] Una máquina de estados de dos estados y una transición.
  \item[(b)] Una RdP equivalente a la máquina de (a).
  \item[(c)] Una RdP con una transición con múltiples arcos de entrada y de
  salida.
\end{itemize}

Se puede extraer como consecuencia directa de esta extensión de la semántica de
un autómata que en una Red de Petri:
\begin{itemize}
  \item Múltiples tokens pueden existir en el modelo al mismo tiempo, y
  particularmente en una plaza.
  \item No existe un estado global explícito.
  \item El estado global del sistema es el conjunto de todos los estados
  parciales, representados por las plazas y sus tokens. A este conjunto se lo
  llama el \textbf{marcado} de la red.
\end{itemize}

\subsubsection{Definición Formal de Red de Petri}
A fin de eliminar ambigüedades, se presenta una serie de definiciones sobre
Redes de Petri.

\begin{itemize}
  \item [\underline{Definición 1}:] Una Red de Petri R está definida por la
  tupla $(P, T, Pre, Post)$ donde:
  \begin{itemize}
    \item $ P = \{ p_1, p_2, \ldots, p_p \} $ un conjunto de plazas.\footnote{Se
    utiliza $p$ como la cantidad de plazas de la RdP en todo momento dentro de este informe por simplicidad para el lector}
    \item $ T = \{ t_1, t_2, \ldots, t_t \} $ un conjunto de transiciones, donde
    $ P \cap T = \emptyset $. \footnote{Se utiliza $t$ como la cantidad de
    transiciones de la RdP en todo momento dentro de este informe por
    simplicidad para el lector}
    \item $ Pre: P \times T \rightarrow \mathbb{N}^{p} $ aplicación de
    precedencia.\footnote{Se toma la definición de números naturales incluyendo
    el cero por simplicidad de notación.}
    \item $ Post: P \times T \rightarrow \mathbb{N}^{p} $ aplicación de
    incidencia.
  \end{itemize}
  $ Pre (p_i, t_j) $ contiene el peso del arco que va de $ p_i $ a $ t_j $, y
  $ Post (p_i, t_j) $ contiene el peso del arco que va de $ t_j $ a $ p_i $

  \item [\underline{Definición 2}:] Una Red de Petri Marcada está
  definida por el par $(R, M)$, donde R es una RdP y $ M : P \rightarrow
  \mathbb{N}^{p} $ (siendo $P$ el conjunto de plazas de dimensión $n$) es una aplicación llamada \textit{marcado}.\\
  $m(R)$, o más simplemente $m$ si la red es conocida, define el marcado de la
  RdP y $m(p_{i})$ o $mp_{i}$ indica el marcado de la plaza $p_{i}$, es decir,
  el número de tokens contenido en la plaza $p_{i}$.\\
  La marca inicial se denota $m_{0}$ y da la cantidad inicial de tokens en todas
  las plazas de la red, por lo que especifica el estado inicial del sistema.
  
  \item [\underline{Definición 3}:] Para una marca $m$, una transición $t_{j}$
  está sensibilizada, y por lo tanto es disparable, si y solo si:\\
  $$ \forall p_{i} \in P, m(p_i) \geq Pre(p_{i}, t_{j}) $$
  Conceptualmente, una transición está sensibilizada si todas sus plazas de
  entrada contienen al menos la cantidad de tokens que indica el peso de los
  arcos que las unen.

  En la figura \ref{fig:transiciones_no_sensibilizadas} se observa gráficamente esta definición mediante dos casos de transiciones no sensibilizadas. Nótese
  el peso de los arcos.

  \begin{figure}[h]
    \centering
    \includegraphics[height=40mm]{Transiciones_No_Sensibilizadas}
    \caption{Ejemplos de transiciones no sensibilizadas.}
    \label{fig:transiciones_no_sensibilizadas}
  \end{figure}
  
  \item [\underline{Definición 4}:] La estructura de una Red de Petri
  se denota $ N = \{P, T, F, W\} $ donde,
  \begin{itemize}
    \item $P$ es en conjunto de plazas.
    \item $T$ es el conjunto de transiciones, donde se cumple que $ P \cap T =
    \emptyset $
    \item $F$ es el conjunto de arcos, donde se cumple que $ F \subseteq (P
    \times T) \cup (F \times P) $.
    \item $W$ es la función de peso de los arcos.
  \end{itemize}

  \item [\underline{Definición 5}:] Conjunto de transición y plaza de entrada y
  de salida.
  \begin{itemize}
    \item[] El conjunto de las plazas de entrada a la transición $t$ se denota
    $\bullet t$ y se define,
    $$ \bullet t = \{ p \in P : (p, t) \in F \} $$
    \item[] El conjunto de las plazas de salida de la transición $t$ se denota $
    t \bullet$ y se define,
    $$ t \bullet = \{ p \in P : (t, p) \in F \} $$
    \item[] El conjunto de las transiciones de entrada a la plaza $p$ se
    denota $\bullet p$ y se define,
    $$ \bullet p = \{ t \in T : (t, p) \in F \} $$
    \item[] El conjunto de las transiciones de salida de la plaza $p$ se denota
    $ p \bullet$ y se define,
    $$ p \bullet = \{ t \in T : (p, t) \in F \} $$
  \end{itemize}
\end{itemize}

\subsubsection{Disparo de una Transición}

La condición de disparo relacionada a $Pre(p_{i}, t_{j})$ significa que para
todas las plazas $p_{i}$ de entrada a $t_{j}$, es decir, todas las plazas que
tienen arcos que apuntan hacia $t_{j}$, el número de tokens presentes debe ser
mayor o igual al peso de dicho arco.

\begin{itemize}
  \item [\underline{Definición 6}:] En una RdP, dada una marca $ m_{n}(p) $,
  cualquier transición $ t_{j} $ que se encuentre sensibilizada puede ser
  disparada, y su disparo lleva a una marca $ m_{n+1}(p)$ dada por:
  $$ m_{n+1}(p) = m_{n}(p) + Post(p_{i}, t_{j}) - Pre(p_{i}, t_{j}), \forall
  p_{i} \in P $$
  Como se indica en la ecuación, al disparar la transición $ t_{j} $, se quitan
  tantos tokens de $ \bullet t $ como indiquen los arcos que las unen a $ t_{j}
  $, y se añaden a $ t \bullet $ la cantidad de tokens que indiquen los arcos
  que unen a $ t_{j} $ con ellas.\\
  El disparo de una transición $ t_{j} $ se denota $ m_{n}\rightarrow t_{j}
  \rightarrow m_{n+1} $

  En la figura {\ref{fig:disparo_transicion}} se observa el estado de una RdP
  antes y después del disparo de una transición.
  \begin{figure}[h]
    \centering
    \subfigure[$t_{0}$ sensibilizada]{\includegraphics[height=40mm]{Red_Sensibilizada}}
    \subfigure[Disparo de $t_{0}$]{\includegraphics[height=40mm]{Red_Disparada}}
    \caption{Disparo de una transición}
    \label{fig:disparo_transicion}
  \end{figure}
  
  \item  [\underline{Definición 7}:] Matriz de Incidencia.\\
  La matriz de incidencia de una RdP se define como,
  $$ I = Post - Pre $$
  \textbf{Notas:}
  \begin{itemize}
    \item El disparo de una transición se reformula como, $$ m_{n+1}(p) =
    m_{n}(p) + I(p_{i}, t_{j}), \forall p_{i} \in P $$
    \item A partir de las matrices $Pre$ y $Post$ se puede reconstruir la
    estructura de la red, a partir de $I$ no es posible.
  \end{itemize}
\end{itemize}

\subsubsection{Sucesión de Disparos}

Si en lugar del disparo de una transición se requiere disparar múltiples
transiciones, se puede reescribir la ecuación de cambio de estado de la red de
la siguiente forma,
$$ m_{n+1} = m_{n} + I \times \sigma $$
En esta ecuación, $\sigma$ representa la sucesión de disparos a realizar. Se
cumple $\sigma \in \mathbb{N}^{t}$ y el elemento $\sigma_{i}$ contiene la
cantidad de disparos a realizar sobre $t_{i}$.\\
Si se comienza a realizar la sucesión de disparos $\sigma_{i}$ a partir del
marcado inicial $m_{0}$ y todos los disparos son exitosos, se llega a un marcado
$m_{i}$ y se dice que $m_{i}$ es \textit{alcanzable}.\\
De la misma forma, si existe un marcado $m_{j}$ alcanzable desde $m_{0}$, debe
exitir una sucesión de disparos $\sigma_{j}$ que permita alcanzarlo.


        \chapter{Paradigmas de Programación}
            \section{Paradigma Dataflow}

El paradigma de programación \textit{Dataflow} se basa en la idea de evitar que
el programador piense en términos del flujo de control del programa y se centre
en el flujo de los datos que son procesados.
De esta manera, las aplicaciones son representadas como un conjunto de nodos (o
bloques) con puertos de entrada y/o salida. Estos nodos pueden ser productores,
consumidores o bloques de procesamiento de información que fluye por el sistema. Los nodos
están conectados por aristas que definen el flujo de información por el
sistema. La mayoría de los lenguajes de programación visuales que usan una
arquitectura basada en bloques están basados en el paradigma dataflow.
\cite{DataflowTiagoSousa}

Los nodos son ejecutados únicamente cuando reciben y/o envían mensajes, lo que
sucede asíncronamente respecto de los demás nodos. Por esto, las aplicaciones
dataflow son inherentemente paralelas.\cite{DataflowRichardHarter}

La programación dataflow es capaz de proveer paralelismo sin la complejidad de
la gestión de hilos. Esto es posible gracias a que cada nodo es un bloque de
procesamiento independiente de los demás y no produce efectos colaterales
\cite{DataflowTiagoSousa}

Hay una amplia variedad de lenguajes dataflow, variando de hojas de cálculo,
Labview, hasta Erlang. Muchos son gráficos. La programación se hace alterando
diagramas de flujo. Una característica que tienen todos en común es que tienen
un sistema de ejecución (runtime system).\cite{DataflowRichardHarter}

Los programas imperativos tradicionales están compuestos de rutinas que se
llaman entre sí, por ejemplo, cuando una llamada hace que el llamador construya
un paquete de datos (secuencia de llamada) y transfiere el control y el paquete
de datos a la rutina llamada. Cuando la rutina llamada termina, contruye un
paquete de datos para pasar de vuelta al llamador y le transfiere nuevamente el control.

En los programas dataflow las “rutinas” no se llaman entre sí, en su lugar son
activadas por el sistema de ejecución cuando hay entrada para ellos. Cuando se
crean salidas, el sistema de ejecución se hace cargo de mover la salida al
destino que requiere esas salidas. Cuando las “rutinas” terminan, transfieren el
control de vuelta al sistema de ejecución.

Una diferencia entre la programación imperativa y dataflow es
la semántica utilizada. Mientras la programación imperativa utiliza semántica
LIFO, la dataflow usa semántica FIFO \cite{DataflowRichardHarter}. Eso es, un
programa imperativo pone datos en una pila y obtiene datos desde la misma pila.
En cambio en programas dataflow, cada elemento obtiene datos de una cola y pone
datos en otras colas.
Otra diferencia es que la conectividad de los programas procedurales está
embebida en el código. Para pasar datos de la rutina $A$ a $B$, $A$ debe
llamar explícitamente a $B$, es decir que un llamado tiene que especificar el
destino de los datos.
Por otro lado, en programas dataflow la conectividad puede estar separada
del código, $A$ no pasa datos directamente a $B$; en su lugar, le pasa datos al
sistema de ejecución, quien le pasa los datos a $B$.
El llamador no tiene que especificar hacia dónde van los datos y hasta puede
no saberlo. \cite{DataflowRichardHarter}

\subsection*{Ventajas y Desventajas del paradigma Dataflow}

Entre las ventajas de utilizar el paradigma dataflow se encuentran:

\begin{itemize}
  \item La concurrencia y paralelismo son naturales. El código se puede distribuir entre cores y a través de redes. Algunos
  problemas relacionados a hilos desaparecen
  \item Las redes dataflow son representaciones naturales e intuitivas para
  representar procesos.
  \item El paso de mensajes permite deshacerse de problemas asociados a memoria compartida y locks.
  \item Los programas dataflow son más extensibles que programas tradicionales.
  Los elementos pueden ser agrupados en elementos compuestos.
\end{itemize}

Por otro lado, resulta poco ventajoso utilizar este paradigma por los siguientes
motivos:

\begin{itemize}
  \item El modelo de pensamiento de programación dataflow es poco familiar para
  la mayoría de los programadores profesionales.
  \item La mayoría de los lenguajes de programación dataflow son lenguajes de
  un nicho usado por programadores no profesionales.
  \item La intervención del sistema de ejecución puede tener aparejado un alto
  costo computacional. La gran ventaja de la semántica LIFO es que se implementa
  en código de manera inmediata y poco costosa.
  \item No utilizar memoria compartida tiene sus costos. Los mensajes deben ser
  copiados o deben ser inmutables.
  \item Usar programación dataflow requiere que sea utilizada del principio. De
  esta manera, convertir programas tradicionales en programas dataflow es
  difícil porque la estructura es diferente.
\end{itemize}

\section{Paradigma Reactivo}

El paradigma reactivo es un paradigma de programación construído en torno a
flujos de datos, y la propagación de los cambios sobre ellos. Esto significa que
los lenguajes que implementan este paradigma deben permitir expresar flujos de
datos de manera estática o dinámica con facilidad, y el modelo de ejecución
debe propagar automáticamente los cambios en los datos cuando ocurran,
actualizando todos los valores correspondientes de manera transparente para el
programador.

A fin de comprender las características principales de este paradigma se
presenta el siguiente ejemplo:

\begin{figure}[h!]
\centering
\begin{minted}{perl}
a = 1
b = 2
c = a + b
a = 3
\end{minted}
\end{figure}

En programación imperativa, terminada la ejecución de esta sección de código,
$c$ vale $3$ y así se mantendrá indefinidamente o hasta que el programador le
asigne un nuevo valor. En cambio en programación reactiva el valor de $c$ se
mantiene siempre actualizado, es decir, la expresión declarada como $c$
se vuelve a computar automáticamente ante un cambio en $a$ o en $b$, y en este
ejemplo pasa a valer $5$. Se dice que $c$ es \textit{dependiente} de $a$ y $b$.
\cite{Bainomugisha:2013:SRP:2501654.2501666}

Al igual que en el paradigma dataflow, en el paradigma reactivo son los datos
los que fluyen por el programa en lugar del control. La diferencia radica en
que, bajo el paradigma reactivo, las ``conexiones'' de datos pueden ser
alteradas dinámicamente en tiempo de ejecución.
Además se introducen restricciones de tiempo real blando, para lo cual se
definen dos conceptos:
\begin{itemize}
  \item \textit{Behaviours (Comportamientos)} representan eventos de variación
  contínua en el tiempo. El beahaviour por excelencia es el tiempo, de hecho los
  lenguajes reactivos ofrecen primitivas para representar al tiempo.
  \item \textit{Events (Eventos)} representan eventos discretos. Suelen estar
  representados en forma de flujos de cambios de valores. A diferencia de los
  behaviours, los eventos cambian en instantes puntuales del tiempo. Los
  lenguajes reactivos ofrecen primitivas para combinar y procesar eventos.
\end{itemize}
\cite{Bainomugisha:2013:SRP:2501654.2501666}

\subsection*{El Paradigma Reactivo y El Patrón Observer}

El patrón de diseño \textit{observer} \cite{Gamma:1995:DPE:186897} nace de la
necesidad de mantener consistencia de datos en sistemas particionados, sin generar
acoplamiento entre capas de dichos sistemas.
Permite que un \textit{sujeto} publique cambios en su estado a sus
\textit{observadores}, quienes se susbribieron previamente a estas
actualizaciones.

El patrón observer se debe utilizar en alguna de las siguientes situaciones:
\begin{itemize}
  \item Cuando una abstracción tiene dos partes, una dependiente de la otra.
  \item Cuando el cambio en un objeto implica el cambio en otros, y no se sabe
  de antemano cuántos ni quiénes deben aplicar estos cambios.
  \item Cuando un objeto debe notificar a otros sin conocer nada de ellos, es
  decir sin generar acomplamiento.
\end{itemize}
\cite{Gamma:1995:DPE:186897}

Analizando este patrón de diseño se encuentran similitudes con el paradigma
reactivo.
La programación reactiva es capaz de explicitar mayor granularidad, pudiendo
describir flujos de datos a nivel de clases, miembros de éstas y hasta
variables, mientras que el patrón observer lo puede hacer a nivel de clases
únicamente.

En programación reactiva, cuando se forma una expresión dependiente de otras, se
genera una suscripción implícita de manera automática y el modelo de ejecución
es el encargado de propagar los cambios de manera transparente para el
programador.


            \section{Programación Orientada a Aspectos}
\label{sec:aop}

\subsection{Concepto}

La programación orientada a aspectos es un paradigma de programación que tiene
como objetivo incrementar la modularidad mediante la separación de intereses
transversales (cross-cutting concerns). 
Los intereses transversales son aspectos de un programa que afectan a otros
intereses. Son partes de un programa que afectan o dependen de muchas
otras partes del sistema.
Estos intereses usualmente no pueden separarse claramente del resto del
sistema, y pueden resultar en duplicación de código o un alto grado de
dependencia entre partes del sistema.


Los intereses transversales son la base para el desarrollo de aspectos. Estos no
pueden ser representados claramente en los paradigmas de programación orientado
a objetos o programación procedural. \cite{AspectJInAction}
La separación de intereses transversales se realiza añadiendo comportamientos
adicionales al código existente, llamados advices o consejos, sin modificar el
mismo. Para lograrlo, se especifican puntos de ejecución (mediante la definición
de pointcuts) donde se aplican los advices previamente mencionados.

La programación orientada a aspectos complementa a la programación orientada a
objetos al permitir al desarrollador modificar dinamicamente el modelo estático
orientado a objetos para crear un sistema que puede crecer para cumplir nuevos
requerimientos. Tal como los objetos en el mundo real pueden cambiar sus estados
a lo largo de su vida, una aplicación puede adoptar nuevas características a
medida que se va desarrollando. \cite{Introduction_To_Aspect}


\subsection{Terminología}
\label{sec:aop_terminologia}
\begin{itemize}
  \item Intereses Transversales (Cross-cutting concerns): Aunque la mayoría de
  las clases en un modelo orientado a objetos está destinada a perfeccionar una función única y
  específica, usualmente comparten requerimientos secundarios en común con otras
  clases. Por ejemplo, se puede desear añadir mecanismos de logueo a las clases
  dentro de la capa de acceso de datos y también a las clases en la capa de
  interfaz de usuario cada vez que un hilo entre o salga de un método. Aunque la
  funcionalidad principal de cada clase es muy diferente, el código necesario
  para realizar la tarea secundaria es usualmente
  idéntico.\cite{Introduction_To_Aspect}
  
  \item Consejos (Advices): Es el código adicional que se desea aplicar al
  modelo existente. Siguiendo con el ejemplo anterior, es el código de logueo
  que se quiere aplicar cada vez que un hilo ingrese o salga de un
  método.\cite{Introduction_To_Aspect}
  
  \item Punto de unión (Join-point): Es el término que se le otorga al punto
  de ejecución en la aplicación en el cual los intereses transversales deben ser
  aplicados. En el ejemplo, un punto de unión es alcanzado cuando un hilo
  ingresa a un método, y un segundo punto de unión es alcanzado cuando un hilo
  sale de un método.
  
  \item Punto de corte (Point-cut): Un punto de corte es un conjunto de puntos
  de unión. Un point-cut permite definir dónde aplicar exactamente un consejo,
  lo cual permite la separación de intereses y ayuda a modularizar la lógica de
  negocios \cite{Classification_Of_Pointcut_Language_Constructs}.
  
  \item Aspecto (Aspect): La combinación de un punto de corte y un consejo se
  denomina aspecto. \cite{Introduction_To_Aspect}
  
  \item Tejido (Weaving): Proceso de aplicar aspectos a los objetos
  destinatarios para crear los nuevos objetos resultantes en los puntos de
  unión especificados. De acuerdo al momento del ciclo de vida del sistema en
  el cual se aplica el tejido, se realiza la siguiente clasificación:
  	\begin{itemize}
		\item Aspectos en Tiempo de Compilación.
		\item Aspectos en Tiempo de Carga.
		\item Aspectos en Tiempo de Ejecución.
	\end{itemize}
\end{itemize} 
            \section{Programación Orientada a Objetos}

No es del interés de este proyecto integrador ahondar en el campo de la
programación orientada a objetos, y se asume que el lector tiene conocimientos
afianzados sobre este paradigma de programación. Aún así, es necesario nombrarlo
para poder introducir los conceptos desarrollados en la sección \ref{reflection}

\subsection{Reflection}
\label{reflection}

Existen escenarios en la programación donde es útil tener la opción de conocer
los datos disponibles y las operaciones que se pueden aplicar sobre estos datos
en tiempo de ejecución. Además, resulta ventajoso poder tomar decisiones sobre
estos datos y operaciones en base al flujo del programa para modificar el
comportamiento del mismo. Tener esta posibilidad permite escribir software
flexible, reutilizable y capaz de adaptarse a múltiples escenarios. Estas
capacidades se pueden obtener por medio de la programación basada en la
\textit{reflexión} del programa.

Reflexión (o reflection) es la habilidad de un programa de examinarse a sí
mismo y a su entorno en tiempo de ejecución, y de cambiar su comportamiento
dependiendo de lo que encuentra.

Para realizar esta autoexaminación, un programa necesita tener una
representación de sí mismo. Esta información se llama \textit{metainformación} o
\textit{metadata}. En particular, en un entorno de programación orientada a
objetos la metadata se organiza en objetos, llamados \textit{metaobjetos}. La
revisión en tiempo de ejecución de los metaobjetos se llama
\textit{introspección} o \textit{introspection}.
\cite{Forman04javareflection}

En general, la introspección está seguida de un cambio del comportamiento.
Existen tres técnicas que una interfaz de programación de reflection puede
ofrecer para generar cambios de comportamiento:
\begin{itemize}
    \item Modificación de los metaobjetos
    \item Operaciones con la metadata: como la invocación dinámica de métodos 
    \item Intercesión: Se le permite al código interceder en varias fases de la
    ejecución para alterar el comportamiento del programa.
\end{itemize}

Estas características hacen que el uso de reflection permita diseñar software
más flexible, que se adapte más fácilmente a cambio de requerimientos y que a
la vez mantenga una estructura ordenada y buena legibilidad de código.
Esto favorece a la mantenibilidad.

Para poder hacer introspection, un programa que aplique reflection debe poder
acceder a su metainformación. Por esto, esta representación es el elemento
estructural más importante de un sistema reflectivo. Examinando su
autorepresentación, un programa puede obtener información acerca de su
estructura y comportamiento para tomar decisiones importantes.

Existen tres problemas relacionados al uso de reflection en el diseño de un
programa que deben ser tenidos en cuenta para poder asegurar la calidad del
mismo. Estos son:
\begin{itemize}
    \item Seguridad
    \item Complejidad del código
    \item Performance
\end{itemize}
Todos ellos se pueden mitigar mediante el uso de buenas prácticas de programación y un correcto diseño del software.
        \chapter{Frameworks y Generación de Código}
            \section{Generación de Código Fuente}

La idea de generación automática de código fuente y de código ejecutable es casi
tan antigua como la programación en sí misma. Debido a que ahorra mucho
tiempo y costo de desarrollo de sistemas, ha sido y sigue siendo foco de
investigación.

La generación automática de código fuente está englobada por el concepto de
\textit{Programación Automática}. El significado de este término ha avanzado
junto a la programación a lo largo de los años:
\begin{itemize}
  \item En la década de 1940, se llamó de esta forma a la automatización del
  proceso de perforar cintas de papel para escribir el programa
  \cite{AutomaticProgrammingGorn}. Lo que Gorn llamó programación automática, es
  en realidad un lenguaje assembler.
  \item En el comienzo de los lenguajes de alto nivel se le llamó de esta manera
  a los compiladores. Tanto es así que uno de los primeros compiladores se llamó
  Autocode.
  \item Actualmente se identifica este término como la generación de código
  fuente escrito en un lenguaje de programación (compilable o interpretable a
  código máquina) a partir de una descripción de más alto nivel.
\end{itemize}

Algunos ejemplos de generadores de código son:
\begin{itemize}
  \item \underline{Apache Thrift:} Desarrollado por Facebook y actualmente
  liberado bajo licencia Apache, Thrift es un generador de servicios para múltiples lenguajes
  orientado a la comunicación por medio de llamadas a procedimiento remoto
  (remote procedure call – RPC). Para lograr esto expone un lenguaje de
  definición de interfaces (interface definition language – IDL) propio,
  utilizado para describir el servicio que luego Thrift generará en alguno de
  los múltiples lenguajes que soporta.\cite{ApacheThrift}
  \item \underline{Acceleo}: Es un generador de código que implementa el
  estándar \textit{MOFM2T} desarrollado por The Eclipse Foundation y su código fuente
  está liberado bajo licencia EPL. Acceleo permite la especificación del
  software en modelos como UML (v1 y v2), EMF (eclipse model framework) y
  lenguajes de modelado personalizados (DSL). Además permite especificar
  plantillas definidas por el usuario. Genera código en múltiples lenguajes de
  programación.\cite{Acceleo}
  \item \underline{Actifsource:} Desarrollado por actifsource GmbH y de código
  cerrado, es un generador de código a partir de modelos similares a UML. Soporta la
  creación de múltiples modelos y la unión de estos, y la utilización de modelos
  generados en cualquier software que tolere formato ecore, definido por el
  Eclipse Modeling Framework. Está desarrollado como un plugin para
  Eclipse.\cite{Actifsource}
  \item \underline{Spring Roo:} Desarrollado conjuntamente por DISID y Pitvotal
  bajo licencia Apache 2.0, es un generador enfocado al desarrollo acelerado de
  software empresarial en Java. La aplicación generada utiliza tecnologías Java
  comunes como Spring Framework, Java Persistence API, Apache Maven, etc. A
  diferencia de otros generadores de código, Roo expone una interfaz por línea
  de comandos con sus propios comandos.\cite{SpringRoo}
  \item \underline{GeneXus:} Desarrollado por ARTech bajo licencia cerrada y con
  primer lanzamiento en 1988, es un generador de código fuente a partir de un
  lenguaje declarativo de alto nivel. A partir de este lenguaje, se genera
  código fuente en C\#, COBOL, Java, Objective-C, RPG, Visual Basic,
  Ruby y Visual FoxPro. Además tolera múltiples lenguajes para gestión de bases de
  datos como Microsoft SQL, Oracle, DB2, Informix, PostgreSQL y
  MySQL.\cite{Genexus}
  \end{itemize}
  
La programación automática siempre ha sido un eufemismo para la
programación con un lenguaje de más alto nivel del disponible para el
programador. Investigar en programación automática es simplemente desarrollar
la implementación de lenguajes de programación de más alto nivel
\cite{Parnas:1985:SAS:214956.214961}.

En conclusión, los generadores automáticos de código fuente son en realidad
traductores de un lenguaje de programación a otro. Esto brinda un mayor nivel
de abstracción para el programador pero lo obliga a especificar su software en
el lenguaje provisto por el generador.

            
\section{Frameworks}


\subsection{Definición}

Los frameworks son una técnica de reutilización de prácticas, conceptos y
criterios orientadas a facilitar la solución de un tipo de problemáticas en
particular. Son estructuras concretas de software, que proveen una manera
estándar de construir aplicaciones. Sirven como base para el diseño y
desarrollo de software orientado a resolver problemas específicos.
De acuerdo a \cite{Johnson97} dos de las definiciones más comunes de framework
son:
\begin{itemize}
  \item ``Un framework es un diseño reusable de todo o parte de un sistema que es
  representado por un conjunto de clases abstractas y la forma en que sus
  instancias interactúan''
  \item ``Un framework es el esqueleto de una aplicación que puede ser
  personalizado por un desarrollador de aplicaciones.''
\end{itemize} 

 La primer definición describe la estructura de un framework, mientras que la
 segunda describe su propósito.

Un framework es una técnica de reutilización de código porque facilita la
creación de una aplicación a partir de una biblioteca de componentes existentes.
Es posible la creación de nuevos componentes extendiendo los provistos por el
framework. Una característica que distingue a los frameworks de otras
técnicas de reutilización es la inversión de control (ver
sección\ref{sec:inversion_control}).

Debe pensarse en frameworks y componentes de software como tecnologías
diferentes, pero que cooperan entre sí \cite{JohnsonFeb97}:
\begin{itemize}
  \item Un framework provee un contexto reusable para los componentes.
  \item Un framework es más abstracto y flexible que los componentes.
\end{itemize} 

Por otro lado, los frameworks son más concretos y simples de reutilizar
que un diseño puro \cite{JohnsonFeb97}.

\subsection{Inversión de Control}
\label{sec:inversion_control}
La inversión de control es una característica principal de los
frameworks. Es un principio de diseño en el cual porciones de código
personalizado por el usuario son controladas por un framework.

Al implementar un sistema sin utilizar un framework, generalmente el
desarrollador escribe el código de un programa principal que realiza llamadas a
componentes de una biblioteca. El desarrollador decide en el código cuándo
llamar al componente y se encarga de la estructura y el flujo de control del
programa.

En un software basado en un framework el programa principal es reutilizado. 
El desarrollador solamente conecta componentes existentes al framework, o
implementa nuevos componentes para conectar. Las porciones de código del
desarrollador son llamadas por el framework. De esta manera, el framework
determina la estructura y el flujo de control del programa.

La inversión de control sirve para los siguientes propósitos de diseño:
\begin{itemize}
  \item Desacoplar la ejecución de una tarea de su implementación.
  \item Mantener el foco en la tarea para la que fue diseñado el módulo.
  \item Guiar el diseño respetando las interfaces entre módulos.
  \item Evitar efectos colaterales al reemplazar un módulo.
\end{itemize}

\subsection {Ventajas de los frameworks}
\begin{itemize}
	\item Al utilizar un framework se aplican técnicas de reutilización de
	software y de diseño.
	
	\item Son personalizables: Los frameworks son más
	personalizables que la mayoría de los componentes. Tienen interfaces más
	complejas.
	
	\item Sirven para múltiples aplicaciones: Un framework está orientado a
	facilitar la implementación de aplicaciones de un tipo determinado. En
	consecuencia, puede ser utilizado para implementar diversas aplicaciones que
	pertenezcan a dicho tipo.
	
	\item Facilitan el trabajo del desarrollador.
	
	\item La uniformidad reduce los costos de mantener el código: Los programadores
	encargados de mantenerlo pueden cambiar de una aplicación a otra que utiliza el
	mismo framework sin tener que aprender un nuevo diseño.
	
	\item Los frameworks obligan al usuario a respetar patrones de diseño en las
	aplicaciones.

\end{itemize}

\subsection {Desventajas de los frameworks}
\begin{itemize}
    \item Curva de Aprendizaje: Los programadores deben aprender las interfaces
    antes de poder utilizar el framework. Generalmente aprender un nuevo
    framework es difícil.
    
    \item Restricción de elección del lenguaje de programación: Uno de los
    problemas de utilizar un framework implementado en un lenguaje en particular
    es que restringe a los sistemas a utilizar dicho lenguaje. La relación
    efectividad-costo es baja al construir una aplicación en un lenguaje con un
    framework escrito en otro.
    
    \item Debido a que los frameworks son descritos con lenguajes de
    programación, es difícil para los desarrolladores aprender los patrones
    colaborativos de un framework mediante la lectura del código.
    
\end{itemize}

\subsection{Frameworks desde la perspectiva del usuario}
\label{sec:tipos_framework}
    Según \cite{JohnsonFeb97} existen tres formas de utilizar un framework
    desde la perspectiva de un usuario desarrollador de software: 
\begin{itemize}
    \item Black-Box Frameworks: Consiste en conectar componentes ya existentes.
    De esta forma no se modifica el framework ni se crean nuevas clases
    concretas sino que se reutilizan las interfaces del framework y sus reglas
    para interconectar componentes. Este método es similar a la construcción de
    un circuito eléctrico. El desarrollador necesita conocer la interfaz de
    conexión entre un objeto A y un objeto B, pero no es necesario que conozca
    la especificación exacta de A o B.

    \item White-Box Frameworks: Consiste en definir clases concretas, que
    extienden de clases abstractas definidas en el framework, y utilizarlas
    para implementar una aplicación. Las subclases están estrechamente
    acopladas a sus superclases. De esta forma se requiere más conocimiento
    acerca de la implementación de las clases que conforman el framework.
	
	\item Extensión o Modificación del núcleo del framework:  Consiste en extender
	el framework cambiando las clases abstractas que forman su núcleo para añadir
	nuevas variables u operaciones. Requiere conocimientos avanzados acerca del
	diseño del framework. Cambiar las clases abstractas puede provocar fallos en
	las clases concretas existentes. Este modo de utilización no es aplicable si
	el propósito es crear un sistema abierto.
\end{itemize}

Entre las formas de utilización mencionadas existen combinaciones intermedias. Es común
que los frameworks se utilicen como Black-Box la mayor parte del tiempo y
sean extendidos cuando la ocasión lo demande.

\section{Comparación entre Frameworks y APIs}

En la tabla \ref {tab:comparacion_frameworks_apis} se observa una comparación
entre frameworks y APIs.
\begin{table}
	\centering
	\begin{tabularx}{\textwidth}{ | p{2.5cm} | X | X | }
	\hline
	\textbf{Categoría} & \textbf{Framework} & \textbf{API (Biblioteca)} \\[10pt]
	\hline
    \textbf{Gestión del Flujo Principal} & El framework toma el
    flujo principal del software & A cargo del programador\\[10pt] \hline
    \textbf{Confiabilidad} & El flujo principal está ampliamente testeado por todos
    los usuarios del framework & No brinda ninguna garantía de flujo\\[10pt] \hline
	\hline \textbf{Extensibilidad} & Por parte de los desarrolladores del
	framework.
	Si es de código abierto cualquiera puede extenderlo. & Por parte del fabricante de la
	librería.
	O cualquiera si es de código abierto \\[10pt] \hline
	\textbf{Reusabilidad} & Objetivo principal del diseño de un framework.
	Se aplica a nivel de arquitectura de software & Es reutilizable a nivel
	de llamada a métodos\\[10pt] \hline 
	\textbf{Complejidad de Uso} & Gran
	complejidad al principio, se simplifica a medida que el usuario aprende el framework & Complejidad inicial menor que un
	framework\\[10pt] \hline 
	\textbf{Aplicación de Patrones de diseño} & Usualmente un framework fuerza al
	usuario a utilizar uno o varios patrones & No obliga al usuario a utilizar ningún patrón
	de diseño\\[10pt] \hline 
	\textbf{Especificidad / Generalización} & Son de uso específico,
	están diseñados para resolver una familia de problemas. Por esto mantienen una
	arquitectura & De uso general donde una funcionalidad pueda ser
	utilizada\\[10pt] \hline 
	\textbf{Restricciones de lenguaje} & Obliga al usuario a
	desarrollar en el mismo lenguaje en el que está hecho el framework & No restringe a un lenguaje.
	Permite llamadas desde cualquier lenguaje mientras se respete la firma de las funciones expuestas\\[10pt] 
	\hline
	\end{tabularx}
	\caption{Comparación entre Frameworks y APIs}
	\label{tab:comparacion_frameworks_apis}
\end{table}


        \chapter{Concurrencia}
            \section{Introducción}
\label{IntroduccionConcurrencia}

``La idea de programación concurrente siempre estuvo asociada al mundo de los
\textit{Sistemas Operativos}. No en vano, los primeros programas concurrentes
fueron los propios Sistemas Operativos de multiprogramación en los que un solo
procesador debía repartir su tiempo entre muchos usuarios.''\cite{PalmaConcurrente}

En las últimas dos décadas, la programación concurrente ganó gran interés y
actualmente está presente en la mayoría de las aplicaciones.
Esto se debe principalmente a algunos grandes hitos en la programación:
\begin{itemize}
	\item La aparición del concepto de \textit{hilo} o \textit{thread}. Permiten la
	ejecución de programas de manera más rápida y eficiente que los programas
	basados en procesos.
	\item La aparición de lenguajes de alto nivel con soporte para
	programación de hilos y de procesos.
	\item La aparición de internet, entorno donde la concurrencia se hace necesaria
	en todo aspecto.
	\item El desarrollo y gran avance de hardware capaz de ejecutar múltiples hilos
	y procesos de forma paralela. Esto permite aprovechar las ventajas de
	performance de la concurrencia. Las principales arquitecturas capaces de
	explotar el paralelismo a nivel de hilo y/o de proceso son
	\begin{itemize}
	    \item Procesadores Multi-Core
	    \item Procesadores Many-Core
	    \item Procesadores con soporte Multi-Thread
    \end{itemize}
\end{itemize}

\section{Programación Concurrente}
\label{ProgramacionConcurrente}

La \textit{programación concurrente} es la disciplina que se encarga del estudio
de las notaciones que permiten especificar la ejecución concurrente de las
acciones de un programa, así como resolver los problemas inherentes a la
ejecución concurrente (ver \ref{ProblemasConcurrencia}). Es de interés
formalizar el concepto de ejecución concurrente y de ejecución paralela a fin
de poder diferenciarlos:

\begin{itemize}
	\stepcounter{definitionsCounter}
	\item [\underline{Definición \thedefinitionsCounter :} ] Dos hilos
	\footnote{En adelante, se hablará de concurrencia de hilos dado que los
	sistemas de referencia son de memoria compartida. En el caso que corresponda
	hablar de procesos se lo mencionará explícitamente.} son \textit{concurrentes}
	si la primera instrucción de uno de ellos se ejecuta después de la primera del
	otro y antes de la última.
	\stepcounter{definitionsCounter}
	\item [\underline{Definición \thedefinitionsCounter :} ] Dos hilos 	se están
	ejecutando de manera \textit{paralela} si son concurrentes y la ejecución de
	ambos se da al mismo tiempo.
\end{itemize}

Para que dos hilos sean concurrentes no es necesario que se ejecuten al mismo
tiempo, es suficiente que exista un intercalado entre la ejecución de sus
instrucciones \cite{PalmaConcurrente}. En este proyecto integrador, es de
interés fundamentalmente la ejecución concurrente.

Anteriormente en esta sección se mencionó que existen problemas aparejados a la
programación de sistemas concurrentes. Sabiendo esto resulta necesario conocer
las ventajas de la programación concurrente, que justifiquen su uso por encima de
las dificultades que genera.

\subsection{Ventajas de la Programación Concurrente}

Los beneficios de programar de manera concurrente pueden englobarse en tres
categorías:

\begin{itemize}
	\item \underline{Incremento en la velocidad de ejecución:} Cuando se ejecuta un
	programa concurrente en un entorno multiprocesador, los distintos hilos que
	lo forman pueden ejecutarse de manera paralela, con lo que el tiempo total de
	ejecución se reduce. Esto es especialmente ventajoso en programas de cálculo
	numérico.
	\item \underline{Solución de problemas inherentemente concurrentes:} Existen
	problemas cuya naturaleza es concurrente, por lo que un modelo de programación
	de este tipo se adapta más naturalmente a la resolución de estos problemas.
	\item \underline{Mejor aprovechamiento del tiempo de CPU:} Un sistema operativo
	con un ambiente de multiprogramación que permita la concurrencia es capaz de
	desalojar a un hilo que está esperando por un evento y no está haciendo uso
	de la CPU para brindarle este tiempo a otro hilo que lo requiera.
\end{itemize}

\subsection{Problemas y Propiedades de la Concurrencia}
\label{ProblemasConcurrencia}

Como se introdujo en la sección \ref{ProgramacionConcurrente}, existen
problemas que aparecen al programar de manera concurrente. Esto lleva a que los
programas concurrentes deban satisfacer una serie de propiedades (además de su
especificación técnica del dominio del problema) para funcionar correctamente.

Estas propiedades se dividen en dos grupos:

\subsubsection*{Propiedades de Seguridad}
Las propiedades de seguridad aseguran que ``nada malo'' va a pasar en la
ejecución del programa \cite{PalmaConcurrente}.
Estas son:

\begin{itemize}
    \item \underline{Exclusión Mutua:} Existen recursos que no pueden ser
    accedidos concurrentemente para evitar problemas de coherencia. Por esto se
    debe garantizar que a lo sumo un hilo está accediendo a un recurso de este
    tipo en un instante dado.
    \item \underline{Condición de Sincronización:} Se pueden dar situaciones
    donde un hilo debe esperar la ocurrencia de un evento para poder
    continuar su flujo. Ante estos casos se debe garantizar que el hilo
    espere por dicha ocurrencia, de otro modo el resultado puede ser indefinido
    o inesperado.
    \item \underline{Interbloqueo \textit{(Deadlock)}:} Sucede cuando dos o más
    hilos están esperando a que suceda un evento que nunca ocurrirá para
    continuar sus flujos de ejecución. El evento no ocurre porque las
    condiciones para que suceda están bloqueadas por los propios hilos.\\
    Para que el interbloqueo suceda efectivamente se tienen que cumplir las
    siguientes condiciones:
    \begin{itemize} 
        \item Exclusión Mutua: si no se exige exlusión mutua, no puede haber
        interdependencia entre los hilos.
        \item Retención y Espera: cada hilo debe retener un recurso y esperar
        a que se libere otro.
        \item No Apropiación: no se puede forzar a un hilo a que desaloje un
        recurso
        \item Circulo Vicioso de Espera: Se forma una cadena cerrada de
        hilos, donde cada uno retiene al menos un recurso que necesita el
        próximo hilo de la cadena.
    \end{itemize}
    Las tres primeras condiciones son necesarias pero no suficientes para que
    efectivamente ocurra el interbloqueo. La cuarta condición nace como
    consecuencia de las tres primeras, siempre que se produzca una secuencia de
    eventos que desemboque en un círculo de espera irresoluble
    \cite{SistOpStallings}.
\end{itemize}

\subsubsection*{Propiedades de Vivacidad}

Si se aseguran las propiedades de vivacidad, ``algo bueno'' pasará eventualmente
en la ejecución del programa.

\begin{itemize}
    \item \underline{Interbloqueo Activo \textit{(Livelock)}:} Se produce un
    interbloqueo activo cuando un sistema ejecuta una serie de instrucciones sin
    hacer ningún progreso. Esto se da cuando $N$ hilos necesitan $N$ recursos
    y se los intercambian sin obtener nunca el conjunto completo.
    \item \underline{Inanición \textit{(Starvation)}:} Se da cuando al menos una
    parte del sistema nunca recibe los recursos necesarios para continuar, o
    demora lo suficiente en recibirlos como para que no sean útiles para obtener
    el resultado esperado. No es necesario que todo el sistema se bloquee para
    estar en una situación de inanición.
\end{itemize}

\section{Mecanismos de Sincronización}

A fin de garantizar el cumplimiento de las propiedades introducidas en la
sección \ref{ProblemasConcurrencia}, es necesario sincronizar la ejecución de
los hilos. De lo contrario, se puede caer en problemas de coherencia
y/o consistencia de datos, o corrupción.

\subsection{Cooperación vs Competencia}

La sincronización de hilos se puede implementar basada en dos principios
\cite{SistOpStallings}:
\begin{itemize}
    \item \underline{Cooperación:} Los hilos se comunican entre ellos para
    cooperar en la compartición de recursos. A su vez, existen dos tipos de
    cooperación:
    \begin{itemize}
        \item \underline{Cooperación por Compartición:} Los hilos interactúan
        para gestionar los recursos. No tienen conocimiento explícito de los demás.
        \item \underline{Cooperación por Comunicación:} Los hilos interactúan
        para gestionar los recursos mediante el paso explícito de mensajes entre
        ellos.
    \end{itemize}
    \item \underline{Competencia:} Los hilos compiten entre sí por los
    recursos. La gestión de los recursos se efectúa por otra entidad, como lo
    puede ser el sistema operativo.
\end{itemize}

\subsubsection{Cooperación por Compartición de Recursos}

Los hilos inteactúan entre ellos sin tener conocimiento explícito de la
existencia de los demás.

Existen regiones de almacenamiento de datos compartidas (espacios de memoria,
archivos, bases de datos, etc) que pueden ser leidas y escritas por múltiples
hilos.

Si bien un hilo no hace referencia a ningún otro, es conciente de que los
datos compartidos pueden ser accedidos y modificados por los demás. Por lo que
el conjunto debe cooperar para asegurar que los datos compartidos se gestionen
correctamente. Es responsabilidad de los mecanismos de control asegurar la
integridad de los datos compartidos \cite{SistOpStallings}.

Como los datos se almacenan en recursos compartidos, existen los problemas de
exclusión mutua, interbloqueo e inanición vistos en la sección
\ref{ProblemasConcurrencia}. La principal diferencia es que existen dos modos de
acceder a los datos: para \textit{lectura} y para \textit{escritura}. Únicamente
se debe asegurar la exclusión mutua para operaciones de escritura ya que sólo
estas pueden romper la \textit{coherencia} y \textit{consistencia} de los datos.

Un conjunto de datos son coherentes si independientemente de quién haya sido el
último escritor, cualquier lector obtiene el último conjunto de valores escrito.
Por otro lado, un dato es consistente si un lector obtiene un valor que fue
realmente escrito por un escritor y no un dato corrupto.

Algunos mecanismos para gestionar el uso de los datos compartidos son:
\begin{itemize}
    \item Semáforos: desarrollado en la sección \ref{semaforos}
    \item Monitores: desarrollado en la sección \ref{monitores}
\end{itemize}

\subsubsection{Cooperación por Comunicación entre Hilos o Procesos}

Cuando los hilos o procesos cooperan por comunicación, participan en alcanzar un
objetivo en común. La comunicación es una manera de sincronizar o coordinar las
distintas actividades.

La comunicación está formada por está formada por el emisor, el receptor, el
canal y el mensaje. El envío y recepción de mensajes son explícitos.
Las herramientas para este paso de mensajes están dadas por el lenguaje de
programación, alguna biblioteca o por el sistema operativo.

Al no haber compartición de datos entre los hilos o procesos, no es necesaria la
ejecución en exclusión mutua. Pese a esto, el interbloqueo y la inanición siguen
siendo problemas que pueden afectar a los hilos o proceos
\cite{SistOpStallings}.

\subsubsection{Competencia entre Hilos}

Los hilos no tienen forma de comunicarse entre ellos para gestionar los
recursos.

Si dos hilos desean acceder a un mismo recurso, el sistema operativo se lo
asignará a uno de ellos y el otro tendrá que esperar. Se debe garantizar:
\begin{itemize}
    \item La toma de los recursos en exlusión mutua.
    \item Correcta gestión de los recursos para evitar interbloqueos.
    \item La reactivación de los hilos bloqueados en un tiempo prudente a fin de
    evitar su inanición.
\end{itemize}
El control de la competencia involucra al sistema operativo
inevitablemente, porque es él quien asigna los recursos del sistema.
Además, los hilos deben ser capaces por sí mismos de expresar de algún
modo los requisitos de exclusión mutua, como puede ser bloqueando los
recursos antes de usarlos. Cualquier solución conlleva alguna ayuda del
sistema operativo, como la provisión del servicio de
bloqueo.\cite{SistOpStallings}

\begin{framed}
\textbf{Nota:}A los fines de este proyecto integrador sólo es de interés la
concurrencia basada en memoria compartida. Dentro de este modelo se destacan dos
mecanismos de sincronización por competencia: los \textit{semáforos} y los
\textit{monitores}.
\end{framed}

\subsection{Semáforos}
\label{semaforos}

Los semáforos fueron el primer mecanismo de sincronización de hilos por
cooperación. Fueron desarollados por E. Dijkstra en 1965 como mecanismos
eficientes y fiables para dar soporte a la cooperación de hilos en un sistema
operativo.

El principio en el que se basan es simple. Un conjunto de hilos pueden
cooperar utilizando señales, de manera que se pueda obligar a un hilo a
detener su ejecución en un punto específico hasta recibir una señal conocida.
La señalización está a cargo de los semáforos.

Para transmitir una señal sobre el semáforo $s$, el hilo $p$ debe ejecutar
$signal(s)$, y para recibir una señal de $s$, debe ejecutar $wait(s)$. Si la
señal no fue transmitida, $p$ se bloquea hasta recibir la señal.

Efectivamente, las operaciones sobre los semáforos son tres:
\begin{itemize}
    \item \underline{$init(sem\ s,\ uint\ n)$:} inicializa al semáforo $s$ con
    un entero positivo $n$.
    \item \underline{$wait(sem\ s)$:} decrementa el valor del semáforo. Si se
    hace negativo, el hilo que realiza la llamada se bloquea. También se la
    llama \textit{acquire}.
    \item \underline{$signal(sem\ s)$:} incrementa el valor del semáforo. Si
    había un hilo bloqueado por una llamada a $wait(s)$, se desbloquea. También
    se la llama \textit{release}.
\end{itemize}

Las llamadas a $signal(s)$ y $wait(s)$ son atómicas para asegurar la
modificación del contador del semáforo en eclusión mutua.

Los hilos que esperan una señal luego de bloquearse por una llamada a
$wait(s)$ deben hacerlo en una cola de espera. Esta cola implementa una política
que decide cuál hilo bloqueado se libera ante la llegada de una señal. El
caso más típico es FIFO, pero se puede implementar otro. Sea cual fuera la
política implementada, se debe asegurar que ningún hilo bloqueado sufrirá
inanición por ella.

Los semáforos descriptos hasta este punto son de tipo \textit{semáforo
general}.
Existe una versión más reducida que sólo puede adquirir valores $0$ y $1$ llamada
\textit{semáforo binario}. Los semáforos binarios son de implementación más
simple que los generales y se demuestra que tienen la misma potencia de
expresividad. \cite{SistOpStallings}

\subsection{Monitores}
\label{monitores}

Los semáforos son herramientas simples y potentes para la gestión de la
concurrencia. Permiten gestionar la ejecución en exclusión mutua y coordinar
hilos. El problema de los semáforos radica en que las operaciones signal(s)
y wait(s) están distribuidas por el código de todos los hilos que lo usan,
con lo que resulta muy difícil entender y predecir el efecto de una operación
sobre todos los hilos que dependen del mismo semáforo.

Para solucionar este problema, C. Hoare definió el concepto de monitor en su
artículo “Monitors: An Operating System Structuring Concept.” en 1974.

Los monitores, al igual que los semáforos, son herramientas de gestión de la
concurrencia entre hilos. Los hilos los usan para asegurar el acceso en
exclusión mutua a recursos y para sincronizar y comunicarse con otros
hilos.\\
El propósito de un monitor de concurrencia es centralizar la gestión de los
recursos compartidos en una sección del código del programa. De esta manera, la
responsabilidad de sincronizar a los hilos para evitar problemas de concurrencia
es enteramente del monitor y no de cada hilo que quiera acceder a un recurso.

Un monitor consiste en un grupo de datos y un conjunto de rutinas exportadas
(llamadas \textit{rutinas de entrada}). Estas rutinas realizan operaciones sobre
los datos. Los datos del monitor representan recursos compartidos para múltiples
hilos (ya sean de software o de hardware) y pueden ser modificados únicamente
dentro de las rutinas del monitor.

La forma que tiene un monitor de gestionar concurrencia es:
\begin{itemize}
    \item Asegurando la ejecución de sus rutinas en exclusión mutua.
    \item Gestión de los recursos de forma implícita o explícita
\end{itemize}

Para asegurar la ejecución en exclusión mutua, sólo se permite que un único
hilo pueda ejecutar una rutina del monitor a la vez. Este hilo recibe el
nombre de \textit{hilo activo}. El hilo activo bloquea la entrada al
monitor cuando ejecuta una rutina y la desbloquea cuando cede
\textit{voluntariamente} el control del monitor. Si otro hilo llama a una
rutina de entrada mientras el monitor está bloqueado, se bloquea en una cola de
entrada al monitor hasta que este pase a estado desbloqueado.

\subsubsection{Sincronización Explícita}
\label{monitor_sincronizacion_explicita}

En muchas ocasiones resulta necesario no sólo garantizar la exclusión mutua
dentro del monitor sino sincronizar hilos dentro de él para la correcta
gestión de los recursos compartidos (cola de cortesía). Para esto el monitor
representa los recursos con variables de condición (o colas de eventos)

\begin{framed}
\paragraph{Variables de condición:} Son variables especiales, sobre las que se
pueden realizar dos acciones:
\begin{itemize}
    \item delay: Suspende al hilo que la llama, a la espera de una señal.
    \item signal: Levanta el estado de suspensión de un hilo suspendido por
    una llamada a \textit{delay()} sobre ella. Si no hay ningún hilo
    suspendido no tiene efecto.
\end{itemize}
Si existe más de un hilo suspendido en una variable de condición cuando otro
hace una llamada a \textit{signal()}, el hilo a despertar será elegido
aplicando una política determinada. Esta política puede ser FIFO, por
prioridades por hilo, etc.
\end{framed}

El hilo activo del monitor puede suspenderse a sí mismo temporalmente bajo
una condición x ejecutando \textit{delay(x)}. Al suspenderse deja de ser el
hilo activo y se sitúa al final de la cola de la condición x, a la espera de
volver a entrar al monitor cuando la condición cambie. Previo a esto debe
desbloquear la entrada al monitor para no generar un interbloqueo con los demás
hilos que intentan acceder. Por otro lado, otro hilo que sea el activo
puede hacer una llamada a \textit{signal(x)} si detecta un cambio en \textit{x},
desbloqueando un hilo suspendido en su cola de condición asociada.

Es común asociar una variable de condición a una proposición lógica sobre el
estado de un recurso gestionado por el monitor, por ejemplo “El buffer A no
está lleno”. De esta manera esperar por esta condición equivale a esperar a que
el buffer A no esté lleno. Esta asociación suele ser implícita, es decir que le
da semántica al monitor pero no forma parte de su funcionamiento.

\subsubsection{Sincronización Implícita}

Como alternativa a la señalización manual, Hoare propone los monitores de
señalización automática. Este tipo de monitor elimina las variables de
condición modificando la directiva \textit{wait} para que reciba una proposición
lógica.

Un hilo que llame \textit{wait(prop)} se mantiene bloqueado mientras la
proposición \textit{prop} sea falsa. Cuando \textit{prop} cambie a ser
verdadera, los hilos que estén bloqueados se desbloquean automáticamente.
Un inconveniente con este mecanismo es que su implementación suele llevar a la
señalización repetida de muchos hilos, con los consecuentes cambios de contexto.

A los efectos de este proyecto integrador, es de mayor interés el monitor de
sincronización explícita. Por esto, los siguientes apartados al respecto son
referidos únicamente a este tipo de monitor.

\subsubsection{Estructura de un Monitor}
De forma general, un monitor de concurrencia de sincronización explícita está
compuesto por las siguientes partes:
\begin{itemize}
    \item Variables de condición
    \item Colas de condición
    \item Rutinas exportadas o de entrada
    \item Cola de entrada
    \item Cola de espera
    \item Cola de cortesía o del señalizador
\end{itemize}

En la figura \ref{fig:monitor01} se observa la estructura de un monitor de
concurrencia:

\begin{figure}[H]
  \centering
  \makebox[\textwidth][c]{
    \includegraphics[width=140mm]{Monitor}
  }
  \caption{Estructura de un Monitor de Concurrencia}
  \label{fig:monitor01}
\end{figure}

Aunque un hilo puede entrar al monitor llamando a cualquiera de sus
procedimientos expuestos, y puesto que se debe asegurar la ejecución en
exclusión mutua se puede considerar que existe un único punto de entrada al
monitor. De ahí que existe una única cola de entrada.

\subsubsection{Máquina de Estados de un Monitor}
Un monitor no es un proceso en sí mismo, por lo que no tiene un hilo de
ejecución. En su lugar, es ejecutado por los hilos de los procesos que llaman a
alguna de sus rutinas.

El estado del monitor, incluyendo si está o no bloqueado determina si un
hilo que intenta ejecutar una rutina de entrada puede continuar o si se bloquea.

Se puede representar el funcionamiento de un monitor por dos máquinas de
estado. La primera indica si el monitor está bloqueado o desbloqueado. La
segunda representa el estado de las colas del monitor.
\begin{itemize}
    \item Estados del Primer Autómata:
    \begin{itemize}
        \item Bloqueado: Un hilo está ejecutando una rutina del monitor
        \item Desbloqueado: No hay hilo activo en el monitor
    \end{itemize}
    \item Estados del Segundo Autómata: Las colas que influyen en el estado del
    monitor son las internas a este (las de condición, la de espera y la de
    cortesía). La cola de entrada no influye en el estado del monitor porque no
    refleja la situación interna del mismo.\\
    Como los estados que puede adquirir una cola son \textit{“vacía”} y
    \textit{“no vacía”}, los estados del segundo autómata son todas las
    combinaciones posibles de las tres colas internas del monitor en cada uno de
    sus estados.
\end{itemize}

\begin{figure}[H]
  \centering
  \includegraphics[width=100mm]{Primer_Automata_Monitor}
  \caption{Primer Autómata de un Monitor de Concurrencia}
  \label{fig:automata_monitor01}
\end{figure}

\begin{figure}[H]
  \centering
  \makebox[\textwidth][c]{
    \includegraphics[width=180mm]{Segundo_Automata_Monitor}
  }
  \caption{Segundo Autómata de un Monitor de Concurrencia}
  \label{fig:automata_monitor02}
\end{figure}

\begin{framed}
\textbf{Nota:} Como se explicó en la sección \ref{comparacion_rdp_automatas} se
puede obtener un único autómata a partir de estos dos, pero es de la opinión de los autores
que esto resultaría en una explicación más confusa.
\end{framed}

\subsubsection{Políticas de Desbloqueo de Hilos}
\label{politica_monitor}
El desbloqueo de un hilo suspendido en la cola de condición x debe ser hecho
por el hilo que produjo el cambio sobre esta condición. La siguiente acción a
realizar luego del desbloqueo dependerá del tipo de monitor en cuestión.
Se puede generar una clasificación de monitores basándose en el comportamiento
luego del desbloqueo de un hilo. A continuación se presentan los tipos de
monitores introducidos en \cite{PalmaConcurrente}

\begin{framed}
\textbf{Nota:} Para todos los siguientes casos, se considera que el hilo
\textit{A} desbloquea al hilo \textit{B} ejecutando \textit{signal(x)},
condición sobre la que \textit{B} se encuentra bloqueado inicialmente. Por lo
tanto, al comenzar cada párrafo, A está bloqueado en la cola de cortesía y B en
la de espera, a menos que se indique lo contrario.
\end{framed}

\paragraph{Desbloquear y continuar (Signal and Continue)}
Se desbloquea a \textit{A} de la cola de cortesía y continúa su ejecución dentro
del monitor, ya sea hasta terminar la llamada al procedimiento o hasta bloquearse
en una cola de condición. Una vez \textit{A} sale del monitor, \textit{B}
ejecuta la instrucción siguiente al \textit{delay(x)} que lo bloqueó. En este
punto, \textit{B} debe volver a verificar la condición que lo suspendió porque
no se puede garantizar que \textit{A} no la haya modificado luego de la llamada
a \textit{signal(x)}.

\paragraph{Retorno forzado}
Se desbloquea a \textit{A}, quien ejecuta una instrucción de salida del monitor
(\textit{return} o \textit{delay(n)}) justo después. De esta manera, no es
necesario que \textit{B} vuelva a comprobar su condición ya que la exclusión
mutua asegura que no fue modificada.


\paragraph{Desbloquear y esperar}
\textit{A} está en la cola de entrada del monitor en lugar de la de cortesía.\\
Se desbloquea a \textit{B} de la cola de espera para que continúe su ejecución
en el monitor. Este enfoque tiene la ventaja de que \textit{B} no necesita
comprobar su condición de bloqueo una vez desbloqueado, pero \textit{A} cede su
lugar en el monitor y debe volver a competir por el ingreso para poder terminar
su ejecución.


\paragraph{Desbloquear y espera urgente}
Esta política soluciona el problema de inequidad de \textit{Desbloquear y
Esperar}.\\
Se desbloquea a \textit{B}, pero \textit{A} se suspende en la cola de cortesía.
De esta manera, el desbloqueo de \textit{A} tendrá prioridad sobre cualquier
hilo que intente entrar al monitor.

\paragraph{Clasificación Generalizada de Políticas de Desbloqueo:}
En \cite{MonitorClassification} el autor hace un análisis más exhaustivo de las
posibilidades existentes para diseñar una política de desbloqueo de hilos.
Dadas las tres colas de donde se puede elegir un hilo para desbloquear se
plantea una prioridad para cada una, resultando en:
\begin{itemize}
    \item EP: prioridad de la cola de entrada (entry queue priority)
    \item WP: prioridad de la cola de espera (waiting queue priority)
    \item SP: prioridad de la cola de cortesía (signaler queue priority)
\end{itemize}
Asignando pesos relativos a las tres prioridades se llega a que existen 13
distintas posibilidades.

En la tabla \ref{tab:prioridades_monitores} se Enumeran las posibilidades. La
tercera columna se refiere a los monitores definidos en [Howard, J. "Proving
Monitors"]

\begin{table}[H]
\centering
\begin{tabular}{|c|c|c|}
\hline
 & Prioridades relativas & Monitor Tradicional Correspondiente \\ \hline
1 & $EP = WP = SP$ & Random\\ \hline
2 & $EP = WP < SP$ & Wait and Notify\\ \hline
3 & $EP = SP < WP$ & Signal and Wait\\ \hline
4 & $EP < WP = SP$ & \\ \hline
5 & $EP < WP < SP$ & Signal and Continue\\ \hline
6 & $EP < SP < WP$ & Signal and Urgent Wait\\ \hline
7 & $EP > WP = SP$ & RECHAZADO\\ \hline
8 & $EP = SP > WP$ & RECHAZADO\\ \hline
9 & $SP > EP > WP$ & RECHAZADO\\ \hline
10 & $EP = WP > SP$ & RECHAZADO\\ \hline
11 & $WP > EP > SP$ & RECHAZADO\\ \hline
12 & $EP > SP > WP$ & RECHAZADO\\ \hline
13 & $EP > WP > SP$ & RECHAZADO\\ \hline
\end{tabular}
\caption{Tipos de monitores según las prioridades relativas de sus colas}
\label{tab:prioridades_monitores}
\end{table}

Las propuestas 7 a 13 son rechazadas porque si la prioridad de entrada es mayor
que cualquiera de las otras dos, ante un flujo constante de hilos de
entrada, habría al menos una cola que nunca sería atendida, lo que lleva a
posible inanición de los hilos que esperan en ella.

\subsubsection{Uso de un Monitor}
En el diagrama \ref{fig:actividad_hilo_monitor} se describen las actividades
de un hilo que accede a un monitor.

\begin{figure}[H]
  \centering
  \makebox[\textwidth][c]{
    \includegraphics[width=180mm]{Actividad_Proceso_Monitor}
  }
  \caption{Diagrama de actividades UML de un hilo ejecutándo una rutina de
  un monitor}
  \label{fig:actividad_hilo_monitor}
\end{figure}

El diagrama de la figura \ref{fig:actividad_hilo_monitor} sugiere que un
hilo puede optar por uno de dos caminos al ejecutar una rutina del monitor:
tomar o devolver un recurso.

Existe otra opción que es tomar y devolver un recurso en la misma rutina. Este
caso no está especificado en el diagrama por simplicidad y por tratarse de una
superposición de los otros dos casos.

\subsubsection{Conclusión}
Como con los semáforos, es posible cometer errores en la sincronización de los
monitores. La ventaja que tienen los monitores sobre los semáforos es que todas
las funciones de sincronización están confinadas dentro del monitor. De este
modo, es más sencillo verificar que la sincronización se ha realizado
correctamente y detectar los fallos. Es más, una vez que un monitor está
correctamente programado, el acceso al recurso protegido es correcto para todos
los hilos. Con los semáforos, en cambio, el acceso al recurso es correcto
sólo si \textbf{todos los hilos} que acceden al recurso están correctamente
programados.\cite{SistOpStallings} Por otro lado, las políticas de desbloqueo
permiten especificar prioridades de ejecución para los hilos, priorizando ya
sea por orden de llegada o por algún otro criterio. Si a su vez, se construye
el monitor de forma modular para cambiar la política de manera simple, se puede
alterar la planificación de los hilos de acuerdo a cada caso, según sea
necesario. Si esto se intenta hacer utilizando semáforos, resultaría mucho más
difícil.
Por estas razones, un punto fuerte a favor de los monitores frente a los semáforos es la mantenibilidad del código.
Por otro lado, al ser el único punto del programa donde se toman decisiones, un monitor se convierte en un cuello de botella para el sistema. Este impacto se puede mitigar con una arquitectura de monitores jerárquicos.


    \part{Desarrollo}
        \chapter{Investigación}
            \section{Requerimientos}

\begin{enumerate}
	\item El sistema debe delegar flujo de ejecución a un motor de
	petri.
	\begin{itemize}
		\item El sistema debe mapear transiciones de una red de petri a eventos
		especificados por los usuarios. Un evento puede ser equivalente a un conjunto de transiciones.
		\item Cuando un evento es desencadenado por el disparo de un conjunto de
		transiciones el sistema debe ejecutar todas las tareas que se encuentran registradas al evento.
		\item Cuando un suceso definido por el usuario ocurre, el sistema debe
		notificar todos los eventos asociados a este suceso al motor de petri.
		\item El sistema debe proveer una interface para que el usuario pueda
		suscribir sucesos, tareas y fines de tareas a eventos especificados por el usuario.
		\item El sistema debe proveer una interface para que el usuario pueda definir
		eventos.
		\item Cuando una tarea termina el sistema debe notificar al motor de petri
		acerca de todos los eventos asociados a la finalización de la tarea.
	\end{itemize}
	\item Para un usuario con conocimiento intermedio en Java y Redes de Petri, el
	framework puede aprender a usarse en una semana o menos.
	\begin{itemize}
	    \item La utilización del sistema puede incorporar como máximo diez
	    conceptos nuevos a aprender por un usuario con un nivel intermedio en Java
	    y redes de Petri.
	    \item El sistema debe ser acompañado con al menos dos ejemplos de uso en
	    los cuales se muestre al menos un 80\% de las funciones del mismo.
	\end{itemize}
	\item El sistema debe ser compatible con las versiones actuales de motores de
	Petri desarrollados en el Laboratorio de Arquitectura de Computadoras de la
	Facultad de Ciencias Exactas y Naturales de la Universidad Nacional de Córdoba.
	\begin{itemize}
	    \item El sistema debe proveer una interfaz para que el usuario ingrese un
	    archivo PNML con la descripción de una red de Petri.
	    \item El sistema puede instanciar un entorno de ejecución de redes de
	    Petri dado que el usuario ha ingresado un archivo PNML conteniendo la
	    descripción de la red y ha elegido el motor de Petri que desea usar.
	    \item El sistema debe utilizar la interface expuesta por el motor de
	    petri.
	\end{itemize}
	\item El sistema quiere tener una interfaz gráfica de usuario para inicializar
	un nuevo proyecto.
	\begin{itemize}
	    \item La interfaz de usuario quiere contener una pantalla 'PNML Loader'
	    	\begin{itemize}
	    	    \item La pantalla debe dejar al usuario buscar en su disco local y
	    	    elegir un archivo.
	    	    \item La pantalla debe dejar al usuario ingresar la dirección a un
	    	    archivo manualmente mediante la escritura con el teclado.
	    	    \item La pantalla debe permitir confirmar la elección de un archivo.
	    	    \item Si el usuario confirma un archivo y el archivo es un PNML válido
	    	    entonces puede usarse para configurar el entorno de ejecución de
	    	    Petri.
	    	    \item Si el usuario confirma un archivo y el archivo no es un PNML
	    	    válido entonces debe mostrarse un error en pantalla y el usuario debe
	    	    ser capaz de elegir otro archivo.
	    	\end{itemize}
	    \item La interfaz de usuario quiere contener una pantalla de creación de
	    eventos
	    \begin{itemize}
	    	    \item La pantalla debe dejar al usuario definir un evento y asociarlo
	    	    con una o más transiciones definidas en un archivo PNML cargado
	    	    previamente por el usuario.
	    	    \item La pantalla de creación de eventos quiere permitir guardar las
	    	    decisiones del usuario en un archivo.
	    	    \item La pantalla de creación de eventos quiere permitir al usuario
	    	    cargar configuraciones a partir de un archivo.
	    	    \item Si un archivo guardado previamente se selecciona para ser
	    	    cargado y su contenido tiene un formato inválido, la pantalla
	    	    quiere mostrar un texto de error especificando el problema y el
	    	    archivo no debe ser cargado.
	    	    \item Si un archivo guardado previamente se selecciona para ser
	    	    cargado y el contenido del archivo contiene uno omás eventos que
	    	    mapean a transiciones inexistentes, la pantalla quiere mostrar un
	    	    texto de error especificando el problema y sólo debe cargarse la
	    	    configuración de los eventos fuera de conflicto.
	    \end{itemize}
	     \item La interfaz de usuario quiere contener una pantalla de selección de
	     el motor de Petri.
	     \begin{itemize}
	         \item La pantalla debe permitir elegir entre un motor de Petri Java,
	         un motor de Petri en hardware o un motor de Petri en driver.
	         \item La pantalla debe comunicar la decisión del usuario para preparar
	         el entorno de ejecución de Petri de acuerdo al motor elegido.
	     \end{itemize}
	         
	\end{itemize}
\end{enumerate}
        \chapter{Monitor de Concurrencia con Redes de Petri}
            \newcommand{\javapetriconcurrencymonitor}{Java Petri Concurrency Monitor }

\section{\javapetriconcurrencymonitor}

\subsection{Introducción}

\javapetriconcurrencymonitor  es un monitor de concurrencia que ejecuta redes
de petri, hecho en lenguaje de programación java.

\subsection{Características Principales}
Entre las principales características de \javapetriconcurrencymonitor están:
\begin{itemize}
  \item Soporte para Redes de Petri:
  \begin{itemize}
    \item Plaza-Transición
    \item Temporales
  \end{itemize}
  
  \item Soporte para tipos de arcos;
  \begin{itemize}
    \item Normal
    \item Lector o de Prueba
    \item Inhibidor
    \item Reset
  \end{itemize}
  
  \item Soporte para guardas
  \item Soporte para transitiones automáticas
  \item Soporte para subscripción a eventos en transiciones informadas
  \item Soporte de políticas intercambiables y extensibles de prioridad de
  disparo de transiciones

\end{itemize}

\subsection{Implementación}

\subsection{Manual de Uso}
        \chapter{\nombreFramework \  Framework}
            \section{Análisis de Modelos con Redes de Petri}
\subsection{Petición de ejecución versus Aviso de ejecución.}
\subsubsection{Estudio de la Red de Petri de una cinta transportadora}
Cinta transportadora con 3 estaciones. Piezas son depositadas en la primer
estación  de manera asincrónica. Cuando esto sucede, la cinta avanza a la
estación 1, donde un operario realiza una transformación a la pieza. Una vez el
operario realizó la transformación, presiona un pulsador y la cinta avanza a la
estación 2, donde un segundo operario empaqueta la pieza. Este segundo operario
presiona otro pulsador al finalizar su tarea y luego la cinta avanza una vez
más y la pieza cae en un contenedor.

\begin{figure}[H]
    \centering
    \includegraphics[height=100mm]{Petri_Cinta_Transportadora_1}
    \caption{Red de Petri de una cinta transportadora}
    \label{fig:petri_cinta_transportadora_1}
\end{figure}

\subsubsection{Análisis de ejecución del caso de estudio en framework Chimp} 
De acuerdo al modo de ejecución implementado por Chimp ~\cite{chimp}, el
framework da aviso de eventos al monitor, desencadenando la ejecución de las tareas:
\begin{enumerate}
    \item Debe insertarse un evento en la cola de entrada de “t0” cuando el framework
		detecte la llegada de una nueva pieza. Si la cinta Transportadora se encuentra
		disponible, el monitor de petri dispara “t0” y se genera un evento que se
		deposita en la cola de salida de “t0”.
    \item Chimp lee el evento de salida de “t0” y realiza la acción “moverEst1”, que
		mueve la pieza a la estación 1 y espera la acción del operador. Una vez que el
		operador realiza su acción, presiona el pulsador generando un evento que Chimp
		envía a la cola de  entrada de “t1”. El monitor de petri dispara “t1” y se
		genera un evento que se deposita en la cola de salida de “t1”.
    \item Chimp lee el evento de salida de “t1” y realiza la acción “moverEst2”, que
		mueve la pieza a la estación 2 y espera la acción del operador. Una vez que el
		operador realiza su acción, presiona el pulsador generando un evento que Chimp
		envía a la cola de  entrada de “t2”. El monitor de petri dispara “t1” y se
		genera un evento que se deposita en la cola de salida de “t2”.
    \item Chimp lee el evento de salida de “t2 y realiza la accion “moverACont”, que
		mueve la pieza al contenedor. Una vez terminada esa acción envía un evento a
		la cola de entrada de “t3”. El monitor de petri dispara “t3” y libera la
		cinta Transportadora para procesar otra pieza.
\end{enumerate}

Esta forma de ejecución limita la funcionalidad de sincronización de la red de
Petri, acotando su funcionamiento únicamente a tareas sincrónicas. El
framework Chimp no contempla el caso de eventos externos asincrónicos que
desencadenen disparos en la red. Además, el framework está asumiendo una parte
del rol de control de flujo de ejecución, lo cual debería delegarse por completo
al monitor de redes de Petri.

\subsubsection{Análisis de ejecución del caso de estudio por
petición de ejecución al motor de petri} 
Una alternativa al método de ejecución de Chimp consiste en que los hilos
encargados de realizar las tareas realicen una petición de ejecución al monitor
sin tener en cuenta el estado actual de la red de Petri.
 El monitor es el encargado de dormir aquellas tareas que no pueden ser
 ejecutadas. Una vez que las condiciones son las adecuadas para realizar la
 tarea, el monitor se encarga de despertar al hilo encargado de ejecutarla.
\begin{enumerate}
    \item Se generan eventos que se encolan en la cola de entrada en “t0, t1,
    	t2 y t3”.
    \item El monitor duerme los hilos que generaron eventos para “t1, t2 y t3”
    	por no estar sensibilizadas las transiciones en ese momento.
    \item El monitor ejecuta “t0”. Y se envía un evento a la cola de salida de
    	“t0”.
    \item Chimp lee el evento de salida de “t0” y ejecuta “moverEst1”.
    \item Existe un problema, ya que al disparar “t0”, el monitor tiene
    	permitido disparar “t1”, pero la operación “moverEst1” aun no ha
    	finalizado.
\end{enumerate}
Tras el análisis  del ejemplo anterior se llega a una serie de
conclusiones. En primer lugar, es necesario que el framework de aviso al monitor
cuando una tarea debe ser realizada de forma atómica. Además, el
sistema de peticiones es más adecuado para una arquitectura manejada por un
monitor, sin embargo, dada la red de petri
Figura~\ref{fig:petri_cinta_transportadora_1} surgen problemas de
sincronización. Un ejemplo de estos problemas se origina al realizar una
petición de ejecución de la tarea “moverEst2”, el monitor permite ejecutar esta
tarea de forma inmediata, sin tener en cuenta si la tarea ``moverEst1'' ha
finalizado.

\subsubsection{Sincronización por Plaza-Transición}
Una posible solución a los problemas planteados en la sección anterior es
modelar la red de la siguiente forma:\\

\begin{figure}[H]
    \centering
    \includegraphics[height=100mm]{Petri_Cinta_Transportadora_2}
    \caption{Red de Petri de una cinta transportadora sincronizada por inserción
    de plaza-transición}
    \label{fig:petri_cinta_transportadora_2}
\end{figure}


Donde:\\
\begin{enumerate}
	\item Se generan eventos que se encolan en la cola de entrada en “t0, t2 y
		t4”.
	\item El monitor duerme los hilos que generaron eventos para “t0, t2 y t4” por
		no estar sensibilizadas las transiciones en ese momento.
	\item Llega una pieza y se genera un evento de entrada en “t6”
	\item El monitor dispara “t6” y se coloca un token en “piezaDisp”,
		sensibilizando “t0”.
	\item El monitor despierta el hilo dormido en “t0” ya que ahora tiene permiso
		de ejecución.
	\item Se ejecuta “moverEst1”. Una vez finalizado se genera un evento que se
		envía a la cola de entrada de “t1”.
	\item Como “t1” está sensibilizada el monitor la dispara y se coloca un token
		en “piezaLista1”, sensibilizando “t2”.
	\item El monitor despierta el hilo dormido en “t2” ya que ahora tiene permiso
		de ejecución.
	\item Se ejecuta “moverEst2”. Una vez finalizado se genera un evento que se
		envía a la cola de entrada de “t3”.
	\item Como “t3” está sensibilizada el monitor la dispara y se coloca un token
		en “piezaLista2”, sensibilizando “t4”.
	\item El monitor despierta el hilo dormido en “t4” ya que ahora tiene permiso
		de ejecución.
	\item Se ejecuta “moverACont”. Una vez finalizado se genera un evento que se
		envía a la cola de entrada de “t5”
	\item Como “t5” está sensibilizada el monitor la dispara y se coloca un token
		en ``piezaEnCont''.
	\item Se ejecuta la transición ``t7'', que es automática, y se libera el
		recurso ``cintaTransp''.
\end{enumerate}
La principal ventaja de este método es que no modifica la semántica de la red y
no añade nuevos conceptos ni cambios en su forma de ejecución.
La desventaja consiste en que realizar este tipo de sincronización puede
conllevar un incremento considerable de la cantidad de plazas y transiciones de
la red, lo que conlleva el procesamiento de matrices de mayor tamaño.

\subsubsection{Sincronización por Guardas}
Este método consiste en la utilización de una guarda como forma de
sincronización entre tareas consecutivas. Ver Figura ~\ref{fig:petri_cinta_transportadora_3}

\begin{figure}[H]
    \centering
    \includegraphics[height=100mm]{Petri_Cinta_Transportadora_3}
    \caption{Red de Petri de una cinta transportadora sincronizada por guardas.}
    \label{fig:petri_cinta_transportadora_3}
\end{figure}

\begin{enumerate}
    \item Se generan eventos que se encolan en la cola de entrada en “t0, t1 y
    t2”.
	\item El monitor duerme los hilos que generaron eventos para “t1 y t2” por
	no estar sensibilizadas las transiciones en ese momento.
	\item Se dispara ``t0'' y se coloca un token en ``moverEst1''. Comienza la
	ejecución de la tarea ``moverEst1''. La transición ``t1'' no se encuentra
	sensibilizada dado que la guarda ``Fin\_moverEst1'' tiene estado ``false''.
	\item Finaliza la ejecución de ``moverEst1'' y se setea la guarda
	``Fin\_moverEst1'' con estado ``true''.
	\item Al estar sensibilizada ``t1'', se dispara y se despierta el hilo que
	duerme en su cola de condición. Se coloca un token en ``moverEst2'' y
	comienza la ejecución de esta tarea. La transición ``t2'' no se
	encuentra sensibilizada dado que la guarda ``Fin\_moverEst2'' tiene estado
	``false''. Debe setearse la guarda ``Fin\_moverEst1'' a ``false'' nuevamente.
	\item Finaliza la ejecución de ``moverEst2'' y se setea la guarda
	``Fin\_moverEst2'' con estado ``true''.
	\item Al estar sensibilizada ``t2'', se dispara y se despierta el hilo que
	duerme en su cola de condición. Se coloca un token en ``moverACont'' y
	comienza la ejecución de esta tarea. La transición ``t3'' no se
	encuentra sensibilizada dado que la guarda ``Fin\_moverACont'' tiene estado
	``false''. Debe setearse la guarda ``Fin\_moverEst2'' a ``false'' nuevamente.
	\item Finaliza la ejecución de ``moverACont'' y se setea la guarda
	``Fin\_moverACont'' con estado ``true''.
	\item Al estar sensibilidada, se dispara la transición ``t3'', que es
	automática, y se libera el recurso ``cintaTransp''. Debe setearse la guarda
	``Fin\_moverACont'' a ``false'' nuevamente.
\end{enumerate}

La ventaja de este método es que permite resolver el problema de sincronización
sin aumentar la cantidad de componentes de la red de Petri.
Como desventaja se puede mencionar que modifica la semántica de la
red, complicando su demostración matemática. Además, el diseño del monitor de
petri soporta una única guarda por transición, por lo tanto esta solución impide
la utilización de la guarda para otros propósitos en una situación de tareas
consecutivas. Por último, una desventaja importante de la utilización de guardas
es que al ser un valor binario no se puede saber cuantas veces ha sido seteada
la guarda.
Se supone un caso donde una ``tarea A'' es realizada por multiples hilos de manera
independiente, y cada hilo realiza la ``tarea A'' en su totalidad. A su
vez una ``tarea B'', que debe realizarse luego de la finalización de la ``tarea
A'', es ejecutada por un único hilo. En este caso la utilización de guardas
podría llevar a una pérdida de eventos de finalización de la ``tarea A'' debido
a la condición binaria de la guarda. Ver Figura ~\ref{fig:ejecucion_multiples_hilos_guardas}

\begin{figure}[H]
    \centering
    \includegraphics[height=60mm]{Ejecucion_Tarea_Multiples_Hilos_Guardas}
    \caption{RdP: Problema de sincronización de tareas sincrónicas usando
    guardas debido a su condición binaria}
    \label{fig:ejecucion_multiples_hilos_guardas}
\end{figure}

En esta red, un máximo de 5 hilos puede ejecutar la ``tarea A'' al mismo
tiempo. En el estado que  muestra la Figura
~\ref{fig:ejecucion_multiples_hilos_guardas} existen tres hilos corriendo la
``tarea A''. De acuerdo a lo supuesto en el planteo de este problema, la ``tarea
B'' es ejecutada por un único hilo. Si dos o más hilos finalizan la ``tarea A''
y setean la guarda ``Fin\_TareaA'' entonces, cuando se dispare ``t1" y antes de
comenzar la ejecución de la ``tareaB'', se debe modificar el valor de la guarda
``Fin\_TareaA'' a ``false'', y de esta manera existe la posibilidad de perder
eventos de finalización de la ``tarea A''.

\subsubsection{Sincronización por Disparo Perenne de Aviso de
Finalización de Tarea}
Esta forma de solucionar la sincronización de tareas sincrónicas consecutivas
supone añadir una nueva propiedad ``P'' a las transiciones. Los hilos que
duermen en la cola de condición de una transición con propiedad ``P'' sólo
se despiertan cuando la transición se encuentra habilitada y además un hilo
externo realiza un disparo perenne sobre la transición.

\begin{figure}[H]
    \centering
    \includegraphics[height=100mm]{Petri_Cinta_Transportadora_4}
    \caption{Red de Petri de una cinta transportadora sincronizada por
    propiedad ``P''.}
    \label{fig:petri_cinta_transportadora_4}
\end{figure}

\begin{enumerate}
    \item Se generan eventos que se encolan en la cola de entrada en “t0, t1,
    t2 y t3”.
	\item El monitor duerme los hilos que generaron eventos para “t1, t2 y t3” por
	no estar sensibilizadas las transiciones en ese momento.
	\item Se dispara ``t0'' y se coloca un token en ``moverEst1''. Comienza la
	ejecución de la tarea ``moverEst1''. La transición ``t1'' no se dispara ya que
	es de tipo ``P'' y solo puede dispararse de forma perenne por un hilo externo.
	\item Finaliza la ejecución de ``moverEst1'' y un hilo dispara ``t1'' de forma
	perenne para dar aviso de la finalización de la tarea.
	\item Se despierta el hilo que
	duerme en cola de condición de ``t1''. Se coloca un token en ``moverEst2'' y
	comienza la ejecución de esta tarea. La transición ``t2'' no se dispara ya que
	es de tipo ``P'' y solo puede dispararse de forma perenne por un hilo externo.
	\item Finaliza la ejecución de ``moverEst2'' y un hilo dispara ``t2'' de forma
	perenne para dar aviso de la finalización de la tarea.
	\item  Se despierta el hilo que
	duerme en cola de condición de ``t2''. Se coloca un token en ``moverACont'' y
	comienza la ejecución de esta tarea. La transición ``t3'' no se dispara ya que
	es de tipo ``P'' y solo puede dispararse de forma perenne por un hilo externo.
	\item Finaliza la ejecución de ``moverACont'' y un hilo dispara ``t3'' de forma
	perenne para dar aviso de la finalización de la tarea.
	\item Se libera el recurso ``cintaTransp''.
\end{enumerate}

Esta solución tiene como principal ventaja mantener la cantidad de componentes
de la red de Petri.
Su principal desventaja consiste en que añade un nuevo componente a la
petri, dificultando su demostración matemática. Esta solución supone añadir una
interfaz al monitor de petri para dormir hilos en una cola de condición de una
transición tipo P y que los hilos bloqueados en esta cola de condición sólo
puedan liberarse por medio de un disparo perenne ocasionado por un hilo externo.
\emph{\color{red} Los hilos peticionarios (que solicitan permiso de ejecución)
a una transición tipo P no debieran utilizar la interfaz fire. En cambio, deben
utilizar una interfaz sleep, que haga una operacion acquire sobre el semáforo
de la cola de condición.
Los hilos que dan aviso de una finalización de tarea deben utilizar la interfaz
fire de modo perenne para realizar una operación release sobre el semáforo de
la cola de condición. Esta operación release debe darse por más que no existan
hilos esperando en la cola de condición de la transición tipo P, porque sino se
tendría una pérdida de eventos de finalización de tarea.}


            \section{Introducción al Sistema de Manejo de Eventos de \nombreFramework}
Uno de los principales objetivos de este proyecto es que una red de Petri esté a
cargo del manejo completo de la concurrencia y el asincronismo de un sistema informático. Para
logralo, se implementó el monitor de petri por software descripto en la
sección \emph{\color{red} CITA REQUERIDA seccion monitor}.
El mismo está basado en la implementación previa realizada en \cite{codegen}.
Este monitor permite delegar la concurrencia y asincronismo del
sistema a una red de Petri. Por ejemplo, fue utilizado con éxito en
\cite{Bentivegna-Ludemann}.

Sin embargo, la utilización directa del monitor es engorrosa y genera un
alto grado de acoplamiento entre el software de usuario y la red de Petri. La
red queda asociada directamente a los eventos del mundo real. La principal
desventaja de tener un sistema acoplado a la red de petri es que disminuye la
escalabilidad del sistema. Esto se debe a que la sustitución de la red (o su modificación) implica
también un cambio en el código del software, dificultando el desarrollo
iterativo de un sistema. Por otro lado, impide la realización de redes
genéricas, más versátiles, útiles para resolver diferentes problemas de las
mismas características.

Otro objetivo del proyecto consiste en la facilidad de uso. Como se explicó
anteriormente, la utilización directa del monitor es complicada y puede
favorecer a la generación de errores, ya que deben crearse todos los hilos de
ejecución y deben programarse los disparos de transición de forma manual en el
código. Ante un cambio en la red de petri deben modificarse todos los disparos
de transición, pudiendo llevar a una incorrecta sincronización de los hilos si
no se realiza con especial precaución.

Finalmente, otro objetivo importante es el de crear una herramienta que permita
al usuario estructurar el proyecto utilizando patrones de diseño de forma
simple. En el caso de utilizar el monitor de forma directa, el diseño del
software queda en manos del usuario desarrollador. Esto significa que la
decisión de utilizar o no patrones de diseño es tomada por el usuario. De
esta forma se desaprovecha la oportunidad de reutilizar ciertos aspectos de
diseño que se consideran pertinentes. Por ejemplo el patrón command descripto en
\cite{chimp}, o algunas caracteristicas similares a un diseño MVC que se pueden
observar en los sistemas sincronizados por Petri. Este último ejemplo se explica
con mayor detalle en la sección \emph{\color{red} CITA REQUERIDA Patrón MVC
en el framework?}.

Como resultado de este análisis, se llegó a la conclusión de que se puede
aprovechar mejor el potencial del monitor si se encuentra embebido en un
framework que se encargue de desacoplar el código de usuario de la lógica de
disparos y de sincronización de hilos. 
Una conclusión de similares características se desprende de
\cite{Bentivegna-Ludemann}, donde los autores expresan: ``La debilidad encontrada en el proceso de elaboración del
software, es que resultó problemático vincular los hilos con las transiciones de
la RdP. Esto se debe a que, en el código de las acciones, quedó embebida la
relación de éstas con las transiciones. Por lo cual, queda en evidencia que es
necesaria la existencia de un framework para automatizar y facilitar la
vinculación entre eventos, acciones y transiciones.''


\section{Red de Petri como procesador de Eventos}
Se puede definir a un sistema desarrollado utilizando el monitor como un
programa de software que intercambia eventos con la red de Petri y con su
entorno físico.

\begin{figure}[h]
	\centering
	\includegraphics[width=75mm]{eventos_petri-programa-mundo}
	\caption{Intercambio de eventos en un programa sincronizado por Red de Petri}
	\label{fig:eventos_petri-programa-mundo}
\end{figure}

El programa puede acceder a hardware del mundo físico, ya sea para realizar una
acción (por ejemplo utilizando actuadores) o  para obtener eventos del mundo
exterior y comunicarselos a la red de Petri (utilizando sensores, por ejemplo).

La red toma los eventos del mundo exterior y los combina con condiciones del
problema y de sincronización para emitir eventos de salida hacia el programa. La
red es básicamente un procesador de eventos.\cite{chimp}

Este concepto se amplía en la sección Eventos Físicos y Eventos
Lógicos de \cite{chimp}. En esta sección se distingue la existencia de los dos
tipos de eventos mencionados, y se los define como:
  \begin{itemize}
    \item Eventos Lógicos: eventos que son comprensibles por el monitor de
    redes de Petri, y están inherentemente asociados a transiciones de la red
    misma y sus colas.
    \item Eventos Físicos: suceden en el mundo físico y representan sucesos del
    dominio del problema
  \end{itemize}

Tras la incorporación del concepto de eventos lógicos y físicos, los autores
proponen en \cite{chimp} un diagrama de arquitectura de alto nivel como el
siguiente:

\begin{figure}[h]
	\centering
	\includegraphics[width=75mm]{eventos_fisicos-logicos}
	\caption{Arquitectura con Eventos Físicos y Lógicos}
	\label{fig:eventos_fisicos-logicos}
\end{figure}





\section{Requerimientos del Sistema de Manejo de Eventos de \nombreFramework}
A continuación se listan los requerimientos establecidos para el nuevo sistema
de manejo de eventos.

\begin{itemize}
  \item 
\end{itemize}



Para explicar el esquema de comunicación de \nombreFramework , primero deben
detallarse otros conceptos inherentes a la misma.


\subsection{Condiciones de Ejecución Síncronas y Asíncronas}
Las condiciones para la ejecución de una acción pueden ser:
  \begin{itemize}
	\item Síncronas: Por ejemplo, condiciones booleanas derivadas del estado del
		sistema que realizan cambios en el flujo de instrucciones del mismo (saltos
		condicionales).
	\item Asíncronas: Por ejemplo, eventos provenientes del mundo exterior o
		señales / mensajes entre hilos / procesos.
  \end{itemize}
  
\subsection{Eventos Lógicos, Programáticos y Físicos}
Los eventos de interés para el sistema pueden ser de tres tipos:
  \begin{itemize}
    \item Eventos Lógicos: eventos que son comprensibles por el monitor de
    redes de Petri, y están inherentemente asociados a transiciones de la red
    misma y sus colas
    \item Eventos del Software: eventos que son comprensibles por el
    subsistema de manejo de eventos y están asociados a acciones del
    software de usuario.
    \item Eventos Físicos: eventos reales que suceden en el mundo físico y
    representan sucesos del dominio del problema. Pueden ser eventos fisicos de
    salida (por ejemplo, al mover un actuador utilizando el programa), o de
    entrada (por ejemplo el cambio de una variable física siendo monitoreada por
    el programa).
  \end{itemize}
  
  
\subsection{Componentes de un sistema desarrollado con \nombreFramework}
Se puede dividir un sistema desarrollado con \nombreFramework en tres partes.
\begin{itemize}
\item Por un lado existe un monitor de redes de Petri, encargado de analizar el
cumplimiento de las condiciones para la ejecución de ciertas partes de código
(acciones). 
\item La segunda parte esta compuesta por un subsistema de manejo de eventos
programáticos, que permite la separación de los eventos lógicos y programáticos
del sistema, la suscripción a dichos eventos programáticos, y la sincronización
y ejecución de las acciones de software suscriptas. El mismo actúa como un
intermediario entre el código del usuario y el monitor de redes de Petri,
permitiendo desacoplar la red de Petri del código de usuario. 
\item Por último, el programa de usuario, que contiene todas las acciones
concretas a realizar, con sus correspondientes suscripciones a eventos
programáticos.
\end{itemize}
  
  
 
La propuesta para lograr el objetivo planteado es un modelo donde el monitor de
red de Petri asume la responsabilidad de verificar las condiciones asíncronas y
de mantener el estado del sistema. De esta forma será la red quien conozca y
analice ese tipo de condiciones.
Los tipos de condiciones que entran bajo consideración del monitor de Petri son:
\begin{itemize}
\item El disparo de eventos provenientes de sistemas externos que llegan en cualquier momento
durante la ejecución y sin ningún orden establecido. Así los estados locales de la red son
mantenidos en causalidad de los procesos que se ejecutan y explicitan los eventos
externos.
\item Condiciones de sincronización que permitan ordenar la realización de tareas en el tiempo.
\item Condiciones impuestas por el dominio del problema. Así la red mantiene su estado en
causalidad de las restricciones del problema.
\end{itemize} \cite{chimp}
\\
Por otro lado, el subsistema manejador de eventos programáticos se encarga de
desacoplar el software del usuario de la red de Petri para relacionar eventos
físicos con eventos lógicos. De esta manera se consigue que las transiciones no
queden asociadas directamente a acciones del programa y, por ende, no se le dé
a la red semántica de un dominio de problema en particular.
Esto trae aparejadas las siguientes ventajas:
  \begin{itemize}
	\item Las redes de Petri se vuelven más genéricas y por lo tanto reutilizables,
    puesto que no hay semántica de un dominio de problema en ellas. Ergo una
    misma red se puede utilizar para resolver problemas diferentes.
	\item El modelo de red de Petri puede cambiar por completo sin afectar al
       programa.
 	Simplemente reconfigurando los mapas de relación de eventos el sistema
 	continúa funcionando
  \end{itemize} \cite{chimp}
\\

 El programa de usuario sólo se limitará a describir las acciones(que pueden
 desencadenar eventos físicos de salida o manejar eventos físicos de entrada)
 y asociarlas a los eventos programáticos que correspondan.
 \\

En este proyecto se utilizan redes de Petri no autónomas. Se dispone de las
etiquetas especificadas en \emph{\color{red} CITA REQUERIDA}. La comunicación
entre el subsistema de manejo de eventos programáticos y el monitor de Petri se
realiza utilizando las interfaces proporcionadas por el monitor.

    \bibliography{./bibliografia}
	\bibliographystyle{plain}
\end{document}