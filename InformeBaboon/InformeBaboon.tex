\documentclass{report}
\usepackage[spanish]{babel}
\usepackage[utf8]{inputenc}
\usepackage{amsmath, array}
\usepackage{amssymb}
\usepackage{tabularx}
\usepackage{minted}
\RecustomVerbatimEnvironment{Verbatim}{BVerbatim}{}
\usepackage{cite}
\usepackage{graphicx}
\usepackage{graphbox}
\usepackage{color}
\usepackage{float}
\usepackage{subfigure}
\usepackage{amsfonts}
\usepackage[section]{placeins}
\usepackage{hyperref}
\hypersetup{
     colorlinks,
     citecolor=black,
     filecolor=black,
     linkcolor=black,
     urlcolor=black
}
\date{}

\begin{document}
    %Nombre templetizado para poder cambiarlo fácil hasta tenerlo definido
    \newcommand{\nombreTesis}{Framework de Sincronización de Tareas Coordinado
    por Redes de Petri}
    %Nombre templetizada por si lo tenemos que cambiar
    \newcommand{\nombreFramework}{Baboon}
    \title{\nombreTesis}
    \author{Ariel Iván Rabinovich \\ \href{mailto:airabinovich@gmail.com}{airabinovich@gmail.com}
        \and Juan José Arce Giacobbe \\ \href{mailto:juanjo.arce7546@gmail.com}{juanjo.arce7456@gmail.com}}
    \graphicspath{ {resources/images/} }
    
    \maketitle
    
    \tableofcontents
    
    \listoffigures
    \listoftables
    
    \part{Marco Teórico}
        \chapter{Modelos}
            \section{Autómatas y Redes de Petri}

\subsection{Autómatas o Máquinas de Estado}

Existen muchas formas de modelar el comportamiento de los sistemas, y el uso de
máquinas de estado finitas es una de las más antiguas y más conocidas.
Las máquinas de estado finitas o autómatas nos permiten pensar acerca del
``estado'' de un sistema en un instante en particular y caracterizar el comportamiento de dicho
sistema basado en ese estado. El uso de esta técnica de modelado no está
limitada al desarrollo de sistemas de software.\cite{FSM_Wright}

\subsubsection{Definición Conceptual de Máquina de Estado}

Si una máquina de estados M, en un instante dado, se encuentra en el estado
$E_{0}$ y ocurre un evento $e_{0}$ que lleva a M al estado $E_{1}$, se
dice que ocurrió una \textit{transición} del estado $E_{0}$ al estado
$E_{1}$.
A partir de esto se puede deducir que M no puede estar en $E_{0}$ y $E_{1}$
a la vez, y por lo tanto los estados de una máquina de estados, son
\textbf{estados globales} del sistema modelado.

Analizando la semántica de las máquinas de estado, se pueden
identificar algunas características clave de un sistema que puede ser modelado con máquinas de
estados finitas:
\begin{itemize}
  \item El sistema debe ser descripto por conjunto finito de estados.
  \item El sistema debe tener una cantidad finita de entradas y/o eventos que
  puedan disparar transiciones entre estados.
  \item El comportamiento del sistema en un instante dado depende del estado
  actual y de sus entradas o eventos que ocurran en ese instante.
  \item Para cada estado posible en que el sistema pueda encontrarse existe un
  comportamiento definido para cada posible entrada o evento.
  \item El sistema tiene un estado inicial único y definido.
\end{itemize} \cite{FSM_Wright}

\subsubsection{Definición Formal de Máquina de Estado}

A fin de eliminar la ambigüedad existente en una definición conceptual, se
introduce una definición formal de Autómata Finito:
\newline\newline\emph{Definición:} Un autómata finito M está definido por una
tupla $(\Sigma, Q, q_{0}, F, \sigma)$, donde:
\begin{itemize}    
  \item $\Sigma$ es el conjunto de símbolos de entrada de M
  \item $Q$ es el conjunto de estados de M
  \item $q_{0}$ es el estado inicial de M
  \item $F \subseteq Q$ es el conjunto de estados finales de M
  \item $\sigma : Q  \times \Sigma \rightarrow Q$ es la función de
  transición
\end{itemize} \cite{FSM_Wright}

\subsection{Redes de Petri}

Tomando el concepto de transición en una máquina de estados, se lo puede
extender a una entidad propia.
Esta transitión $t_{i}$ será denotada por una barra, un rectángulo o un
cuadrado, y puede tener múltiples arcos de entrada (entrantes) y de salida
(salientes) a la vez. Esta transición, representa la \textit{transición} básica
de una Red de Petri (RdP).\cite{PetriNetsFundamentals}

De la misma forma que en una máquina de estados los círculos denotan estados
del sistema, en una RdP se utilizan círculos para denotar las \textit{plazas} o
\textit{lugares} de la red. Estas plazas no representan estados globales, sino
\textbf{estados locales}. \cite{PetriNetsFundamentals}

El estado local de una plaza, está dado por la cantidad de \textit{tokens} o
\textit{marcas} que esta contiene.

Como consecuencia de su estructura, una Red de Retri puede ser representada como
un grafo bipartito, donde los tipos de nodo existentes son \textit{plazas} y
\textit{transiciones}. Estos nodos se unen entre dos de distinto tipo
únicamente (de ahí el calificativo de bipartito), utilizando \textit{arcos}.\\

\begin{figure}[h]
	\centering
	\includegraphics[width=75mm]{Partes_De_Una_Red}
	\caption{Partes de una Red de Petri}
	\label{fig:partes_de_una_red}
\end{figure}

Se pueden visualizar las partes de una Red de Petri en la figura
\ref{fig:partes_de_una_red}.\\


\begin{figure}[h]
    \centering
    \includegraphics[height=40mm]{Automata_Y_Petri}
    \caption{Equivalencia entre una Máquina de Estados y una Red de Petri}
    \label{fig:automata_y_petri}
\end{figure}

En la figura \ref{fig:automata_y_petri} se aprecia:\\
\begin{itemize}
  \item[(a)] Una máquina de estados de dos estados y una transición.
  \item[(b)] Una RdP equivalente a la máquina de (a).
  \item[(c)] Una RdP con una transición con múltiples arcos de entrada y de
  salida.
\end{itemize}

Se puede extraer como consecuencia directa de esta extensión de la semántica de
un autómata que en una Red de Petri:
\begin{itemize}
  \item Múltiples tokens pueden existir en el modelo al mismo tiempo, y
  particularmente en una plaza.
  \item No existe un estado global explícito.
  \item El estado global del sistema es el conjunto de todos los estados
  parciales, representados por las plazas y sus tokens. A este conjunto se lo
  llama el \textbf{marcado} de la red.
\end{itemize}

\subsubsection{Definición Formal de Red de Petri}
A fin de eliminar ambigüedades, se presenta una serie de definiciones sobre
Redes de Petri.

\begin{itemize}
  \item [\underline{Definición 1}:] Una Red de Petri R está definida por la
  tupla $(P, T, Pre, Post)$ donde:
  \begin{itemize}
    \item $ P = \{ p_1, p_2, \ldots, p_p \} $ un conjunto de plazas.\footnote{Se
    utiliza $p$ como la cantidad de plazas de la RdP en todo momento dentro de este informe por simplicidad para el lector}
    \item $ T = \{ t_1, t_2, \ldots, t_t \} $ un conjunto de transiciones, donde
    $ P \cap T = \emptyset $. \footnote{Se utiliza $t$ como la cantidad de
    transiciones de la RdP en todo momento dentro de este informe por
    simplicidad para el lector}
    \item $ Pre: P \times T \rightarrow \mathbb{N}^{p} $ aplicación de
    precedencia.\footnote{Se toma la definición de números naturales incluyendo
    el cero por simplicidad de notación.}
    \item $ Post: P \times T \rightarrow \mathbb{N}^{p} $ aplicación de
    incidencia.
  \end{itemize}
  $ Pre (p_i, t_j) $ contiene el peso del arco que va de $ p_i $ a $ t_j $, y
  $ Post (p_i, t_j) $ contiene el peso del arco que va de $ t_j $ a $ p_i $

  \item [\underline{Definición 2}:] Una Red de Petri Marcada está
  definida por el par $(R, M)$, donde R es una RdP y $ M : P \rightarrow
  \mathbb{N}^{p} $ (siendo $P$ el conjunto de plazas de dimensión $n$) es una aplicación llamada \textit{marcado}.\\
  $m(R)$, o más simplemente $m$ si la red es conocida, define el marcado de la
  RdP y $m(p_{i})$ o $mp_{i}$ indica el marcado de la plaza $p_{i}$, es decir,
  el número de tokens contenido en la plaza $p_{i}$.\\
  La marca inicial se denota $m_{0}$ y da la cantidad inicial de tokens en todas
  las plazas de la red, por lo que especifica el estado inicial del sistema.
  
  \item [\underline{Definición 3}:] Para una marca $m$, una transición $t_{j}$
  está sensibilizada, y por lo tanto es disparable, si y solo si:\\
  $$ \forall p_{i} \in P, m(p_i) \geq Pre(p_{i}, t_{j}) $$
  Conceptualmente, una transición está sensibilizada si todas sus plazas de
  entrada contienen al menos la cantidad de tokens que indica el peso de los
  arcos que las unen.

  En la figura \ref{fig:transiciones_no_sensibilizadas} se observa gráficamente esta definición mediante dos casos de transiciones no sensibilizadas. Nótese
  el peso de los arcos.

  \begin{figure}[h]
    \centering
    \includegraphics[height=40mm]{Transiciones_No_Sensibilizadas}
    \caption{Ejemplos de transiciones no sensibilizadas.}
    \label{fig:transiciones_no_sensibilizadas}
  \end{figure}
  
  \item [\underline{Definición 4}:] La estructura de una Red de Petri
  se denota $ N = \{P, T, F, W\} $ donde,
  \begin{itemize}
    \item $P$ es en conjunto de plazas.
    \item $T$ es el conjunto de transiciones, donde se cumple que $ P \cap T =
    \emptyset $
    \item $F$ es el conjunto de arcos, donde se cumple que $ F \subseteq (P
    \times T) \cup (F \times P) $.
    \item $W$ es la función de peso de los arcos.
  \end{itemize}

  \item [\underline{Definición 5}:] Conjunto de transición y plaza de entrada y
  de salida.
  \begin{itemize}
    \item[] El conjunto de las plazas de entrada a la transición $t$ se denota
    $\bullet t$ y se define,
    $$ \bullet t = \{ p \in P : (p, t) \in F \} $$
    \item[] El conjunto de las plazas de salida de la transición $t$ se denota $
    t \bullet$ y se define,
    $$ t \bullet = \{ p \in P : (t, p) \in F \} $$
    \item[] El conjunto de las transiciones de entrada a la plaza $p$ se
    denota $\bullet p$ y se define,
    $$ \bullet p = \{ t \in T : (t, p) \in F \} $$
    \item[] El conjunto de las transiciones de salida de la plaza $p$ se denota
    $ p \bullet$ y se define,
    $$ p \bullet = \{ t \in T : (p, t) \in F \} $$
  \end{itemize}
\end{itemize}

\subsubsection{Disparo de una Transición}

La condición de disparo relacionada a $Pre(p_{i}, t_{j})$ significa que para
todas las plazas $p_{i}$ de entrada a $t_{j}$, es decir, todas las plazas que
tienen arcos que apuntan hacia $t_{j}$, el número de tokens presentes debe ser
mayor o igual al peso de dicho arco.

\begin{itemize}
  \item [\underline{Definición 6}:] En una RdP, dada una marca $ m_{n}(p) $,
  cualquier transición $ t_{j} $ que se encuentre sensibilizada puede ser
  disparada, y su disparo lleva a una marca $ m_{n+1}(p)$ dada por:
  $$ m_{n+1}(p) = m_{n}(p) + Post(p_{i}, t_{j}) - Pre(p_{i}, t_{j}), \forall
  p_{i} \in P $$
  Como se indica en la ecuación, al disparar la transición $ t_{j} $, se quitan
  tantos tokens de $ \bullet t $ como indiquen los arcos que las unen a $ t_{j}
  $, y se añaden a $ t \bullet $ la cantidad de tokens que indiquen los arcos
  que unen a $ t_{j} $ con ellas.\\
  El disparo de una transición $ t_{j} $ se denota $ m_{n}\rightarrow t_{j}
  \rightarrow m_{n+1} $

  En la figura {\ref{fig:disparo_transicion}} se observa el estado de una RdP
  antes y después del disparo de una transición.
  \begin{figure}[h]
    \centering
    \subfigure[$t_{0}$ sensibilizada]{\includegraphics[height=40mm]{Red_Sensibilizada}}
    \subfigure[Disparo de $t_{0}$]{\includegraphics[height=40mm]{Red_Disparada}}
    \caption{Disparo de una transición}
    \label{fig:disparo_transicion}
  \end{figure}
  
  \item  [\underline{Definición 7}:] Matriz de Incidencia.\\
  La matriz de incidencia de una RdP se define como,
  $$ I = Post - Pre $$
  \textbf{Notas:}
  \begin{itemize}
    \item El disparo de una transición se reformula como, $$ m_{n+1}(p) =
    m_{n}(p) + I(p_{i}, t_{j}), \forall p_{i} \in P $$
    \item A partir de las matrices $Pre$ y $Post$ se puede reconstruir la
    estructura de la red, a partir de $I$ no es posible.
  \end{itemize}
\end{itemize}

\subsubsection{Sucesión de Disparos}

Si en lugar del disparo de una transición se requiere disparar múltiples
transiciones, se puede reescribir la ecuación de cambio de estado de la red de
la siguiente forma,
$$ m_{n+1} = m_{n} + I \times \sigma $$
En esta ecuación, $\sigma$ representa la sucesión de disparos a realizar. Se
cumple $\sigma \in \mathbb{N}^{t}$ y el elemento $\sigma_{i}$ contiene la
cantidad de disparos a realizar sobre $t_{i}$.\\
Si se comienza a realizar la sucesión de disparos $\sigma_{i}$ a partir del
marcado inicial $m_{0}$ y todos los disparos son exitosos, se llega a un marcado
$m_{i}$ y se dice que $m_{i}$ es \textit{alcanzable}.\\
De la misma forma, si existe un marcado $m_{j}$ alcanzable desde $m_{0}$, debe
exitir una sucesión de disparos $\sigma_{j}$ que permita alcanzarlo.


        \chapter{Paradigmas de Programación}
            \section{Paradigma Dataflow}

El paradigma de programación \textit{Dataflow} se basa en la idea de evitar que
el programador piense en términos del flujo de control del programa y se centre
en el flujo de los datos que son procesados.
De esta manera, las aplicaciones son representadas como un conjunto de nodos (o
bloques) con puertos de entrada y/o salida. Estos nodos pueden ser productores,
consumidores o bloques de procesamiento de información que fluye por el sistema. Los nodos
están conectados por aristas que definen el flujo de información por el
sistema. La mayoría de los lenguajes de programación visuales que usan una
arquitectura basada en bloques están basados en el paradigma dataflow.
\cite{DataflowTiagoSousa}

Los nodos son ejecutados únicamente cuando reciben y/o envían mensajes, lo que
sucede asíncronamente respecto de los demás nodos. Por esto, las aplicaciones
dataflow son inherentemente paralelas.\cite{DataflowRichardHarter}

La programación dataflow es capaz de proveer paralelismo sin la complejidad de
la gestión de hilos. Esto es posible gracias a que cada nodo es un bloque de
procesamiento independiente de los demás y no produce efectos colaterales
\cite{DataflowTiagoSousa}

Hay una amplia variedad de lenguajes dataflow, variando de hojas de cálculo,
Labview, hasta Erlang. Muchos son gráficos. La programación se hace alterando
diagramas de flujo. Una característica que tienen todos en común es que tienen
un sistema de ejecución (runtime system).\cite{DataflowRichardHarter}

Los programas imperativos tradicionales están compuestos de rutinas que se
llaman entre sí, por ejemplo, cuando una llamada hace que el llamador construya
un paquete de datos (secuencia de llamada) y transfiere el control y el paquete
de datos a la rutina llamada. Cuando la rutina llamada termina, contruye un
paquete de datos para pasar de vuelta al llamador y le transfiere nuevamente el control.

En los programas dataflow las “rutinas” no se llaman entre sí, en su lugar son
activadas por el sistema de ejecución cuando hay entrada para ellos. Cuando se
crean salidas, el sistema de ejecución se hace cargo de mover la salida al
destino que requiere esas salidas. Cuando las “rutinas” terminan, transfieren el
control de vuelta al sistema de ejecución.

Una diferencia entre la programación imperativa y dataflow es
la semántica utilizada. Mientras la programación imperativa utiliza semántica
LIFO, la dataflow usa semántica FIFO \cite{DataflowRichardHarter}. Eso es, un
programa imperativo pone datos en una pila y obtiene datos desde la misma pila.
En cambio en programas dataflow, cada elemento obtiene datos de una cola y pone
datos en otras colas.
Otra diferencia es que la conectividad de los programas procedurales está
embebida en el código. Para pasar datos de la rutina $A$ a $B$, $A$ debe
llamar explícitamente a $B$, es decir que un llamado tiene que especificar el
destino de los datos.
Por otro lado, en programas dataflow la conectividad puede estar separada
del código, $A$ no pasa datos directamente a $B$; en su lugar, le pasa datos al
sistema de ejecución, quien le pasa los datos a $B$.
El llamador no tiene que especificar hacia dónde van los datos y hasta puede
no saberlo. \cite{DataflowRichardHarter}

\subsection*{Ventajas y Desventajas del paradigma Dataflow}

Entre las ventajas de utilizar el paradigma dataflow se encuentran:

\begin{itemize}
  \item La concurrencia y paralelismo son naturales. El código se puede distribuir entre cores y a través de redes. Algunos
  problemas relacionados a hilos desaparecen
  \item Las redes dataflow son representaciones naturales e intuitivas para
  representar procesos.
  \item El paso de mensajes permite deshacerse de problemas asociados a memoria compartida y locks.
  \item Los programas dataflow son más extensibles que programas tradicionales.
  Los elementos pueden ser agrupados en elementos compuestos.
\end{itemize}

Por otro lado, resulta poco ventajoso utilizar este paradigma por los siguientes
motivos:

\begin{itemize}
  \item El modelo de pensamiento de programación dataflow es poco familiar para
  la mayoría de los programadores profesionales.
  \item La mayoría de los lenguajes de programación dataflow son lenguajes de
  un nicho usado por programadores no profesionales.
  \item La intervención del sistema de ejecución puede tener aparejado un alto
  costo computacional. La gran ventaja de la semántica LIFO es que se implementa
  en código de manera inmediata y poco costosa.
  \item No utilizar memoria compartida tiene sus costos. Los mensajes deben ser
  copiados o deben ser inmutables.
  \item Usar programación dataflow requiere que sea utilizada del principio. De
  esta manera, convertir programas tradicionales en programas dataflow es
  difícil porque la estructura es diferente.
\end{itemize}

\section{Paradigma Reactivo}

El paradigma reactivo es un paradigma de programación construído en torno a
flujos de datos, y la propagación de los cambios sobre ellos. Esto significa que
los lenguajes que implementan este paradigma deben permitir expresar flujos de
datos de manera estática o dinámica con facilidad, y el modelo de ejecución
debe propagar automáticamente los cambios en los datos cuando ocurran,
actualizando todos los valores correspondientes de manera transparente para el
programador.

A fin de comprender las características principales de este paradigma se
presenta el siguiente ejemplo:

\begin{figure}[h!]
\centering
\begin{minted}{perl}
a = 1
b = 2
c = a + b
a = 3
\end{minted}
\end{figure}

En programación imperativa, terminada la ejecución de esta sección de código,
$c$ vale $3$ y así se mantendrá indefinidamente o hasta que el programador le
asigne un nuevo valor. En cambio en programación reactiva el valor de $c$ se
mantiene siempre actualizado, es decir, la expresión declarada como $c$
se vuelve a computar automáticamente ante un cambio en $a$ o en $b$, y en este
ejemplo pasa a valer $5$. Se dice que $c$ es \textit{dependiente} de $a$ y $b$.
\cite{Bainomugisha:2013:SRP:2501654.2501666}

Al igual que en el paradigma dataflow, en el paradigma reactivo son los datos
los que fluyen por el programa en lugar del control. La diferencia radica en
que, bajo el paradigma reactivo, las ``conexiones'' de datos pueden ser
alteradas dinámicamente en tiempo de ejecución.
Además se introducen restricciones de tiempo real blando, para lo cual se
definen dos conceptos:
\begin{itemize}
  \item \textit{Behaviours (Comportamientos)} representan eventos de variación
  contínua en el tiempo. El beahaviour por excelencia es el tiempo, de hecho los
  lenguajes reactivos ofrecen primitivas para representar al tiempo.
  \item \textit{Events (Eventos)} representan eventos discretos. Suelen estar
  representados en forma de flujos de cambios de valores. A diferencia de los
  behaviours, los eventos cambian en instantes puntuales del tiempo. Los
  lenguajes reactivos ofrecen primitivas para combinar y procesar eventos.
\end{itemize}
\cite{Bainomugisha:2013:SRP:2501654.2501666}

\subsection*{El Paradigma Reactivo y El Patrón Observer}

El patrón de diseño \textit{observer} \cite{Gamma:1995:DPE:186897} nace de la
necesidad de mantener consistencia de datos en sistemas particionados, sin generar
acoplamiento entre capas de dichos sistemas.
Permite que un \textit{sujeto} publique cambios en su estado a sus
\textit{observadores}, quienes se susbribieron previamente a estas
actualizaciones.

El patrón observer se debe utilizar en alguna de las siguientes situaciones:
\begin{itemize}
  \item Cuando una abstracción tiene dos partes, una dependiente de la otra.
  \item Cuando el cambio en un objeto implica el cambio en otros, y no se sabe
  de antemano cuántos ni quiénes deben aplicar estos cambios.
  \item Cuando un objeto debe notificar a otros sin conocer nada de ellos, es
  decir sin generar acomplamiento.
\end{itemize}
\cite{Gamma:1995:DPE:186897}

Analizando este patrón de diseño se encuentran similitudes con el paradigma
reactivo.
La programación reactiva es capaz de explicitar mayor granularidad, pudiendo
describir flujos de datos a nivel de clases, miembros de éstas y hasta
variables, mientras que el patrón observer lo puede hacer a nivel de clases
únicamente.

En programación reactiva, cuando se forma una expresión dependiente de otras, se
genera una suscripción implícita de manera automática y el modelo de ejecución
es el encargado de propagar los cambios de manera transparente para el
programador.


            \section{Programación Orientada a Aspectos}
\label{sec:aop}

\subsection{Concepto}

La programación orientada a aspectos es un paradigma de programación que tiene
como objetivo incrementar la modularidad mediante la separación de intereses
transversales (cross-cutting concerns). 
Los intereses transversales son aspectos de un programa que afectan a otros
intereses. Son partes de un programa que afectan o dependen de muchas
otras partes del sistema.
Estos intereses usualmente no pueden separarse claramente del resto del
sistema, y pueden resultar en duplicación de código o un alto grado de
dependencia entre partes del sistema.


Los intereses transversales son la base para el desarrollo de aspectos. Estos no
pueden ser representados claramente en los paradigmas de programación orientado
a objetos o programación procedural. \cite{AspectJInAction}
La separación de intereses transversales se realiza añadiendo comportamientos
adicionales al código existente, llamados advices o consejos, sin modificar el
mismo. Para lograrlo, se especifican puntos de ejecución (mediante la definición
de pointcuts) donde se aplican los advices previamente mencionados.

La programación orientada a aspectos complementa a la programación orientada a
objetos al permitir al desarrollador modificar dinamicamente el modelo estático
orientado a objetos para crear un sistema que puede crecer para cumplir nuevos
requerimientos. Tal como los objetos en el mundo real pueden cambiar sus estados
a lo largo de su vida, una aplicación puede adoptar nuevas características a
medida que se va desarrollando. \cite{Introduction_To_Aspect}


\subsection{Terminología}
\label{sec:aop_terminologia}
\begin{itemize}
  \item Intereses Transversales (Cross-cutting concerns): Aunque la mayoría de
  las clases en un modelo orientado a objetos está destinada a perfeccionar una función única y
  específica, usualmente comparten requerimientos secundarios en común con otras
  clases. Por ejemplo, se puede desear añadir mecanismos de logueo a las clases
  dentro de la capa de acceso de datos y también a las clases en la capa de
  interfaz de usuario cada vez que un hilo entre o salga de un método. Aunque la
  funcionalidad principal de cada clase es muy diferente, el código necesario
  para realizar la tarea secundaria es usualmente
  idéntico.\cite{Introduction_To_Aspect}
  
  \item Consejos (Advices): Es el código adicional que se desea aplicar al
  modelo existente. Siguiendo con el ejemplo anterior, es el código de logueo
  que se quiere aplicar cada vez que un hilo ingrese o salga de un
  método.\cite{Introduction_To_Aspect}
  
  \item Punto de unión (Join-point): Es el término que se le otorga al punto
  de ejecución en la aplicación en el cual los intereses transversales deben ser
  aplicados. En el ejemplo, un punto de unión es alcanzado cuando un hilo
  ingresa a un método, y un segundo punto de unión es alcanzado cuando un hilo
  sale de un método.
  
  \item Punto de corte (Point-cut): Un punto de corte es un conjunto de puntos
  de unión. Un point-cut permite definir dónde aplicar exactamente un consejo,
  lo cual permite la separación de intereses y ayuda a modularizar la lógica de
  negocios \cite{Classification_Of_Pointcut_Language_Constructs}.
  
  \item Aspecto (Aspect): La combinación de un punto de corte y un consejo se
  denomina aspecto. \cite{Introduction_To_Aspect}
  
  \item Tejido (Weaving): Proceso de aplicar aspectos a los objetos
  destinatarios para crear los nuevos objetos resultantes en los puntos de
  unión especificados. De acuerdo al momento del ciclo de vida del sistema en
  el cual se aplica el tejido, se realiza la siguiente clasificación:
  	\begin{itemize}
		\item Aspectos en Tiempo de Compilación.
		\item Aspectos en Tiempo de Carga.
		\item Aspectos en Tiempo de Ejecución.
	\end{itemize}
\end{itemize} 
        \chapter{Frameworks y Genración de Código}
            \section{Generación de Código Fuente}

La idea de generación automática de código fuente y de código ejecutable es casi
tan antigua como la programación en sí misma. Debido a que ahorra mucho
tiempo y costo de desarrollo de sistemas, ha sido y sigue siendo foco de
investigación.

La generación automática de código fuente está englobada por el concepto de
\textit{Programación Automática}. El significado de este término ha avanzado
junto a la programación a lo largo de los años:
\begin{itemize}
  \item En la década de 1940, se llamó de esta forma a la automatización del
  proceso de perforar cintas de papel para escribir el programa
  \cite{AutomaticProgrammingGorn}. Lo que Gorn llamó programación automática, es
  en realidad un lenguaje assembler.
  \item En el comienzo de los lenguajes de alto nivel se le llamó de esta manera
  a los compiladores. Tanto es así que uno de los primeros compiladores se llamó
  Autocode.
  \item Actualmente se identifica este término como la generación de código
  fuente escrito en un lenguaje de programación (compilable o interpretable a
  código máquina) a partir de una descripción de más alto nivel.
\end{itemize}

Algunos ejemplos de generadores de código son:
\begin{itemize}
  \item \underline{Apache Thrift:} Desarrollado por Facebook y actualmente
  liberado bajo licencia Apache, Thrift es un generador de servicios para múltiples lenguajes
  orientado a la comunicación por medio de llamadas a procedimiento remoto
  (remote procedure call – RPC). Para lograr esto expone un lenguaje de
  definición de interfaces (interface definition language – IDL) propio,
  utilizado para describir el servicio que luego Thrift generará en alguno de
  los múltiples lenguajes que soporta.\cite{ApacheThrift}
  \item \underline{Acceleo}: Es un generador de código que implementa el
  estándar \textit{MOFM2T} desarrollado por The Eclipse Foundation y su código fuente
  está liberado bajo licencia EPL. Acceleo permite la especificación del
  software en modelos como UML (v1 y v2), EMF (eclipse model framework) y
  lenguajes de modelado personalizados (DSL). Además permite especificar
  plantillas definidas por el usuario. Genera código en múltiples lenguajes de
  programación.\cite{Acceleo}
  \item \underline{Actifsource:} Desarrollado por actifsource GmbH y de código
  cerrado, es un generador de código a partir de modelos similares a UML. Soporta la
  creación de múltiples modelos y la unión de estos, y la utilización de modelos
  generados en cualquier software que tolere formato ecore, definido por el
  Eclipse Modeling Framework. Está desarrollado como un plugin para
  Eclipse.\cite{Actifsource}
  \item \underline{Spring Roo:} Desarrollado conjuntamente por DISID y Pitvotal
  bajo licencia Apache 2.0, es un generador enfocado al desarrollo acelerado de
  software empresarial en Java. La aplicación generada utiliza tecnologías Java
  comunes como Spring Framework, Java Persistence API, Apache Maven, etc. A
  diferencia de otros generadores de código, Roo expone una interfaz por línea
  de comandos con sus propios comandos.\cite{SpringRoo}
  \item \underline{GeneXus:} Desarrollado por ARTech bajo licencia cerrada y con
  primer lanzamiento en 1988, es un generador de código fuente a partir de un
  lenguaje declarativo de alto nivel. A partir de este lenguaje, se genera
  código fuente en C\#, COBOL, Java, Objective-C, RPG, Visual Basic,
  Ruby y Visual FoxPro. Además tolera múltiples lenguajes para gestión de bases de
  datos como Microsoft SQL, Oracle, DB2, Informix, PostgreSQL y
  MySQL.\cite{Genexus}
  \end{itemize}
  
La programación automática siempre ha sido un eufemismo para la
programación con un lenguaje de más alto nivel del disponible para el
programador. Investigar en programación automática es simplemente desarrollar
la implementación de lenguajes de programación de más alto nivel
\cite{Parnas:1985:SAS:214956.214961}.

En conclusión, los generadores automáticos de código fuente son en realidad
traductores de un lenguaje de programación a otro. Esto brinda un mayor nivel
de abstracción para el programador pero lo obliga a especificar su software en
el lenguaje provisto por el generador.

            
\section{Frameworks}


\subsection{Definición}

Los frameworks son una técnica de reutilización de prácticas, conceptos y
criterios orientadas a facilitar la solución de un tipo de problemáticas en
particular. Son estructuras concretas de software, que proveen una manera
estándar de construir aplicaciones. Sirven como base para el diseño y
desarrollo de software orientado a resolver problemas específicos.
De acuerdo a \cite{Johnson97} dos de las definiciones más comunes de framework
son:
\begin{itemize}
  \item ``Un framework es un diseño reusable de todo o parte de un sistema que es
  representado por un conjunto de clases abstractas y la forma en que sus
  instancias interactúan''
  \item ``Un framework es el esqueleto de una aplicación que puede ser
  personalizado por un desarrollador de aplicaciones.''
\end{itemize} 

 La primer definición describe la estructura de un framework, mientras que la
 segunda describe su propósito.

Un framework es una técnica de reutilización de código porque facilita la
creación de una aplicación a partir de una biblioteca de componentes existentes.
Es posible la creación de nuevos componentes extendiendo los provistos por el
framework. Una característica que distingue a los frameworks de otras
técnicas de reutilización es la inversión de control (ver
sección\ref{sec:inversion_control}).

Debe pensarse en frameworks y componentes de software como tecnologías
diferentes, pero que cooperan entre sí \cite{JohnsonFeb97}:
\begin{itemize}
  \item Un framework provee un contexto reusable para los componentes.
  \item Un framework es más abstracto y flexible que los componentes.
\end{itemize} 

Por otro lado, los frameworks son más concretos y simples de reutilizar
que un diseño puro \cite{JohnsonFeb97}.

\subsection{Inversión de Control}
\label{sec:inversion_control}
La inversión de control es una característica principal de los
frameworks. Es un principio de diseño en el cual porciones de código
personalizado por el usuario son controladas por un framework.

Al implementar un sistema sin utilizar un framework, generalmente el
desarrollador escribe el código de un programa principal que realiza llamadas a
componentes de una biblioteca. El desarrollador decide en el código cuándo
llamar al componente y se encarga de la estructura y el flujo de control del
programa.

En un software basado en un framework el programa principal es reutilizado. 
El desarrollador solamente conecta componentes existentes al framework, o
implementa nuevos componentes para conectar. Las porciones de código del
desarrollador son llamadas por el framework. De esta manera, el framework
determina la estructura y el flujo de control del programa.

La inversión de control sirve para los siguientes propósitos de diseño:
\begin{itemize}
  \item Desacoplar la ejecución de una tarea de su implementación.
  \item Mantener el foco en la tarea para la que fue diseñado el módulo.
  \item Guiar el diseño respetando las interfaces entre módulos.
  \item Evitar efectos colaterales al reemplazar un módulo.
\end{itemize}

\subsection {Ventajas de los frameworks}
\begin{itemize}
	\item Al utilizar un framework se aplican técnicas de reutilización de
	software y de diseño.
	
	\item Son personalizables: Los frameworks son más
	personalizables que la mayoría de los componentes. Tienen interfaces más
	complejas.
	
	\item Sirven para múltiples aplicaciones: Un framework está orientado a
	facilitar la implementación de aplicaciones de un tipo determinado. En
	consecuencia, puede ser utilizado para implementar diversas aplicaciones que
	pertenezcan a dicho tipo.
	
	\item Facilitan el trabajo del desarrollador.
	
	\item La uniformidad reduce los costos de mantener el código: Los programadores
	encargados de mantenerlo pueden cambiar de una aplicación a otra que utiliza el
	mismo framework sin tener que aprender un nuevo diseño.
	
	\item Los frameworks obligan al usuario a respetar patrones de diseño en las
	aplicaciones.

\end{itemize}

\subsection {Desventajas de los frameworks}
\begin{itemize}
    \item Curva de Aprendizaje: Los programadores deben aprender las interfaces
    antes de poder utilizar el framework. Generalmente aprender un nuevo
    framework es difícil.
    
    \item Restricción de elección del lenguaje de programación: Uno de los
    problemas de utilizar un framework implementado en un lenguaje en particular
    es que restringe a los sistemas a utilizar dicho lenguaje. La relación
    efectividad-costo es baja al construir una aplicación en un lenguaje con un
    framework escrito en otro.
    
    \item Debido a que los frameworks son descritos con lenguajes de
    programación, es difícil para los desarrolladores aprender los patrones
    colaborativos de un framework mediante la lectura del código.
    
\end{itemize}

\subsection{Frameworks desde la perspectiva del usuario}
\label{sec:tipos_framework}
    Según \cite{JohnsonFeb97} existen tres formas de utilizar un framework
    desde la perspectiva de un usuario desarrollador de software: 
\begin{itemize}
    \item Black-Box Frameworks: Consiste en conectar componentes ya existentes.
    De esta forma no se modifica el framework ni se crean nuevas clases
    concretas sino que se reutilizan las interfaces del framework y sus reglas
    para interconectar componentes. Este método es similar a la construcción de
    un circuito eléctrico. El desarrollador necesita conocer la interfaz de
    conexión entre un objeto A y un objeto B, pero no es necesario que conozca
    la especificación exacta de A o B.

    \item White-Box Frameworks: Consiste en definir clases concretas, que
    extienden de clases abstractas definidas en el framework, y utilizarlas
    para implementar una aplicación. Las subclases están estrechamente
    acopladas a sus superclases. De esta forma se requiere más conocimiento
    acerca de la implementación de las clases que conforman el framework.
	
	\item Extensión o Modificación del núcleo del framework:  Consiste en extender
	el framework cambiando las clases abstractas que forman su núcleo para añadir
	nuevas variables u operaciones. Requiere conocimientos avanzados acerca del
	diseño del framework. Cambiar las clases abstractas puede provocar fallos en
	las clases concretas existentes. Este modo de utilización no es aplicable si
	el propósito es crear un sistema abierto.
\end{itemize}

Entre las formas de utilización mencionadas existen combinaciones intermedias. Es común
que los frameworks se utilicen como Black-Box la mayor parte del tiempo y
sean extendidos cuando la ocasión lo demande.

\section{Comparación entre Frameworks y APIs}

En la tabla \ref {tab:comparacion_frameworks_apis} se observa una comparación
entre frameworks y APIs.
\begin{table}
	\centering
	\begin{tabularx}{\textwidth}{ | p{2.5cm} | X | X | }
	\hline
	\textbf{Categoría} & \textbf{Framework} & \textbf{API (Biblioteca)} \\[10pt]
	\hline
    \textbf{Gestión del Flujo Principal} & El framework toma el
    flujo principal del software & A cargo del programador\\[10pt] \hline
    \textbf{Confiabilidad} & El flujo principal está ampliamente testeado por todos
    los usuarios del framework & No brinda ninguna garantía de flujo\\[10pt] \hline
	\hline \textbf{Extensibilidad} & Por parte de los desarrolladores del
	framework.
	Si es de código abierto cualquiera puede extenderlo. & Por parte del fabricante de la
	librería.
	O cualquiera si es de código abierto \\[10pt] \hline
	\textbf{Reusabilidad} & Objetivo principal del diseño de un framework.
	Se aplica a nivel de arquitectura de software & Es reutilizable a nivel
	de llamada a métodos\\[10pt] \hline 
	\textbf{Complejidad de Uso} & Gran
	complejidad al principio, se simplifica a medida que el usuario aprende el framework & Complejidad inicial menor que un
	framework\\[10pt] \hline 
	\textbf{Aplicación de Patrones de diseño} & Usualmente un framework fuerza al
	usuario a utilizar uno o varios patrones & No obliga al usuario a utilizar ningún patrón
	de diseño\\[10pt] \hline 
	\textbf{Especificidad / Generalización} & Son de uso específico,
	están diseñados para resolver una familia de problemas. Por esto mantienen una
	arquitectura & De uso general donde una funcionalidad pueda ser
	utilizada\\[10pt] \hline 
	\textbf{Restricciones de lenguaje} & Obliga al usuario a
	desarrollar en el mismo lenguaje en el que está hecho el framework & No restringe a un lenguaje.
	Permite llamadas desde cualquier lenguaje mientras se respete la firma de las funciones expuestas\\[10pt] 
	\hline
	\end{tabularx}
	\caption{Comparación entre Frameworks y APIs}
	\label{tab:comparacion_frameworks_apis}
\end{table}



    \part{Desarrollo}
        \chapter{Investigación}
            \section{Requerimientos}

\begin{enumerate}
	\item El sistema debe delegar flujo de ejecución a un motor de
	petri.
	\begin{itemize}
		\item El sistema debe mapear transiciones de una red de petri a eventos
		especificados por los usuarios. Un evento puede ser equivalente a un conjunto de transiciones.
		\item Cuando un evento es desencadenado por el disparo de un conjunto de
		transiciones el sistema debe ejecutar todas las tareas que se encuentran registradas al evento.
		\item Cuando un suceso definido por el usuario ocurre, el sistema debe
		notificar todos los eventos asociados a este suceso al motor de petri.
		\item El sistema debe proveer una interface para que el usuario pueda
		suscribir sucesos, tareas y fines de tareas a eventos especificados por el usuario.
		\item El sistema debe proveer una interface para que el usuario pueda definir
		eventos.
		\item Cuando una tarea termina el sistema debe notificar al motor de petri
		acerca de todos los eventos asociados a la finalización de la tarea.
	\end{itemize}
	\item Para un usuario con conocimiento intermedio en Java y Redes de Petri, el
	framework puede aprender a usarse en una semana o menos.
	\begin{itemize}
	    \item La utilización del sistema puede incorporar como máximo diez
	    conceptos nuevos a aprender por un usuario con un nivel intermedio en Java
	    y redes de Petri.
	    \item El sistema debe ser acompañado con al menos dos ejemplos de uso en
	    los cuales se muestre al menos un 80\% de las funciones del mismo.
	\end{itemize}
	\item El sistema debe ser compatible con las versiones actuales de motores de
	Petri desarrollados en el Laboratorio de Arquitectura de Computadoras de la
	Facultad de Ciencias Exactas y Naturales de la Universidad Nacional de Córdoba.
	\begin{itemize}
	    \item El sistema debe proveer una interfaz para que el usuario ingrese un
	    archivo PNML con la descripción de una red de Petri.
	    \item El sistema puede instanciar un entorno de ejecución de redes de
	    Petri dado que el usuario ha ingresado un archivo PNML conteniendo la
	    descripción de la red y ha elegido el motor de Petri que desea usar.
	    \item El sistema debe utilizar la interface expuesta por el motor de
	    petri.
	\end{itemize}
	\item El sistema quiere tener una interfaz gráfica de usuario para inicializar
	un nuevo proyecto.
	\begin{itemize}
	    \item La interfaz de usuario quiere contener una pantalla 'PNML Loader'
	    	\begin{itemize}
	    	    \item La pantalla debe dejar al usuario buscar en su disco local y
	    	    elegir un archivo.
	    	    \item La pantalla debe dejar al usuario ingresar la dirección a un
	    	    archivo manualmente mediante la escritura con el teclado.
	    	    \item La pantalla debe permitir confirmar la elección de un archivo.
	    	    \item Si el usuario confirma un archivo y el archivo es un PNML válido
	    	    entonces puede usarse para configurar el entorno de ejecución de
	    	    Petri.
	    	    \item Si el usuario confirma un archivo y el archivo no es un PNML
	    	    válido entonces debe mostrarse un error en pantalla y el usuario debe
	    	    ser capaz de elegir otro archivo.
	    	\end{itemize}
	    \item La interfaz de usuario quiere contener una pantalla de creación de
	    eventos
	    \begin{itemize}
	    	    \item La pantalla debe dejar al usuario definir un evento y asociarlo
	    	    con una o más transiciones definidas en un archivo PNML cargado
	    	    previamente por el usuario.
	    	    \item La pantalla de creación de eventos quiere permitir guardar las
	    	    decisiones del usuario en un archivo.
	    	    \item La pantalla de creación de eventos quiere permitir al usuario
	    	    cargar configuraciones a partir de un archivo.
	    	    \item Si un archivo guardado previamente se selecciona para ser
	    	    cargado y su contenido tiene un formato inválido, la pantalla
	    	    quiere mostrar un texto de error especificando el problema y el
	    	    archivo no debe ser cargado.
	    	    \item Si un archivo guardado previamente se selecciona para ser
	    	    cargado y el contenido del archivo contiene uno omás eventos que
	    	    mapean a transiciones inexistentes, la pantalla quiere mostrar un
	    	    texto de error especificando el problema y sólo debe cargarse la
	    	    configuración de los eventos fuera de conflicto.
	    \end{itemize}
	     \item La interfaz de usuario quiere contener una pantalla de selección de
	     el motor de Petri.
	     \begin{itemize}
	         \item La pantalla debe permitir elegir entre un motor de Petri Java,
	         un motor de Petri en hardware o un motor de Petri en driver.
	         \item La pantalla debe comunicar la decisión del usuario para preparar
	         el entorno de ejecución de Petri de acuerdo al motor elegido.
	     \end{itemize}
	         
	\end{itemize}
\end{enumerate}
        \chapter{Monitor de Concurrencia con Redes de Petri}
            \newcommand{\javapetriconcurrencymonitor}{Java Petri Concurrency Monitor }

\section{\javapetriconcurrencymonitor}

\subsection{Introducción}

\javapetriconcurrencymonitor  es un monitor de concurrencia que ejecuta redes
de petri, hecho en lenguaje de programación java.

\subsection{Características Principales}
Entre las principales características de \javapetriconcurrencymonitor están:
\begin{itemize}
  \item Soporte para Redes de Petri:
  \begin{itemize}
    \item Plaza-Transición
    \item Temporales
  \end{itemize}
  
  \item Soporte para tipos de arcos;
  \begin{itemize}
    \item Normal
    \item Lector o de Prueba
    \item Inhibidor
    \item Reset
  \end{itemize}
  
  \item Soporte para guardas
  \item Soporte para transitiones automáticas
  \item Soporte para subscripción a eventos en transiciones informadas
  \item Soporte de políticas intercambiables y extensibles de prioridad de
  disparo de transiciones

\end{itemize}

\subsection{Implementación}

\subsection{Manual de Uso}
        \chapter{\nombreFramework \  Framework}
            \section{Análisis de Modelos con Redes de Petri}
\subsection{Petición de ejecución versus Aviso de ejecución.}
\subsubsection{Estudio de la Red de Petri de una cinta transportadora}
Cinta transportadora con 3 estaciones. Piezas son depositadas en la primer
estación  de manera asincrónica. Cuando esto sucede, la cinta avanza a la
estación 1, donde un operario realiza una transformación a la pieza. Una vez el
operario realizó la transformación, presiona un pulsador y la cinta avanza a la
estación 2, donde un segundo operario empaqueta la pieza. Este segundo operario
presiona otro pulsador al finalizar su tarea y luego la cinta avanza una vez
más y la pieza cae en un contenedor.

\begin{figure}[H]
    \centering
    \includegraphics[height=100mm]{Petri_Cinta_Transportadora_1}
    \caption{Red de Petri de una cinta transportadora}
    \label{fig:petri_cinta_transportadora_1}
\end{figure}

\subsubsection{Análisis de ejecución del caso de estudio en framework Chimp} 
De acuerdo al modo de ejecución implementado por Chimp ~\cite{chimp}, el
framework da aviso de eventos al monitor, desencadenando la ejecución de las tareas:
\begin{enumerate}
    \item Debe insertarse un evento en la cola de entrada de “t0” cuando el framework
		detecte la llegada de una nueva pieza. Si la cinta Transportadora se encuentra
		disponible, el monitor de petri dispara “t0” y se genera un evento que se
		deposita en la cola de salida de “t0”.
    \item Chimp lee el evento de salida de “t0” y realiza la acción “moverEst1”, que
		mueve la pieza a la estación 1 y espera la acción del operador. Una vez que el
		operador realiza su acción, presiona el pulsador generando un evento que Chimp
		envía a la cola de  entrada de “t1”. El monitor de petri dispara “t1” y se
		genera un evento que se deposita en la cola de salida de “t1”.
    \item Chimp lee el evento de salida de “t1” y realiza la acción “moverEst2”, que
		mueve la pieza a la estación 2 y espera la acción del operador. Una vez que el
		operador realiza su acción, presiona el pulsador generando un evento que Chimp
		envía a la cola de  entrada de “t2”. El monitor de petri dispara “t1” y se
		genera un evento que se deposita en la cola de salida de “t2”.
    \item Chimp lee el evento de salida de “t2 y realiza la accion “moverACont”, que
		mueve la pieza al contenedor. Una vez terminada esa acción envía un evento a
		la cola de entrada de “t3”. El monitor de petri dispara “t3” y libera la
		cinta Transportadora para procesar otra pieza.
\end{enumerate}

Esta forma de ejecución limita la funcionalidad de sincronización de la red de
Petri, acotando su funcionamiento únicamente a tareas sincrónicas. El
framework Chimp no contempla el caso de eventos externos asincrónicos que
desencadenen disparos en la red. Además, el framework está asumiendo una parte
del rol de control de flujo de ejecución, lo cual debería delegarse por completo
al monitor de redes de Petri.

\subsubsection{Análisis de ejecución del caso de estudio por
petición de ejecución al motor de petri} 
Una alternativa al método de ejecución de Chimp consiste en que los hilos
encargados de realizar las tareas realicen una petición de ejecución al monitor
sin tener en cuenta el estado actual de la red de Petri.
 El monitor es el encargado de dormir aquellas tareas que no pueden ser
 ejecutadas. Una vez que las condiciones son las adecuadas para realizar la
 tarea, el monitor se encarga de despertar al hilo encargado de ejecutarla.
\begin{enumerate}
    \item Se generan eventos que se encolan en la cola de entrada en “t0, t1,
    	t2 y t3”.
    \item El monitor duerme los hilos que generaron eventos para “t1, t2 y t3”
    	por no estar sensibilizadas las transiciones en ese momento.
    \item El monitor ejecuta “t0”. Y se envía un evento a la cola de salida de
    	“t0”.
    \item Chimp lee el evento de salida de “t0” y ejecuta “moverEst1”.
    \item Existe un problema, ya que al disparar “t0”, el monitor tiene
    	permitido disparar “t1”, pero la operación “moverEst1” aun no ha
    	finalizado.
\end{enumerate}
Tras el análisis  del ejemplo anterior se llega a una serie de
conclusiones. En primer lugar, es necesario que el framework de aviso al monitor
cuando una tarea debe ser realizada de forma atómica. Además, el
sistema de peticiones es más adecuado para una arquitectura manejada por un
monitor, sin embargo, dada la red de petri
Figura~\ref{fig:petri_cinta_transportadora_1} surgen problemas de
sincronización. Un ejemplo de estos problemas se origina al realizar una
petición de ejecución de la tarea “moverEst2”, el monitor permite ejecutar esta
tarea de forma inmediata, sin tener en cuenta si la tarea ``moverEst1'' ha
finalizado.

\subsubsection{Sincronización por Plaza-Transición}
Una posible solución a los problemas planteados en la sección anterior es
modelar la red de la siguiente forma:\\

\begin{figure}[H]
    \centering
    \includegraphics[height=100mm]{Petri_Cinta_Transportadora_2}
    \caption{Red de Petri de una cinta transportadora sincronizada por inserción
    de plaza-transición}
    \label{fig:petri_cinta_transportadora_2}
\end{figure}


Donde:\\
\begin{enumerate}
	\item Se generan eventos que se encolan en la cola de entrada en “t0, t2 y
		t4”.
	\item El monitor duerme los hilos que generaron eventos para “t0, t2 y t4” por
		no estar sensibilizadas las transiciones en ese momento.
	\item Llega una pieza y se genera un evento de entrada en “t6”
	\item El monitor dispara “t6” y se coloca un token en “piezaDisp”,
		sensibilizando “t0”.
	\item El monitor despierta el hilo dormido en “t0” ya que ahora tiene permiso
		de ejecución.
	\item Se ejecuta “moverEst1”. Una vez finalizado se genera un evento que se
		envía a la cola de entrada de “t1”.
	\item Como “t1” está sensibilizada el monitor la dispara y se coloca un token
		en “piezaLista1”, sensibilizando “t2”.
	\item El monitor despierta el hilo dormido en “t2” ya que ahora tiene permiso
		de ejecución.
	\item Se ejecuta “moverEst2”. Una vez finalizado se genera un evento que se
		envía a la cola de entrada de “t3”.
	\item Como “t3” está sensibilizada el monitor la dispara y se coloca un token
		en “piezaLista2”, sensibilizando “t4”.
	\item El monitor despierta el hilo dormido en “t4” ya que ahora tiene permiso
		de ejecución.
	\item Se ejecuta “moverACont”. Una vez finalizado se genera un evento que se
		envía a la cola de entrada de “t5”
	\item Como “t5” está sensibilizada el monitor la dispara y se coloca un token
		en ``piezaEnCont''.
	\item Se ejecuta la transición ``t7'', que es automática, y se libera el
		recurso ``cintaTransp''.
\end{enumerate}
La principal ventaja de este método es que no modifica la semántica de la red y
no añade nuevos conceptos ni cambios en su forma de ejecución.
La desventaja consiste en que realizar este tipo de sincronización puede
conllevar un incremento considerable de la cantidad de plazas y transiciones de
la red, lo que conlleva el procesamiento de matrices de mayor tamaño.

\subsubsection{Sincronización por Guardas}
Este método consiste en la utilización de una guarda como forma de
sincronización entre tareas consecutivas. Ver Figura ~\ref{fig:petri_cinta_transportadora_3}

\begin{figure}[H]
    \centering
    \includegraphics[height=100mm]{Petri_Cinta_Transportadora_3}
    \caption{Red de Petri de una cinta transportadora sincronizada por guardas.}
    \label{fig:petri_cinta_transportadora_3}
\end{figure}

\begin{enumerate}
    \item Se generan eventos que se encolan en la cola de entrada en “t0, t1 y
    t2”.
	\item El monitor duerme los hilos que generaron eventos para “t1 y t2” por
	no estar sensibilizadas las transiciones en ese momento.
	\item Se dispara ``t0'' y se coloca un token en ``moverEst1''. Comienza la
	ejecución de la tarea ``moverEst1''. La transición ``t1'' no se encuentra
	sensibilizada dado que la guarda ``Fin\_moverEst1'' tiene estado ``false''.
	\item Finaliza la ejecución de ``moverEst1'' y se setea la guarda
	``Fin\_moverEst1'' con estado ``true''.
	\item Al estar sensibilizada ``t1'', se dispara y se despierta el hilo que
	duerme en su cola de condición. Se coloca un token en ``moverEst2'' y
	comienza la ejecución de esta tarea. La transición ``t2'' no se
	encuentra sensibilizada dado que la guarda ``Fin\_moverEst2'' tiene estado
	``false''. Debe setearse la guarda ``Fin\_moverEst1'' a ``false'' nuevamente.
	\item Finaliza la ejecución de ``moverEst2'' y se setea la guarda
	``Fin\_moverEst2'' con estado ``true''.
	\item Al estar sensibilizada ``t2'', se dispara y se despierta el hilo que
	duerme en su cola de condición. Se coloca un token en ``moverACont'' y
	comienza la ejecución de esta tarea. La transición ``t3'' no se
	encuentra sensibilizada dado que la guarda ``Fin\_moverACont'' tiene estado
	``false''. Debe setearse la guarda ``Fin\_moverEst2'' a ``false'' nuevamente.
	\item Finaliza la ejecución de ``moverACont'' y se setea la guarda
	``Fin\_moverACont'' con estado ``true''.
	\item Al estar sensibilidada, se dispara la transición ``t3'', que es
	automática, y se libera el recurso ``cintaTransp''. Debe setearse la guarda
	``Fin\_moverACont'' a ``false'' nuevamente.
\end{enumerate}

La ventaja de este método es que permite resolver el problema de sincronización
sin aumentar la cantidad de componentes de la red de Petri.
Como desventaja se puede mencionar que modifica la semántica de la
red, complicando su demostración matemática. Además, el diseño del monitor de
petri soporta una única guarda por transición, por lo tanto esta solución impide
la utilización de la guarda para otros propósitos en una situación de tareas
consecutivas. Por último, una desventaja importante de la utilización de guardas
es que al ser un valor binario no se puede saber cuantas veces ha sido seteada
la guarda.
Se supone un caso donde una ``tarea A'' es realizada por multiples hilos de manera
independiente, y cada hilo realiza la ``tarea A'' en su totalidad. A su
vez una ``tarea B'', que debe realizarse luego de la finalización de la ``tarea
A'', es ejecutada por un único hilo. En este caso la utilización de guardas
podría llevar a una pérdida de eventos de finalización de la ``tarea A'' debido
a la condición binaria de la guarda. Ver Figura ~\ref{fig:ejecucion_multiples_hilos_guardas}

\begin{figure}[H]
    \centering
    \includegraphics[height=60mm]{Ejecucion_Tarea_Multiples_Hilos_Guardas}
    \caption{RdP: Problema de sincronización de tareas sincrónicas usando
    guardas debido a su condición binaria}
    \label{fig:ejecucion_multiples_hilos_guardas}
\end{figure}

En esta red, un máximo de 5 hilos puede ejecutar la ``tarea A'' al mismo
tiempo. En el estado que  muestra la Figura
~\ref{fig:ejecucion_multiples_hilos_guardas} existen tres hilos corriendo la
``tarea A''. De acuerdo a lo supuesto en el planteo de este problema, la ``tarea
B'' es ejecutada por un único hilo. Si dos o más hilos finalizan la ``tarea A''
y setean la guarda ``Fin\_TareaA'' entonces, cuando se dispare ``t1" y antes de
comenzar la ejecución de la ``tareaB'', se debe modificar el valor de la guarda
``Fin\_TareaA'' a ``false'', y de esta manera existe la posibilidad de perder
eventos de finalización de la ``tarea A''.

\subsubsection{Sincronización por Disparo Perenne de Aviso de
Finalización de Tarea}
Esta forma de solucionar la sincronización de tareas sincrónicas consecutivas
supone añadir una nueva propiedad ``P'' a las transiciones. Los hilos que
duermen en la cola de condición de una transición con propiedad ``P'' sólo
se despiertan cuando la transición se encuentra habilitada y además un hilo
externo realiza un disparo perenne sobre la transición.

\begin{figure}[H]
    \centering
    \includegraphics[height=100mm]{Petri_Cinta_Transportadora_4}
    \caption{Red de Petri de una cinta transportadora sincronizada por
    propiedad ``P''.}
    \label{fig:petri_cinta_transportadora_4}
\end{figure}

\begin{enumerate}
    \item Se generan eventos que se encolan en la cola de entrada en “t0, t1,
    t2 y t3”.
	\item El monitor duerme los hilos que generaron eventos para “t1, t2 y t3” por
	no estar sensibilizadas las transiciones en ese momento.
	\item Se dispara ``t0'' y se coloca un token en ``moverEst1''. Comienza la
	ejecución de la tarea ``moverEst1''. La transición ``t1'' no se dispara ya que
	es de tipo ``P'' y solo puede dispararse de forma perenne por un hilo externo.
	\item Finaliza la ejecución de ``moverEst1'' y un hilo dispara ``t1'' de forma
	perenne para dar aviso de la finalización de la tarea.
	\item Se despierta el hilo que
	duerme en cola de condición de ``t1''. Se coloca un token en ``moverEst2'' y
	comienza la ejecución de esta tarea. La transición ``t2'' no se dispara ya que
	es de tipo ``P'' y solo puede dispararse de forma perenne por un hilo externo.
	\item Finaliza la ejecución de ``moverEst2'' y un hilo dispara ``t2'' de forma
	perenne para dar aviso de la finalización de la tarea.
	\item  Se despierta el hilo que
	duerme en cola de condición de ``t2''. Se coloca un token en ``moverACont'' y
	comienza la ejecución de esta tarea. La transición ``t3'' no se dispara ya que
	es de tipo ``P'' y solo puede dispararse de forma perenne por un hilo externo.
	\item Finaliza la ejecución de ``moverACont'' y un hilo dispara ``t3'' de forma
	perenne para dar aviso de la finalización de la tarea.
	\item Se libera el recurso ``cintaTransp''.
\end{enumerate}

Esta solución tiene como principal ventaja mantener la cantidad de componentes
de la red de Petri.
Su principal desventaja consiste en que añade un nuevo componente a la
petri, dificultando su demostración matemática. Esta solución supone añadir una
interfaz al monitor de petri para dormir hilos en una cola de condición de una
transición tipo P y que los hilos bloqueados en esta cola de condición sólo
puedan liberarse por medio de un disparo perenne ocasionado por un hilo externo.
\emph{\color{red} Los hilos peticionarios (que solicitan permiso de ejecución)
a una transición tipo P no debieran utilizar la interfaz fire. En cambio, deben
utilizar una interfaz sleep, que haga una operacion acquire sobre el semáforo
de la cola de condición.
Los hilos que dan aviso de una finalización de tarea deben utilizar la interfaz
fire de modo perenne para realizar una operación release sobre el semáforo de
la cola de condición. Esta operación release debe darse por más que no existan
hilos esperando en la cola de condición de la transición tipo P, porque sino se
tendría una pérdida de eventos de finalización de tarea.}



    \bibliography{./bibliografia}
	\bibliographystyle{plain}
\end{document}