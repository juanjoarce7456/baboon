\documentclass{report}
\usepackage[spanish]{babel}
\usepackage[utf8]{inputenc}
\usepackage{amsmath, array}
\usepackage{amssymb}
\usepackage{tabularx}
\usepackage{minted}
\RecustomVerbatimEnvironment{Verbatim}{BVerbatim}{}
\usepackage{cite}
\usepackage{graphicx}
\usepackage{graphbox}
\usepackage{color}
\usepackage{enumitem}
\usepackage{float}
\usepackage{subfigure}
\usepackage{amsfonts}
\usepackage[section]{placeins}
\usepackage{hyperref}
\usepackage{blindtext}
\usepackage{scrextend}
\usepackage{enumitem}
\usepackage{subcaption}
\addtokomafont{labelinglabel}{\sffamily}
\hypersetup{
     colorlinks=true,
     citecolor=black,
     filecolor=black,
     linkcolor=black,
     urlcolor=blue
}
\date{}
\setcounter{secnumdepth}{3} % Add number to subsubsections
\usepackage{framed}

% counter used for definitions. Every time you want to add a definition, use
% \stepcounter{definitionsCounter} before the definition
\newcounter{definitionsCounter}

\begin{document}
    %Nombre templetizado para poder cambiarlo fácil hasta tenerlo definido
    \newcommand{\nombreTesis}{Framework de Sincronización de Tareas Coordinado
    por Redes de Petri}
    %Nombre templetizada por si lo tenemos que cambiar
    \newcommand{\nombreFramework}{Baboon}
    %Nombre del monitor templetizado
    \newcommand{\javapetriconcurrencymonitor}{Java Petri Concurrency Monitor }
    
    \newcommand{\repoMonitor}{https://github.com/airabinovich/java_petri_engine.git}
    \newcommand{\repoFramework}{https://github.com/juanjoarce7456/petri_event_dispatcher.git}

    % reset subfigure counter
    \newcommand{\ResetCounter}{\setcounter{subfigure}{0}}

    \title{\nombreTesis}
    \author{Ariel Iván Rabinovich \\ \href{mailto:airabinovich@gmail.com}{airabinovich@gmail.com}
        \and Juan José Arce Giacobbe \\ \href{mailto:juanjo.arce7546@gmail.com}{juanjo.arce7456@gmail.com}}
    \graphicspath{ {resources/images/} }
    
    \maketitle
    
    \tableofcontents
    
    \listoffigures
    \listoftables
    
    \setcounter{definitionsCounter}{0}
    \part{Introducción y Objetivos}
        \chapter{Introducción}
        \section{Introducción}
En este capítulo se detalla el diseño de \nombreFramework \ Framework.
En primer lugar se fundamenta la decisión de elaborar un Framework teniendo en
cuenta el análisis de experiencias previas. Se realiza una clasificación de los
eventos que se intercambian en sistemas reactivos desarrollados utilizando el
monitor de RdP.
Se detalla el diseño de la arquitectura del framework en base a dicho
intercambio de eventos.

Se realiza un análisis de las formas en que se pueden sincronizar las
acciones de un sistema utilizando un monitor de RdP. Este análisis tiene el
objetivo de definir el modo de sincronización más adecuado para la arquitectura
del framework.

Se define el concepto de controlador de acción y sus clasificaciones. A su vez,
se define el concepto de Guard Provider. Finalmente se define la relación entre
los eventos y los controladores de acción, formalizando el intercambio de
eventos entre el software de usuario y el framework.



        \chapter{Objetivos}
        \section{Objetivo General}
El objetivo general de este proyecto integrador es diseñar e implementar un
framework que permita desarrollar sistemas reactivos utilizando modelos basados
en Redes de Petri no autónomas. El framework resultante debe aislar el control
de la aplicación en un módulo que ejecute a la RdP. Este módulo debe manejar el
asincronismo del sistema de forma transparente al código de software de la
aplicación, delegando las decisiones a la RdP, tanto de la lógica del sistema
como de la política.


\section{Objetivos Secundarios}
\label{sec:objetivos_secundarios}
A continuación se mencionan los objetivos secundarios de este proyecto:
\begin{itemize}
  \item Reutilizar en la etapa de implementación de un sistema reactivo el
  modelo lógico del mismo para garantizar su correcto funcionamiento.
  \item Separar la lógica del sistema del código que implementa sus
  funcionalidades.
  \item Estudiar Redes de Petri ordinarias, orientadas a procesos y no
  autónomas.
  \item Investigar implementaciones en Proyectos Integradores previos y analizar
  la reutilización o reimplementación del código existente.
  \item Obtener un software capaz de ejecutar RdP utilizando la ecuación de
  estado generalizada descrita en \cite{Ecuacion_generalizada_LAC}.
  \item Implementar la inversión de control del framework.
  \item Priorizar la mantenibilidad del código generado respetando estándares y
  estilos de programación.
  \item Generar tests automáticos para garantizar el correcto funcionamiento de
  las funcionalidades del software generado
  \item Documentar detalladamente el código fuente siguiendo una metodología
  estándar.
  \item Ofrecer interfaces de programación sencillas.
  \item Resolver problemas conocidos de programación concurrente para poner a
  prueba el framework.
  \item Resolver un problema real de concurrencia utilizando el framework
  desarrollado.
  \item Documentar el proceso de desarrollo de aplicaciones particulares
  utilizando el framework.
  \item Ofrecer ejemplos de uso del framework para facilitar la asimilación de
  los usuarios.
  \item Ofrecer el software resultante de forma pública y accesible para
  cualquier usuario en cualquier parte del mundo.
\end{itemize}

\section{Análisis Previo}
Antes de la definición del objetivo general se realizó un trabajo de
investigación y comparación para determinar el tipo de herramienta a
desarrollar (Biblioteca de software, API, Framework, Generación de código,
etc). (ver capítulo \ref{cap:investigacion})
    \part{Marco Teórico}
        \label{cap:marco_teorico}
        \chapter{Introducción al Marco Teórico}
        \section{Introducción}
En este capítulo se detalla el diseño de \nombreFramework \ Framework.
En primer lugar se fundamenta la decisión de elaborar un Framework teniendo en
cuenta el análisis de experiencias previas. Se realiza una clasificación de los
eventos que se intercambian en sistemas reactivos desarrollados utilizando el
monitor de RdP.
Se detalla el diseño de la arquitectura del framework en base a dicho
intercambio de eventos.

Se realiza un análisis de las formas en que se pueden sincronizar las
acciones de un sistema utilizando un monitor de RdP. Este análisis tiene el
objetivo de definir el modo de sincronización más adecuado para la arquitectura
del framework.

Se define el concepto de controlador de acción y sus clasificaciones. A su vez,
se define el concepto de Guard Provider. Finalmente se define la relación entre
los eventos y los controladores de acción, formalizando el intercambio de
eventos entre el software de usuario y el framework.



        \chapter{Modelos}
            \section{Autómatas y Redes de Petri}

\subsection{Autómatas o Máquinas de Estado}

Existen muchas formas de modelar el comportamiento de los sistemas, y el uso de
máquinas de estado finitas es una de las más antiguas y más conocidas.
Las máquinas de estado finitas o autómatas nos permiten pensar acerca del
``estado'' de un sistema en un instante en particular y caracterizar el comportamiento de dicho
sistema basado en ese estado. El uso de esta técnica de modelado no está
limitada al desarrollo de sistemas de software.\cite{FSM_Wright}

\subsubsection{Definición Conceptual de Máquina de Estado}

Si una máquina de estados M, en un instante dado, se encuentra en el estado
$E_{0}$ y ocurre un evento $e_{0}$ que lleva a M al estado $E_{1}$, se
dice que ocurrió una \textit{transición} del estado $E_{0}$ al estado
$E_{1}$.
A partir de esto se puede deducir que M no puede estar en $E_{0}$ y $E_{1}$
a la vez, y por lo tanto los estados de una máquina de estados, son
\textbf{estados globales} del sistema modelado.

Analizando la semántica de las máquinas de estado, se pueden
identificar algunas características clave de un sistema que puede ser modelado con máquinas de
estados finitas:
\begin{itemize}
  \item El sistema debe ser descripto por conjunto finito de estados.
  \item El sistema debe tener una cantidad finita de entradas y/o eventos que
  puedan disparar transiciones entre estados.
  \item El comportamiento del sistema en un instante dado depende del estado
  actual y de sus entradas o eventos que ocurran en ese instante.
  \item Para cada estado posible en que el sistema pueda encontrarse existe un
  comportamiento definido para cada posible entrada o evento.
  \item El sistema tiene un estado inicial único y definido.
\end{itemize} \cite{FSM_Wright}

\subsubsection{Definición Formal de Máquina de Estado}

A fin de eliminar la ambigüedad existente en una definición conceptual, se
introduce una definición formal de Autómata Finito:
\newline\newline\emph{Definición:} Un autómata finito M está definido por una
tupla $(\Sigma, Q, q_{0}, F, \sigma)$, donde:
\begin{itemize}    
  \item $\Sigma$ es el conjunto de símbolos de entrada de M
  \item $Q$ es el conjunto de estados de M
  \item $q_{0}$ es el estado inicial de M
  \item $F \subseteq Q$ es el conjunto de estados finales de M
  \item $\sigma : Q  \times \Sigma \rightarrow Q$ es la función de
  transición
\end{itemize} \cite{FSM_Wright}

\subsection{Redes de Petri}

Tomando el concepto de transición en una máquina de estados, se lo puede
extender a una entidad propia.
Esta transitión $t_{i}$ será denotada por una barra, un rectángulo o un
cuadrado, y puede tener múltiples arcos de entrada (entrantes) y de salida
(salientes) a la vez. Esta transición, representa la \textit{transición} básica
de una Red de Petri (RdP).\cite{PetriNetsFundamentals}

De la misma forma que en una máquina de estados los círculos denotan estados
del sistema, en una RdP se utilizan círculos para denotar las \textit{plazas} o
\textit{lugares} de la red. Estas plazas no representan estados globales, sino
\textbf{estados locales}. \cite{PetriNetsFundamentals}

El estado local de una plaza, está dado por la cantidad de \textit{tokens} o
\textit{marcas} que esta contiene.

Como consecuencia de su estructura, una Red de Retri puede ser representada como
un grafo bipartito, donde los tipos de nodo existentes son \textit{plazas} y
\textit{transiciones}. Estos nodos se unen entre dos de distinto tipo
únicamente (de ahí el calificativo de bipartito), utilizando \textit{arcos}.\\

\begin{figure}[h]
	\centering
	\includegraphics[width=75mm]{Partes_De_Una_Red}
	\caption{Partes de una Red de Petri}
	\label{fig:partes_de_una_red}
\end{figure}

Se pueden visualizar las partes de una Red de Petri en la figura
\ref{fig:partes_de_una_red}.\\


\begin{figure}[h]
    \centering
    \includegraphics[height=40mm]{Automata_Y_Petri}
    \caption{Equivalencia entre una Máquina de Estados y una Red de Petri}
    \label{fig:automata_y_petri}
\end{figure}

En la figura \ref{fig:automata_y_petri} se aprecia:\\
\begin{itemize}
  \item[(a)] Una máquina de estados de dos estados y una transición.
  \item[(b)] Una RdP equivalente a la máquina de (a).
  \item[(c)] Una RdP con una transición con múltiples arcos de entrada y de
  salida.
\end{itemize}

Se puede extraer como consecuencia directa de esta extensión de la semántica de
un autómata que en una Red de Petri:
\begin{itemize}
  \item Múltiples tokens pueden existir en el modelo al mismo tiempo, y
  particularmente en una plaza.
  \item No existe un estado global explícito.
  \item El estado global del sistema es el conjunto de todos los estados
  parciales, representados por las plazas y sus tokens. A este conjunto se lo
  llama el \textbf{marcado} de la red.
\end{itemize}

\subsubsection{Definición Formal de Red de Petri}
A fin de eliminar ambigüedades, se presenta una serie de definiciones sobre
Redes de Petri.

\begin{itemize}
  \item [\underline{Definición 1}:] Una Red de Petri R está definida por la
  tupla $(P, T, Pre, Post)$ donde:
  \begin{itemize}
    \item $ P = \{ p_1, p_2, \ldots, p_p \} $ un conjunto de plazas.\footnote{Se
    utiliza $p$ como la cantidad de plazas de la RdP en todo momento dentro de este informe por simplicidad para el lector}
    \item $ T = \{ t_1, t_2, \ldots, t_t \} $ un conjunto de transiciones, donde
    $ P \cap T = \emptyset $. \footnote{Se utiliza $t$ como la cantidad de
    transiciones de la RdP en todo momento dentro de este informe por
    simplicidad para el lector}
    \item $ Pre: P \times T \rightarrow \mathbb{N}^{p} $ aplicación de
    precedencia.\footnote{Se toma la definición de números naturales incluyendo
    el cero por simplicidad de notación.}
    \item $ Post: P \times T \rightarrow \mathbb{N}^{p} $ aplicación de
    incidencia.
  \end{itemize}
  $ Pre (p_i, t_j) $ contiene el peso del arco que va de $ p_i $ a $ t_j $, y
  $ Post (p_i, t_j) $ contiene el peso del arco que va de $ t_j $ a $ p_i $

  \item [\underline{Definición 2}:] Una Red de Petri Marcada está
  definida por el par $(R, M)$, donde R es una RdP y $ M : P \rightarrow
  \mathbb{N}^{p} $ (siendo $P$ el conjunto de plazas de dimensión $n$) es una aplicación llamada \textit{marcado}.\\
  $m(R)$, o más simplemente $m$ si la red es conocida, define el marcado de la
  RdP y $m(p_{i})$ o $mp_{i}$ indica el marcado de la plaza $p_{i}$, es decir,
  el número de tokens contenido en la plaza $p_{i}$.\\
  La marca inicial se denota $m_{0}$ y da la cantidad inicial de tokens en todas
  las plazas de la red, por lo que especifica el estado inicial del sistema.
  
  \item [\underline{Definición 3}:] Para una marca $m$, una transición $t_{j}$
  está sensibilizada, y por lo tanto es disparable, si y solo si:\\
  $$ \forall p_{i} \in P, m(p_i) \geq Pre(p_{i}, t_{j}) $$
  Conceptualmente, una transición está sensibilizada si todas sus plazas de
  entrada contienen al menos la cantidad de tokens que indica el peso de los
  arcos que las unen.

  En la figura \ref{fig:transiciones_no_sensibilizadas} se observa gráficamente esta definición mediante dos casos de transiciones no sensibilizadas. Nótese
  el peso de los arcos.

  \begin{figure}[h]
    \centering
    \includegraphics[height=40mm]{Transiciones_No_Sensibilizadas}
    \caption{Ejemplos de transiciones no sensibilizadas.}
    \label{fig:transiciones_no_sensibilizadas}
  \end{figure}
  
  \item [\underline{Definición 4}:] La estructura de una Red de Petri
  se denota $ N = \{P, T, F, W\} $ donde,
  \begin{itemize}
    \item $P$ es en conjunto de plazas.
    \item $T$ es el conjunto de transiciones, donde se cumple que $ P \cap T =
    \emptyset $
    \item $F$ es el conjunto de arcos, donde se cumple que $ F \subseteq (P
    \times T) \cup (F \times P) $.
    \item $W$ es la función de peso de los arcos.
  \end{itemize}

  \item [\underline{Definición 5}:] Conjunto de transición y plaza de entrada y
  de salida.
  \begin{itemize}
    \item[] El conjunto de las plazas de entrada a la transición $t$ se denota
    $\bullet t$ y se define,
    $$ \bullet t = \{ p \in P : (p, t) \in F \} $$
    \item[] El conjunto de las plazas de salida de la transición $t$ se denota $
    t \bullet$ y se define,
    $$ t \bullet = \{ p \in P : (t, p) \in F \} $$
    \item[] El conjunto de las transiciones de entrada a la plaza $p$ se
    denota $\bullet p$ y se define,
    $$ \bullet p = \{ t \in T : (t, p) \in F \} $$
    \item[] El conjunto de las transiciones de salida de la plaza $p$ se denota
    $ p \bullet$ y se define,
    $$ p \bullet = \{ t \in T : (p, t) \in F \} $$
  \end{itemize}
\end{itemize}

\subsubsection{Disparo de una Transición}

La condición de disparo relacionada a $Pre(p_{i}, t_{j})$ significa que para
todas las plazas $p_{i}$ de entrada a $t_{j}$, es decir, todas las plazas que
tienen arcos que apuntan hacia $t_{j}$, el número de tokens presentes debe ser
mayor o igual al peso de dicho arco.

\begin{itemize}
  \item [\underline{Definición 6}:] En una RdP, dada una marca $ m_{n}(p) $,
  cualquier transición $ t_{j} $ que se encuentre sensibilizada puede ser
  disparada, y su disparo lleva a una marca $ m_{n+1}(p)$ dada por:
  $$ m_{n+1}(p) = m_{n}(p) + Post(p_{i}, t_{j}) - Pre(p_{i}, t_{j}), \forall
  p_{i} \in P $$
  Como se indica en la ecuación, al disparar la transición $ t_{j} $, se quitan
  tantos tokens de $ \bullet t $ como indiquen los arcos que las unen a $ t_{j}
  $, y se añaden a $ t \bullet $ la cantidad de tokens que indiquen los arcos
  que unen a $ t_{j} $ con ellas.\\
  El disparo de una transición $ t_{j} $ se denota $ m_{n}\rightarrow t_{j}
  \rightarrow m_{n+1} $

  En la figura {\ref{fig:disparo_transicion}} se observa el estado de una RdP
  antes y después del disparo de una transición.
  \begin{figure}[h]
    \centering
    \subfigure[$t_{0}$ sensibilizada]{\includegraphics[height=40mm]{Red_Sensibilizada}}
    \subfigure[Disparo de $t_{0}$]{\includegraphics[height=40mm]{Red_Disparada}}
    \caption{Disparo de una transición}
    \label{fig:disparo_transicion}
  \end{figure}
  
  \item  [\underline{Definición 7}:] Matriz de Incidencia.\\
  La matriz de incidencia de una RdP se define como,
  $$ I = Post - Pre $$
  \textbf{Notas:}
  \begin{itemize}
    \item El disparo de una transición se reformula como, $$ m_{n+1}(p) =
    m_{n}(p) + I(p_{i}, t_{j}), \forall p_{i} \in P $$
    \item A partir de las matrices $Pre$ y $Post$ se puede reconstruir la
    estructura de la red, a partir de $I$ no es posible.
  \end{itemize}
\end{itemize}

\subsubsection{Sucesión de Disparos}

Si en lugar del disparo de una transición se requiere disparar múltiples
transiciones, se puede reescribir la ecuación de cambio de estado de la red de
la siguiente forma,
$$ m_{n+1} = m_{n} + I \times \sigma $$
En esta ecuación, $\sigma$ representa la sucesión de disparos a realizar. Se
cumple $\sigma \in \mathbb{N}^{t}$ y el elemento $\sigma_{i}$ contiene la
cantidad de disparos a realizar sobre $t_{i}$.\\
Si se comienza a realizar la sucesión de disparos $\sigma_{i}$ a partir del
marcado inicial $m_{0}$ y todos los disparos son exitosos, se llega a un marcado
$m_{i}$ y se dice que $m_{i}$ es \textit{alcanzable}.\\
De la misma forma, si existe un marcado $m_{j}$ alcanzable desde $m_{0}$, debe
exitir una sucesión de disparos $\sigma_{j}$ que permita alcanzarlo.


        \chapter{Paradigmas de Programación}
            \section{Paradigma Dataflow}

El paradigma de programación \textit{Dataflow} se basa en la idea de evitar que
el programador piense en términos del flujo de control del programa y se centre
en el flujo de los datos que son procesados.
De esta manera, las aplicaciones son representadas como un conjunto de nodos (o
bloques) con puertos de entrada y/o salida. Estos nodos pueden ser productores,
consumidores o bloques de procesamiento de información que fluye por el sistema. Los nodos
están conectados por aristas que definen el flujo de información por el
sistema. La mayoría de los lenguajes de programación visuales que usan una
arquitectura basada en bloques están basados en el paradigma dataflow.
\cite{DataflowTiagoSousa}

Los nodos son ejecutados únicamente cuando reciben y/o envían mensajes, lo que
sucede asíncronamente respecto de los demás nodos. Por esto, las aplicaciones
dataflow son inherentemente paralelas.\cite{DataflowRichardHarter}

La programación dataflow es capaz de proveer paralelismo sin la complejidad de
la gestión de hilos. Esto es posible gracias a que cada nodo es un bloque de
procesamiento independiente de los demás y no produce efectos colaterales
\cite{DataflowTiagoSousa}

Hay una amplia variedad de lenguajes dataflow, variando de hojas de cálculo,
Labview, hasta Erlang. Muchos son gráficos. La programación se hace alterando
diagramas de flujo. Una característica que tienen todos en común es que tienen
un sistema de ejecución (runtime system).\cite{DataflowRichardHarter}

Los programas imperativos tradicionales están compuestos de rutinas que se
llaman entre sí, por ejemplo, cuando una llamada hace que el llamador construya
un paquete de datos (secuencia de llamada) y transfiere el control y el paquete
de datos a la rutina llamada. Cuando la rutina llamada termina, contruye un
paquete de datos para pasar de vuelta al llamador y le transfiere nuevamente el control.

En los programas dataflow las “rutinas” no se llaman entre sí, en su lugar son
activadas por el sistema de ejecución cuando hay entrada para ellos. Cuando se
crean salidas, el sistema de ejecución se hace cargo de mover la salida al
destino que requiere esas salidas. Cuando las “rutinas” terminan, transfieren el
control de vuelta al sistema de ejecución.

Una diferencia entre la programación imperativa y dataflow es
la semántica utilizada. Mientras la programación imperativa utiliza semántica
LIFO, la dataflow usa semántica FIFO \cite{DataflowRichardHarter}. Eso es, un
programa imperativo pone datos en una pila y obtiene datos desde la misma pila.
En cambio en programas dataflow, cada elemento obtiene datos de una cola y pone
datos en otras colas.
Otra diferencia es que la conectividad de los programas procedurales está
embebida en el código. Para pasar datos de la rutina $A$ a $B$, $A$ debe
llamar explícitamente a $B$, es decir que un llamado tiene que especificar el
destino de los datos.
Por otro lado, en programas dataflow la conectividad puede estar separada
del código, $A$ no pasa datos directamente a $B$; en su lugar, le pasa datos al
sistema de ejecución, quien le pasa los datos a $B$.
El llamador no tiene que especificar hacia dónde van los datos y hasta puede
no saberlo. \cite{DataflowRichardHarter}

\subsection*{Ventajas y Desventajas del paradigma Dataflow}

Entre las ventajas de utilizar el paradigma dataflow se encuentran:

\begin{itemize}
  \item La concurrencia y paralelismo son naturales. El código se puede distribuir entre cores y a través de redes. Algunos
  problemas relacionados a hilos desaparecen
  \item Las redes dataflow son representaciones naturales e intuitivas para
  representar procesos.
  \item El paso de mensajes permite deshacerse de problemas asociados a memoria compartida y locks.
  \item Los programas dataflow son más extensibles que programas tradicionales.
  Los elementos pueden ser agrupados en elementos compuestos.
\end{itemize}

Por otro lado, resulta poco ventajoso utilizar este paradigma por los siguientes
motivos:

\begin{itemize}
  \item El modelo de pensamiento de programación dataflow es poco familiar para
  la mayoría de los programadores profesionales.
  \item La mayoría de los lenguajes de programación dataflow son lenguajes de
  un nicho usado por programadores no profesionales.
  \item La intervención del sistema de ejecución puede tener aparejado un alto
  costo computacional. La gran ventaja de la semántica LIFO es que se implementa
  en código de manera inmediata y poco costosa.
  \item No utilizar memoria compartida tiene sus costos. Los mensajes deben ser
  copiados o deben ser inmutables.
  \item Usar programación dataflow requiere que sea utilizada del principio. De
  esta manera, convertir programas tradicionales en programas dataflow es
  difícil porque la estructura es diferente.
\end{itemize}

\section{Paradigma Reactivo}

El paradigma reactivo es un paradigma de programación construído en torno a
flujos de datos, y la propagación de los cambios sobre ellos. Esto significa que
los lenguajes que implementan este paradigma deben permitir expresar flujos de
datos de manera estática o dinámica con facilidad, y el modelo de ejecución
debe propagar automáticamente los cambios en los datos cuando ocurran,
actualizando todos los valores correspondientes de manera transparente para el
programador.

A fin de comprender las características principales de este paradigma se
presenta el siguiente ejemplo:

\begin{figure}[h!]
\centering
\begin{minted}{perl}
a = 1
b = 2
c = a + b
a = 3
\end{minted}
\end{figure}

En programación imperativa, terminada la ejecución de esta sección de código,
$c$ vale $3$ y así se mantendrá indefinidamente o hasta que el programador le
asigne un nuevo valor. En cambio en programación reactiva el valor de $c$ se
mantiene siempre actualizado, es decir, la expresión declarada como $c$
se vuelve a computar automáticamente ante un cambio en $a$ o en $b$, y en este
ejemplo pasa a valer $5$. Se dice que $c$ es \textit{dependiente} de $a$ y $b$.
\cite{Bainomugisha:2013:SRP:2501654.2501666}

Al igual que en el paradigma dataflow, en el paradigma reactivo son los datos
los que fluyen por el programa en lugar del control. La diferencia radica en
que, bajo el paradigma reactivo, las ``conexiones'' de datos pueden ser
alteradas dinámicamente en tiempo de ejecución.
Además se introducen restricciones de tiempo real blando, para lo cual se
definen dos conceptos:
\begin{itemize}
  \item \textit{Behaviours (Comportamientos)} representan eventos de variación
  contínua en el tiempo. El beahaviour por excelencia es el tiempo, de hecho los
  lenguajes reactivos ofrecen primitivas para representar al tiempo.
  \item \textit{Events (Eventos)} representan eventos discretos. Suelen estar
  representados en forma de flujos de cambios de valores. A diferencia de los
  behaviours, los eventos cambian en instantes puntuales del tiempo. Los
  lenguajes reactivos ofrecen primitivas para combinar y procesar eventos.
\end{itemize}
\cite{Bainomugisha:2013:SRP:2501654.2501666}

\subsection*{El Paradigma Reactivo y El Patrón Observer}

El patrón de diseño \textit{observer} \cite{Gamma:1995:DPE:186897} nace de la
necesidad de mantener consistencia de datos en sistemas particionados, sin generar
acoplamiento entre capas de dichos sistemas.
Permite que un \textit{sujeto} publique cambios en su estado a sus
\textit{observadores}, quienes se susbribieron previamente a estas
actualizaciones.

El patrón observer se debe utilizar en alguna de las siguientes situaciones:
\begin{itemize}
  \item Cuando una abstracción tiene dos partes, una dependiente de la otra.
  \item Cuando el cambio en un objeto implica el cambio en otros, y no se sabe
  de antemano cuántos ni quiénes deben aplicar estos cambios.
  \item Cuando un objeto debe notificar a otros sin conocer nada de ellos, es
  decir sin generar acomplamiento.
\end{itemize}
\cite{Gamma:1995:DPE:186897}

Analizando este patrón de diseño se encuentran similitudes con el paradigma
reactivo.
La programación reactiva es capaz de explicitar mayor granularidad, pudiendo
describir flujos de datos a nivel de clases, miembros de éstas y hasta
variables, mientras que el patrón observer lo puede hacer a nivel de clases
únicamente.

En programación reactiva, cuando se forma una expresión dependiente de otras, se
genera una suscripción implícita de manera automática y el modelo de ejecución
es el encargado de propagar los cambios de manera transparente para el
programador.


            \section{Programación Orientada a Aspectos}
\label{sec:aop}

\subsection{Concepto}

La programación orientada a aspectos es un paradigma de programación que tiene
como objetivo incrementar la modularidad mediante la separación de intereses
transversales (cross-cutting concerns). 
Los intereses transversales son aspectos de un programa que afectan a otros
intereses. Son partes de un programa que afectan o dependen de muchas
otras partes del sistema.
Estos intereses usualmente no pueden separarse claramente del resto del
sistema, y pueden resultar en duplicación de código o un alto grado de
dependencia entre partes del sistema.


Los intereses transversales son la base para el desarrollo de aspectos. Estos no
pueden ser representados claramente en los paradigmas de programación orientado
a objetos o programación procedural. \cite{AspectJInAction}
La separación de intereses transversales se realiza añadiendo comportamientos
adicionales al código existente, llamados advices o consejos, sin modificar el
mismo. Para lograrlo, se especifican puntos de ejecución (mediante la definición
de pointcuts) donde se aplican los advices previamente mencionados.

La programación orientada a aspectos complementa a la programación orientada a
objetos al permitir al desarrollador modificar dinamicamente el modelo estático
orientado a objetos para crear un sistema que puede crecer para cumplir nuevos
requerimientos. Tal como los objetos en el mundo real pueden cambiar sus estados
a lo largo de su vida, una aplicación puede adoptar nuevas características a
medida que se va desarrollando. \cite{Introduction_To_Aspect}


\subsection{Terminología}
\label{sec:aop_terminologia}
\begin{itemize}
  \item Intereses Transversales (Cross-cutting concerns): Aunque la mayoría de
  las clases en un modelo orientado a objetos está destinada a perfeccionar una función única y
  específica, usualmente comparten requerimientos secundarios en común con otras
  clases. Por ejemplo, se puede desear añadir mecanismos de logueo a las clases
  dentro de la capa de acceso de datos y también a las clases en la capa de
  interfaz de usuario cada vez que un hilo entre o salga de un método. Aunque la
  funcionalidad principal de cada clase es muy diferente, el código necesario
  para realizar la tarea secundaria es usualmente
  idéntico.\cite{Introduction_To_Aspect}
  
  \item Consejos (Advices): Es el código adicional que se desea aplicar al
  modelo existente. Siguiendo con el ejemplo anterior, es el código de logueo
  que se quiere aplicar cada vez que un hilo ingrese o salga de un
  método.\cite{Introduction_To_Aspect}
  
  \item Punto de unión (Join-point): Es el término que se le otorga al punto
  de ejecución en la aplicación en el cual los intereses transversales deben ser
  aplicados. En el ejemplo, un punto de unión es alcanzado cuando un hilo
  ingresa a un método, y un segundo punto de unión es alcanzado cuando un hilo
  sale de un método.
  
  \item Punto de corte (Point-cut): Un punto de corte es un conjunto de puntos
  de unión. Un point-cut permite definir dónde aplicar exactamente un consejo,
  lo cual permite la separación de intereses y ayuda a modularizar la lógica de
  negocios \cite{Classification_Of_Pointcut_Language_Constructs}.
  
  \item Aspecto (Aspect): La combinación de un punto de corte y un consejo se
  denomina aspecto. \cite{Introduction_To_Aspect}
  
  \item Tejido (Weaving): Proceso de aplicar aspectos a los objetos
  destinatarios para crear los nuevos objetos resultantes en los puntos de
  unión especificados. De acuerdo al momento del ciclo de vida del sistema en
  el cual se aplica el tejido, se realiza la siguiente clasificación:
  	\begin{itemize}
		\item Aspectos en Tiempo de Compilación.
		\item Aspectos en Tiempo de Carga.
		\item Aspectos en Tiempo de Ejecución.
	\end{itemize}
\end{itemize} 
            \section{Programación Orientada a Objetos}

No es del interés de este proyecto integrador ahondar en el campo de la
programación orientada a objetos, y se asume que el lector tiene conocimientos
afianzados sobre este paradigma de programación. Aún así, es necesario nombrarlo
para poder introducir los conceptos desarrollados en la sección \ref{reflection}

\subsection{Reflection}
\label{reflection}

Existen escenarios en la programación donde es útil tener la opción de conocer
los datos disponibles y las operaciones que se pueden aplicar sobre estos datos
en tiempo de ejecución. Además, resulta ventajoso poder tomar decisiones sobre
estos datos y operaciones en base al flujo del programa para modificar el
comportamiento del mismo. Tener esta posibilidad permite escribir software
flexible, reutilizable y capaz de adaptarse a múltiples escenarios. Estas
capacidades se pueden obtener por medio de la programación basada en la
\textit{reflexión} del programa.

Reflexión (o reflection) es la habilidad de un programa de examinarse a sí
mismo y a su entorno en tiempo de ejecución, y de cambiar su comportamiento
dependiendo de lo que encuentra.

Para realizar esta autoexaminación, un programa necesita tener una
representación de sí mismo. Esta información se llama \textit{metainformación} o
\textit{metadata}. En particular, en un entorno de programación orientada a
objetos la metadata se organiza en objetos, llamados \textit{metaobjetos}. La
revisión en tiempo de ejecución de los metaobjetos se llama
\textit{introspección} o \textit{introspection}.
\cite{Forman04javareflection}

En general, la introspección está seguida de un cambio del comportamiento.
Existen tres técnicas que una interfaz de programación de reflection puede
ofrecer para generar cambios de comportamiento:
\begin{itemize}
    \item Modificación de los metaobjetos
    \item Operaciones con la metadata: como la invocación dinámica de métodos 
    \item Intercesión: Se le permite al código interceder en varias fases de la
    ejecución para alterar el comportamiento del programa.
\end{itemize}

Estas características hacen que el uso de reflection permita diseñar software
más flexible, que se adapte más fácilmente a cambio de requerimientos y que a
la vez mantenga una estructura ordenada y buena legibilidad de código.
Esto favorece a la mantenibilidad.

Para poder hacer introspection, un programa que aplique reflection debe poder
acceder a su metainformación. Por esto, esta representación es el elemento
estructural más importante de un sistema reflectivo. Examinando su
autorepresentación, un programa puede obtener información acerca de su
estructura y comportamiento para tomar decisiones importantes.

Existen tres problemas relacionados al uso de reflection en el diseño de un
programa que deben ser tenidos en cuenta para poder asegurar la calidad del
mismo. Estos son:
\begin{itemize}
    \item Seguridad
    \item Complejidad del código
    \item Performance
\end{itemize}
Todos ellos se pueden mitigar mediante el uso de buenas prácticas de programación y un correcto diseño del software.
        \chapter{Generación de Código Frameworks y APIs}
            \label{generacion_frameworks_apis}
            \section{Generación de Código Fuente}

La idea de generación automática de código fuente y de código ejecutable es casi
tan antigua como la programación en sí misma. Debido a que ahorra mucho
tiempo y costo de desarrollo de sistemas, ha sido y sigue siendo foco de
investigación.

La generación automática de código fuente está englobada por el concepto de
\textit{Programación Automática}. El significado de este término ha avanzado
junto a la programación a lo largo de los años:
\begin{itemize}
  \item En la década de 1940, se llamó de esta forma a la automatización del
  proceso de perforar cintas de papel para escribir el programa
  \cite{AutomaticProgrammingGorn}. Lo que Gorn llamó programación automática, es
  en realidad un lenguaje assembler.
  \item En el comienzo de los lenguajes de alto nivel se le llamó de esta manera
  a los compiladores. Tanto es así que uno de los primeros compiladores se llamó
  Autocode.
  \item Actualmente se identifica este término como la generación de código
  fuente escrito en un lenguaje de programación (compilable o interpretable a
  código máquina) a partir de una descripción de más alto nivel.
\end{itemize}

Algunos ejemplos de generadores de código son:
\begin{itemize}
  \item \underline{Apache Thrift:} Desarrollado por Facebook y actualmente
  liberado bajo licencia Apache, Thrift es un generador de servicios para múltiples lenguajes
  orientado a la comunicación por medio de llamadas a procedimiento remoto
  (remote procedure call – RPC). Para lograr esto expone un lenguaje de
  definición de interfaces (interface definition language – IDL) propio,
  utilizado para describir el servicio que luego Thrift generará en alguno de
  los múltiples lenguajes que soporta.\cite{ApacheThrift}
  \item \underline{Acceleo}: Es un generador de código que implementa el
  estándar \textit{MOFM2T} desarrollado por The Eclipse Foundation y su código fuente
  está liberado bajo licencia EPL. Acceleo permite la especificación del
  software en modelos como UML (v1 y v2), EMF (eclipse model framework) y
  lenguajes de modelado personalizados (DSL). Además permite especificar
  plantillas definidas por el usuario. Genera código en múltiples lenguajes de
  programación.\cite{Acceleo}
  \item \underline{Actifsource:} Desarrollado por actifsource GmbH y de código
  cerrado, es un generador de código a partir de modelos similares a UML. Soporta la
  creación de múltiples modelos y la unión de estos, y la utilización de modelos
  generados en cualquier software que tolere formato ecore, definido por el
  Eclipse Modeling Framework. Está desarrollado como un plugin para
  Eclipse.\cite{Actifsource}
  \item \underline{Spring Roo:} Desarrollado conjuntamente por DISID y Pitvotal
  bajo licencia Apache 2.0, es un generador enfocado al desarrollo acelerado de
  software empresarial en Java. La aplicación generada utiliza tecnologías Java
  comunes como Spring Framework, Java Persistence API, Apache Maven, etc. A
  diferencia de otros generadores de código, Roo expone una interfaz por línea
  de comandos con sus propios comandos.\cite{SpringRoo}
  \item \underline{GeneXus:} Desarrollado por ARTech bajo licencia cerrada y con
  primer lanzamiento en 1988, es un generador de código fuente a partir de un
  lenguaje declarativo de alto nivel. A partir de este lenguaje, se genera
  código fuente en C\#, COBOL, Java, Objective-C, RPG, Visual Basic,
  Ruby y Visual FoxPro. Además tolera múltiples lenguajes para gestión de bases de
  datos como Microsoft SQL, Oracle, DB2, Informix, PostgreSQL y
  MySQL.\cite{Genexus}
  \end{itemize}
  
La programación automática siempre ha sido un eufemismo para la
programación con un lenguaje de más alto nivel del disponible para el
programador. Investigar en programación automática es simplemente desarrollar
la implementación de lenguajes de programación de más alto nivel
\cite{Parnas:1985:SAS:214956.214961}.

En conclusión, los generadores automáticos de código fuente son en realidad
traductores de un lenguaje de programación a otro. Esto brinda un mayor nivel
de abstracción para el programador pero lo obliga a especificar su software en
el lenguaje provisto por el generador.

            
\section{Frameworks}


\subsection{Definición}

Los frameworks son una técnica de reutilización de prácticas, conceptos y
criterios orientadas a facilitar la solución de un tipo de problemáticas en
particular. Son estructuras concretas de software, que proveen una manera
estándar de construir aplicaciones. Sirven como base para el diseño y
desarrollo de software orientado a resolver problemas específicos.
De acuerdo a \cite{Johnson97} dos de las definiciones más comunes de framework
son:
\begin{itemize}
  \item ``Un framework es un diseño reusable de todo o parte de un sistema que es
  representado por un conjunto de clases abstractas y la forma en que sus
  instancias interactúan''
  \item ``Un framework es el esqueleto de una aplicación que puede ser
  personalizado por un desarrollador de aplicaciones.''
\end{itemize} 

 La primer definición describe la estructura de un framework, mientras que la
 segunda describe su propósito.

Un framework es una técnica de reutilización de código porque facilita la
creación de una aplicación a partir de una biblioteca de componentes existentes.
Es posible la creación de nuevos componentes extendiendo los provistos por el
framework. Una característica que distingue a los frameworks de otras
técnicas de reutilización es la inversión de control (ver
sección\ref{sec:inversion_control}).

Debe pensarse en frameworks y componentes de software como tecnologías
diferentes, pero que cooperan entre sí \cite{JohnsonFeb97}:
\begin{itemize}
  \item Un framework provee un contexto reusable para los componentes.
  \item Un framework es más abstracto y flexible que los componentes.
\end{itemize} 

Por otro lado, los frameworks son más concretos y simples de reutilizar
que un diseño puro \cite{JohnsonFeb97}.

\subsection{Inversión de Control}
\label{sec:inversion_control}
La inversión de control es una característica principal de los
frameworks. Es un principio de diseño en el cual porciones de código
personalizado por el usuario son controladas por un framework.

Al implementar un sistema sin utilizar un framework, generalmente el
desarrollador escribe el código de un programa principal que realiza llamadas a
componentes de una biblioteca. El desarrollador decide en el código cuándo
llamar al componente y se encarga de la estructura y el flujo de control del
programa.

En un software basado en un framework el programa principal es reutilizado. 
El desarrollador solamente conecta componentes existentes al framework, o
implementa nuevos componentes para conectar. Las porciones de código del
desarrollador son llamadas por el framework. De esta manera, el framework
determina la estructura y el flujo de control del programa.

La inversión de control sirve para los siguientes propósitos de diseño:
\begin{itemize}
  \item Desacoplar la ejecución de una tarea de su implementación.
  \item Mantener el foco en la tarea para la que fue diseñado el módulo.
  \item Guiar el diseño respetando las interfaces entre módulos.
  \item Evitar efectos colaterales al reemplazar un módulo.
\end{itemize}

\subsection {Ventajas de los frameworks}
\begin{itemize}
	\item Al utilizar un framework se aplican técnicas de reutilización de
	software y de diseño.
	
	\item Son personalizables: Los frameworks son más
	personalizables que la mayoría de los componentes. Tienen interfaces más
	complejas.
	
	\item Sirven para múltiples aplicaciones: Un framework está orientado a
	facilitar la implementación de aplicaciones de un tipo determinado. En
	consecuencia, puede ser utilizado para implementar diversas aplicaciones que
	pertenezcan a dicho tipo.
	
	\item Facilitan el trabajo del desarrollador.
	
	\item La uniformidad reduce los costos de mantener el código: Los programadores
	encargados de mantenerlo pueden cambiar de una aplicación a otra que utiliza el
	mismo framework sin tener que aprender un nuevo diseño.
	
	\item Los frameworks obligan al usuario a respetar patrones de diseño en las
	aplicaciones.

\end{itemize}

\subsection {Desventajas de los frameworks}
\begin{itemize}
    \item Curva de Aprendizaje: Los programadores deben aprender las interfaces
    antes de poder utilizar el framework. Generalmente aprender un nuevo
    framework es difícil.
    
    \item Restricción de elección del lenguaje de programación: Uno de los
    problemas de utilizar un framework implementado en un lenguaje en particular
    es que restringe a los sistemas a utilizar dicho lenguaje. La relación
    efectividad-costo es baja al construir una aplicación en un lenguaje con un
    framework escrito en otro.
    
    \item Debido a que los frameworks son descritos con lenguajes de
    programación, es difícil para los desarrolladores aprender los patrones
    colaborativos de un framework mediante la lectura del código.
    
\end{itemize}

\subsection{Frameworks desde la perspectiva del usuario}
\label{sec:tipos_framework}
    Según \cite{JohnsonFeb97} existen tres formas de utilizar un framework
    desde la perspectiva de un usuario desarrollador de software: 
\begin{itemize}
    \item Black-Box Frameworks: Consiste en conectar componentes ya existentes.
    De esta forma no se modifica el framework ni se crean nuevas clases
    concretas sino que se reutilizan las interfaces del framework y sus reglas
    para interconectar componentes. Este método es similar a la construcción de
    un circuito eléctrico. El desarrollador necesita conocer la interfaz de
    conexión entre un objeto A y un objeto B, pero no es necesario que conozca
    la especificación exacta de A o B.

    \item White-Box Frameworks: Consiste en definir clases concretas, que
    extienden de clases abstractas definidas en el framework, y utilizarlas
    para implementar una aplicación. Las subclases están estrechamente
    acopladas a sus superclases. De esta forma se requiere más conocimiento
    acerca de la implementación de las clases que conforman el framework.
	
	\item Extensión o Modificación del núcleo del framework:  Consiste en extender
	el framework cambiando las clases abstractas que forman su núcleo para añadir
	nuevas variables u operaciones. Requiere conocimientos avanzados acerca del
	diseño del framework. Cambiar las clases abstractas puede provocar fallos en
	las clases concretas existentes. Este modo de utilización no es aplicable si
	el propósito es crear un sistema abierto.
\end{itemize}

Entre las formas de utilización mencionadas existen combinaciones intermedias. Es común
que los frameworks se utilicen como Black-Box la mayor parte del tiempo y
sean extendidos cuando la ocasión lo demande.

\section{Comparación entre Frameworks y APIs}

En la tabla \ref {tab:comparacion_frameworks_apis} se observa una comparación
entre frameworks y APIs.
\begin{table}
	\centering
	\begin{tabularx}{\textwidth}{ | p{2.5cm} | X | X | }
	\hline
	\textbf{Categoría} & \textbf{Framework} & \textbf{API (Biblioteca)} \\[10pt]
	\hline
    \textbf{Gestión del Flujo Principal} & El framework toma el
    flujo principal del software & A cargo del programador\\[10pt] \hline
    \textbf{Confiabilidad} & El flujo principal está ampliamente testeado por todos
    los usuarios del framework & No brinda ninguna garantía de flujo\\[10pt] \hline
	\hline \textbf{Extensibilidad} & Por parte de los desarrolladores del
	framework.
	Si es de código abierto cualquiera puede extenderlo. & Por parte del fabricante de la
	librería.
	O cualquiera si es de código abierto \\[10pt] \hline
	\textbf{Reusabilidad} & Objetivo principal del diseño de un framework.
	Se aplica a nivel de arquitectura de software & Es reutilizable a nivel
	de llamada a métodos\\[10pt] \hline 
	\textbf{Complejidad de Uso} & Gran
	complejidad al principio, se simplifica a medida que el usuario aprende el framework & Complejidad inicial menor que un
	framework\\[10pt] \hline 
	\textbf{Aplicación de Patrones de diseño} & Usualmente un framework fuerza al
	usuario a utilizar uno o varios patrones & No obliga al usuario a utilizar ningún patrón
	de diseño\\[10pt] \hline 
	\textbf{Especificidad / Generalización} & Son de uso específico,
	están diseñados para resolver una familia de problemas. Por esto mantienen una
	arquitectura & De uso general donde una funcionalidad pueda ser
	utilizada\\[10pt] \hline 
	\textbf{Restricciones de lenguaje} & Obliga al usuario a
	desarrollar en el mismo lenguaje en el que está hecho el framework & No restringe a un lenguaje.
	Permite llamadas desde cualquier lenguaje mientras se respete la firma de las funciones expuestas\\[10pt] 
	\hline
	\end{tabularx}
	\caption{Comparación entre Frameworks y APIs}
	\label{tab:comparacion_frameworks_apis}
\end{table}


        \chapter{Concurrencia}
            \section{Introducción}
\label{IntroduccionConcurrencia}

``La idea de programación concurrente siempre estuvo asociada al mundo de los
\textit{Sistemas Operativos}. No en vano, los primeros programas concurrentes
fueron los propios Sistemas Operativos de multiprogramación en los que un solo
procesador debía repartir su tiempo entre muchos usuarios.''\cite{PalmaConcurrente}

En las últimas dos décadas, la programación concurrente ganó gran interés y
actualmente está presente en la mayoría de las aplicaciones.
Esto se debe principalmente a algunos grandes hitos en la programación:
\begin{itemize}
	\item La aparición del concepto de \textit{hilo} o \textit{thread}. Permiten la
	ejecución de programas de manera más rápida y eficiente que los programas
	basados en procesos.
	\item La aparición de lenguajes de alto nivel con soporte para
	programación de hilos y de procesos.
	\item La aparición de internet, entorno donde la concurrencia se hace necesaria
	en todo aspecto.
	\item El desarrollo y gran avance de hardware capaz de ejecutar múltiples hilos
	y procesos de forma paralela. Esto permite aprovechar las ventajas de
	performance de la concurrencia. Las principales arquitecturas capaces de
	explotar el paralelismo a nivel de hilo y/o de proceso son
	\begin{itemize}
	    \item Procesadores Multi-Core
	    \item Procesadores Many-Core
	    \item Procesadores con soporte Multi-Thread
    \end{itemize}
\end{itemize}

\section{Programación Concurrente}
\label{ProgramacionConcurrente}

La \textit{programación concurrente} es la disciplina que se encarga del estudio
de las notaciones que permiten especificar la ejecución concurrente de las
acciones de un programa, así como resolver los problemas inherentes a la
ejecución concurrente (ver \ref{ProblemasConcurrencia}). Es de interés
formalizar el concepto de ejecución concurrente y de ejecución paralela a fin
de poder diferenciarlos:

\begin{itemize}
	\stepcounter{definitionsCounter}
	\item [\underline{Definición \thedefinitionsCounter :} ] Dos hilos
	\footnote{En adelante, se hablará de concurrencia de hilos dado que los
	sistemas de referencia son de memoria compartida. En el caso que corresponda
	hablar de procesos se lo mencionará explícitamente.} son \textit{concurrentes}
	si la primera instrucción de uno de ellos se ejecuta después de la primera del
	otro y antes de la última.
	\stepcounter{definitionsCounter}
	\item [\underline{Definición \thedefinitionsCounter :} ] Dos hilos 	se están
	ejecutando de manera \textit{paralela} si son concurrentes y la ejecución de
	ambos se da al mismo tiempo.
\end{itemize}

Para que dos hilos sean concurrentes no es necesario que se ejecuten al mismo
tiempo, es suficiente que exista un intercalado entre la ejecución de sus
instrucciones \cite{PalmaConcurrente}. En este proyecto integrador, es de
interés fundamentalmente la ejecución concurrente.

Anteriormente en esta sección se mencionó que existen problemas aparejados a la
programación de sistemas concurrentes. Sabiendo esto resulta necesario conocer
las ventajas de la programación concurrente, que justifiquen su uso por encima de
las dificultades que genera.

\subsection{Ventajas de la Programación Concurrente}

Los beneficios de programar de manera concurrente pueden englobarse en tres
categorías:

\begin{itemize}
	\item \underline{Incremento en la velocidad de ejecución:} Cuando se ejecuta un
	programa concurrente en un entorno multiprocesador, los distintos hilos que
	lo forman pueden ejecutarse de manera paralela, con lo que el tiempo total de
	ejecución se reduce. Esto es especialmente ventajoso en programas de cálculo
	numérico.
	\item \underline{Solución de problemas inherentemente concurrentes:} Existen
	problemas cuya naturaleza es concurrente, por lo que un modelo de programación
	de este tipo se adapta más naturalmente a la resolución de estos problemas.
	\item \underline{Mejor aprovechamiento del tiempo de CPU:} Un sistema operativo
	con un ambiente de multiprogramación que permita la concurrencia es capaz de
	desalojar a un hilo que está esperando por un evento y no está haciendo uso
	de la CPU para brindarle este tiempo a otro hilo que lo requiera.
\end{itemize}

\subsection{Problemas y Propiedades de la Concurrencia}
\label{ProblemasConcurrencia}

Como se introdujo en la sección \ref{ProgramacionConcurrente}, existen
problemas que aparecen al programar de manera concurrente. Esto lleva a que los
programas concurrentes deban satisfacer una serie de propiedades (además de su
especificación técnica del dominio del problema) para funcionar correctamente.

Estas propiedades se dividen en dos grupos:

\subsubsection*{Propiedades de Seguridad}
Las propiedades de seguridad aseguran que ``nada malo'' va a pasar en la
ejecución del programa \cite{PalmaConcurrente}.
Estas son:

\begin{itemize}
    \item \underline{Exclusión Mutua:} Existen recursos que no pueden ser
    accedidos concurrentemente para evitar problemas de coherencia. Por esto se
    debe garantizar que a lo sumo un hilo está accediendo a un recurso de este
    tipo en un instante dado.
    \item \underline{Condición de Sincronización:} Se pueden dar situaciones
    donde un hilo debe esperar la ocurrencia de un evento para poder
    continuar su flujo. Ante estos casos se debe garantizar que el hilo
    espere por dicha ocurrencia, de otro modo el resultado puede ser indefinido
    o inesperado.
    \item \underline{Interbloqueo \textit{(Deadlock)}:} Sucede cuando dos o más
    hilos están esperando a que suceda un evento que nunca ocurrirá para
    continuar sus flujos de ejecución. El evento no ocurre porque las
    condiciones para que suceda están bloqueadas por los propios hilos.\\
    Para que el interbloqueo suceda efectivamente se tienen que cumplir las
    siguientes condiciones:
    \begin{itemize} 
        \item Exclusión Mutua: si no se exige exlusión mutua, no puede haber
        interdependencia entre los hilos.
        \item Retención y Espera: cada hilo debe retener un recurso y esperar
        a que se libere otro.
        \item No Apropiación: no se puede forzar a un hilo a que desaloje un
        recurso
        \item Circulo Vicioso de Espera: Se forma una cadena cerrada de
        hilos, donde cada uno retiene al menos un recurso que necesita el
        próximo hilo de la cadena.
    \end{itemize}
    Las tres primeras condiciones son necesarias pero no suficientes para que
    efectivamente ocurra el interbloqueo. La cuarta condición nace como
    consecuencia de las tres primeras, siempre que se produzca una secuencia de
    eventos que desemboque en un círculo de espera irresoluble
    \cite{SistOpStallings}.
\end{itemize}

\subsubsection*{Propiedades de Vivacidad}

Si se aseguran las propiedades de vivacidad, ``algo bueno'' pasará eventualmente
en la ejecución del programa.

\begin{itemize}
    \item \underline{Interbloqueo Activo \textit{(Livelock)}:} Se produce un
    interbloqueo activo cuando un sistema ejecuta una serie de instrucciones sin
    hacer ningún progreso. Esto se da cuando $N$ hilos necesitan $N$ recursos
    y se los intercambian sin obtener nunca el conjunto completo.
    \item \underline{Inanición \textit{(Starvation)}:} Se da cuando al menos una
    parte del sistema nunca recibe los recursos necesarios para continuar, o
    demora lo suficiente en recibirlos como para que no sean útiles para obtener
    el resultado esperado. No es necesario que todo el sistema se bloquee para
    estar en una situación de inanición.
\end{itemize}

\section{Mecanismos de Sincronización}

A fin de garantizar el cumplimiento de las propiedades introducidas en la
sección \ref{ProblemasConcurrencia}, es necesario sincronizar la ejecución de
los hilos. De lo contrario, se puede caer en problemas de coherencia
y/o consistencia de datos, o corrupción.

\subsection{Cooperación vs Competencia}

La sincronización de hilos se puede implementar basada en dos principios
\cite{SistOpStallings}:
\begin{itemize}
    \item \underline{Cooperación:} Los hilos se comunican entre ellos para
    cooperar en la compartición de recursos. A su vez, existen dos tipos de
    cooperación:
    \begin{itemize}
        \item \underline{Cooperación por Compartición:} Los hilos interactúan
        para gestionar los recursos. No tienen conocimiento explícito de los demás.
        \item \underline{Cooperación por Comunicación:} Los hilos interactúan
        para gestionar los recursos mediante el paso explícito de mensajes entre
        ellos.
    \end{itemize}
    \item \underline{Competencia:} Los hilos compiten entre sí por los
    recursos. La gestión de los recursos se efectúa por otra entidad, como lo
    puede ser el sistema operativo.
\end{itemize}

\subsubsection{Cooperación por Compartición de Recursos}

Los hilos inteactúan entre ellos sin tener conocimiento explícito de la
existencia de los demás.

Existen regiones de almacenamiento de datos compartidas (espacios de memoria,
archivos, bases de datos, etc) que pueden ser leidas y escritas por múltiples
hilos.

Si bien un hilo no hace referencia a ningún otro, es conciente de que los
datos compartidos pueden ser accedidos y modificados por los demás. Por lo que
el conjunto debe cooperar para asegurar que los datos compartidos se gestionen
correctamente. Es responsabilidad de los mecanismos de control asegurar la
integridad de los datos compartidos \cite{SistOpStallings}.

Como los datos se almacenan en recursos compartidos, existen los problemas de
exclusión mutua, interbloqueo e inanición vistos en la sección
\ref{ProblemasConcurrencia}. La principal diferencia es que existen dos modos de
acceder a los datos: para \textit{lectura} y para \textit{escritura}. Únicamente
se debe asegurar la exclusión mutua para operaciones de escritura ya que sólo
estas pueden romper la \textit{coherencia} y \textit{consistencia} de los datos.

Un conjunto de datos son coherentes si independientemente de quién haya sido el
último escritor, cualquier lector obtiene el último conjunto de valores escrito.
Por otro lado, un dato es consistente si un lector obtiene un valor que fue
realmente escrito por un escritor y no un dato corrupto.

Algunos mecanismos para gestionar el uso de los datos compartidos son:
\begin{itemize}
    \item Semáforos: desarrollado en la sección \ref{semaforos}
    \item Monitores: desarrollado en la sección \ref{monitores}
\end{itemize}

\subsubsection{Cooperación por Comunicación entre Hilos o Procesos}

Cuando los hilos o procesos cooperan por comunicación, participan en alcanzar un
objetivo en común. La comunicación es una manera de sincronizar o coordinar las
distintas actividades.

La comunicación está formada por está formada por el emisor, el receptor, el
canal y el mensaje. El envío y recepción de mensajes son explícitos.
Las herramientas para este paso de mensajes están dadas por el lenguaje de
programación, alguna biblioteca o por el sistema operativo.

Al no haber compartición de datos entre los hilos o procesos, no es necesaria la
ejecución en exclusión mutua. Pese a esto, el interbloqueo y la inanición siguen
siendo problemas que pueden afectar a los hilos o proceos
\cite{SistOpStallings}.

\subsubsection{Competencia entre Hilos}

Los hilos no tienen forma de comunicarse entre ellos para gestionar los
recursos.

Si dos hilos desean acceder a un mismo recurso, el sistema operativo se lo
asignará a uno de ellos y el otro tendrá que esperar. Se debe garantizar:
\begin{itemize}
    \item La toma de los recursos en exlusión mutua.
    \item Correcta gestión de los recursos para evitar interbloqueos.
    \item La reactivación de los hilos bloqueados en un tiempo prudente a fin de
    evitar su inanición.
\end{itemize}
El control de la competencia involucra al sistema operativo
inevitablemente, porque es él quien asigna los recursos del sistema.
Además, los hilos deben ser capaces por sí mismos de expresar de algún
modo los requisitos de exclusión mutua, como puede ser bloqueando los
recursos antes de usarlos. Cualquier solución conlleva alguna ayuda del
sistema operativo, como la provisión del servicio de
bloqueo.\cite{SistOpStallings}

\begin{framed}
\textbf{Nota:}A los fines de este proyecto integrador sólo es de interés la
concurrencia basada en memoria compartida. Dentro de este modelo se destacan dos
mecanismos de sincronización por competencia: los \textit{semáforos} y los
\textit{monitores}.
\end{framed}

\subsection{Semáforos}
\label{semaforos}

Los semáforos fueron el primer mecanismo de sincronización de hilos por
cooperación. Fueron desarollados por E. Dijkstra en 1965 como mecanismos
eficientes y fiables para dar soporte a la cooperación de hilos en un sistema
operativo.

El principio en el que se basan es simple. Un conjunto de hilos pueden
cooperar utilizando señales, de manera que se pueda obligar a un hilo a
detener su ejecución en un punto específico hasta recibir una señal conocida.
La señalización está a cargo de los semáforos.

Para transmitir una señal sobre el semáforo $s$, el hilo $p$ debe ejecutar
$signal(s)$, y para recibir una señal de $s$, debe ejecutar $wait(s)$. Si la
señal no fue transmitida, $p$ se bloquea hasta recibir la señal.

Efectivamente, las operaciones sobre los semáforos son tres:
\begin{itemize}
    \item \underline{$init(sem\ s,\ uint\ n)$:} inicializa al semáforo $s$ con
    un entero positivo $n$.
    \item \underline{$wait(sem\ s)$:} decrementa el valor del semáforo. Si se
    hace negativo, el hilo que realiza la llamada se bloquea. También se la
    llama \textit{acquire}.
    \item \underline{$signal(sem\ s)$:} incrementa el valor del semáforo. Si
    había un hilo bloqueado por una llamada a $wait(s)$, se desbloquea. También
    se la llama \textit{release}.
\end{itemize}

Las llamadas a $signal(s)$ y $wait(s)$ son atómicas para asegurar la
modificación del contador del semáforo en eclusión mutua.

Los hilos que esperan una señal luego de bloquearse por una llamada a
$wait(s)$ deben hacerlo en una cola de espera. Esta cola implementa una política
que decide cuál hilo bloqueado se libera ante la llegada de una señal. El
caso más típico es FIFO, pero se puede implementar otro. Sea cual fuera la
política implementada, se debe asegurar que ningún hilo bloqueado sufrirá
inanición por ella.

Los semáforos descriptos hasta este punto son de tipo \textit{semáforo
general}.
Existe una versión más reducida que sólo puede adquirir valores $0$ y $1$ llamada
\textit{semáforo binario}. Los semáforos binarios son de implementación más
simple que los generales y se demuestra que tienen la misma potencia de
expresividad. \cite{SistOpStallings}

\subsection{Monitores}
\label{monitores}

Los semáforos son herramientas simples y potentes para la gestión de la
concurrencia. Permiten gestionar la ejecución en exclusión mutua y coordinar
hilos. El problema de los semáforos radica en que las operaciones signal(s)
y wait(s) están distribuidas por el código de todos los hilos que lo usan,
con lo que resulta muy difícil entender y predecir el efecto de una operación
sobre todos los hilos que dependen del mismo semáforo.

Para solucionar este problema, C. Hoare definió el concepto de monitor en su
artículo “Monitors: An Operating System Structuring Concept.” en 1974.

Los monitores, al igual que los semáforos, son herramientas de gestión de la
concurrencia entre hilos. Los hilos los usan para asegurar el acceso en
exclusión mutua a recursos y para sincronizar y comunicarse con otros
hilos.\\
El propósito de un monitor de concurrencia es centralizar la gestión de los
recursos compartidos en una sección del código del programa. De esta manera, la
responsabilidad de sincronizar a los hilos para evitar problemas de concurrencia
es enteramente del monitor y no de cada hilo que quiera acceder a un recurso.

Un monitor consiste en un grupo de datos y un conjunto de rutinas exportadas
(llamadas \textit{rutinas de entrada}). Estas rutinas realizan operaciones sobre
los datos. Los datos del monitor representan recursos compartidos para múltiples
hilos (ya sean de software o de hardware) y pueden ser modificados únicamente
dentro de las rutinas del monitor.

La forma que tiene un monitor de gestionar concurrencia es:
\begin{itemize}
    \item Asegurando la ejecución de sus rutinas en exclusión mutua.
    \item Gestión de los recursos de forma implícita o explícita
\end{itemize}

Para asegurar la ejecución en exclusión mutua, sólo se permite que un único
hilo pueda ejecutar una rutina del monitor a la vez. Este hilo recibe el
nombre de \textit{hilo activo}. El hilo activo bloquea la entrada al
monitor cuando ejecuta una rutina y la desbloquea cuando cede
\textit{voluntariamente} el control del monitor. Si otro hilo llama a una
rutina de entrada mientras el monitor está bloqueado, se bloquea en una cola de
entrada al monitor hasta que este pase a estado desbloqueado.

\subsubsection{Sincronización Explícita}
\label{monitor_sincronizacion_explicita}

En muchas ocasiones resulta necesario no sólo garantizar la exclusión mutua
dentro del monitor sino sincronizar hilos dentro de él para la correcta
gestión de los recursos compartidos (cola de cortesía). Para esto el monitor
representa los recursos con variables de condición (o colas de eventos)

\begin{framed}
\paragraph{Variables de condición:} Son variables especiales, sobre las que se
pueden realizar dos acciones:
\begin{itemize}
    \item delay: Suspende al hilo que la llama, a la espera de una señal.
    \item signal: Levanta el estado de suspensión de un hilo suspendido por
    una llamada a \textit{delay()} sobre ella. Si no hay ningún hilo
    suspendido no tiene efecto.
\end{itemize}
Si existe más de un hilo suspendido en una variable de condición cuando otro
hace una llamada a \textit{signal()}, el hilo a despertar será elegido
aplicando una política determinada. Esta política puede ser FIFO, por
prioridades por hilo, etc.
\end{framed}

El hilo activo del monitor puede suspenderse a sí mismo temporalmente bajo
una condición x ejecutando \textit{delay(x)}. Al suspenderse deja de ser el
hilo activo y se sitúa al final de la cola de la condición x, a la espera de
volver a entrar al monitor cuando la condición cambie. Previo a esto debe
desbloquear la entrada al monitor para no generar un interbloqueo con los demás
hilos que intentan acceder. Por otro lado, otro hilo que sea el activo
puede hacer una llamada a \textit{signal(x)} si detecta un cambio en \textit{x},
desbloqueando un hilo suspendido en su cola de condición asociada.

Es común asociar una variable de condición a una proposición lógica sobre el
estado de un recurso gestionado por el monitor, por ejemplo “El buffer A no
está lleno”. De esta manera esperar por esta condición equivale a esperar a que
el buffer A no esté lleno. Esta asociación suele ser implícita, es decir que le
da semántica al monitor pero no forma parte de su funcionamiento.

\subsubsection{Sincronización Implícita}

Como alternativa a la señalización manual, Hoare propone los monitores de
señalización automática. Este tipo de monitor elimina las variables de
condición modificando la directiva \textit{wait} para que reciba una proposición
lógica.

Un hilo que llame \textit{wait(prop)} se mantiene bloqueado mientras la
proposición \textit{prop} sea falsa. Cuando \textit{prop} cambie a ser
verdadera, los hilos que estén bloqueados se desbloquean automáticamente.
Un inconveniente con este mecanismo es que su implementación suele llevar a la
señalización repetida de muchos hilos, con los consecuentes cambios de contexto.

A los efectos de este proyecto integrador, es de mayor interés el monitor de
sincronización explícita. Por esto, los siguientes apartados al respecto son
referidos únicamente a este tipo de monitor.

\subsubsection{Estructura de un Monitor}
De forma general, un monitor de concurrencia de sincronización explícita está
compuesto por las siguientes partes:
\begin{itemize}
    \item Variables de condición
    \item Colas de condición
    \item Rutinas exportadas o de entrada
    \item Cola de entrada
    \item Cola de espera
    \item Cola de cortesía o del señalizador
\end{itemize}

En la figura \ref{fig:monitor01} se observa la estructura de un monitor de
concurrencia:

\begin{figure}[H]
  \centering
  \makebox[\textwidth][c]{
    \includegraphics[width=140mm]{Monitor}
  }
  \caption{Estructura de un Monitor de Concurrencia}
  \label{fig:monitor01}
\end{figure}

Aunque un hilo puede entrar al monitor llamando a cualquiera de sus
procedimientos expuestos, y puesto que se debe asegurar la ejecución en
exclusión mutua se puede considerar que existe un único punto de entrada al
monitor. De ahí que existe una única cola de entrada.

\subsubsection{Máquina de Estados de un Monitor}
Un monitor no es un proceso en sí mismo, por lo que no tiene un hilo de
ejecución. En su lugar, es ejecutado por los hilos de los procesos que llaman a
alguna de sus rutinas.

El estado del monitor, incluyendo si está o no bloqueado determina si un
hilo que intenta ejecutar una rutina de entrada puede continuar o si se bloquea.

Se puede representar el funcionamiento de un monitor por dos máquinas de
estado. La primera indica si el monitor está bloqueado o desbloqueado. La
segunda representa el estado de las colas del monitor.
\begin{itemize}
    \item Estados del Primer Autómata:
    \begin{itemize}
        \item Bloqueado: Un hilo está ejecutando una rutina del monitor
        \item Desbloqueado: No hay hilo activo en el monitor
    \end{itemize}
    \item Estados del Segundo Autómata: Las colas que influyen en el estado del
    monitor son las internas a este (las de condición, la de espera y la de
    cortesía). La cola de entrada no influye en el estado del monitor porque no
    refleja la situación interna del mismo.\\
    Como los estados que puede adquirir una cola son \textit{“vacía”} y
    \textit{“no vacía”}, los estados del segundo autómata son todas las
    combinaciones posibles de las tres colas internas del monitor en cada uno de
    sus estados.
\end{itemize}

\begin{figure}[H]
  \centering
  \includegraphics[width=100mm]{Primer_Automata_Monitor}
  \caption{Primer Autómata de un Monitor de Concurrencia}
  \label{fig:automata_monitor01}
\end{figure}

\begin{figure}[H]
  \centering
  \makebox[\textwidth][c]{
    \includegraphics[width=180mm]{Segundo_Automata_Monitor}
  }
  \caption{Segundo Autómata de un Monitor de Concurrencia}
  \label{fig:automata_monitor02}
\end{figure}

\begin{framed}
\textbf{Nota:} Como se explicó en la sección \ref{comparacion_rdp_automatas} se
puede obtener un único autómata a partir de estos dos, pero es de la opinión de los autores
que esto resultaría en una explicación más confusa.
\end{framed}

\subsubsection{Políticas de Desbloqueo de Hilos}
\label{politica_monitor}
El desbloqueo de un hilo suspendido en la cola de condición x debe ser hecho
por el hilo que produjo el cambio sobre esta condición. La siguiente acción a
realizar luego del desbloqueo dependerá del tipo de monitor en cuestión.
Se puede generar una clasificación de monitores basándose en el comportamiento
luego del desbloqueo de un hilo. A continuación se presentan los tipos de
monitores introducidos en \cite{PalmaConcurrente}

\begin{framed}
\textbf{Nota:} Para todos los siguientes casos, se considera que el hilo
\textit{A} desbloquea al hilo \textit{B} ejecutando \textit{signal(x)},
condición sobre la que \textit{B} se encuentra bloqueado inicialmente. Por lo
tanto, al comenzar cada párrafo, A está bloqueado en la cola de cortesía y B en
la de espera, a menos que se indique lo contrario.
\end{framed}

\paragraph{Desbloquear y continuar (Signal and Continue)}
Se desbloquea a \textit{A} de la cola de cortesía y continúa su ejecución dentro
del monitor, ya sea hasta terminar la llamada al procedimiento o hasta bloquearse
en una cola de condición. Una vez \textit{A} sale del monitor, \textit{B}
ejecuta la instrucción siguiente al \textit{delay(x)} que lo bloqueó. En este
punto, \textit{B} debe volver a verificar la condición que lo suspendió porque
no se puede garantizar que \textit{A} no la haya modificado luego de la llamada
a \textit{signal(x)}.

\paragraph{Retorno forzado}
Se desbloquea a \textit{A}, quien ejecuta una instrucción de salida del monitor
(\textit{return} o \textit{delay(n)}) justo después. De esta manera, no es
necesario que \textit{B} vuelva a comprobar su condición ya que la exclusión
mutua asegura que no fue modificada.


\paragraph{Desbloquear y esperar}
\textit{A} está en la cola de entrada del monitor en lugar de la de cortesía.\\
Se desbloquea a \textit{B} de la cola de espera para que continúe su ejecución
en el monitor. Este enfoque tiene la ventaja de que \textit{B} no necesita
comprobar su condición de bloqueo una vez desbloqueado, pero \textit{A} cede su
lugar en el monitor y debe volver a competir por el ingreso para poder terminar
su ejecución.


\paragraph{Desbloquear y espera urgente}
Esta política soluciona el problema de inequidad de \textit{Desbloquear y
Esperar}.\\
Se desbloquea a \textit{B}, pero \textit{A} se suspende en la cola de cortesía.
De esta manera, el desbloqueo de \textit{A} tendrá prioridad sobre cualquier
hilo que intente entrar al monitor.

\paragraph{Clasificación Generalizada de Políticas de Desbloqueo:}
En \cite{MonitorClassification} el autor hace un análisis más exhaustivo de las
posibilidades existentes para diseñar una política de desbloqueo de hilos.
Dadas las tres colas de donde se puede elegir un hilo para desbloquear se
plantea una prioridad para cada una, resultando en:
\begin{itemize}
    \item EP: prioridad de la cola de entrada (entry queue priority)
    \item WP: prioridad de la cola de espera (waiting queue priority)
    \item SP: prioridad de la cola de cortesía (signaler queue priority)
\end{itemize}
Asignando pesos relativos a las tres prioridades se llega a que existen 13
distintas posibilidades.

En la tabla \ref{tab:prioridades_monitores} se Enumeran las posibilidades. La
tercera columna se refiere a los monitores definidos en [Howard, J. "Proving
Monitors"]

\begin{table}[H]
\centering
\begin{tabular}{|c|c|c|}
\hline
 & Prioridades relativas & Monitor Tradicional Correspondiente \\ \hline
1 & $EP = WP = SP$ & Random\\ \hline
2 & $EP = WP < SP$ & Wait and Notify\\ \hline
3 & $EP = SP < WP$ & Signal and Wait\\ \hline
4 & $EP < WP = SP$ & \\ \hline
5 & $EP < WP < SP$ & Signal and Continue\\ \hline
6 & $EP < SP < WP$ & Signal and Urgent Wait\\ \hline
7 & $EP > WP = SP$ & RECHAZADO\\ \hline
8 & $EP = SP > WP$ & RECHAZADO\\ \hline
9 & $SP > EP > WP$ & RECHAZADO\\ \hline
10 & $EP = WP > SP$ & RECHAZADO\\ \hline
11 & $WP > EP > SP$ & RECHAZADO\\ \hline
12 & $EP > SP > WP$ & RECHAZADO\\ \hline
13 & $EP > WP > SP$ & RECHAZADO\\ \hline
\end{tabular}
\caption{Tipos de monitores según las prioridades relativas de sus colas}
\label{tab:prioridades_monitores}
\end{table}

Las propuestas 7 a 13 son rechazadas porque si la prioridad de entrada es mayor
que cualquiera de las otras dos, ante un flujo constante de hilos de
entrada, habría al menos una cola que nunca sería atendida, lo que lleva a
posible inanición de los hilos que esperan en ella.

\subsubsection{Uso de un Monitor}
En el diagrama \ref{fig:actividad_hilo_monitor} se describen las actividades
de un hilo que accede a un monitor.

\begin{figure}[H]
  \centering
  \makebox[\textwidth][c]{
    \includegraphics[width=180mm]{Actividad_Proceso_Monitor}
  }
  \caption{Diagrama de actividades UML de un hilo ejecutándo una rutina de
  un monitor}
  \label{fig:actividad_hilo_monitor}
\end{figure}

El diagrama de la figura \ref{fig:actividad_hilo_monitor} sugiere que un
hilo puede optar por uno de dos caminos al ejecutar una rutina del monitor:
tomar o devolver un recurso.

Existe otra opción que es tomar y devolver un recurso en la misma rutina. Este
caso no está especificado en el diagrama por simplicidad y por tratarse de una
superposición de los otros dos casos.

\subsubsection{Conclusión}
Como con los semáforos, es posible cometer errores en la sincronización de los
monitores. La ventaja que tienen los monitores sobre los semáforos es que todas
las funciones de sincronización están confinadas dentro del monitor. De este
modo, es más sencillo verificar que la sincronización se ha realizado
correctamente y detectar los fallos. Es más, una vez que un monitor está
correctamente programado, el acceso al recurso protegido es correcto para todos
los hilos. Con los semáforos, en cambio, el acceso al recurso es correcto
sólo si \textbf{todos los hilos} que acceden al recurso están correctamente
programados.\cite{SistOpStallings} Por otro lado, las políticas de desbloqueo
permiten especificar prioridades de ejecución para los hilos, priorizando ya
sea por orden de llegada o por algún otro criterio. Si a su vez, se construye
el monitor de forma modular para cambiar la política de manera simple, se puede
alterar la planificación de los hilos de acuerdo a cada caso, según sea
necesario. Si esto se intenta hacer utilizando semáforos, resultaría mucho más
difícil.
Por estas razones, un punto fuerte a favor de los monitores frente a los semáforos es la mantenibilidad del código.
Por otro lado, al ser el único punto del programa donde se toman decisiones, un monitor se convierte en un cuello de botella para el sistema. Este impacto se puede mitigar con una arquitectura de monitores jerárquicos.


    \part{Desarrollo}
        \chapter{Investigación}
            \label{cap:investigacion}
            \section{Introducción}
En este capítulo se describe el proceso de investigación realizado con el
objetivo de determinar las características de la herramienta de software a
desarrollar en este Proyecto Integrador.

\section{Objetivos de la Investigación}
Con la intención de definir la orientación del proyecto, se realizó
una primera definición aproximada del objetivo principal:\\
\emph{``El sistema a desarrollar es una herramienta para facilitar a
desarrolladores de software la utilización de una Red de Petri como lógica de su
producto.''}

En este sentido, se realizó un listado de herramientas candidatas para
cumplir con dicho objetivo:
\begin{itemize}
  \item Generación de código
  \item APIs
  \item Frameworks
\end{itemize}

Se desea que la herramienta resultante posea las siguientes condiciones:
\begin{itemize}
    \item La lógica del sistema debe quedar expresada en una Red de Petri.
    \item El grado de acoplamiento entre el código de usuario
    y la Red de Petri debe ser el mínimo.
    \item La arquitectura de la herramienta tiene que contemplar el uso de
    patrones de diseño en el código de usuario.
    \item El flujo de ejecución debe quedar definido por la herramienta.
    \item Determinar el grado de escalabilidad de los sistemas desarrollados
    utilizando la herramienta.
    \item La herramienta debe favorecer la mantenibilidad del código de usuario.
    \item En caso de ser posible, analizar experiencias previas de sistemas
    similares utilizando dichas herramientas.
\end{itemize}

\section{Desarrollo de la Investigación}
\label{sec:investigacion_desarrollo}
La investigación se basa en el análisis de generación de código, APIs y
Frameworks expuestas en el capítulo~\ref{generacion_frameworks_apis}. Además se
estudiaron las experiencias previas desarrolladas en \cite{codegen}, \cite{chimp} y
\cite{Bentivegna-Ludemann}:
\begin{itemize}
  \item En \cite{codegen} se propone una solución basada en la generación de
  código. Tras analizar los ejemplos de uso de la herramienta se pudieron
  detectar desventajas importantes. El código generado utiliza las interfaces
  del monitor directamente, provocando un alto grado de acoplamiento entre el
  software del usuario y la Red de Petri. Esto provoca que los sistemas
  desarrollados tienen reducida escalabilidad y mantenibilidad.
  \item En \cite{chimp} se propone una solución basada en el desarrollo de un
  framework como herramienta superadora a la generación de código. Los ejemplos
  de uso muestran un claro avance respecto a \cite{codegen}. Los
  sistemas desarrollados son mantenibles y la herramienta permite la
  utilización de patrones de diseño en las aplicaciónes.
  Sin embargo, la implementación presenta problemas de acoplamiento a la Red de
  Petri ya que carece de una capa de abstracción entre eventos físicos y eventos
  lógicos (ver sección~\ref{sec:sincronizacion_RdP_por_eventos}).
  \item En \cite{Bentivegna-Ludemann} se desarrolla un sistema dirigido por RdP
  y se utilizan las interfaces del monitor de Petri como una API o
  biblioteca de funciones. Si bien se logra obtener un sistema funcional, en la
  conclusión del proyecto sus autores expresan que es necesario un framework
  para desacoplar el código de la red. Además el flujo de ejecución es
  determinado por el desarrollador, aumentando el grado de acoplamiento.
\end{itemize}

\section{Conclusión de la Investigación}
Tras lo expuesto en la sección~\ref{sec:investigacion_desarrollo} se determinó
que la herramienta a desarrollar debe tener la forma de un Framework por las
siguientes razones:
\begin{itemize}
  \item Funciona como una capa de abstracción entre las acciones de software,
  los eventos lógicos de la red de Petri, los eventos físicos del mundo
  exterior al sistema, las secuencias de acciones de software, las políticas de
  prioridad y los estados de la Red de Petri.
  \item El flujo de control queda contenido dentro de la estructura del
  Framework.
  \item El desacoplamiento facilita la mantenibilidad y legibilidad del sistema.
\end{itemize} 

        \chapter{Requerimientos}
            \section{Requerimientos}

\begin{enumerate}
	\item El sistema debe delegar flujo de ejecución a un motor de
	petri.
	\begin{itemize}
		\item El sistema debe mapear transiciones de una red de petri a eventos
		especificados por los usuarios. Un evento puede ser equivalente a un conjunto de transiciones.
		\item Cuando un evento es desencadenado por el disparo de un conjunto de
		transiciones el sistema debe ejecutar todas las tareas que se encuentran registradas al evento.
		\item Cuando un suceso definido por el usuario ocurre, el sistema debe
		notificar todos los eventos asociados a este suceso al motor de petri.
		\item El sistema debe proveer una interface para que el usuario pueda
		suscribir sucesos, tareas y fines de tareas a eventos especificados por el usuario.
		\item El sistema debe proveer una interface para que el usuario pueda definir
		eventos.
		\item Cuando una tarea termina el sistema debe notificar al motor de petri
		acerca de todos los eventos asociados a la finalización de la tarea.
	\end{itemize}
	\item Para un usuario con conocimiento intermedio en Java y Redes de Petri, el
	framework puede aprender a usarse en una semana o menos.
	\begin{itemize}
	    \item La utilización del sistema puede incorporar como máximo diez
	    conceptos nuevos a aprender por un usuario con un nivel intermedio en Java
	    y redes de Petri.
	    \item El sistema debe ser acompañado con al menos dos ejemplos de uso en
	    los cuales se muestre al menos un 80\% de las funciones del mismo.
	\end{itemize}
	\item El sistema debe ser compatible con las versiones actuales de motores de
	Petri desarrollados en el Laboratorio de Arquitectura de Computadoras de la
	Facultad de Ciencias Exactas y Naturales de la Universidad Nacional de Córdoba.
	\begin{itemize}
	    \item El sistema debe proveer una interfaz para que el usuario ingrese un
	    archivo PNML con la descripción de una red de Petri.
	    \item El sistema puede instanciar un entorno de ejecución de redes de
	    Petri dado que el usuario ha ingresado un archivo PNML conteniendo la
	    descripción de la red y ha elegido el motor de Petri que desea usar.
	    \item El sistema debe utilizar la interface expuesta por el motor de
	    petri.
	\end{itemize}
	\item El sistema quiere tener una interfaz gráfica de usuario para inicializar
	un nuevo proyecto.
	\begin{itemize}
	    \item La interfaz de usuario quiere contener una pantalla 'PNML Loader'
	    	\begin{itemize}
	    	    \item La pantalla debe dejar al usuario buscar en su disco local y
	    	    elegir un archivo.
	    	    \item La pantalla debe dejar al usuario ingresar la dirección a un
	    	    archivo manualmente mediante la escritura con el teclado.
	    	    \item La pantalla debe permitir confirmar la elección de un archivo.
	    	    \item Si el usuario confirma un archivo y el archivo es un PNML válido
	    	    entonces puede usarse para configurar el entorno de ejecución de
	    	    Petri.
	    	    \item Si el usuario confirma un archivo y el archivo no es un PNML
	    	    válido entonces debe mostrarse un error en pantalla y el usuario debe
	    	    ser capaz de elegir otro archivo.
	    	\end{itemize}
	    \item La interfaz de usuario quiere contener una pantalla de creación de
	    eventos
	    \begin{itemize}
	    	    \item La pantalla debe dejar al usuario definir un evento y asociarlo
	    	    con una o más transiciones definidas en un archivo PNML cargado
	    	    previamente por el usuario.
	    	    \item La pantalla de creación de eventos quiere permitir guardar las
	    	    decisiones del usuario en un archivo.
	    	    \item La pantalla de creación de eventos quiere permitir al usuario
	    	    cargar configuraciones a partir de un archivo.
	    	    \item Si un archivo guardado previamente se selecciona para ser
	    	    cargado y su contenido tiene un formato inválido, la pantalla
	    	    quiere mostrar un texto de error especificando el problema y el
	    	    archivo no debe ser cargado.
	    	    \item Si un archivo guardado previamente se selecciona para ser
	    	    cargado y el contenido del archivo contiene uno omás eventos que
	    	    mapean a transiciones inexistentes, la pantalla quiere mostrar un
	    	    texto de error especificando el problema y sólo debe cargarse la
	    	    configuración de los eventos fuera de conflicto.
	    \end{itemize}
	     \item La interfaz de usuario quiere contener una pantalla de selección de
	     el motor de Petri.
	     \begin{itemize}
	         \item La pantalla debe permitir elegir entre un motor de Petri Java,
	         un motor de Petri en hardware o un motor de Petri en driver.
	         \item La pantalla debe comunicar la decisión del usuario para preparar
	         el entorno de ejecución de Petri de acuerdo al motor elegido.
	     \end{itemize}
	         
	\end{itemize}
\end{enumerate}
        \chapter{Monitor de Concurrencia con Redes de Petri}
        	\label{cap:petri_monitor}
            \newcommand{\javapetriconcurrencymonitor}{Java Petri Concurrency Monitor }

\section{\javapetriconcurrencymonitor}

\subsection{Introducción}

\javapetriconcurrencymonitor  es un monitor de concurrencia que ejecuta redes
de petri, hecho en lenguaje de programación java.

\subsection{Características Principales}
Entre las principales características de \javapetriconcurrencymonitor están:
\begin{itemize}
  \item Soporte para Redes de Petri:
  \begin{itemize}
    \item Plaza-Transición
    \item Temporales
  \end{itemize}
  
  \item Soporte para tipos de arcos;
  \begin{itemize}
    \item Normal
    \item Lector o de Prueba
    \item Inhibidor
    \item Reset
  \end{itemize}
  
  \item Soporte para guardas
  \item Soporte para transitiones automáticas
  \item Soporte para subscripción a eventos en transiciones informadas
  \item Soporte de políticas intercambiables y extensibles de prioridad de
  disparo de transiciones

\end{itemize}

\subsection{Implementación}

\subsection{Manual de Uso}
            \section{Diseño y Funcionamiento}

En primera instancia se analizó la reutilización y expansión de las
funcionalidades del monitor construido en el desarrollo de \cite{codegen}, y
reutilizado en \cite{chimp}. Este software no satisface los requerimientos, por
lo cual es necesario diseñarlo y construirlo completamente en este proyecto
integrador.

\subsection{Arquitectura de Alto Nivel}
\label{JPCM_arq_alto_nivel}

Según las clasificaciones vistas en las secciones
\ref{monitor_sincronizacion_explicita} y \ref{politica_monitor}, Java Petri
Cconcurrency Monitor (JPCM) es un monitor de sincronización explícita y aplica
una política de desbloqueo de hilos de retorno forzado. Teniendo en cuenta los
requerimientos expresados en el apartado anterior se determinan los principales
componentes de JPCM:
\begin{itemize}
  \item \underline{PetriNet:} contiene la información de la RdP a utilizar y la
  lógica de la ejecución
  \item \underline{PnmlParser + PetriNetFactory:} su responsabilidad es
  convertir la información del archivo que describe a la RdP en un formato
  ejecutable para JPCM.
  \item \underline{PetriMonitor:} es el monitor en sí mismo. Expone las
  interfaces de programación y tiene la responsabilidad de gestionar los hilos,
  para evitar la ejecución concurrente de la RdP.
  \item \underline{TransitionsPolicy:} representa la política de gestión de los
  recursos del monitor. No es la misma política analizada en la sección
  \ref{politica_monitor}. Esta política permite decidir cuál transición, de un
  conjunto dado, es la próxima a ser disparada.
\end{itemize}

La figura \ref{fig:JPCM_Arquitectura} es un diagrama de arquitectura de JPCM,
mostrando sus principales bloques e interacciones.

\begin{figure}[H]
  \centering
  \includegraphics[width=125mm]{JPCM_Arquitectura}
  \caption{Arquitectura de JPCM}
  \label{fig:JPCM_Arquitectura}
\end{figure}

\subsection{Gestión de los Recursos con RdP}
\label{JPCM_gestion_rec_rdp}
En la sección \ref{monitores} se analiza cómo se gestionan los recursos de un
monitor con variables de condición en un monitor de sincronización explícita.
En dicha sección se explicó cómo un hilo que desea tomar o devolver un recurso
debe señalizar a una variable de condición, y bloquearse en caso de no tener el
recurso disponible.

Por otro lado, en la sección \ref{POPN} se explica cómo una RdP representa
procesos. De esta manera, para ejecutar un proceso \textit{A}, un hilo debe
señalizar el comienzo y fin de la ejecución en la RdP disparando un subconjunto
de transiciones de inicio y otro subconjunto de transiciones de finalización.
 
De esta forma se establece una relación entre las operaciones de una variable de
condición y los disparos de una transición. Se logra obtener el mismo
comportamiento sobre los hilos que las acceden con la ayuda de colas auxiliares. 

El intento de disparo sobre una transición no sensibilizada es equivalente a una
llamada a \textit{signal()} sobre una variable de condición (ver sección
\ref{variables_condicion}).
De la misma manera, el disparo de una transición que resulta en la
sensibilización de otra, es equivalente a una llamada a \textit{signal()} sobre
una variable de condición.

Por inducción podemos decir que una variable de condición verdadera equivale a
disparar una transición y una variable de condición falsa equivale a no poder
disparar una transición.

A aprtir de esta semejanza queda en evidencia que las operaciones a realizar
dentro de un monitor que ejecuta una RdP consisten consisten del disparo de
transiciones para evolucionar el estado de los recursos y operaciones del
sistema modelado.

\subsection{Estructura Interna de JPCM}
Basándose en el diagrama de la figura \ref{fig:JPCM_Arquitectura}, se puede
dividir a JPCM en dos secciones:
\begin{itemize}
    \item Modelo: Tiene como eje central a la clase \textit{PetriNet}.
    Dentro de esta sección está el modelo a ejecutar. Contiene las matrices de
    la RdP (parte estática del modelo) y al vector de estado (parte dinámica).
    Define y expone el método de cambio de estado (disparo de una transición).
    \item Conducción: Tiene como eje central a la clase \textit{PetriMonitor}.
    Se encarga de conducir a los hilos que ejecutan el modelo.
    Incluye clases auxiliares para implementar las políticas del monitor (de
    transiciones y de colas).
\end{itemize}

\subsubsection{Sección Modelo}
En el diagrama de la figura \ref{fig:JPCM_PetriNet_Structure} se observan las
clases que componen a esta sección, sus relaciones y colaboraciones:

\begin{figure}[H]
  \hspace*{-3cm}
  \includegraphics[width=180mm]{JPCM_PetriNet_Structure_Simple}
  \caption{Diagrama de clases de la sección \textit{Modelo}}
  \label{fig:JPCM_PetriNet_Structure}
\end{figure}

Se observan las clases que modelan a los componentes de una RdP estructural
(ver sección \ref{def_formal_petri}) (\textit{Arc}, \textit{Transition},
\textit{Place}), la especializaciones concretas de \textit{PetriNet}
(\textit{PlaceTransitionPetriNet} para RdP plaza-transición y
\textit{TimedPetriNet} para RdP temporales), las clases \textit{PetriNetFactory}
y \textit{PnmlParser} analizadas en la sección \ref{JPCM_arq_alto_nivel} y la
especialización de \textit{PnmlParser} que comprende el dialecto PNML de TINA,
\textit{TinaPnmlParser}. Además se observan las relaciones entre estas clases
que colaboran entre sí.

\subsubsection{Sección Conducción}
En el diagrama de la figura \ref{fig:JPCM_PetriMonitor_Structure} se observan
las clases que componen a esta sección, sus relaciones y colaboraciones:

\begin{figure}[H]
  \hspace*{-2.5cm}
  \includegraphics[width=180mm]{JPCM_PetriMonitor_Structure_Simple}
  \caption{Diagrama de clases de la sección \textit{Conducción}}
  \label{fig:JPCM_PetriMonitor_Structure}
\end{figure}

Junto a la clase \textit{PetriMonitor} se puede ver por un lado la subsección
que aplica la política de transiciones. Esta subsección contiene la clase
abstracta \textit{TransitionsPolicy}, cuya responsabilidad es elegir cuál
transición disparar dado un conjunto de transiciones sensibilizadas.
Se proveen dos especializaciones: 
\begin{itemize}
  \item \textit{RandomPolicy}: Política aleatoria. Basa su decisión en un
  generador de números aleatorios.
  \item \textit{FirstInLinePolicy}: Política de primer transición en la lista.
  Elige siempre a la primer transición sensibilizada que encuentre en la lista.
\end{itemize}

Por otro lado, la clase \textit{PriorityBinaryLock} implementa la política de
colas por prioridad (ver sección \ref{JPCM_solucion_inv_prioridad}) y es
utilizada tanto para la cola de entrada como para las colas de condición.

Las colas de condición asociadas a las transiciones (ver sección
\ref{JPCM_gestion_rec_rdp}) están descriptas en la interfaz
\textit{VarCondQueue} e implementadas en la clase \textit{FairQueue}.


\subsection{Interfaces de Programación}

JPCM ofrece una interfaz de entrada al monitor y dos de salida en tiempo de
ejecución, y dos interfaces de carga de datos de configuración en tiempo de
inicialización.

En tiempo de ejecución, la interfaz de entrada es el método público 
\mint{java}|PetriMonitor.fireTransition()|. Este método permite disparar una
transición de la RdP dentro del monitor de concurrencia, ya sea utilizando el
objeto \textit{Transition} (transición) de la RdP, o el nombre de la misma.

Las interfaces salida son dos:
\begin{itemize}
  \item \underline{Retorno del disparo de una transición:} la finalización de
  ejecución del método \mint{java}|PetriMonitor.fireTransition()| asegura que el
  hilo que hizo la llamada, efectivamente disparó de la transición deseada
  (excepto en disparos no-perennes donde no se desea esta garantía, ver sección
  \ref{disparos_perennes})
  \item \underline{Informes de disparo:} ante el disparo de una
  transición informada, se emite un evento con información sobre la transición
  disparada. Todos los observadores suscritos a estos eventos lo recibirán.
\end{itemize}

En tiempo de inicialización se proveen dos interfaces al usuario:
\begin{itemize} 
  \item \underline{Carga de una RdP:} Se hace mediante un archivo descriptor
  en formato PNML. El bloque \textit{PnmlParser + PetriNetFactory} lo utiliza
  para generar una RdP ejecutable.
  \item \underline{Carga de una política personalizada:} Es opcional. Permite
  al usuario definir una política de transiciones que se ajuste al problema a
  resolver por su software.
\end{itemize}

Estas interfaces proveen al usuario de los mecanismos necesarios para
inicializar el monitor, y para luego utilizarlo.

\subsection{Inicialización de JPCM}
Los pasos para inicializar JPCM son:
\begin{itemize}
  \item Instanciar la clase \textit{PnmlParser} con la ruta al archivo de
  descripción de la RdP a utilizar
  \item Instanciar la clase \textit{PetriNetFactory} con el objeto
  \textit{PnmlParser}
  \item Utilizar el objeto \textit{PetriNetFactory} para generar una instancia
  de \textit{PetriNet}
  \item Instanciar una política (clases \textit{FirstInLinePolicy} o
  \textit{RandomPolicy} o la desarrollada por el usuario)
  \item Instanciar la clase \textit{PetriMonitor} con el objeto
  \textit{PetriNet} y el objeto de la política
  \item Crear los hilos que van a ejecutar las acciones del sistema
  \item Inicializar la RdP llamando a \mint{java}|PetriNet.initializePetriNet()|
  sobre la instancia de \textit{PetriNet}
  \item Lanzar los hilos creados anteriormente
  \item Evitar que el hilo principal termine antes que los hilos trabajadores
\end{itemize}
 
Luego de realizar todas estas tareas, el sistema se ejecuta guiado por la RdP.

\subsubsection{Generación de una RdP Ejecutable a Partir de un Archivo PNML}
Como se dijo anteriormente, JPCM requiere de un archivo PNML con la descripción
de la RdP a utilizar. Los pasos a seguir para obtener una RdP ejecutable a
partir del archivo PNML son los siguientes:
\begin{itemize}
  \item Una instancia de PnmlParser analiza el archivo para obtener los
  componentes de RdP contenidos en él (Plazas, Arcos y Transiciones)
  \item Con la información obtenida genera objetos de tipo \textit{Place},
  \textit{Transition} o \textit{Arc} dependiendo del caso
  \item Ordena los objetos generados en tres arrays, uno por cada tipo de
  componente y los empaqueta en una tupla
  \item Cuando se llama a \mint{java}|PetriNetFactory.makePetriNet()| sobre la
  instancia de \textit{PetriNetFactory}, ésta obtiene la tupla de arrays de
  componentes generada por \textit{PnmlParser}
  \item Con los objetos componentes de la RdP, \textit{PetriNetFactory} calcula
  y almacena las matrices de precedencia, pos-incidencia, incidencia, inhibición,
  reset y lectura, y el vector del marcado inicial
  \item Con los objetos componentes, las matrices de la RdP y el tipo de RdP a
  generar, crea una instancia de la subclase de \textit{PetriNet}
  correspondiente (\textit{TimedPetriNet} para RdP temporales y
  \textit{PlaceTransitionPetriNet} para RdP ordinarias)
\end{itemize}

El objeto resultante es una RdP ejecutable que se utiliza en conjunto con el
monitor.

\subsection{Disparo de una Transición en JPCM}
El disparo de una transición en el monitor mediante la ejecución del método
\mint{java}|PetriMonitor.fireTransition(t)| desencadena las siguientes acciones
sobre el hilo que realiza la llamada:

\begin{itemize}
  \item \underline{Verifica que la transición t no sea automática:} En cuyo caso
  falla con un error explicando la situación
  \item \underline{Verifica que la red esté inicializada:} En caso contrario
  falla con un error explicando la situación
  \item \underline{Toma el lock sobre la entrada del monitor:} Si no lo puede
  tomar se bloquea en la cola de entrada hasta poder tomarlo
  \item  \underline{Dispara la transición en la RdP:} Devuelve un código de
  estado indicando el resultado (ver sección \ref{codigos_de_estado_disparo})
  \item \underline{Maneja el resultado del disparo:} se analiza si se debe
  liberar el lock de entrada, disparar una transición automática, liberar un
  hilo bloqueado o bloquearse por una condición
  \item \underline{Libera el lock de entrada:} Sólo si es necesario. Algunas
  situaciones requieren que no se permita la entrada de un nuevo hilo, como la
  liberación de un hilo bloqueado en una cola de condición
\end{itemize}
 
La figura \ref{fig:JPCM_Fire_General} es un diagrama de secuencias donde se
muestra el flujo de un hilo que realiza un disparo de una transición:

\begin{figure}[H]
  \hspace*{-3cm}
  \includegraphics[width=190mm]{JPCM_Fire_General}
  \caption{Disparo de una transición}
  \label{fig:JPCM_Fire_General}
\end{figure}

\subsection*{Manejo del código de estado del disparo}
\label{codigos_de_estado_disparo}

Cuando se realiza el disparo efectivo de la transición en la Red de Petri, ésta
devuelve un código de estado, que es el resultado del disparo. Estos pueden
ser:
\begin{itemize}
  \item \textit{SUCCESS:} el disparo fue exitoso
  \item \textit{NOT\_ENABLED:} la transición no está sensbilizada
  \item \textit{TIMED\_BEFORE\_TIMESTAMP:} Sólo para transiciones
  temporales. El instante de disparo es anterior al \textit{instante menor de
  disparo} $(\alpha)$ de la transición (ver sección \ref{semantica_tiempo_debil})
  \item \textit{TIMED\_AFTER\_TIMESTAMP:} Sólo para transiciones
  temporales. El instante de disparo es posterior al \textit{instante mayor de
  disparo} $(\beta)$ de la transición (ver sección \ref{semantica_tiempo_debil})
\end{itemize}

\subsubsection*{Caso del disparo exitoso}
Ante un disparo exitoso, la llamada a \mint{java}|PetriNet.fire(Transition t)|
devuelve un código \textit{SUCCESS}. Una vez que esto ocurre, el hilo que
realizó el disparo debe:
\begin{itemize}
  \item Emitir un evento para todos los suscriptores (en el caso que la
  transición sea informada).
  \item Determinar si existen nuevas transiciones sensibilizadas producto del
  último disparo
      \begin{itemize}
        \item Si no existen, libera el lock de entrada y abandona el monitor
        \item Si existen:
            \begin{itemize}
                \item Selecciona a la próxima transición $t_{n}$ a ser
                disparada, de acuerdo con la política de transiciones.
                \item Si $t_{n}$ es de tipo \textit{automática} el disparo lo
                hace el mismo hilo, por lo que itera para realizar un nuevo
                disparo sobre dicha transición.
                \item Si $t_{n}$ es de tipo \textit{disparada} se debe
                señalizar al hilo que esté esperando en su cola de condición y
                abandonar el monitor inmediatamente sin liberar el lock de entrada.
            \end{itemize}
      \end{itemize}
\end{itemize}

En el diagrama de secuencias de la figura \ref{fig:JPCM_Fire_SUCCESS} se observa
en detalle el flujo de un disparo exitoso de una transición:

\begin{figure}[H]
  \centering
  \includegraphics[width=140mm]{JPCM_Fire_SUCCESS}
  \caption{Manejo del disparo exitoso de una transición}
  \label{fig:JPCM_Fire_SUCCESS}
\end{figure}

\subsubsection*{Caso del disparo no exitoso}

Cuando la transición no está sensibilizada, su disparo devuelve
\textit{NOT\_ENABLED}. Ante este caso, un hilo que realizó un disparo perenne
debe esperar en la cola de condición asociada a la transición. A su vez, si en
una iteración anterior realizó una espera temporal por la transición, debe
bloquearse con alta prioridad.
 
En el diagrama de secuencias de la figura \ref{fig:JPCM_Fire_NOT_ENABLED} se muestra
detallado el caso del disparo no exitoso.

\begin{figure}[H]
  \hspace*{-2cm}
  \includegraphics[width=170mm]{JPCM_Fire_NOT_ENABLED}
  \caption{Manejo del disparo no exitoso de una transición}
  \label{fig:JPCM_Fire_NOT_ENABLED}
\end{figure}

\subsubsection*{Caso del disparo temporal antes del intervalo}
Si un hilo intenta disparar una transición temporal antes del instante menor de
disparo, la llamada a \mint{java}|PetriNet.fire(t)| devuelve un código de estado
\textit{TIMED\_BEFORE\_TIMESPAN}.
En este caso, el procedimiento a seguir es el siguiente:
\begin{itemize}
  \item Si no hay hilo bloqueado por t:
      \begin{itemize}
        \item Liberar el lock de entrada
        \item Esperar temporalmente hasta el \textit{instante menor de disparo}
        \item Tomar el lock de entrada con alta prioridad (ver
        sección~\ref{sec:inversion_prioridad})
        \item Señalizar que ya se suspendió temporalmente para evitar futuras
        inversiones de prioridad
        \item Iterar sobre el disparo para reintentarlo
      \end{itemize}
  \item Si existe un hilo bloqueado por t:
      \begin{itemize}
        \item Si el disparo es no-perenne
                \begin{itemize}
                  \item Liberar el lock de entrada
                  \item Abandonar el monitor
                \end{itemize}
        \item Si el disparo es perenne
                \begin{itemize}
                  \item Liberar el lock de entrada
                  \item Bloquearse en la cola de condición de t. Si ya se suspendió
                  temporalmente por t antes, lo hace con alta prioridad (ver
                  sección~\ref{sec:inversion_prioridad})
                  \item Cuando se desbloquea itera para reintentar el disparo
                \end{itemize}
      \end{itemize}
\end{itemize}
En el diagrama de secuencias de la figura
\ref{fig:JPCM_Fire_TIMED_BEFORE_TIMESPAN} se detalla el flujo de este caso de
resultado de disparo.

\begin{figure}[H]
  \centering
  \includegraphics[width=140mm]{JPCM_Fire_TIMED_BEFORE_TIMESPAN}
  \caption{Manejo del disparo de una transición temporal antes del instante
  menor de disparo}
  \label{fig:JPCM_Fire_TIMED_BEFORE_TIMESPAN}
\end{figure}

\subsubsection*{Caso del disparo temporal después del intervalo}

Ante el intento de disparo de una transición temporal en un instante posterior
al instante mayor de disparo, el resultado es \textit{TIMED\_AFTER\_TIMESPAN}.
Al no cumplirse la condición de sensibilización temporal, este intento de
disparo es equivalente al caso del disparo no exitoso. Por esto, la rutina de
manejo de ambos casos es la misma y correponde al diagrama de secuencias de la
figura \ref{fig:JPCM_Fire_NOT_ENABLED}.

\subsection{Problema de la Inversión de Prioridades}
\label{sec:inversion_prioridad}
Durante el desarrollo de JPCM se detectaron dos casos donde se produce una
inversión de prioridades entre los hilos gestionados. A continuación se
analizan estos casos.

\subsubsection{Inversión de prioridades en la cola de entrada}
\label{inversion_prioridad_cola_entrada}
Existe inversión de prioridad en la cola de entrada si ocurren los siguientes
eventos en orden:
\begin{itemize}
  \item Se sensibiliza la transición temporal $t_{0}$ con intervalo $[a,b]$ en
  el instante $t_{i}$
  \item El hilo $th_{0}$ intenta disparar $t_{0}$ en $t_{d} < (t_{i} + a)$
  \item El hilo $th_{0}$ libera la entrada y se bloquea temporalmente
  \item Entra otro hilo a realizar un disparo y bloquea la entrada
  \item Llegan al monitor $N$ hilos a intentar realizar disparos y se encolan a
  la entrada
  \item Cuando $th_{0}$ se desbloquea, intenta tomar el lock de entrada y va al
  final de la cola
\end{itemize}

En este caso, $th_{0}$ cedió su prioridad a los $N$ hilos que llegaron después
que él. Esto provoca que el intervalo de disparo termine antes de que $th_{0}$
recupere su turno dentro del monitor, provocando su inanición.

La figura \ref{fig:Inv_prior_entrada} explica esta secuencia. Está formada por
una sucesión de subfiguras, una por cada instante de interés. Los elementos que
aparecen en todas ellas son:
\begin{itemize}
    \item El monitor de concurrencia (rectángulo mayor)
    \item La cola de entrada
    \item Las colas de condición
    \item Una Red de Petri con tres plazas y cuatro transiciones, donde $t_{0}$
    es temporal con intervalo $[\alpha,\beta]$
    \item Un reloj que indica en instante de cada subfigura
    \item Un área donde se simboliza el bloqueo temporal de un hilo (rectángulo
    redondeado con líneas de punto)
    \item Uno o más hilos. El hilo rojo es el de interés
\end{itemize}

Las subfiguras grafican los siguientes eventos:
\begin{enumerate}[label=\alph*)]
    \item La transición $t_{0}$ se sensibiliza en el instante $t_{i}$. Luego, el
    hilo $th_{0}$ llama a \mint{java}|petriMonitor.fireTransition("t0")|. 
    \item Como no hay hilo activo en el monitor y la cola de entrada está vacía,
    $th_{0}$ toma el lock de entrada e intenta disparar $t_{0}$. Mientras tanto
    se encolan algunos hilos en la cola de entrada
    \item La llamada a \mint{java}|petri.fire("t0")| retornó un código
    \mint{java}|TIMED_BEFORE_TIMESPAN|, por lo que $th_{0}$ se bloquea
    temporalmente hasta el instante $t_{i}+a$.
    \item Antes de bloquearse, $th_{0}$ libera la entrada al monitor. Otro hilo
    ingresa al monitor a disparar a $t_{1}$
    \item Se alcanza el instante $t_{i}+a$ por lo que $th_{0}$ se desbloquea y
    reintenta el disparo. Como la cola de entrada no está vacía, se encola al
    final
    \item Otro hilo ingresa al monitor a disparar a $t_{2}$
    \item Se superó el instante $t_{i}+b$ (fin del intervalo de disparo de
    $t_{0}$) sin que $th_{0}$ haya podido hacer el disparo, provocando su
    inanición
\end{enumerate}

\begin{figure}[H]
    \centering
    \ContinuedFloat
    \subfigure[] {
      \includegraphics[width=\textwidth]{InversionPrioridad/Inversion_Prioridad_01}
    }
    \phantomcaption
    \label{fig:Inv_prior_entrada}
\end{figure}
\begin{figure}[H]
    \ContinuedFloat
    \subfigure[] {
      \includegraphics[width=\textwidth]{InversionPrioridad/Inversion_Prioridad_02}
    }
    \label{fig:Inv_prior_entrada}
\end{figure}
\begin{figure}[H]
    \ContinuedFloat
    \subfigure[] {
      \includegraphics[width=\textwidth]{InversionPrioridad/Inversion_Prioridad_03}
    }
\end{figure}
\begin{figure}[H]
    \ContinuedFloat
    \subfigure[] {
      \includegraphics[width=\textwidth]{InversionPrioridad/Inversion_Prioridad_04}
    }
    \label{fig:Inv_prior_entrada}
\end{figure}
\begin{figure}[H]
    \ContinuedFloat
    \subfigure[] {
      \includegraphics[width=\textwidth]{InversionPrioridad/Inversion_Prioridad_05}
    }
\end{figure}
\begin{figure}[H]
    \ContinuedFloat
    \subfigure[] {
      \includegraphics[width=\textwidth]{InversionPrioridad/Inversion_Prioridad_06}
    }
    \label{fig:Inv_prior_entrada}
\end{figure}
\begin{figure}[H]
    \ContinuedFloat
    \subfigure[] {
      \includegraphics[width=\textwidth]{InversionPrioridad/Inversion_Prioridad_07}
    }
    \caption{Inversión de prioridades en la cola de entrada del monitor}
    \label{fig:Inv_prior_entrada}
\end{figure}

\subsubsection{Inversión de prioridades en una cola de condición}

Existe inversión de prioridad en la cola de condición de una transición si
ocurren los siguientes eventos en orden:
\begin{itemize}
  \item Se sensibiliza la transición temporal $t_{0}$ con intervalo $[a,b]$ en
  el instante $t_{i}$
  \item El hilo $th_{0}$ intenta disparar $t_{0}$ en $td < (t_{i} + a)$
  \item El hilo $th_{0}$ libera la entrada y se bloquea temporalmente
  \item Llegan $N$ hilos que intentan disparar a $t_{0}$ y se encolan en su cola
  de condición
  \item El hilo $th_{1}$ toma el lock de entrada y dispara la transición
  $t_{1}$, que deshabilita a $t_{0}$
  \item Cuando $th_{0}$ se desbloquea, toma el lock de entrada e intenta
  disparar $t_{0}$. Como la transición no está sensibilizada, $th_{0}$ se
  bloquea en la cola de condición de $t_{0}$
\end{itemize}
 
En este caso, $th_{0}$ se encola al final de la cola de condición, cediéndole
incorrectamente su prioridad a los $N$ hilos que llegaron después que él al
monitor a disparar a $t_{0}$.

\subsection{Solución a los problemas de inversión de prioridad}
\label{JPCM_solucion_inv_prioridad}

Del análisis de los casos de inversión de prioridades expuestos en la sección
anterior, se llega a la conclusión de que ambos problemas se solucionan mediante
una modificación de la política de prioridad de desbloqueo de hilos en las colas
de entrada y de condición. De esta manera, un hilo que espera por la
sensibilización temporal de una transición temporizada tendrá máxima prioridad
al momento de reintentar un disparo fallido.

Para implementar colas de prioridad, \mint{java}|PetriMonitor| utiliza una
instancia de la clase \mint{java}|PriorityBinaryLock| para la cola de entrada,
y \mint{java}|FairQueue| utiliza una para las colas de condición. (ver figura
\ref{fig:JPCM_PetriMonitor_Structure}).
Esta clase define un lock que implementa dos métodos básicos para su uso:
\begin{itemize}
  \item \mint{java}|lock(LockPriority priority)|: Su parámetro es opcional y por
  defecto es de baja prioridad. Permite a un hilo tomar el lock. En caso de ya
  estar tomado, el hilo se bloquea en la cola asociada.
  \item \mint{java}|unlock()|: Libera el lock. Si existe al menos un hilo en la
  cola despierta al primero.
\end{itemize}

Internamente, \mint{java}|PriorityBinaryLock| utiliza una cola de prioridades
para gestionar los hilos bloqueados que almacena. De esta forma, siendo $A$ y $B$
dos hilos bloqueados en la cola de prioridades, $A$ estará más próximo a ser
desbloqueado si:
\begin{itemize}
  \item $A$ tiene mayor prioridad que $B$
  \item $A$ y $B$ tienen la misma prioridad pero el instante de bloqueo de $A$
  es menor al de $B$
\end{itemize}

De esta manera, dentro de un mismo rango de prioridades se respeta el esquema
FIFO utilizado comúnmente en una cola.

El ordenamiento de la cola es equivalente a que existan dos colas FIFO, una con
prioridad alta y otra baja. Así, cada vez que se libere el lock, se prefiere
desbloquear a un hilo en la cola de alta prioridad.

En la figura \ref{fig:Inv_prior_entrada} se observa la secuencia de pasos que
lleva a la solución del caso de inversión de prioridades de la sección
\ref{inversion_prioridad_cola_entrada}.

Las subfiguras grafican los siguientes eventos:
\begin{enumerate}[label=\alph*)]
    \item La transición $t_{0}$ se sensibiliza en el instante $t_{i}$. El hilo
    $th_{0}$ llama a \mint{java}|petriMonitor.fireTransition("t0")|
    \item Como no hay hilo activo en el monitor y la cola de entrada está vacía,
    $th_{0}$ toma el lock de entrada e intenta disparar $t_{0}$. Mientras tanto
    se encolan algunos hilos en la cola de entrada
    \item La llamada a \mint{java}|petri.fire("t0")| retornó un código
    \mint{java}|TIMED_BEFORE_TIMESPAN|, por lo que $th_{0}$ se bloquea
    temporalmente hasta el instante $t_{i}+a$
    \item Antes de bloquearse, $th_{0}$ libera la entrada al monitor. Otro hilo
    ingresa al monitor a disparar a $t_{1}$
    \item Se alcanza el instante $t_{i}+a$ por lo que $th_{0}$ se desbloquea y
    reintenta el disparo. Como existe un hilo activo se bloquea con alta
    prioridad.
    \item El hilo activo abandona el monitor y libera la entrada. Se habilita a
    $th_{0}$ por estar en la cola de alta prioridad.
    \item $th_{0}$ ingresa al monitor dentro del intervalo de sensibilización de
    $t_{0}$ para reintentar el disparo
    \item $th_0$ logró disparar a $t_{0}$ dentro del intervalo de
    sensibilización y abandona el monitor, liberando la entrada para un nuevo
    hilo
\end{enumerate}

\begin{figure}[H]
    \centering
    \ResetCounter
    \ContinuedFloat
    \subfigure[] {
      \includegraphics[width=\textwidth]{InversionPrioridad/solucion/Solucion_Inversion_Prioridad_01}
    }
    \label{fig:Inv_prior_entrada}
    \phantomcaption
\end{figure}
\begin{figure}[H]
    \centering
    \ContinuedFloat
    \subfigure[] {
      \includegraphics[width=\textwidth]{InversionPrioridad/solucion/Solucion_Inversion_Prioridad_02}
    }
    \label{fig:Inv_prior_entrada}
    \phantomcaption
\end{figure}
\begin{figure}[H]
    \centering
    \ContinuedFloat
    \subfigure[] {
      \includegraphics[width=\textwidth]{InversionPrioridad/solucion/Solucion_Inversion_Prioridad_03}
    }
    \label{fig:Inv_prior_entrada}
    \phantomcaption
\end{figure}
\begin{figure}[H]
    \centering
    \ContinuedFloat
    \subfigure[] {
      \includegraphics[width=\textwidth]{InversionPrioridad/solucion/Solucion_Inversion_Prioridad_04}
    }
    \label{fig:Inv_prior_entrada}
\end{figure}
\begin{figure}[H]
    \centering
    \ContinuedFloat
    \subfigure[] {
      \includegraphics[width=\textwidth]{InversionPrioridad/solucion/Solucion_Inversion_Prioridad_05}
    }
    \label{fig:Inv_prior_entrada}
    \phantomcaption
\end{figure}
\begin{figure}[H]
    \centering
    \ContinuedFloat
    \subfigure[] {
      \includegraphics[width=\textwidth]{InversionPrioridad/solucion/Solucion_Inversion_Prioridad_06}
    }
    \label{fig:Inv_prior_entrada}
\end{figure}
\begin{figure}[H]
    \centering
    \ContinuedFloat
    \subfigure[] {
      \includegraphics[width=\textwidth]{InversionPrioridad/solucion/Solucion_Inversion_Prioridad_07}
    }
    \label{fig:Inv_prior_entrada}
    \phantomcaption
\end{figure}
\begin{figure}[H]
    \centering
    \ContinuedFloat
    \subfigure[] {
      \includegraphics[width=\textwidth]{InversionPrioridad/solucion/Solucion_Inversion_Prioridad_08}
    }
    \caption{Inversión de prioridades en la cola de entrada del monitor}
    \label{fig:Inv_prior_entrada}
\end{figure}

\subsection{Informes de Disparo de una Transición}

Existen casos en los que es de interés del programador de un sistema que
utilice a JPCM recibir una notificación cuando se produce el disparo de una
transición. Para esto, JPCM ofrece una interfaz de suscripción a eventos de
disparo de transición.

Como se puede observar en el diagrama de la figura \ref{fig:JPCM_Fire_SUCCESS},
luego de un disparo exitoso se envía un evento de disparo. Para que un
observador reciba este evento debe haberse suscrito previamente a los eventos de
la transición en cuestión. La suscripción se realiza con una llamada al método
\mint{java}|PetriMonitor.subscribeToTransition()|.

Un intento de suscripción a una transición que no sea de tipo \textit{informada}
resulta en el lanzamiento de una excepción \mint{java}|IllegalArgumentException|
con un mensaje que explica la situación. De la misma manera, no se realizan
envíos de eventos para transiciones que no sean de tipo \textit{informada}.

\subsection{Guardas}

Como se explicó en la sección \ref{guardas}, a una transición se pueden asociar
valores booleanos que modifican su semántica de sensibilización. JPCM provee
soporte para guardas con habilitación por valor \mint{java}|true| o por valor
\mint{java}|false|.

Las guardas son cargadas en tiempo de inicialización durante la construcción
del objeto \textit{PetriNet} y son inicializadas con valor \mint{java}|false|.

En cualquier momento un hilo puede modificar el valor de una guarda accediendo
al método \mint{java}|PetriMonitor.setGuard()| con el nombre de la guarda a
modificar y su nuevo valor. Esto es coordinado por el lock de entrada de forma
conjunta con el disparo de una transición para evitar problemas de acceso
concurrente sobre la RdP. Se contempla la posible sensibilización de
transiciones, ya sean de tipo \textit{automática} (donde el hilo que hizo el
cambio de valor de la guarda realiza el disparo) o de tipo \textit{disparada}
(en cuyo caso desbloquea a un hilo que estuviera esperando en su cola de
condición si lo hubiera) de la misma forma que en el disparo de una transición.

            \section{Manual de Uso}

\subsection{Formato del Archivo}

\javapetriconcurrencymonitor utiliza el formato estándar PNML para descripción
de redes de Petri, específicamente el dialecto del editor TINA\cite{TinaSite}.

Para más información sobre el estándar PNML, el sitio online de referencia del
lenguaje es \cite{PnmlSite}.

\begin{framed}
\paragraph{Nota:}
Como TINA no tiene soporte para arcos de reset, el usuario debe especificar
este tipo de conexión modificando el archivo PNML con un editor de texto.
Se recomienda utilizar otro tipo de arco en TINA y luego cambiar el valor del
campo \textit{``type''} del arco correspondiente en el archivo PNML.

Ej:
\begin{figure}[H]
\centering
\begin{minted}{xml}
<arc id="e-1" source="p-1" target="t-1">
  <type value="reset"/>
</arc>
\end{minted}
\end{figure}
\end{framed}

\subsection{Etiquetas}

Las etiquetas especifican atributos para una transición. En TINA, una etiqueta
se agrega a una transición como un atributo ``label''.
Dentro de las etiquetas existen tres atributos:

\begin{itemize}
    \item \underline{Automática:} Una transición automática no se puede disparar
    manualmente. En su lugar, se dispara automáticamente cuando la lógica de
    sensibilizado y la política lo indican.
    \item \underline{Informada:} Una transición informada acepta peticiones de
    suscripción a informes, y envía informes a sus suscriptores ante un disparo.
    Ver sección \ref{mensaje_eventos}
    \item \underline{Guarda:} Es el nombre de una guarda que afecta
    al sensibilizado de una transición. Ver sección \ref{guardas}
\end{itemize}

La sintaxis a utilizar es la siguiente

\begin{figure}[H]
\centering
\begin{minted}{perl}
<valor_automatica,valor_informada,(nombre_guarda)>
\end{minted}
\end{figure}

Donde:
\begin{itemize}
    \item \textbf{valor\_automatica} es un string no sensible a mayúsculas que
    puede adoptar valor \textit{``A''} para una transición automática, o
    \textit{``F''} o \textit{``D''} para una disparada.
    \item \textbf{valor\_informada} es un string no sensible a mayúsculas que
    puede adoptar valor \textit{``I''} para una transición informada, y
    \textit{``N''} para no informada.
    \item \textbf{nombre\_guarda} es el nombre de la guarda asociada a la
    transición etiquetada. Las guardas pueden estar compartidas por muchas
    transiciones. Se puede asociar por habilitación por \textit{false} si antes
    del nombre se agrega el símbolo $!$ ó $\mathtt{\sim}$.\\
    El nombre de una guarda debe respetar las limitaciones del nombramiento de
    variables de Java.
\end{itemize}

Un ejemplo de etiqueta es:

\begin{figure}[H]
\centering
\begin{minted}{perl}
<D,I,(!foobar)>
\end{minted}
\end{figure}

Esta guarda especifica que la transición etiquetada es \textit{disparada},
\textit{informada}, y que tiene una guarda asociada llamada \textit{foobar}, que
la habilita por valor \textbf{false}.

Todos los valores de las etiquetas son opcionales, pero para poder especificar
cualquiera, todos los valores a su izquierda deben estar explicitados. Los
valores por defecto son:
\begin{itemize}
    \item valor\_automatica: \textit{``D''}
    \item valor\_informada: \textit{``N''}
    \item nombre\_guarda: Ninguno
\end{itemize}

\subsection{Mensajes de Eventos}
\label{mensaje_eventos}

Cuando se dispara una transición informada, si existe al menos un observador
suscripto a sus eventos, se envía un mensaje de evento.\\
El mensaje de evento se envía en formato JSON y contiene información sobre la
transición disparada. La información es la siguiente:

\begin{itemize}
    \item \underline{Nombre de la transición:} Provisto por el usuario o
    asignado automáticamente por el editor TINA.
    \item \underline{Id de la transición:} Asignado automáticamente por TINA.
    \item \underline{Índice de la transición:} Índice correspondiente a la
    columna de la matriz de incidencia de la RdP para la transición disparada.
    Este índice se computa internamente y es útil para el depurado de la red.
\end{itemize}

El formato del mensaje es el siguiente:

\begin{figure}[H]
\centering
\begin{minted}{json}
{
  "name": "nombre_de_la_transición",
  "id": "id_de_la_transición",
  "index": "índice_de_la_transición"
}
\end{minted}
\end{figure}

\paragraph{Generar y Anular una Suscripción a Eventos:}
Para generar una suscripción a los eventos de una transición debe haber una
implementación concreta de \mint{java}|rx.Observer<String>|
(\url{http://reactivex.io/RxJava/javadoc/rx/Observer.html}), que en su
implementación de \mint{java}|onNext()| 
(\url{http://reactivex.io/RxJava/javadoc/rx/Observer.html#onNext(T)})
maneje el mensaje de evento descripto en esta sección.

A su vez, la suscripción puede hacerse con el objeto \textit{Transition} o
alternativamente con el nombre de la transición.

A modo de ejemplo se asume que existe una clase \textit{ConcreteObserver} que se
ajusta a las restricciones antes mencionadas, y se genera una suscripción por
objeto \textit{Transition} y otra por nombre de transición.

\begin{figure}[H]
\centering
\begin{minted}{java}
  // código
  
  // transición a la que se quiere suscribir
  Transition t0 = petri.getTransitions()[0]; 
  
  // otra transición a la que se quiere suscribir
  String t1Name = "start_process_01";
  
  Observer<String> observer = new ConcreteObserver();
  
  // suscripción mediante objeto Transition
  Subscription subscription0 = monitor.subscribeToTransition(t0, observer);
  
  //suscripción mediante nombre de la transición
  Subscription subscription1 = monitor.subscribeToTransition(t1Name, observer);
  
  // más código
\end{minted}
\end{figure}

La suscripción a un evento devuelve un objeto \mint{java}|Subscription|
(\url{http://reactivex.io/RxJava/javadoc/rx/Subscription.html}),
que es usado para cancelar la suscripción.
Una llamada a \mint{java}|subscription.unsubscribe()| cancela la suscripción.

\begin{framed}
\textbf{Nota:} La generación y cancelación de suscripciones utiliza la
biblioteca RxJava. Más información en \cite{RxJava} y \cite{RxJavaJavadoc}
\end{framed}

\subsection{Guardas}

Las guardas son variables booleanas asociadas a una o más transiciones (ver
sección \ref{guardas}).

Si una guarda está asociada a diferentes transiciones se la puede utilizar para
decidir un camino a tomar. En la figura \ref{fig:guarda_como_decision} se
muestra cómo la guarda \textit{``fooGuard''} puede decidir si disparar $t_{0}$ ó
$t_{1}$.

\begin{figure}[H]
  \centering
  \includegraphics[width=60mm]{Guarda_Como_Decision}
  \caption{Guarda como toma de una decisión.}
  \label{fig:guarda_como_decision}
\end{figure}

\begin{framed}
\textbf{Nota:} Todas las guardas comienzan con valor \mint{java}|false| al
principio de la ejecución del programa.
\end{framed}

\paragraph{Cambiar el valor de una Guarda:} En el siguiente fragmento de código
se muestra cómo cambiar el valor de una guarda \textit{fooGuard}.

\begin{figure}[H]
\centering
\begin{minted}{java}
  // código
  
  // ya fue declarado un PetriMonitor llamado "monitor"
  
  monitor.setGuard("fooGuard", true);
  
  // más código
  
  monitor.setGuard("fooGuard", false);
  
  // más código
\end{minted}
\end{figure}

\subsection{Inicialización del Monitor de Redes de Petri}

Antes de poder operar con el monitor, el usuario debe inicializar algunos
objetos con la información del modelo.

A modo de ejemplo, se asume que existe un archivo PNML que describe la RdP a
utilizar en la dirección ``/path/a/mi/red/de/petri.pnml''. El siguiente
fragmento de código inicializa  el entorno:

\begin{figure}[H]
\centering
\begin{minted}{java}
public void setUp() {
  PetriNetFactory factory = new PetriNetFactory("/path/a/mi/red/de/petri.pnml");
  PetriNet petri = (PlaceTransitionPetriNet) factory.makePetriNet(petriNetType.PT);
  TransitionsPolicy policy = new FirstInLinePolicy();
  PetriMonitor monitor = new PetriMonitor(petri, policy);
  
  // declarar e los hilos de trabajo aquí
  
  // se debe inicializar la RdP antes de poder utilizarla
  petri.initializePetriNet();
  
  // iniciar la ejecución de los hilos de trabajo
  
  // el hilo principal puede realizar otra tarea
  // mientras los hilos de trabajo se ejecutan
  // en este ejemplo, se imprime el marcado de la red cada 5 segundos
  
  while(true){
    try{
      Thread.sleep(5000);
      System.out.println(petri.getCurrentMarking());
    } catch (InterruptedException e){
    }
  }
}
\end{minted}
\end{figure}

\subsection{Disparo de una Transición}

El disparo de una transición se realiza en exclusión mutua dentro del monitor,
es decir de una instancia de la clase \textit{``PetriMonitor''}. La forma de
hacerlo es llamando a alguna de las alternativas del método
\mint{java}|PetriMonitor.fireTransition()|.

Existen dos formas de disparar una transición: mediante el objeto
\textit{Transition} o por nombre de la transición.
En el siguiente ejemplo se muestra la declaración de un hilo que dispara una
transición por cada una de las variantes.

\begin{figure}[H]
\centering
\begin{minted}{java}
  Thread worker = new Thread( new Runnable() {
  
    @Override
    public void run() {
      try {
      
        // sentencias a ejecutar fuera de la exclusión mutua

        // disparo por nombre de transición
        monitor.fireTransition("NombreDeUnaTransicion");

        // otras tareas

        // disparo por objeto Transition
        Transition t0 = petri.getTransition()[0];
        monitor.fireTransition(t0);
      
        // otras tareas
      
        // quizá disparar otra transición de ser necesario

      } catch (IllegalArgumentException | NotInitializedPetriNetException e) {
        // manejar las excepciones
      }
    }
  });
  worker.start();
\end{minted}
\end{figure}

\subsubsection*{Disparos Perennes}
\label{disparos_perennes}
Existe un segundo parámetro para el disparo de una transición: 
\mint{java}|boolean notPerennialFire|. Tiene valor por defecto
\mint{java}|false| e indica si el disparo a realizar es no-perenne.

Si se intenta disparar una transición no sensibilizada de forma perenne, el hilo
que intenta hacer el disparo se bloquea en la cola de condición asociada. Por
otro lado, si el disparo es no-perenne, el hilo que intenta hacer el disparo no
se bloquea y sale del monitor.

Los disparos no-perenne son útiles cuando el disparo equivale a una acción que
se intenta efectuar únicamente si están dadas las condiciones y no de otra
manera. Un ejemplo de esto es encender una luz; si la luz ya está encendida la
acción no termina, pero el resultado final es el esperado de todas formas.

En el caso del disparo no-perenne de una transición temporal, el hilo
llamador puede bloquearse únicamente si el intento de disparo ocurre antes del
principio del intervalo dinámico de la transición (ver sección
\ref{semantica_tiempo_debil}). Este tipo de bloqueo es temporizado y no requiere
de la activación por medio de otro hilo.

\begin{framed}
\textbf{Nota:} Como el disparo de una transición puede bloquear el hilo que lo
ejecuta, no se recomienda utilizar el hilo principal para realizar disparos
porque puede llevar al bloqueo del programa.
\end{framed}

\begin{framed}
\textbf{Nota:} El disparo explícito de una transición automática es un error
grave, que lleva a que se lance un error
\mint{java}|IllegalTransitionFiringError|.
\end{framed}

\subsection{Política de Transiciones}

Cuando cambia una condición de sensibilización (marcado de la red o valor de
una guarda), algunas transiciones pueden sensibilizarse. Si esto sucede, se
deben disparar las transiciones automáticas sensibilizadas y se debe señalizar a
los hilos que estaban esperando por una transición recientemente sensibilizada.

Si un cambio en las condiciones de sensibilización habilita una sola
transición es trivial cuál transición disparar o cuál hilo señalizar. De otro
modo, la \textit{política} a seguir puede generar o evitar conflictos entre las
transiciones a disparar.
La \textit{política de transiciones} es la que debe decidir qué transición tiene
la mayor prioridad entre el conjunto de todas las transiciones sensibilizadas.

\javapetriconcurrencymonitor incluye dos políticas por defecto. Estas son:
\begin{itemize}
    \item FirstInLinePolicy: Elige la primer transición sensibilizada del
    conjunto dado.
    \item RandomPolicy: Elige una transición de forma aleatoria de entre todas
    las sensibilizadas.
\end{itemize}

Se puede asignar una política de transiciones al monitor de dos formas:

\begin{itemize}
    \item Durante la inicialización, pasando un objeto instancia de
    \textit{TransitionsPolicy} al constructor de \textit{PetriMonitor}.
    \item En tiempo de ejecución, mediante el método\\ {
    \begin{minted}{java}
  PetriMonitor.setTransitionsPolicy(
    TransitionsPolicy _transitionsPolicy)
    \end{minted}
    }
\end{itemize}

\paragraph{Creación de una Política de Transiciones:}
Cualquier instancia de una clase que extienda la clase abstracta
\textit{TransitionsPolicy} puede ser usada como política de transiciones.
La forma de especificar la próxima transición a elegir sobre un conjunto es
implementando el método \mint{java}|public int which(boolean[] enabled)| donde:
\begin{itemize}
    \item El array booleano recibido por parámetro está ordenado por índice de
    las transiciones en la matriz de incidencia, y se corresponde con el orden
    del array de transiciones de \mint{java}|PetriNet.getTransitions()|. Una
    posición con valor \mint{java}|true| indica que la transición cuyo índice
    corresponde con la posición del array, está sensibilizada.
    \item El valor de retorno es el índice de la próxima transición a disparar.
\end{itemize}

En el siguiente ejemplo se muestra cómo definir una política de transiciones
para utilizar con el monitor de RdP. En este caso, se utiliza una clase anónima
de Java y se realiza en el momento de inicialización del monitor:

\begin{figure}[H]
\centering
\begin{minted}{java}
  // código
  
  // se asume que existe un objeto PetriMonitor llamado "petri"
  
  PetriMonitor monitor = new PetriMonitor(petri, new TransitionsPolicy(petri) {
  
    @Override
    public int which(boolean[] enabled) {
      int ret = 0;
      // se debe dar algún valor a "ret"
      // siguiendo el criterio de la política a aplicar
      return ret;
    }
    
  };
  
  // más código
\end{minted}
\end{figure}

En el próximo ejemplo, se muestra el cuerpo de una clase que implementa una
política estática, definiendo un órden de prioridad de las transiciones mediante sus
nombres:

\begin{figure}[H]
\centering
\begin{minted}{java}
public class OrderedPrioritiesPolicy extends TransitionsPolicy {

    private int[] priorityArray = {
        petri.getTransition("fin_proceso_01").getIndex(),
        petri.getTransition("fin_proceso_02").getIndex(),
        petri.getTransition("comienzo_proceso_01").getIndex(),
        petri.getTransition("comienzo_proceso_02").getIndex()
    };

    public OrderedPrioritiesPolicy(PetriNet _petri){
        super(_petri)
    }
    
    @Override
    public int which(boolean[] enabled) {
        for(int index : priorityArray) {
            if(enabled[index]) {
                return index;
            }
        }
        
        return -1;
    }
}
\end{minted}
\end{figure}

\begin{framed}
\textbf{Nota:} la implementación de una política de transiciones de forma
incorrecta puede llevar a situaciones de conflicto que terminen en la inanición
de uno o más hilos de ejecución.
\end{framed}

        \chapter{Diseño de \nombreFramework \ Framework}
        	\label{cap:diseno_framework}
        	\section{Introducción}
En este capítulo se detalla el diseño de \nombreFramework \ Framework.
En primer lugar se fundamenta la decisión de elaborar un Framework teniendo en
cuenta el análisis de experiencias previas. Se realiza una clasificación de los
eventos que se intercambian en sistemas reactivos desarrollados utilizando el
monitor de RdP.
Se detalla el diseño de la arquitectura del framework en base a dicho
intercambio de eventos.

Se realiza un análisis de las formas en que se pueden sincronizar las
acciones de un sistema utilizando un monitor de RdP. Este análisis tiene el
objetivo de definir el modo de sincronización más adecuado para la arquitectura
del framework.

Se define el concepto de controlador de acción y sus clasificaciones. A su vez,
se define el concepto de Guard Provider. Finalmente se define la relación entre
los eventos y los controladores de acción, formalizando el intercambio de
eventos entre el software de usuario y el framework.



            \section{Fundamentos del Framework}
En la sección~\ref{sec:petri_concurrency_monitor_intro} se propone utilizar RdP
como la lógica secuencial de un sistema concurrente. Para logralo, se
implementó el monitor de petri por software descripto en la
sección~\ref{sec:java_petri_concurrency_monitor}.
Este monitor permite delegar la concurrencia y asincronismo del
sistema a una red de Petri \cite{TesisMicolini}. Un ejemplo de uso exitoso se
describe en \cite{Bentivegna-Ludemann}.

La utilización directa del monitor es engorrosa y genera un
alto grado de acoplamiento entre el software de usuario y la red de Petri
puesto que los eventos de la red quedan asociados directamente a los eventos del
mundo real que modela.
La principal desventaja de un sistema acoplado a la red de Petri está dada por la
reducción de la escalabilidad del sistema. Esto se debe a que una modificación
de la lógica, que conlleva una sustitución de la red, implica también un cambio
en el código del software. En consecuencia, dificulta el proceso de desarrollo y
su mantenibilidad. Otra desventaja es que impide la reutilización de redes de
Petri genéricas, útiles para resolver diferentes problemas de
características similares.

Un requerimiento importante de este proyecto consiste en la facilidad de uso.
Como se explicó en el párrafo anterior, la utilización del monitor en forma
directa presenta una complejidad elevada y favorece a la generación de errores,
ya que deben crearse todos los hilos de ejecución y deben programarse los disparos de
transición de forma manual en el código. Este problema se manifiesta, por
ejemplo, en soluciones como CodeGen \cite{codegen}.
Ante un cambio en la Red de Petri deben modificarse algunos o todos los
disparos de transición distribuidos a lo largo del código. En caso de un error
en esta tarea, se genera una sincronización incorrecta de los hilos.

A su vez, para desacoplar las acciones que debe realizar el sistema de los
eventos, es necesario incorporar una entidad encargada de manejar y ejecutar
dichas acciones.

Como resultado de este análisis, se llegó a la conclusión de que es necesario
embeber el monitor de Petri en un framework que se encargue de desacoplar el
código de usuario de la lógica de disparos.
Una conclusión de similares características se desprende de
\cite{Bentivegna-Ludemann}, donde los autores expresan: ``La debilidad
encontrada en el proceso de elaboración del software, es que resultó
problemático vincular los hilos con las transiciones de la RdP. Esto se debe a
que entre las acciones y las transiciones no existe una capa de abstracción
para mapear las mismas.
Por lo cual, queda en evidencia que es necesaria la existencia de un framework
para automatizar y facilitar la vinculación entre eventos, acciones y
transiciones.''


\section{Sincronización por Red de Petri a través de Eventos}
\label{sec:sincronizacion_RdP_por_eventos}
Los sistemas a desarrollar utilizando el monitor descripto en la
sección~\ref{sec:java_petri_concurrency_monitor}, son programas de software que
intercambian eventos con la red de Petri y con su entorno físico.

\begin{figure}[h]
	\centering
	\includegraphics[width=75mm]{eventos_petri-programa-mundo}
	\caption{Intercambio de eventos en un programa sincronizado por Red de Petri}
	\label{fig:eventos_petri-programa-mundo}
\end{figure}

El programa tiene la posibilidad de acceder a hardware del mundo físico, ya sea
para realizar una acción (por ejemplo utilizando actuadores) o  para obtener
eventos del mundo exterior y comunicarlos a la red de Petri (por ejemplo con
sensores).

La red toma los eventos del mundo exterior y, dependiendo de las condiciones del
problema y del estado global, calcula los eventos hacia el programa.
La red es un procesador de eventos. \cite{TesisMicolini}\cite{chimp}

Este concepto se amplía en la sección Eventos Físicos y Eventos
Lógicos de \cite{chimp}. En esta sección se distingue la existencia de los dos
tipos de eventos mencionados, y se los define como:
  \begin{itemize}
    \item Eventos Lógicos: eventos que son comprensibles por el monitor de
    redes de Petri, y están inherentemente asociados a transiciones de la red
    misma y sus colas.
    \item Eventos Físicos: suceden en el mundo físico y representan sucesos del
    dominio del problema. Están conectados con el software.
  \end{itemize}

Tras la incorporación del concepto de eventos lógicos y físicos, en \cite{chimp}
se propone un diagrama de arquitectura de alto nivel como el de la
Figura~\ref{fig:eventos_fisicos-logicos}.

\begin{figure}[h]
	\centering
	\includegraphics[width=75mm]{eventos_fisicos-logicos}
	\caption{Arquitectura con Eventos Físicos y Lógicos}
	\label{fig:eventos_fisicos-logicos}
\end{figure}

Los eventos físicos representados en la
Figura~\ref{fig:eventos_fisicos-logicos} abarcan dos tipos de eventos. Uno de
ellos está efectivamente relacionado con el hardware o software externos al
sistema (mundo físico). El otro tipo de eventos está relacionado con el manejo
de las acciones que van a realizar dichos elementos del mundo exterior, el cual
se ejecuta a través del software del sistema.
Por este motivo se determinó que la clasificación de los
eventos en dos tipos no es suficiente para explicar la comunicación en un
sistema de estas características. 

A partir de lo expuesto en el párrafo previo, se modifica la clasificación
de eventos existente:

\begin{itemize}
    \item Eventos Lógicos: Conserva la definición descripta previamente. Este
    tipo de eventos se comunica utilizando las interfaces proporcionadas por el
    monitor de redes de Petri.
    \item Eventos Físicos: suceden en el mundo físico y representan sucesos del
    dominio del problema. Este tipo de eventos se comunican a través de las
    interfaces que expone el elemento del mundo exterior y las interfaces que
    ofrece el lenguaje de programación utilizado para desarrollar el software.
    Por ejemplo pueden comunicarse a través de Event Listeners, mecanismos
    de IPC, Sockets, Serial, Bluetooth, etc.
    \item Eventos de Acción: Evento intermedio
    entre los eventos físicos y lógicos. Este tipo de eventos es manejado
    por el framework durante la inicialización del sistema. Cumplen una función
    de abstracción entre las acciones de software y los eventos lógicos,
    necesaria para desacoplar la red de Petri del software y permitir la
    inversión de control descripta en la sección~\ref{sec:inversion_control}.
\end{itemize}

A partir de la nueva clasificación de los eventos del sistema emerge una nueva
arquitectura de alto nivel.

\subsection{Arquitectura de alto nivel de \nombreFramework}
\label{sec:arquitectura_alto_nivel}
Un sistema informático consiste en una secuencia de acciones que se ejecutan
ante el cumplimiento de determinadas condiciones. Desde el punto de vista del
programa que analiza dichas condiciones, se las clasifica en síncronas y asíncronas.
  \begin{itemize}
	\item Síncronas: Están sicronizadas con la ejecución del programa. Se
	desencadenan en un momento específico, conocido de antemano.
	Por ejemplo, condiciones booleanas derivadas del estado del sistema que
	realizan cambios en el flujo de instrucciones del mismo (saltos
	condicionales).
	\item Asíncronas: Se desencadenan en cualquier momento, de forma independiente
	a la ejecución del programa. Por ejemplo, eventos provenientes del mundo
	exterior o mensajes entre procesos.
  \end{itemize}

El objetivo de este trabajo es implementar un framework dirigido por redes de
Petri para controlar la ejecución de todas aquellas acciones que respondan a
estos tipos de condiciones.
De esta forma, será el monitor de redes de Petri quien analice las condiciones y
explicite el estado del sistema. Así, es responsabilidad de la red:
\begin{itemize}
  \item Disparo de eventos provenientes de sistemas externos, que pueden llegar
  en cualquier instante de tiempo y sin un orden preestablecido. Los estados
  locales de la red se mantienen en causalidad de las acciones ejecutadas con
  anterioridad. El monitor es el encargado de mantener el estado lógico del
  sistema.
  
  \item Condiciones de sincronización para el ordenamiento de la ejecución de
  acciones en el tiempo. 
  
  \item Condiciones impuestas por el dominio del problema. El monitor es el
  encargado de impedir la ejecución de una acción hasta que la misma pueda ser
  realizada sin riesgos.
\end{itemize}

De acuerdo a lo estudiado en la sección~\ref{sec:inversion_control}, una
característica principal de un framework es la inversión de control. Por este
motivo, el diseño de la arquitectura del framework contempla el control del
flujo de ejecución del código de usuario.

\begin{figure}[H]
	\centering
	\includegraphics[width=120mm]{arquitectura_framework}
	\caption{Diagrama de Arquitectura de Alto Nivel}
	\label{fig:arquitectura_petri-manejador-acciones-mundo}
\end{figure}

Como puede apreciarse en la
Figura~\ref{fig:arquitectura_petri-manejador-acciones-mundo}, \nombreFramework
Framework puede dividirse en tres partes a nivel de arquitectura:
\begin{enumerate}
  \item Un conjunto de módulos para la suscripción a eventos de acción (ver
  sección 1) encargado de:
  \begin{itemize}
    \item Ofrecer interfaces para definir los eventos de acción del sistema.
    \item Ofrecer interfaces para suscribir las acciones del sistema a los
    eventos de acción correspondientes.
    \item Ofrecer interfaces para definir las reglas de traducción entre eventos
    de acción y eventos lógicos.
    \item Encapsular las acciones junto a los eventos lógicos necesarios para su
    sincronización.
  \end{itemize}
\item Un monitor de redes de Petri (ver sección 2) cuyas responsabilidades son:
	\begin{itemize}
	  \item Brindar las interfaces para la incorporación del modelo de Red de Petri
	  del sistema, el cual contiene la definición de los eventos
	  lógicos.
	  \item Garantizar el cumplimiento de las condiciones de sincronización y
	  exclusión mutua de la concatenación de acciones.
	\end{itemize}
\item Un sistema ejecutor de las acciones definidas por el usuario (ver sección 3). 
  Se encarga de:
  \begin{itemize}
	  \item Intercambiar eventos lógicos con el monitor para asegurar la
	  sincronización y exclusión mutua en la ejecución de las acciones.
	  \item Ejecutar las acciones cuando las condiciones de ejecución
	  estén dadas.
	  \item Intercambiar eventos lógicos con el monitor para informar acerca de la
	  finalización de la ejecución de una acción.
	\end{itemize}
\end{enumerate}

Por su parte, el programa de usuario puede dividirse en:
\begin{enumerate}
  \item Una Red de Petri conteniendo el modelo de la lógica del sistema.
  \item Tópicos que contienen la definición de los eventos de acción y sus
  reglas de traducción a eventos lógicos.
  \item El código de usuario. Contiene todas las acciones de software
	concretas a realizar, con sus correspondientes suscripciones a eventos
	de acción. Dichas acciones pueden comunicarse con el mundo exterior. Por esta
	razón, \textbf{\emph{el manejo de los eventos físicos es responsabilidad del
	usuario}}.
\end{enumerate}

\begin{framed}
\textbf{Notas:} 
\begin{itemize}
\item Los controladores de acciones se ejecutan cuando el monitor de red de
Petri lo dispone. El monitor es el encargado de bloquear o liberar los hilos de una
acción de acuerdo a las condiciones de sincronizacion. Por otro lado, cuando una
acción finaliza, el sistema de ejecución se encarga de dar aviso al monitor.

\item La definición y el desarrollo de las acciones de software, y su
asociación a los eventos de acción correspondientes quedan a cargo del usuario
desarrollador. 

\item El usuario no decide en qué momento se ejecuta la acción, ya que
con el fin de lograr la inversión de control, dicha responsabilidad es otorgada
al monitor de redes de Petri.
\end{itemize}
\end{framed}
            \section{Modos de Sincronización de Acciones utilizando Redes de Petri}
\label{sec:sincronizacion_cinta_transportadora}
En esta sección se analizan dos formas de llevar la sincronización de las
acciones mediante la utilización las interfaces proporcionadas por el monitor de
Redes de Petri. Estos modos se denominan:
\begin{enumerate}
  \item Sincronización por aviso de ejecución.
  \item Sincronización por petición de ejecución.
\end{enumerate}

Para estudiar los modos de sincronización mencionados se hará uso de un caso de
ejemplo, descripto a continuación:

\begin{labeling}{description}
\item [Ejemplo]
Cinta transportadora con 3 estaciones. Piezas son depositadas en la primer
estación de manera asincrónica. Cuando esto sucede, la cinta avanza a la
estación 1, donde un operario realiza una transformación a la pieza. Una vez el
operario realizó la transformación, presiona un pulsador y la cinta avanza a la
estación 2, donde el mismo operario empaqueta la pieza. El operario
presiona otro pulsador al finalizar su tarea y luego la cinta avanza una vez
más y la pieza cae en un contenedor. Una vez que la pieza llega al contenedor se
habilita el ingreso de una nueva pieza al proceso.
\end{labeling}

\begin{figure}[H]
    \centering
    \includegraphics[height=100mm]{Petri_Cinta_Transportadora_1}
    \caption{Red de Petri de una cinta transportadora}
    \label{fig:petri_cinta_transportadora_1}
\end{figure}

\subsection{Análisis de ejecución del caso de estudio, utilizando
sincronización por aviso de ejecución} 
De acuerdo al modo de ejecución por aviso, es el resto del framework el
responsable de dar aviso de eventos al monitor, desencadenando la ejecución de las acciones:
\begin{enumerate}
    \item Debe insertarse un evento en la cola de entrada de “t0” cuando la
    acción escuchando el sensor detecte la llegada de una nueva pieza. Si la
    cinta Transportadora se encuentra disponible, el monitor de petri dispara
    “t0” y se genera un evento que se deposita en la cola de salida de “t0”.
    \item Un manejador de eventos lee el evento de salida de “t0” y llama a
    ejecutar la acción “moverEst1”, que mueve la pieza a la estación 1 y espera
    el trabajo del operador. Una vez que el operador realiza su trabajo,
    presiona el pulsador generando un evento que el manejador de eventos, a través del
    mapa de eventos envía a la cola de entrada de “t1”. El monitor de petri
    dispara “t1” y se genera un evento que se deposita en la cola de salida de
    “t1”.
    \item El manejador de eventos lee el evento de salida de “t1” y llama a
    ejecutar la acción “moverEst2”, que mueve la pieza a la estación 2 y espera
    el trabajo del operador. Una vez que el operador realiza su trabajo,
    presiona el pulsador generando un evento que el framework envía a la cola
    de  entrada de “t2”. El monitor de petri dispara “t1” y se genera un evento
    que se deposita en la cola de salida de “t2”.
    \item El manejador de eventos lee el evento de salida de “t2 y llama a
    ejecutar la accion “moverACont”, que mueve la pieza al contenedor. Una vez
    terminada esa acción envía un evento a la cola de entrada de “t3”. El
    monitor de petri dispara “t3” y libera la cinta Transportadora para procesar otra pieza.
\end{enumerate}

Se realizó un análisis del modo de sincronización por aviso de ejecución, y se
llegó a la conclusión de que presenta una gran desventaja.
Al utilizar la suscripción a transiciones informadas el módulo
encargado de manejar los eventos provenientes de la red de petri asume
una parte del control de la lógica de ejecución del sistema.
De acuerdo a los objetivos de este trabajo, esta responsabilidad debe pertenecer
por completo al monitor de redes de Petri. Sin embargo, en este caso es el
manejador de eventos quien toma decisiones sobre el bloqueo o liberación de los
hilos a partir del informe recibido desde la red. En el caso
particular del framework realizado en~\cite{chimp}, el uso de la sincronización
por avisos de ejecución lleva a que el bloqueo y liberación de hilos se realice
fuera de la estructura del monitor, desaprovechando una de las principales
ventajas de la arquitectura del sistema.
\\

\subsection{Análisis de ejecución del caso de estudio, utilizando
sincronización por petición de ejecución}
\label{sec:sincronizacion_peticion_ejecucion}
 El modo de sincronización por petición de ejecución es una alternativa
 propuesta al modo por aviso de ejecución.
 Consiste en que los hilos que ejecutan acciones realicen una petición de ejecución al
 monitor, sin tener en cuenta el estado actual de la red de Petri.
 El monitor es el encargado de bloquear aquellos hilos cuyas acciones no
 pueden ser ejecutadas en el momento de la petición del permiso ejecución. Una
 vez que las condiciones son las adecuadas para realizar la acción, el monitor
 se encarga de liberar al hilo encargado de ejecutarla.
 De esta forma, el manejo de la concurrencia del sistema es realizado
 íntegramente dentro del monitor. El sistema de ejecución basado en
 peticiones es más adecuado para una arquitectura controlada por monitor. 

\begin{enumerate}
    \item Se generan eventos que se encolan en la cola de entrada en “t0, t1,
    	t2 y t3”.
    \item El monitor bloquea los hilos que generaron eventos para “t1, t2 y t3”
    	por no estar sensibilizadas las transiciones en ese momento.
    \item El monitor ejecuta “t0”. Y se envía un evento a la cola de salida de
    	“t0”, liberando al hilo durmiendo en dicha transición.
    \item El hilo ejecuta “moverEst1”.
    \item Existe un problema, ya que al disparar “t0”, el monitor tiene
    	permitido disparar “t1”, pero la operación “moverEst1” aun no ha
    	finalizado, generando un problema de sincronización.
\end{enumerate}

Dada la red de petri de la Figura~\ref{fig:petri_cinta_transportadora_1}
surgen problemas de sincronización. Por ejemplo, uno de estos problemas
tiene origen al iniciar la acción ``moverEst1'' cuando existe una petición
de ejecución de la acción “moverEst2”. En este caso el monitor otorga el permiso
de ejecución de “moverEst2” de forma inmediata, sin tener en cuenta si
``moverEst1'' ha finalizado.

Del análisis de este caso se desprenden las siguientes conclusiones:
\begin {itemize}
  \item La petición de ejecución permite concentrar el control del flujo de
  	ejecución en el monitor.
  \item Es necesario que el framework de aviso al
	monitor de un evento de finalización de acción, cuando existen otras partes de
	la red que dependen de este evento.
\end{itemize}

En consecuencia, utilizar un sistema de ejecución basado en peticiones requiere
un nuevo modelo en red de Petri del problema, que sea capaz de sincronizar los
eventos de finalización de acción. A continuación se procede a estudiar tres
formas de sincronizar dichos eventos:
\begin{enumerate}
  \item Evento de finalización de acción por grupo transición-plaza
  \item Evento de finalización de acción por guardas
  \item Evento de finalización de acción por cola de condición de disparo no
  perenne.
\end{enumerate}

\subsubsection{Evento de Finalización de Acción por Grupo
Transición-Plaza}
\label{sec:sincronizacion_peticion_ejecucion_transicion_plaza}
En la Figura~\ref{fig:petri_cinta_transportadora_2} se observa un modelo de
RdP para la sincronización de acciones mediante petición de ejecución, con aviso
de finalización de acción por grupo transición-plaza. Este método consiste en
añadir una transición y una plaza extra por cada acción que
requiera enviar un evento de finalización de acción a la red. Esta transición y
plaza de aviso de finalización deben colocarse en cadena con la plaza que
representa el estado de ejecución de la acción.
\\

\begin{figure}[H]
    \centering
    \includegraphics[height=100mm]{Petri_Cinta_Transportadora_2}
    \caption{Red de Petri de una cinta transportadora sincronizada por inserción
    de plaza-transición}
    \label{fig:petri_cinta_transportadora_2}
\end{figure}


A continuación se detalla la ejecución de la red de la
Figura~\ref{fig:petri_cinta_transportadora_2}:\\
\begin{enumerate}
	\item Se generan eventos que se encolan en la cola de entrada en “t0, t2 y
		t4”.
	\item El monitor bloquea los hilos que generaron eventos para “t0, t2 y t4” por
		no estar sensibilizadas las transiciones en ese momento.
	\item Llega una pieza y se genera un evento de entrada en “t6”
	\item El monitor dispara “t6” y se coloca un token en “piezaDisp”,
		sensibilizando “t0”.
	\item El monitor libera el hilo bloqueado en “t0” ya que ahora tiene permiso
		de ejecución.
	\item Se ejecuta “moverEst1”. Una vez finalizado se genera un evento que se
		envía a la cola de entrada de “t1”.
	\item Como “t1” está sensibilizada el monitor la dispara y se coloca un token
		en “piezaLista1”, sensibilizando “t2”.
	\item El monitor libera el hilo bloqueado en “t2” ya que ahora tiene permiso
		de ejecución.
	\item Se ejecuta “moverEst2”. Una vez finalizado se genera un evento que se
		envía a la cola de entrada de “t3”.
	\item Como “t3” está sensibilizada el monitor la dispara y se coloca un token
		en “piezaLista2”, sensibilizando “t4”.
	\item El monitor libera el hilo bloqueado en “t4” ya que ahora tiene permiso
		de ejecución.
	\item Se ejecuta “moverACont”. Una vez finalizado se genera un evento que se
		envía a la cola de entrada de “t5”
	\item Como “t5” está sensibilizada el monitor la dispara y se coloca un token
		en ``piezaEnCont''.
	\item Se ejecuta la transición ``t7'', que es automática, y se libera el
		recurso ``cintaTransp''.
\end{enumerate}

La principal ventaja de este método consiste en no modificar la semántica de
la red y no añadir nuevos conceptos ni cambios en la forma de ejecución.
La desventaja más importante es que puede provocar un incremento considerable de
la cantidad de plazas y transiciones de la red, lo que conlleva el
procesamiento de matrices de mayor tamaño. En la
sección~\ref{sec:complex_secuential_task_controller} se estudia un método que
permiten contrarrestar el aumento de tamaño de la RdP para procesos
secuenciales.

\subsubsection{Evento de Finalización de Acción por Guardas}
En la Figura~\ref{fig:petri_cinta_transportadora_3} se observa un modelo de
RdP para la sincronización de acciones mediante petición de ejecución, con aviso
de finalización de acción por guardas. Este método consiste en la utilización de
una guarda como forma de sincronización entre acciones consecutivas.

\begin{figure}[H]
    \centering
    \includegraphics[height=100mm]{Petri_Cinta_Transportadora_3}
    \caption{Red de Petri de una cinta transportadora sincronizada por guardas.}
    \label{fig:petri_cinta_transportadora_3}
\end{figure}

A continuación se detalla la ejecución de la red de la
Figura~\ref{fig:petri_cinta_transportadora_3}
\begin{enumerate}
    \item Se generan eventos que se encolan en la cola de entrada en “t0, t1 y
    t2”.
	\item El monitor bloquea los hilos que generaron eventos para “t1 y t2” por
	no estar sensibilizadas las transiciones en ese momento.
	\item Se dispara ``t0'' y se coloca un token en ``moverEst1''. Comienza la
	ejecución de la acción ``moverEst1''. La transición ``t1'' no se encuentra
	sensibilizada dado que la guarda ``Fin\_moverEst1'' tiene estado ``false''.
	\item Finaliza la ejecución de ``moverEst1'' y se setea la guarda
	``Fin\_moverEst1'' con estado ``true''.
	\item Al estar sensibilizada ``t1'', se dispara y se libera el hilo
	bloqueado en su cola de condición. Se coloca un token en ``moverEst2'' y
	comienza la ejecución de esta acción. La transición ``t2'' no se encuentra
	sensibilizada dado que la guarda ``Fin\_moverEst2'' tiene estado ``false''.
	Debe setearse la guarda ``Fin\_moverEst1'' a ``false'' nuevamente.
	\item Finaliza la ejecución de ``moverEst2'' y se setea la guarda
	``Fin\_moverEst2'' con estado ``true''.
	\item Al estar sensibilizada ``t2'', se dispara y se libera el hilo
	bloqueado en su cola de condición. Se coloca un token en ``moverACont'' y
	comienza la ejecución de esta acción. La transición ``t3'' no se encuentra
	sensibilizada dado que la guarda ``Fin\_moverACont'' tiene estado ``false''.
	Debe setearse la guarda ``Fin\_moverEst2'' a ``false'' nuevamente.
	\item Finaliza la ejecución de ``moverACont'' y se setea la guarda
	``Fin\_moverACont'' con estado ``true''.
	\item Al estar sensibilidada, se dispara la transición ``t3'', que es
	automática, y se libera el recurso ``cintaTransp''. Debe setearse la guarda
	``Fin\_moverACont'' a ``false'' nuevamente.
\end{enumerate}

La ventaja de este método es que permite resolver el problema de sincronización
sin aumentar la cantidad de componentes de la red de Petri.
Como desventaja se puede mencionar que el diseño
del monitor de petri soporta una única guarda por transición, por lo tanto esta
solución impide la utilización de la guarda para otros propósitos. Otra
desventaja importante de la utilización de guardas es que al ser un valor
binario, lleva a una pérdida de eventos de finalización cuando
múltiples hilos realizan una misma acción.

Con el fin de ejemplificar la pérdida de eventos de finalización al utilizar
sincronización de finalización de acción por guardas se analiza el siguiente
caso de ejemplo:
\begin{labeling}{description}
\item [Ejemplo]
	Una ``tarea A'' es realizada por multiples hilos de manera independiente, y cada
	hilo realiza la ``tarea A'' en su totalidad. A su vez una ``tarea B'', que
	debe realizarse luego de la finalización de la ``tarea A'', es ejecutada por
	un único hilo. En este caso la utilización de guardas podría llevar a una
	pérdida de eventos de finalización de la ``tarea A'' debido a la condición
	binaria de la guarda. Ver Figura ~\ref{fig:ejecucion_multiples_hilos_guardas}
	En esta red, un máximo de 5 hilos puede ejecutar la ``tarea A'' al mismo
	tiempo. En el estado que  muestra la
	Figura~\ref{fig:ejecucion_multiples_hilos_guardas} existen tres hilos
	corriendo la ``tarea A''. De acuerdo al planteo de este problema, la ``tarea
	B'' es ejecutada por un único hilo. Cuando dos o más hilos finalizan la ``tarea
	A'' y setean la guarda ``Fin\_TareaA'' en ``true'', existe la posibilidad de
	que otro hilo dispare ``t1'' antes de comenzar la ejecución de
	la ``tareaB''. En este momento, el hilo que dispara ``t1'' modifica el
	valor de la guarda ``Fin\_TareaA'' a ``false'', sobreescribiendo el aviso de
	finalización de acción de los hilos que ya habían seteado ``Fin\_TareaA'' en
	``true''.
\end{labeling}

\begin{figure}[H]
    \centering
    \includegraphics[height=60mm]{Ejecucion_Tarea_Multiples_Hilos_Guardas}
    \caption{RdP: Problema de sincronización de acciones dependientes usando
    guardas, debido a su condición binaria}
    \label{fig:ejecucion_multiples_hilos_guardas}
\end{figure}


\subsubsection{Evento de Finalización de Acción por Cola de Condición de
Disparo No Perenne}
En la Figura~\ref{fig:petri_cinta_transportadora_4} se observa un modelo de
RdP para la sincronización de acciones mediante petición de ejecución, con aviso
de finalización de acción por cola de condición de disparo no perenne. 
Esta forma de solucionar la sincronización de acciones dependientes
supone añadir una nueva propiedad ``P'' a las transiciones. Los hilos
bloqueados en la cola de condición de una transición con propiedad ``P'' sólo
se liberan cuando la transición se encuentra habilitada y además un hilo
externo realiza un disparo no perenne sobre la transición.

\begin{figure}[H]
    \centering
    \includegraphics[height=100mm]{Petri_Cinta_Transportadora_4}
    \caption{Red de Petri de una cinta transportadora sincronizada por
    propiedad ``P''.}
    \label{fig:petri_cinta_transportadora_4}
\end{figure}

\begin{enumerate}
    \item Se generan eventos que se encolan en la cola de entrada en “t0, t1,
    t2 y t3”.
	\item El monitor bloquea los hilos que generaron eventos para “t1, t2 y t3” por
	no estar sensibilizadas las transiciones en ese momento.
	\item Se dispara ``t0'' y se coloca un token en ``moverEst1''. Comienza la
	ejecución de la acción ``moverEst1''. La transición ``t1'' no se dispara ya que
	es de tipo ``P'' y solo puede dispararse de forma no perenne por un hilo
	externo.
	\item Finaliza la ejecución de ``moverEst1'' y un hilo dispara ``t1'' de forma
	no perenne para dar aviso de la finalización de la acción.
	\item Se libera el hilo bloqueado en cola de condición de ``t1''. Se coloca un
	token en ``moverEst2'' y comienza la ejecución de esta acción. La transición
	``t2'' no se dispara ya que es de tipo ``P'' y solo puede dispararse de forma
	no perenne por un hilo externo.
	\item Finaliza la ejecución de ``moverEst2'' y un hilo dispara ``t2'' de forma
	no perenne para dar aviso de la finalización de la acción.
	\item  Se libera el hilo bloqueado en cola de condición de ``t2''. Se coloca un
	token en ``moverACont'' y comienza la ejecución de esta acción. La transición
	``t3'' no se dispara ya que es de tipo ``P'' y solo puede dispararse de forma
	no perenne por un hilo externo.
	\item Finaliza la ejecución de ``moverACont'' y un hilo dispara ``t3'' de forma
	no perenne para dar aviso de la finalización de la acción.
	\item Se libera el recurso ``cintaTransp''.
\end{enumerate}

Esta solución tiene como principal ventaja mantener la cantidad de componentes
de la red de Petri.
Su principal desventaja consiste en que añade una nueva etiqueta a la
RdP, dificultando su demostración matemática. Esta solución supone añadir
una interfaz al monitor de Petri para bloquear hilos en una cola de condición de
una transición tipo P y que los hilos bloqueados en esta cola de condición sólo
puedan liberarse por medio de un disparo no perenne ocasionado por un hilo
externo.
El hilo que realiza el disparo sobre la transición debe realizar una operación
release sobre la cola de condición de disparos no perennes (sin importar si
existen o no hilos bloqueados en la cola) para evitar la pérdida de eventos de
finalización de acción.

\subsection{Resumen de Modos de Sincronización}
\label{sec:resumen_sincronizacion}
Existen dos maneras de coordinar la ejecución de las acciones a partir de una
red de Petri.\\ 
\begin{enumerate}
  \item \textbf{Sincronización por aviso de ejecución: } Consiste en suscribir
  una acción a una transición, que informa cuando es disparada. Ante un informe
  de esta transición, la acción suscripta comienza su ejecución.
  Al finalizar una acción, se dispara la transición a la cual está suscripta la
  siguiente acción a ejecutar.
  La principal desventaja de este método consiste en la descentralización del
  manejo de los hilos por parte del monitor, lo cual va en contra de los objetivos del
  proyecto.
  \item \textbf{Sincronización por petición de ejecución: } Consiste en que los
  hilos que ejecutan acciones realicen una petición de ejecución al
  monitor (mediante el disparo de transiciones), sin tener en cuenta el estado
  actual de la red de Petri. De este modo el monitor bloquea los hilos que no
  pueden ejecutarse y libera los hilos que cumplan con las condiciones de
  ejecución. El manejo de la concurrencia es llevado a cabo íntegramente por el
  monitor.
  Para sincronizar acciones dependientes entre sí (una debe comenzar luego de
  la finalización de la otra) se requiere un modo de dar aviso al monitor de
  la finalización de una acción. Se analizan tres opciones:
  \begin{itemize}
      \item Evento de finalización de acción por grupo transición-plaza
	  \item Evento de finalización de acción por guardas
	  \item Evento de finalización de acción por cola de condición de disparo no
	  perenne.
  \end{itemize}
  Se optó por adoptar el evento de finalización de acción por grupo
  transición-plaza. Esta forma presenta la ventaja de no añadir
  conceptos nuevos a la RdP, facilitando el entendimiento de la misma. Además
  no presenta desventajas como la pérdida de eventos de finalización, presente
  en el evento de finalización de acción por guardas.
  La principal desventaja del grupo transición-plaza consiste en el incremento
  del tamaño de la RdP. En procesos secuenciales, puede contrarrestarse mediante
  el uso del controlador de acciones descripto en la
  sección~\ref{sec:complex_secuential_task_controller}.
\end{enumerate}
 

 
 
 
 

            \section{Clasificación de Eventos Físicos: \\ Eventos Task y Eventos Happening}
\label{sec:clasificacion_eventos_fisicos} 

El monitor de RdP es el único responsable del bloqueo o habilitación de un hilo
que ejecuta una acción emisora de eventos físicos de salida. El hilo realiza el
disparo de petición de ejecución al inicio de la ejecución del programa, sin
tener en cuenta el estado actual de la red (ver
sección~\ref{sec:sincronizacion_peticion_ejecucion}). En consecuencia, el hilo
queda bloqueado en el monitor hasta que el mismo decide desbloquearlo,
dependiendo del estado de la red y la política de prioridades (ver
Figura~\ref{fig:actividades_evento_task}).

\begin{figure}[H]
	\centering
	\includegraphics[width=0.9\textwidth]{secuencia_evento_task}
	\caption{Diagrama de Secuencia de la Ejecución de una Acción que Emite un
	Evento Físico de Salida}
	\label{fig:actividades_evento_task}
\end{figure}

En la Figura~\ref{fig:actividades_evento_happening} se observa que
la ejecución de una acción encargada de recibir un evento físico de entrada
depende, en primera instancia de la ocurrencia de dicho evento. Luego, también
es sincronizada con la RdP mediante el uso del monitor. De este modo, las
acciones que reciben eventos físicos de entrada presentan una restricción extra
respecto a aquellas que emiten eventos físicos de salida.



\begin{figure}[H]
	\centering
	\includegraphics[width=0.9\textwidth]{secuencia_evento_happening}
	\caption{Diagrama de Secuencia de la Ejecución de una Acción que Recibe un
	Evento Físico de Entrada}
	\label{fig:actividades_evento_happening}
\end{figure}

Desde este punto de vista, se decidió dar una clasificación más significativa a
los eventos físicos. Esta clasificación estará presente a lo largo de todo
el desarrollo del framework.
\begin{itemize}
  \item Eventos Task (Tarea): Eventos físicos de salida. Son
  desencadenados y sincronizados exclusivamente por eventos lógicos que dependen
  de condiciones ya presentes en el monitor de Redes de Petri al momento de su
  emisión. Si bien el monitor no produce directamente eventos físicos, puede
  advertirse que un evento task tiene una relación directa con determinados
  eventos lógicos. Dichos eventos lógicos se encuentran definidos en un tópico
  (evento de acción).
  \item Eventos Happening (Suceso): Eventos físicos de entrada. Son
  desencadenados por el mundo externo de manera totalmente asincrónica respecto
  al sistema. La ejecución de las acciones que reciben y manejan estos eventos
  es sincronizada por el monitor de redes de petri. De esta forma, el monitor
  conserva su responsabilidad frente al manejo del asincronismo del sistema.
  Los eventos lógicos requeridos para la sincronización se encuentran definidos
  en un tópico (evento de acción).
\end{itemize}

\section{Controladores de Acciones: \\ Task Controllers y Happening Controllers}
\label{sec:controladores_de_acciones}

Las acciones que debe realizar el sistema, junto con las interfaces necesarias
para la comunicación de los eventos físicos correspondientes, se encuentran
embebidos dentro de controladores de acción. En la
Figura~\ref{fig:arquitectura_petri-manejador-acciones-mundo} se muestran dichos
controladores.\\

A partir de la clasificación de eventos físicos propuesta en la
sección~\ref{sec:clasificacion_eventos_fisicos}, surge la necesidad de
clasificar los controladores de acción respecto a su condición de receptores de
Eventos Happening, o de emisores de Eventos Task. En consecuencia, emerge un
nuevo requerimiento del framework, relacionado con el requerimiento número 2
definido en la sección~\ref{sec:definicion_reqs}:
\begin{itemize}
    \item El framework debe ofrecer interfaces para especificar si un
    controlador de acción responde a un evento físico de entrada o a un
    evento físico de salida.
\end{itemize}
En respuesta a este requerimiento se definen los controladores de tipo
Happening Controller y Task Controller.

\subsection{Ejecución de un Task Controller}
\label{sec:ejecucion_task_controller}
De acuerdo al diseño del framework, el comportamiento dinámico de los eventos
Task es dirigido únicamente por la evolución de los estados de la RdP. Esto se
debe a las siguientes condiciones:
\begin{itemize}
    \item En la sección~\ref{sec:resumen_sincronizacion} se explicita que el
    modo de sincronización adoptado es el de petición de ejecución al monitor. De
    esta forma se admiten disparos asíncronos a transiciones realizados desde
    diferentes hilos de ejecución, delegando en el monitor la responsabilidad de
    bloquear los hilos que no cumplen con las condiciones necesarias para la ejecución.

    \item En la sección~\ref{sec:clasificacion_eventos_fisicos} se determina la
    existencia de una relación directa entre un Evento Task y un conjunto de
    eventos lógicos.

    \item Un Evento Task es desencadenado por condiciones que estan presentes en el
    monitor al momento de la emisión de este evento
\end{itemize}

Del analisis anterior emerge un nuevo requerimiento, relacionado
con el requerimiento 2 de la sección~\ref{sec:definicion_reqs}:
\begin{itemize}
    \item El framework debe ser responsable de crear y controlar la
    ejecución de los hilos de ejecución para controladores de acción que
    generan eventos físicos de salida.
\end{itemize}

 Como corolario, se determina que la ejecución de un Task Controller debe
 encapsularse en un hilo al momento de inicialización del programa. El hilo se
 encarga de:
\begin{enumerate}
  \item Realizar las peticiones de los permisos de sincronización al
  monitor de forma directa, sin tener en cuenta el estado de la red.
  \item  Ejecutar el código correspondiente al Task Controller, una vez que el
  monitor otorga el permiso.
  \item  Realizar el aviso de finalización de ejecución al monitor de Petri.
  \item Repetir los pasos de ejecución de forma infinita, delegando en el
  monitor el control de la ejecución.
\end{enumerate}

La responsabilidad de la creación de los hilos para ejecutar los Task
Controllers pertenece al framework. En consecuencia, la ejecución de las
acciones encapsuladas en un Task Controller es transparente al usuario
desarrollador.

\begin{figure}[H]
	\centering
	\includegraphics[width=0.9\textwidth]{ejecucion_task_controller_pt1}
\end{figure}
\begin{figure}[H]
	\centering
	\includegraphics[width=0.9\textwidth]{ejecucion_task_controller_pt2}
	\caption{Pasos de la Ejecución de un Task Controller}
	\label{fig:ejecucion_task_controller}
\end{figure}

\newpage

\subsection{Ejecución de un Happening Controller}
\label{sec:ejecucion_happening_controller}
Un Evento Happening se desencadena en el mundo externo y de forma asíncrona al
sistema (ver sección~\ref{sec:clasificacion_eventos_fisicos}). 
En consecuencia, el código de usuario es responsable de realizar la llamada a
ejecución de un Happening Controller, encargado de manejar dicho evento.

En principio, realizar una llamada desde el código de usuario podría generar un
grave problema en cuanto a las responsabilidades de sincronización. Las
acciones del Happening Controller quedarían fuera del mecanismo de
sincronización del monitor de Petri, atentando contra el objetivo de delegar
el asincronismo del sistema en la RdP. \\

Del análisis anterior emerge un nuevo requerimiento, relacionado con el
requerimiento 2 de la sección~\ref{sec:definicion_reqs}:
    \begin{itemize}
        \item El framework debe ser responsable de controlar la ejecución de los
        hilos creados por el usuario, correspondientes a controladores de acción
        que manejan eventos físicos de entrada.
    \end{itemize}

Este requerimiento se cumple mediante la utilización de herramientas de la
programacón orientada a aspectos (ver sección~\ref{sec:aop}). Se optó por
encapsular las instrucciones de sincronización necesarias dentro de advices.
Dichos advices son aplicados en joinpoints, definidos por los puntos de
ejecución del programa donde existen llamadas a ejecución o retornos de
rutinas de tipo Happening Controller.

\begin{framed}
\textbf{Nota:}
	Un controlador de acción que reacciona ante
	eventos Happening (Happening Controller) debe ser ejecutado dentro del
	contexto de un Listener u Observer del evento deseado.
	Para maximizar las libertades de elección del desarrollador en la recepción de
	eventos externos, la responsabilidad de crear estos Listeners se delega en el
	usuario.
\end{framed}


El flujo de ejecución de un Happening Controller es el siguiente:
\begin{enumerate}
  \item Cuando el código de usuario hace un llamado a la ejecución del Happening
  Controller, la aplicación alcanza un joinpoint.
  \item En el momento en que se alcanza el joinpoint se aplica un advice que
  realiza los pedidos de permiso de sincronización al monitor. Dicho advice es
  aplicado automáticamente en tiempo de compilación, por lo tanto el usuario no
  tiene la responsabilidad de realizar la sincronización manualmente. De esta
  manera se logra conservar la inversión de control.
  \item Una vez que el monitor libera el hilo, indicando que la ejecución es
  posible, se procede a ejecutar el código del controller.
  \item Al finalizar la ejecución del Happening Controller, se alcanza un nuevo
  joinpoint.
  \item En el momento en que se alcanza el joinpoint descripto en el punto
  anterior se aplica un advice que realiza el aviso de finalización de
  ejecución al monitor. Este advice se aplica de manera análoga a la descripta
  en el punto 2.
\end{enumerate}

\begin{framed}
\textbf{Nota:} Los advices descriptos en esta sección son ejecutados por el
mismo hilo que ejecuta el código del Happening Controller. Dado que la llamada
al código del Happening Controller es responsabilidad del usuario, es él quien
debe generar el hilo encargado de ejecutar el código. Se recomienda no utilizar
el hilo principal del programa para ejecutar Happening Controllers ya que puede
ocasionar el bloqueo del sistema.
\end{framed}

En conclusión, la sincronización de la ejecución de los Happening Controllers
sigue siendo una responsabilidad del monitor y es realizada de forma
transparente al usuario. Es posible realizar La llamada a un HappeningController
en cualquier punto del código de la aplicación, de esta forma el usuario
tiene libertad para realizar el manejo de eventos físicos de entrada. Se destaca
que aunque el usuario tiene la libertad de llamar al código del Happening
Controller en cualquier momento, el monitor es responsable de determinar si
la ejecución del controlador comenzará al momento de la llamada a ejecución, o
si bloqueará el hilo por no cumplirse las condiciones de ejecución del controlador.

\begin{figure}[H]
	\centering
	\includegraphics[width=0.9\textwidth]{ejecucion_happening_controller_pt1}
\end{figure}
\begin{figure}[H]
	\centering
	\includegraphics[width=0.75\textwidth]{ejecucion_happening_controller_pt2}
	\caption{Pasos de la Ejecución de un Happening Controller}
	\label{fig:ejecucion_happening_controller}
\end{figure}

\section{ComplexSequentialTaskController}
\label{sec:complex_sequential_task_controller}
En esta sección se presenta una optimización de la ejecución de Task Controllers
(ver sección~\ref{sec:ejecucion_task_controller}) para sistemas de caracter
secuencial, como el que se estudia en la
sección~\ref{sec:sincronizacion_cinta_transportadora}.
Debido a que no existe un paralelismo aprovechable en una secuencia de
tareas dependientes resulta innecesario utilizar un hilo
por cada Task Controller correspondiente a cada tarea secuencial.

En respuesta a este problema, se determinó que los Task Controllers de acciones
correspondientes a un proceso secuencial deben ser ejecutados por un mismo hilo.
Esta agrupación de controladores se denomina Complex Sequential Task
Controller.
\\
La utilización de Complex Sequential Task Controllers reduce la
cantidad de hilos necesarios para ejecutar un sistema con tareas secuenciales.
Para lograrlo, se incorpora una etapa de petición de permisos y una etapa de
ejecución de controlador por cada Task Controller que forma parte del
ComplexSequentialTaskController. Cada una de estas etapas debe seguir el orden
preestablecido por la secuencia de acciones que conforman el proceso secuencial.
Dicho orden se establece en el tópico, al definir el evento de acción para la
sincronización de la tarea compleja.

Una ventaja de agrupar tareas secuenciales en un mismo hilo es que permite
eliminar la plaza-transición incorporada por el Modo de Sincronización de
petición de de ejecución (ver sección~\ref{sec:sincronizacion_peticion_ejecucion_transicion_plaza}).

\begin{figure}[H]
	\centering
	\includegraphics[width=0.8\textwidth]{simple_vs_complex_task_petri_net}
	\caption{Comparación del modelo en RdP de un sistema con tres
	acciones secuenciales para Task Controllers simples y para Complex Sequential
	Task Controller.}
	\label{fig:simple_task_vs_complex_task_petri_net}
\end{figure}

\begin{figure}[H]
	\centering
	\includegraphics[width=0.9\textwidth]{ejecucion_complextask_controller_pt1}
\end{figure}

\begin{figure}[H]
	\centering
	\includegraphics[width=0.9\textwidth]{ejecucion_complextask_controller_pt2}
\end{figure}

\begin{figure}[H]
	\centering
	\includegraphics[width=0.9\textwidth]{ejecucion_complextask_controller_pt3}
	\caption{Pasos de la Ejecución de un Complex Sequential Task Controller con
	Dos Acciones Task}
	\label{fig:ejecucion_complextask_controller}
\end{figure}
            \section{Manejo de Guardas: \\ Guard Providers}
\label{sec:guard_providers}
Las guardas son un componente importante de las RdP no autónomas utilizadas para
modelar los sistemas concurrentes (ver
sección~\ref{guardas}). Las mismas
permiten relacionar la RdP con el estado del medio, y representar condiciones de
ejecución síncronas propias del mismo sistema (ver
sección~\ref{sec:arquitectura_alto_nivel}).

El monitor de petri ofrece una interfaz para realizar el seteo del valor
de una guarda. De esta manera se permite al usuario indicar directamente
dicho valor. Sin embargo, el uso de esta alternativa trae aparejada una pérdida
de la inversión de control. Esto se debe a que se puede modificar el estado
de la red de manera directa y en cualquier instante, permitiendo al usuario tomar
parte del control de la ejecución.

Para lidiar con este problema se propone el concepto de Guard Provider. Un Guard
Provider es un método asociado a una guarda que retorna una variable de tipo
boolean. Este método es llamado automáticamente luego de ejecutar un controlador
de acción que requiera modificar dicha guarda. Para lograr la llamada automática
a un Guard Provider se utiliza reflection (ver sección~\ref{reflection}). La
finalidad de un método Guard Provider retornar el valor que debe
tomar la guarda asociada. 

El concepto de Guard Provider permite limitar el acceso a las
guardas. En consecuencia, la modificación de una de ellas se realiza sólo tras
realizar las acciones que deben modificar en la RdP el estado representado por
dicha guarda. El usuario tiene la capacidad de definir la lógica que da el
valor a la guarda, pero el monitor conserva la decisión de ejecución del Guard
Provider, ya que la misma esta asociada a la ejecución de los controladores de acciones (ver
sección~\ref{sec:controladores_de_acciones})

\begin{figure}[H]
	\centering
	\includegraphics[width=120mm]{ejecucion_guard_provider}
	\caption{Pasos de la Ejecución de un Guard Provider asociado a un Task
	Controller }
	\label{fig:ejecucion_guard_provider}
\end{figure}
            \section{Relación entre Eventos Lógicos, Eventos de Acción y Controladores de
Acción}
\label{sec:relacion_evento_controlador}
En la sección~\ref{sec:sincronizacion_RdP_por_eventos} se explica que las
acciones de software, embebidas en controladores de acción, son desencadenadas
por eventos de acción. A su vez, en la sección
\ref{sec:controladores_de_acciones} se definen dos tipos de controladores de
acción, cuyo modo de ejecución se explica en las secciones
\ref{sec:ejecucion_task_controller} y \ref{sec:ejecucion_happening_controller}.

La ejecución de ambos tipos de controladores tiene tres etapas marcadas:
\begin{itemize}
  \item Petición de permiso de ejecución al monitor.
  \item Ejecución de controlador de acción.
  \item Aviso de finalización de ejecución al monitor.
\end{itemize}

Si incorporamos el concepto de Guard Provider definido en la
sección~\ref{sec:guard_providers}, se adiciona una etapa a la ejecución,
resultando en:
\begin{itemize}
  \item Petición de permiso de ejecución al monitor.
  \item Ejecución de controlador de acción.
  \item Ejecución del método Guard Provider y evaluación de la guarda.
  \item Aviso de finalización de ejecución al monitor.
\end{itemize}

A partir del análisis de las etapas de ejecución del controlador de acción,
emergen nuevos requerimientos, relacionados con el requerimiento 2 de la
sección~\ref{sec:definicion_reqs}:
\begin{itemize}
    \item Un evento de acción debe definir todos los eventos lógicos
    necesarios para la sincronización de la ejecución de un controlador de
    acción:
        \begin{itemize}
          \item Permiso de ejecución del controlador de acción (disparo de
          transición).
          \item Cambio de estado del sistema (evaluación de guardas).
          \item Aviso de finalización de ejecución del controlador de acción
          (disparo de transición).
        \end{itemize}
    \item El framework debe ofrecer interfaces para la suscripción de
    controladores de acción a eventos de acción.
\end{itemize}

En consecuencia, se define a un evento de acción como el conjunto de tres
eventos lógicos, claves para definir la sincronización de la ejecución de un
controlador de acción y sus influencias sobre el estado de la RdP:
\begin{labeling}{description}
  \item [Permiso de ejecución: ] Consiste en un evento lógico de disparo
  de transición al monitor de manera bloqueante (perenne).
  \item [Callback de guardas: ] Consiste en un evento lógico de evaluación de
  guarda. El sistema de ejecución obtiene el valor a establecer en la guarda a
  partir de la ejecución automática de un método de tipo Guard Provider. Este
  método está asociado a la guarda y al controlador de acción desencadenado por
  el evento de acción.
  \item [Callback de aviso de finalización: ]
  Consiste en un evento lógico de disparo de transición al monitor de manera no
  bloqueante (no perenne). Se trata de una devolución de recursos al monitor,
  no es una petición de sincronización.
\end{labeling}

\begin{framed}
\textbf{Nota:} Un evento de acción debe contener un permiso de ejecución
obligatoriamente, ya que la petición de ejecución es el principio de la
inversión de control del framework. Sin embargo los callbacks de guardas y de
aviso de finalización son opcionales y están sujetos a las características del
modelo para el controlador de acción correspondiente.
\end{framed}

\subsection{Tópicos}
\label{sec:diseno_topicos}
Un tópico es una representación de un evento de acción. Los controladores de
acciones se suscriben a tópicos.
Un tópico esta compuesto por:
\begin{itemize}
  \item Un nombre único: identifica al tópico y se utiliza al momento de
  realizar la suscripción al mismo.
  \item Una lista ordenada de nombres de transición. Constituye el permiso
  de ejecución de cada controlador de acción suscripto al tópico. En el caso de
  Task Controllers simples y de Happening Controllers, esta lista contiene
  un solo nombre de transición mientras que en el caso de ComplexSequentialTask
  Controllers contiene uno por cada sub tarea.
  \item Una lista ordenada de conjuntos de nombres guardas. Cada conjunto
   dentro de la lista ordenada constituye el callback de guardas de cada
   controlador de acción suscripto al tópico. En un ComplexSequentialTask
   Controller las guardas se evalúan al finalizar cada una de las acciones
   individuales que componen la tarea compleja. En el caso de
  Task Controllers simples y de Happening Controllers, la lista ordenada
  contiene un único conjunto de nombres de guardas mientras que en el caso de
  ComplexSequentialTask Controllers contiene un conjunto por cada sub tarea.
  \item Un conjunto de nombres de transiciones que constituye el callback de
  aviso de finalización de ejecución. Contiene los nombres de todas las
  transiciones que se disparan de manera no bloqueante al finalizar la ejecución
  del controlador de acción. En un  ComplexSequentialTask
  Controller el callback de transiciones se dispara luego de finalizar la última
  sub tarea.
\end{itemize}
        \chapter{Implementación de \nombreFramework \ Framework}
            \section{Introducción}
En este capítulo se describe la implementación de los componentes del framework
y se detallan sus interfaces de programación.
Se detalla la implementación de un controlador de acción y el
proceso de ejecución de los mismos. A su vez, se detalla la implementación de
los tópicos.

La implementación se realizó de acuerdo a los aspectos de diseño
expuestos en el capítulo~\ref{cap:diseno_framework}.

\section{Detalles de Implementación de \nombreFramework \ Framework}
\nombreFramework \ Framework es de un tipo intermedio entre caja negra
y caja blanca. Para utilizar la mayor parte de las funcionalidades ofrecidas, el
usuario sólo debe conocer las interfaces que expone el framework para conectar
los componentes. Sin embargo, para determinados casos de uso (por ejemplo para
añadir políticas de disparo de transiciones) se requiere de la extensión de
clases del framework.
(ver sección~\ref{sec:tipos_framework}).
El usuario tiene la responsabilidad de definir los siguientes componentes,
previamente expuestos en el capítulo~\ref{cap:diseno_framework}:
\begin{itemize} 
  \item Controladores de acciones.
  \item Tópicos.
  \item Modelo de la lógica del sistema en RdP.
\end{itemize}

Por su lado, el framework se encarga de:
\begin{itemize} 
  \item Proveer interfaces para interconectar los componentes definidos por el
  usuario.
  \item Controlar el flujo de ejecución del programa.
\end{itemize}

\section{Implementación de un Controlador De Acción}
\label{sec:implementacion_controlador_accion}
La implementación de un controlador de acción queda definida por dos elementos:
\begin{labeling}{description}
  \item [Acción a realizar: ] Método con una anotación de Java
  identificando el tipo de controlador de acción. 
  Dicha anotacion puede ser de dos tipos:
  \begin{itemize}
    \item \emph{@TaskController}
    \item \emph{@HappeningController}
  \end{itemize}
  \item [Ejecutante de la Acción: ] Es la instancia del objeto que ejecuta el
  método. Este objeto pertenece a una clase que contiene la declaración
  del método.
\end{labeling}

\section{Componentes del Framework}
\label{sec:componentes_baboon}
En la Figura~\ref{fig:diagrama_componentes} se pueden observar los
módulos que componen el framework implementado:
\begin{itemize}
  \item \textbf{BaboonConfig: } Este componente se encarga de manejar la
  suscripción de controladores de acciones a los tópicos (eventos de acción).
  Para ello provee interfaces para:
	  \begin{itemize}
	    \item Recibir el archivo de tópicos definido por el usuario
	    \emph{(addTopics)}.
	    \item Suscribir los controladores de acciones a tópicos específicos
	    \emph{(subscribeToTopic, createNewComplexTask, appendTaskToComplexTask)}.
	    \item Consultar las suscripciones existentes
	    \emph{(getTaskControllerSubscription,
	    getHappeningControllerSubscripton)}.
	  \end{itemize}
	  
  \item \textbf{PetriCore: } Este componente es un wrapper sobre el monitor de
  Redes de Petri descripto en el
  Capítulo~\ref{cap:petri_monitor}, y provee las interfaces del
  monitor al resto del framework.
  
  \item \textbf{BaboonApplication: } Es una interfaz Java que define dos
  métodos. Esta interfaz debe ser implementada por una clase de usuario, que
  será utilizada para inicializar el sistema. Los métodos son:
      \begin{itemize}
        \item \textbf{\emph{declare(): }} En este método el usuario debe:
        	\begin{enumerate}
        	  \item Proveer un archivo JSON con la definición de los tópicos.
        	  \item Proveer un archivo PNML con la definición del modelo de RdP.
        	  \item Declarar e inicializar los objetos del sistema. Estos objetos
        	  (desarrollados por el usuario) contienen la implementación
        	  de los controladores de acción.
        	\end{enumerate}
        \item \textbf{\emph{subscribe(): }} En este método el usuario define todas
        las suscripciones de controladores de acción a tópicos.
       \end{itemize}
   
  \item \textbf{DummyThread: } Objeto que implementa la interfaz Callable y, en
  consecuencia, es ejecutable por un hilo. Encapsula el controlador de acción
  con los eventos lógicos necesarios para su sincronización. Se lo denomina
  ``dummy'' (tonto) porque no posee información propia, simplemente envía los
  eventos de sincronización de acuerdo a los contenidos del tópico y ejecuta
  las acciones embebidas en el TaskController.
  
  \item \textbf{DummiesExecutor: } Pool de hilos encargado de ejecutar los
  objetos DummyThread.
  
  \item \textbf{HappeningControllerJoinpointReporter: } Aspect donde se
  definen los puntos de unión del sistema y los advices que se aplican en
  dichos puntos.
  Es observable por aquellos objetos que implementen la interfaz
  JoinPointObserver.
  Los puntos de union se alcanzan:
  	\begin{itemize}
  	  \item  En el momento previo a iniciar la ejecución de un
  	  método anotado con \emph{@HappeningController}.
  	  \item En el momento posterior a finalizar la ejecución de un
  	  método anotado con \emph{@HappeningController}.
  	\end{itemize}
  Durante la ejecución del advice se realiza una actualización sobre los
  observadores suscriptos al objeto, brindando el estado del joinpoint
  alcanzado. El estado del joinpoint permite identificar a un
  HappeningController en particular a partir de la instancia de objeto que hizo
  el llamado a ejecución, y de la firma del método que se llamó a ejecutar (ver
  Sección~\ref{sec:implementacion_controlador_accion}).
  
  \item \textbf{HappeningControllerSyncronizer: } Objeto que implementa la
  interfaz JoinPointObserver. Se suscribe como observador del objeto
  HappeningControllerJoinpoint. Cuando se alcanza un joinpoint, intercambia con
  el monitor los eventos lógicos necesarios para la sincronización del
  HappeningController. Obtiene los eventos lógicos utilizando las
  interfaces de BaboonConfig para consultar las suscripciones a tópicos.
  
  \item \textbf{Main: } Componente que contiene el método principal del
  programa. Al existir inversión de control, el método main forma parte del
  framework y no debe ser implementado por el usuario.
\end{itemize}

\begin{figure}[H]
	\vspace*{-4cm}
	\hspace{-1,60cm}
	\includegraphics[width=150mm]{diagrama_componentes}
	\caption{Diagrama de Componentes de la implementación de \nombreFramework
	Framework}
	\label{fig:diagrama_componentes}
\end{figure}


\section{Ejecución de \nombreFramework \ Framework}

\begin{itemize}
  \item \textbf{Main: } La ejecución de este método consta de
  los siguientes pasos (ver
  Figura~\ref{fig:diagrama_secuencia_implementacion_ejecucion_main}):
  	\begin{enumerate}
  	  \item Obtiene la clase de usuario que implementa la interfaz
  	  BaboonApplication (utilizando Reflection) e instancia un nuevo objeto de
  	  dicha clase.
  	  \item Ejecuta los métodos declare() y subscribe() (en ese orden) del
  	  objeto mencionado en el punto anterior.
  	  \item Crea el Objeto HappeningControllerSyncronizer y lo suscribe
  	  como observer de HappeningControllerJoinPoint.
  	  \item Utiliza las interfaces de BaboonConfig para obtener las
  	  suscripciones a tópicos de los Task Controllers.
  	  \item Encapsula los TaskControllers en DummyThreads y los envía al objeto
  	  DummiesExecutor para su ejecución.
  	\end{enumerate}
  	
 \begin{figure}[H]
	\hspace{-2,90cm}
	\includegraphics[width=180mm]{secuencia_implementacion_ejecucion_main}
	\caption{Diagrama de Secuencia de la Implementación del Método Principal}
	\label{fig:diagrama_secuencia_implementacion_ejecucion_main}
\end{figure}


  \item \textbf{TaskController: } La ejecución del TaskController implementada
  consta de los siguientes pasos (ver
  Figura~\ref{fig:diagrama_secuencia_implementacion_ejecucion_task_controller}):
  	\begin{enumerate}
  	  \item El pool de hilos DummiesExecutor llama a ejecutar el método
  	  \emph{call()} del DummyThread.
  	  \item El método \emph{call()} del DummyThread inicia un bucle infinito
  	  \item Utilizando el tópico de la suscripción se obtiene la transición que
  	  conforma el permiso de ejecución.
  	  \item El DummyThread realiza un disparo perenne de la transición de permiso
  	  de ejecución.
  	  \item Cuando el hilo que ejecuta al DummyThread es liberado por el
  	  monitor, llama a ejecutar el método del TaskController.
  	  \item El código del TaskController emite un evento físico de salida
  	  (Evento Task).
  	  \item Una vez finalizada la ejecución, se obtienen los nombres de las guardas asociadas al
  	  controlador a través del tópico de la suscripción.
  	  \item Se ejecutan los métodos GuardProvider que corresponden a las guardas
  	  asociadas. 
  	  \item El objeto DummyThread setea en el monitor de Petri, para cada guarda asociada, el valor
  	  correspondiente que retornan los métodos GuardProvider.
  	  \item Se obtienen del tópico los nombres de las transiciones de aviso de
  	  finalización de ejecución.
  	  \item El objeto DummyThread realiza disparos no perennes de las
  	  transiciones de aviso de finalización.
  	  \item Se repite el bucle infinito del método \emph{call()} del DummyThread.
  	\end{enumerate}
  	
\begin{figure}[H]
	\hspace{-2,90cm}
	\includegraphics[width=180mm]{secuencia_implementacion_ejecucion_task_controller}
	\caption{Diagrama de Secuencia de la Ejecución Implementada de un
	TaskController}
	\label{fig:diagrama_secuencia_implementacion_ejecucion_task_controller}
\end{figure}

  \item \textbf{HappeningController: } La ejecución implementada de un
  HappeningController es de la siguiente forma (ver
  Figura~\ref{fig:diagrama_secuencia_implementacion_ejecucion_happening_controller}):
  	\begin{enumerate}
  	  \item El código de usuario recibe un evento asincrónico del Mundo Exterior
  	  (Evento Happening).
  	  \item El código de usuario realiza en un hilo el llamado a ejecución del
  	  HappeningController encargado de manejar el Evento Happening.
  	  \item En este punto de la ejecución se alcanza un joinpoint, por lo tanto
  	  antes de ejecutar el HappeningController se ejecuta el advice
  	  \emph{before()} del objeto HappeningControllerJoinPoint.
  	  \item El advice \emph{before()} realiza un update del estado del joinpoint
  	  al objeto HappeningControllerSincronizer.
  	  Dicho estado se compone del nombre del método anotado con
  	  \emph{@HappeningController} que se llamó a ejecución, de la instancia del
  	  objeto que invocó dicho método y de un enum que indica que el joinpoint es
  	  previo a la ejecución del controlador.
  	  \item El HappeningControllerSincronizer obtiene de BaboonConfig la
  	  suscripción al tópico del HappeningController, a partir de los datos
  	  obtenidos del estado del joinpoint.
  	  \item Utilizando el tópico de la suscripción se obtiene la transición que
  	  conforma el permiso de ejecución.
  	  \item El HappeningControllerSincronizer realiza un disparo perenne de la
  	  transición de permiso de ejecución.
  	  \item  Cuando el hilo de ejecución es liberado por el monitor, se ejecuta
  	  el código del HappeningController, donde se maneja el evento recibido.
  	  \item Al finalizar la ejecución del HappeningController se alcanza un
  	  joinpoint, y se ejecuta el advice \emph{after()} del objeto
  	  HappeningControllerJoinPoint.
  	  \item El advice \emph{after()} realiza un update del estado del joinpoint
  	  al objeto HappeningControllerSincronizer. Este estado es similar al
  	  descripto en el advice \emph{before()}, con la diferencia de que en este
  	  caso el enum indica que el joinpoint es posterior a la ejecución del
  	  controlador.
  	  \item El HappeningControllerSincronizer obtiene de BaboonConfig la
  	  suscripción al tópico del HappeningController, a partir de los datos
  	  obtenidos del estado del joinpoint.
  	  \item Se obtienen los nombres de las guardas asociadas al controlador a
  	  través del tópico de la suscripción.
  	  \item Se ejecutan los métodos GuardProvider que corresponden a las guardas
  	  asociadas.
  	  \item El objeto HappeningControllerSincronizer setea en el monitor de
  	  Petri, para cada guarda asociada, el valor correspondiente que retornan los métodos GuardProvider.
  	  \item Se obtienen del tópico los nombres de las transiciones de aviso de
  	  finalización de ejecución.
  	  \item El objeto HappeningControllerSincronizer realiza disparos no perennes
  	  de las transiciones de aviso de finalización.
  	\end{enumerate}
  	
  	\begin{framed}
	\textbf{Nota:} Los advices de AspectJ implementados se ``tejen'' al código de
	usuario en tiempo de compilación (ver Sección~\ref{sec:aop_terminologia}).
	Dichos advices son ejecutados por el mismo hilo que el HappeningController que
	produce el joinpoint. De esta forma, el hilo que realiza los disparos de
	transición desde el objeto HappeningControllerSincronizer es el mismo que
	ejecuta el controlador de acción, permitiendo su sincronización.
	\end{framed}



\begin{figure}[H]
	\hspace{-2,90cm}
	\includegraphics[width=180mm]{secuencia_implementacion_ejecucion_happening_controller}
	\caption{Diagrama de Secuencia de la Ejecución Implementada de un
	HappeningController}
	\label{fig:diagrama_secuencia_implementacion_ejecucion_happening_controller}
\end{figure}

\end{itemize}

\section{Implementación de Tópicos}
Los tópicos, cuyo concepto se explica en la sección \ref{sec:diseno_topicos}, son definidos
por el usuario en un archivo de formato JSON, de la forma que se presenta a
continuación:
\begin{minted}{json}
[

  {
    "name":"custom_name_1",
    "permission":["my_p", "my_p2", "my_p3"],
    "setGuardCallback": [["g1","g3"],["g1","g2"],["g2","g3"]]
    "fireCallback":["fc_1","fc_2"]
  },
  
  {
    "name":"test",
    "permission":["my_permission", "my_permission2", "perm3", "p4"],
    "setGuardCallback": [["g1","guard2","g3"],["guardaCuatro"],[],["g5","g_6"]] 
  },
  
  {
    "name":"test",
    "permission":["my_transition"],
  },
  
  {
    "name":"topicN",
    "permission":["transition0"],
    "fireCallback":["t1","t2","tx"]
  }
]
\end{minted}

El path a este archivo es incluido en el software de usuario al momento de
inicializar \nombreFramework framework. Un parser interno del framework se
encarga de procesar este archivo JSON y asociar el tópico a los eventos
lógicos.

Una vez configurado y cargado en el sistema el archivo de tópicos, el usuario
puede usar el valor de ``name'' del tópico como identificador para suscribir
los controladores de acciones.


    \part{Conclusiones}
        \chapter{Conclusión}
            \section{Conclusión}

En este trabajo se realizó el diseño e implementación
\textit{\nombreFramework}, un framework para el desarrollo de sistemas
reactivos. Como resultado de este desarrollo se logró la centralización de la
gestión de los recursos, la concurrencia y la sincronización de hilos y,
además, se obtuvo la conducción del flujo de ejecución utilizando un modelo de
Red de Petri que implementa la lógica del sistema. A su vez, se desarrolló un
mecanismo de gestión de prioridad de ejecución de los hilos por medio de
políticas configurables por el usuario.

La centralización de gestión de recursos, manejo de concurrencia y
sincronización de hilos mencionada en el párrafo previo se obtuvo mediante:
\begin{itemize}
 \item La transformación del modelo de RdP en código interpretado.

 \item El desarrollo mecanismo de suscripción a disparos de transiciones.

 \item La implementación de colas de espera y suspensión de hilos.
\end{itemize}

Estas características fueron implementadas como parte de Java Petri Concurrency
 Monitor (JPCM), un monitor de concurrencia que ejecuta Redes de Petri haciendo
 uso de la ecuación generalizada desarrollada en
 \cite{Ecuacion_generalizada_LAC}.

Se obtuvo una arquitectura de framework que gestiona los eventos y mecanismos
de comunicación necesarios para desacoplar el modelo de RdP, el código de
usuario y el entorno.
Como resultado, se simplifica el diseño del software de usuario, el que queda
definido por:

\begin{itemize}
    \item El modelo de RdP
    \item El conjunto de acciones: Son porciones de código con una
    responsabilidad concreta y simple. Intercambian eventos con el entorno.
    \item El conjunto de eventos de acción: Contienen las reglas de traducción
    entre los eventos del entorno y eventos comprensibles para el modelo de
    RdP.
    \item El conjunto de suscripciones de acciones a eventos de acción
\end{itemize}

Se logró un diseño de framework no restrictivo sobre las herramientas
que ofrece el lenguaje de programación. En consecuencia, el usuario dispone
de todas las características de la programación orientada a objetos para el
diseño de su sistema.

Se implementó la inversión de control del framework mediante la utilización de
prácticas de \textit{Reflection} y \textit{Aspect Oriented Programming}. Como
resultado, el flujo de control del programa es responsabilidad del framework y
el código de usuario se centra en las funcionalidades concretas del sistema a implementar.

El código del framework se encuentra disponible de forma pública en repositorios
en la red (ver sección \ref{config_baboon_env}). En los repositorios mencionados
se encuentra también la documentación en formato \textit{Javadoc} y los casos de
test automatizados para las funcionalidades implementadas.

La utilización de Baboon Framework en conjunto con el proceso de diseño de
sistemas reactivos expuesto en \cite{Bentivegna-Ludemann} permite el diseño y
desarrollo de sistemas reactivos confiables, mantenibles y portables a
múltiples plataformas.

        \chapter{Trabajo Futuro}
            \section{Trabajo Futuro}
Durante el desarrollo del framework y su posterior utilización, surgieron
nuevos aspectos y mejoras a desarrollar en versiones futuras:
\begin{itemize}
  \item Performance: el uso de técnicas de reflection y de programación
  orientada a aspectos brinda la posibilidad de realizar la inversión de control
  y de ofrecer una interfaz de usuario amigable para el programador de sistemas
  reactivos, pero sacrifica performance. Es de interés investigar la existencia 
  de otras alternativas que permitan la implementación de la inversión de
  control de manera más performante.
  \item Utilización para desarrollo de sistemas distribuidos: analizar la
  posibilidad de centralizar la ejecución de la RdP y exponerla como un servicio
  para la coordinación de ejecución de sistemas remotos.
  \item Agregar soporte para otros dialectos de PNML: existen otros software de
  edición de RdP más potentes que TINA, cada uno con su dialecto de PNML. JPCM
  está preparado para agregar soporte para nuevos dialectos de forma simple.
  \item Soporte para ejecución de múltiples RdP: Se puede paralelizar la toma
  de decisiones mediante el uso de RdP jerárquicas o simplemente mediante el
  uso de múltiples RdP donde cada una modela una parte del sistema.
  \item Soporte para ejecución de otros tipos de modelo: en principio, dentro
  del monitor JPCM se podría ejecutar múltiples tipos de modelos (máquinas de
  estado finitas, máquinas de Turing, etc). Esta capacidad amplía la gama de
  usuarios interesados en la utilización de \textit{\nombreFramework \
  Framework}.
  \item Embeber el manejo e intercambio de datos entre procesos dentro del
  framework: Actualmente, el framework controla el flujo de control de las
  instrucciones del programa, pero el manejo de los datos es responsabilidad
  total del usuario. En la versión actual del framework, el manejo de la
  dinámica de los datos agrega una complejidad innecesaria
  al concepto de acción de software, ya que además de la acción concreta debe
  programarse el manejo del flujo de datos. La implementación de nuevas
  interfaces para el intercambio de datos entre acciones de software
  facilitaría aún más la creación de sistemas reactivos utilizando
  \textit{\nombreFramework \ Framework}.
  \item Agregar una interfaz para la liberación de recursos en tiempo de
  finalización del programa. Esta interfaz debe formar parte de
  \mint{java}|BaboonApplication|. Así, el usuario deberá implementar esta
  funcionalidad en un método que será ejecutado por \textit{\nombreFramework \
  Framework}.
\end{itemize}




    \part{Anexos}
        \chapter{Ejemplo de Uso de \nombreFramework \ Framework}
            \section {Sistema de Clasificación y Lavado de Botellas}
A lo largo de este capítulo se analiza un caso hipotético de uso.
Dicho caso sirve como ejemplo para demostrar las funcionalidades que ofrece el
framework. El mismo consiste en un sistema de clasificación y lavado de
botellas.
A continuación se describen sus características principales:

\begin{itemize}
  \item El ingreso de botellas no es controlado por el sistema.
  \item Las botellas pueden ser depositadas en la máquina en cualquier instante
  de tiempo, de forma asincrónica para el sistema.
  \item La clasificación de las botellas se realiza en un módulo que
  diferencia entre cerveza, gaseosa u otras.
  \item Puede haber un máximo de 20 botellas en la etapa de
  recepción y clasificación.
  \item La recepción de botellas tiene máxima prioridad siempre que pueda
  realizarse.
  \item Puede iniciarse la recepción de una botella antes de finalizar
  la recepción de la botella anterior.
  \item Mientras la máquina se encuentra en proceso de clasificación de
  botellas, no puede iniciar la recepción de una nueva botella.
  \item Cuando la máquina no puede recibir una botella, la misma queda en espera
  para ser procesada.
  \item El proceso completo de lavado para una botella de gaseosa consta de los
  siguientes pasos:
      \begin{itemize} 
        \item Lavado con agua a presión.
        \item Secado.
      \end{itemize}
  \item El proceso completo de lavado para una botella de cerveza consta de los
  siguientes pasos:
      \begin{itemize} 
        \item Lavado con agua a presión y detergente.
        \item Enjuague.
        \item Secado.
      \end{itemize}
  \item Si se inserta una botella de otro tipo, la misma es expulsada de la
    máquina sin lavar.
  \item La máquina sólo puede procesar una botella por vez en cada uno de sus
  módulos.
  \item Al finalizar el proceso de lavado las botellas son devueltas por la
  máquina, utilizando una salida independiente de la entrada.
  
\end{itemize}

\begin{figure}[H]
    \centering
    \includegraphics[width=120mm]{petri_lavadora_botellas}
    \caption{Modelo en Red de Petri de la Lógica de un Sistema de Clasificación y
    Lavado de Botellas}
    \label{fig:petri_lavadora_botellas}
\end{figure}

\subsection {Pasos para el desarrollo del sistema utilizando \nombreFramework}
\label{sec:pasos_desarrollo_lavadora_botellas}
Tras el modelado de la lógica del sistema en la Red de Petri, el
proceso de desarrollo del sistema puede descomponerse en una serie de
pasos a resolver.
De esta manera se comprende fácilmente el proceso para crear un sistema
utilizando el framework.

\begin{enumerate}
\item \textbf{Determinar los objetos para conformar el sistema:}\\
        Como en cualquier desarrollo orientado a objetos, uno de los pasos principales
        del diseño es la identificación de los objetos que intervienen en el sistema,
        para luego obtener las clases.
        En este caso se identifican los siguientes:
            \begin{itemize}
              \item Botellas insertadas.
              \item Máquina lavadora compuesta por:
              \begin{itemize}
                  \item Módulo de clasificación de botellas.
                  \item Módulo de lavado.
                  \item Módulo de enjuague.
                  \item Módulo de secado.
                  \item Módulo de expulsión de botellas incorrectas.
                  \item Colas donde se almacenan las botellas durante el proceso.
              \end{itemize}
            \end{itemize}

\item \textbf{Determinar las acciones de los objetos del sistema:}\\
            Otro paso principal en el desarrollo orientado a objetos es determinar las
            acciones que pueden realizar los objetos del sistema. Estas acciones serán
            embebidas en los métodos controladores de acciones de las clases
            desarrolladas.
            En este caso, las acciones son realizadas por la máquina lavadora, y se
            identifican las siguientes:
            \begin{itemize}
              \item Recibir botellas.
              \item Clasificar botellas.
              \item Lavar botellas de gaseosa.
              \item Secar botellas de gaseosa.
              \item Lavar botellas de cerveza.
              \item Enjuagar botellas de cerveza.
              \item Secar botellas de cerveza.
              \item Expulsar botellas incorrectas.
              \item Devolver botellas limpias.
            \end{itemize}

\item \textbf{Clasificar Happening Controllers y Task Controllers:}\\
            Los controladores de acciones pueden procesar eventos físicos de entrada
            (Eventos Happening) o emitir eventos físicos de salida (Eventos Task). Es
            importante determinar a cual grupo pertenece cada acción del sistema, para
            poder clasificarla como Happening Controller o Task Controller (ver
            sección~\ref{sec:controladores_de_acciones}).
            Esta clasificación se realiza en el software a través de las anotaciones Java
            @HappeningController y @TaskController. En el caso de ejemplo se distinguen:
            \begin{itemize}
              \item Happening Controllers:
                  \begin{itemize}
                    \item Recibir botellas.
                  \end{itemize}
              \item Task Controllers:
                  \begin{itemize}
                  \item Clasificar botellas.
                  \item Lavar botellas de gaseosa.
                  \item Secar botellas de gaseosa.
                  \item Lavar botellas de cerveza.
                  \item Enjuagar botellas de cerveza.
                  \item Secar botellas de cerveza.
                  \item Expulsar botellas incorrectas.
                  \item Devolver botellas limpias.
                  \end{itemize}
            \end{itemize}

\item \textbf{Manejo de guardas:}\\
            En determinadas ocasiones, es necesario representar
            estados externos que no están modelados en la red. Para lograr esto asociamos
            una variable booleana que representa el estado externo a una transición.
            Actuando la misma como un factor de sensibilización externo (ver
            sección~\ref{sec:guardas_monitor}).
            Para alterar el valor de una guarda, debe definirse un método del tipo Guard Provider
            asociado a dicha guarda, utilizando la anotación Java @GuardProvider en el
            código del software (ver sección~\ref{sec:guard_providers}).
    
\item \textbf{Definir la prioridad de las acciones:}\\
            A veces es necesario definir prioridades específicas para la ejecución de los
            controladores de acciones. Estas prioridades se manejan a través del monitor
            de redes de Petri, mediante la utilización de políticas de disparo de
            transición. El usuario puede utilizar las políticas proporcionadas por el
            framework (transición aleatoria o primera transición en línea) o definir una
            política a su medida, mediante la creación de una clase que hereda de
            TransitionsPolicy, donde se puede definir la siguiente transición a
            ejecutar de manera programática mediante la implementación del método
            \emph{which()} (ver sección~\ref{sec:politica_transiciones}). De acuerdo a
            las características descriptas para este caso, la transición
            ``recibirBotella'' debe tener máxima prioridad.

\item \textbf{Definir la cantidad de Hilos de ejecución:}\\
            Si bien no existe una respuesta absoluta, en términos generales se busca
            optimizar la performance minimizando la cantidad de hilos de ejecución
            posibles sin afectar al paralelismo del programa. Para lo cual se determinan
            aquellos Task Controllers secuenciales, que sólo requieren un único hilo para
            su ejecución, y se los agrupa en Complex Secuential Task Controllers (ver
            sección~\ref{sec:complex_secuential_task_controller}).
            Se distinguen los siguientes hilos de ejecución, donde se especifica quién
            está a cargo de la creación e inicialización del hilo (el control de la
            ejecución lo realiza el monitor en todos los casos):
            \begin{itemize}
              \item \textbf{Happening Controllers:}
                  \begin{enumerate}[label=\fbox{\arabic*}]
                    \item Recibir botellas. \emph{(A cargo del usuario).}
                  \end{enumerate}
              \item \textbf{Task Controllers:}
                  \begin{enumerate}[resume , label=\fbox{\arabic*}]
                  \item Clasificar botellas.\emph{(A cargo del framework).}
                  \item Expulsar botellas incorrectas. \emph{(A cargo del framework).}
                  \item Devolver botellas limpias. \emph{(A cargo del framework).}
                  \end{enumerate}
              \item \textbf{Complex Secuential Task Controllers:}
                  \begin{enumerate}[resume, label=\fbox{\arabic*}]
                    \item Proceso de lavado de botellas de gaseosa. \emph{(A cargo del
                    framework).}
                        \begin{itemize}
                          \item Lavar botellas de gaseosa.
                          \item Secar botellas de gaseosa.
                      \end{itemize}
                    \item Proceso de lavado de botellas de cerveza. \emph{(A cargo del
                    framework).}
                        \begin{itemize}
                          \item Lavar botellas de cerveza.
                          \item Enjuagar botellas de cerveza.
                          \item Secar botellas de cerveza.
                      \end{itemize}
                  \end{enumerate}
            \end{itemize}

\item \textbf{Determinar la relación entre la Red de Petri, los métodos y las clases:}\\
            La ejecución de los controladores de acciones está relacionada con el
            disparo de las transiciónes en la Red de Petri. Por eso, es necesario 
            relacionar dichas transiciones con los objetos y métodos correspondientes.
            Dicha tarea se realiza a través de un archivo JSON, configurable por el
            usuario, donde se definen los tópicos (ver
            sección~\ref{sec:relacion_evento_controlador}). Tras crear el archivo, el
            usuario debe suscribir cada objeto y método que sean parte de un controlador
            de acción a un tópico. Esta suscripción se realiza en el código del software.
\end{enumerate}

\subsection {Implementación del sistema utilizando \nombreFramework}

En esta sección se analiza una implementación del sistema de
lavado y clasificación de botellas. En esta implementación en particular, el
mundo físico externo al sistema es representado por una Interfaz Gráfica de
Usuario. 

En la figura~\ref{fig:diagrama_clases_lavadora_botellas} se observa que la
implementación de los objetos y métodos del sistema se basa en el paradigma de
programación orientada a objetos. El usuario es el encargado de definir las
interfaces y herencias que considere necesarias para implementar las
funcionalidades del sistema.
\begin{figure}[H]
    \centering
    \includegraphics[width=120mm]{diagrama_clases_lavadora_botellas}
    \caption{Diagrama de Clases de un Sistema de Clasificación y Lavado de
    Botellas.}
    \label{fig:diagrama_clases_lavadora_botellas}
\end{figure}

Aquellos métodos que forman parte de un controlador de acción deben
clasificarse y etiquetarse en el código con \emph{@HappeningController} o
\emph{@TaskController}. Por ejemplo:\\

\begin{minted}{java}
  @TaskController
  public void lavarGaseosa(){
    BotellaGaseosa botella = botellasGaseosaClasificadas.poll();
    lavar(botella);
    botellasGaseosaLavadas.add(botella);
  }
\end{minted}

\begin{minted}{java}
  @HappeningController
  public void recibirBotella(Botella botella){
    botellasInsertadas.add(botella);
  }
\end{minted}

El manejo de las guardas se realiza mediante la definición de métodos anotados
con \emph{@GuardProvider}. Por ejemplo:\\

\begin{minted}{java}
  @GuardProvider("isCerveza")
  public boolean isCerveza(){
    return ultimoTipoInsertado.equals(TipoBotella.CERVEZA);
  }
\end{minted}

El manejo de prioridades se realiza mediante la implementación de una política
de disparo de transiciones personalizada. En la definición de las
características del problema se estableció que la transición ``recibirBotella''
debe tener máxima prioridad. Para ello se creó una clase
\emph{InsertBottlesFirstPolicy} que hereda de \emph{TransitionsPolicy} e
implementa el método \emph{which()}. 

\begin{framed}
\textbf{Nota:} El usuario no debe instanciar un objeto de la clase
\emph{InsertBottlesFirstPolicy}. Al momento de crear el monitor de petri se
indica cuál es la clase que implementa la política de disparos, y el framework
se encarga luego de generar la instancia que será utilizada. De esta forma el
usuario no tiene acceso directo a la RdP desde el código, previniendo
modificaciones indeseadas del estado de la red.
\end{framed}


\begin{minted}{java}

public class InsertBottlesFirstPolicy extends TransitionsPolicy{
  public InsertBottlesFirstPolicy(PetriNet _petri) {
    super(_petri);
  }
  @Override
  public int which(boolean[] enabled) {
    Integer[] marking  = petri.getCurrentMarking();
    int index = petri.getTransition("recibirBotella").getIndex();
    if(enabled[index]){ //Si recibirBotella está habilitada.
      return index; //retornar el indice de recibirBotella.
    }
    else {
      //Si no, disparar la primer transición 
      //habilitada que encuentre.
      for(int i = 0; i < enabled.length; i++){
        if(enabled[i]){
        return i; 
        }
      }
    }
    return -1;
  }
}
\end{minted}

En la sección~\ref{sec:pasos_desarrollo_lavadora_botellas} se determinaron 6
hilos de ejecución. Cada uno de los hilos corresponde a un controlador de
acción. A su vez, cada controlador de acción debe suscribirse a
un tópico, por lo que es necesario definir en primer lugar dichos tópicos. Los
mismos se definen a través de un archivo JSON, de la siguiente forma:\\
\begin{minted}{json}
[
  {
    "name" : "recepcion_botella",
    "permission": ["recibirBotella"],
    "fireCallback": ["finRecepcionBot"]
  },
  {
    "name" : "clasificacion_botella",
    "permission": ["clasificarBotella"],
    "fireCallback": ["clasificadaGaseosa", "clasificadaCerveza", "clasificadaOtra"],
    "setGuardCallback" : [["guardCerveza","guardGaseosa","guardOtra"]]
  },
  {
    "name" : "lavado_cerveza",
    "permission" : ["lavarCerveza","enjuagarCerveza","secarCerveza"],
    "fireCallback": ["finSecadoCerveza"]
  },
  {
    "name": "lavado_gaseosa",
    "permission" : ["lavarGaseosa","secarGaseosa"],
    "fireCallback" : ["finSecadoGaseosa"]
  },
  {
    "name" : "expulsion_otra",
    "permission": ["expulsarBotella"],
    "fireCallback" : ["finExpulsion"]
  },
  {
    "name" : "devolucion_botella",
    "permission": ["devolverBotellaLimpia"],
    "fireCallback" : ["finDevolucion"]
  }
]
\end{minted}

Un tópico está relacionado con los eventos lógicos de
disparo de transiciones y de modificación de valor de guardas. Dicha relación
se determina a través del nombre de dichos componentes de la red de Petri, como
puede apreciarse en la figura~\ref{fig:topic_petri_relacion}.

\begin{figure}[H]
    \centering
    \includegraphics[width=120mm]{topic_petri_relacion}
    \caption{Relación entre un tópico y los componentes de la Red de Petri.}
    \label{fig:topic_petri_relacion}
\end{figure}

Otro aspecto que el usuario debe tener en cuenta en cuanto a los hilos, es la
creación e inicialización de aquellos que se encargan de ejecutar los
HappeningController. Esto es responsabilidad del cógido de usuario. Por
ejemplo, cuando el EventListener que escucha eventos del mouse en la GUI
detecta un evento, se crea un nuevo hilo que ejecuta el HappeningController
encargado de manejar dicho evento (\emph{recibirBotella()}). A
continuación se observa el código descripto:

\begin{minted}{java}
    @Override
    public void mouseClicked(MouseEvent arg0) {
      new Thread( () -> {
        if(botonCerveza.contains(arg0.getPoint())){
          maquina.recibirBotella(new BotellaCerveza());
        }
        else if(botonGaseosa.contains(arg0.getPoint())){
          maquina.recibirBotella(new BotellaGaseosa());
        }
        else if(botonOtra.contains(arg0.getPoint())){
          maquina.recibirBotella(new BotellaOtra());
        }
      }).start();
    }
\end{minted}


Finalmente, se define la clase \emph{AppSetup}, que implementa la interfaz
\emph{BaboonApplication} (ver sección~\ref{sec:componentes_baboon}), donde se
realizan las siguientes actividades:
\begin{itemize}
  \item Inicializar el monitor de RdP con el archivo PNML y la política de
  disparos.
  \item Añadir el archivo de tópicos para configurar los eventos de acción en
  el sistema.
  \item Declarar e inicializar los objetos del sistema.
  \item Suscribir los controladores de acción a los tópicos.
\end{itemize} 

A continuación se observa la implementación de dicha clase con
las actividades descriptas previamente:

\begin{minted}{java}
public class AppSetup implements BaboonApplication {
  private static Logger LOGGER = Logger.getLogger(AppSetup.class.getName());
  private MaquinaLavadora maquina;
  private View vista;

  @Override
  public void declare() {
    //Inicialización de los objetos del sistema
    maquina = new MaquinaLavadora();
    vista = new View(maquina);
    try {
      //Inicialización del core de petri.
      //Notar que se requiere el objeto Class de la política.
      BaboonFramework.createPetriCore("pnml/lavadoBotellas_v2.pnml",
          petriNetType.PLACE_TRANSITION, InsertBottlesFirstPolicy.class);
    } 
    catch (BadPolicyException e) {
      LOGGER.log(Level.SEVERE, "Error configurando la política de disparos.
                     La aplicación terminará ahora.", e);
      System.exit(1);
    }
    try {
      //Inicialización de los tópicos.
      BaboonFramework.addTopicsFile("topic/topics.json");
    } 
    catch (BadTopicsJsonFormat | NoTopicsJsonFileException e) {
      LOGGER.log(Level.SEVERE, "Error inicializando Baboon Framework.
                     La aplicación terminará ahora.", e);
      System.exit(1);
    }

    //Inicialización de thread de GUI
    new Thread( () -> {
      while(true){
        vista.drawScreen();
        try {
          Thread.sleep(25);
        } catch (InterruptedException e) {
           LOGGER.log(Level.INFO, "InterruptedException en thread de GUI", e);
        }
      }
    }).start();
  }

\end{minted}

\begin{minted}{java}
  @Override
  public void subscribe() {
    try {
      //Suscripción de un HappeningHandler.
      //Notar que se usa un objeto BotellaCerveza como parámetro.
      //El framework lo utiliza para determinar el método correcto a
      //utilizar como controlador de acción
      // En este caso, cualquier objeto que herede de
      //la clase abstracta Botella puede ser utilizado.
      BaboonFramework.subscribeToTopic("recepcion_botella", maquina,
          "recibirBotella", new BotellaCerveza());

      //Suscripción de TaskControllers Simples
      BaboonFramework.subscribeToTopic("clasificacion_botella", 
            maquina, "clasificarBotella");
      BaboonFramework.subscribeToTopic("expulsion_otra", 
            maquina, "expulsarOtra");
      BaboonFramework.subscribeToTopic("devolucion_botella", 
            maquina, "devolverBotellaLimpia");

      //Creación de ComplexSecuentialTaskControllers
      BaboonFramework.createNewComplexTask("lavarCerveza", "lavado_cerveza");
      BaboonFramework.createNewComplexTask("lavarGaseosa", "lavado_gaseosa");

      //Suscripción de TaskControllers a ComplexSecuentialTaskControllers
      BaboonFramework.appendTaskToComplexTask("lavarCerveza", 
            maquina, "lavarCerveza");
      BaboonFramework.appendTaskToComplexTask("lavarCerveza", 
            maquina, "enjuagarCerveza");
      BaboonFramework.appendTaskToComplexTask("lavarCerveza", 
            maquina, "secarCerveza");
      BaboonFramework.appendTaskToComplexTask("lavarGaseosa", 
            maquina, "lavarGaseosa");
      BaboonFramework.appendTaskToComplexTask("lavarGaseosa",
            maquina, "secarGaseosa");

    } catch (NotSubscribableException e) {
      LOGGER.log(Level.SEVERE, "Error suscribiendo acciones a Baboon Framework.
        La aplicación terminará ahora.", e);
      System.exit(1);
    }
  }
}
\end{minted}

Finalmente, para comenzar la ejecución del sistema, debe seleccionarse
\emph{BaboonFramework} como clase principal en las configuraciones:

\begin{figure}[H]
    \centering
    \includegraphics[width=120mm]{baboon_main}
    \caption{Clase Principal de un Sistema Desarrollado con Baboon Framework.}
    \label{fig:baboon_main}
\end{figure}


    \bibliography{./bibliografia}
	\bibliographystyle{alpha}
\end{document}